\chapterimage{ss-observers.jpg}

\chapter{State-space observers}

State-space observers are used to estimate unmeasured \glspl{state}.

\section{Luenberger observer}

\begin{align}
  \dot{\hat{\mtx{x}}} &= \mtx{A}\hat{\mtx{x}} + \mtx{B}\mtx{u} +
    \mtx{L} (\mtx{y} - \hat{\mtx{y}}) \label{eq:s_obsv_x} \\
  \hat{\mtx{y}} &= \mtx{C}\hat{\mtx{x}} + \mtx{D}\mtx{u} \label{eq:s_obsv_y}
\end{align}

\begin{align}
  \hat{\mtx{x}}_{k+1} &= \mtx{A}\hat{\mtx{x}}_k + \mtx{B}\mtx{u}_k +
    \mtx{L}(\mtx{y}_k - \hat{\mtx{y}}_k) \label{eq:z_obsv_x} \\
  \hat{\mtx{y}}_k &= \mtx{C}\hat{\mtx{x}}_k + \mtx{D}\mtx{u}_k
    \label{eq:z_obsv_y} \\ \nonumber
\end{align}

\begin{center}
  \renewcommand{\arraystretch}{1.3}
  \begin{tabulary}{\linewidth}{LLLL}
    $\mtx{A}$ & system matrix      & $\hat{\mtx{x}}$ & state estimate vector \\
    $\mtx{B}$ & input matrix       & $\mtx{u}$ & input vector \\
    $\mtx{C}$ & output matrix      & $\mtx{y}$ & output vector \\
    $\mtx{D}$ & feedthrough matrix & $\hat{\mtx{y}}$ & output estimate vector \\
    $\mtx{L}$ & estimator gain matrix & & \\
  \end{tabulary}
\end{center}

\begin{table}[ht]
  \renewcommand{\arraystretch}{1.5}
  \centering
  \begin{tabular}{|ll|ll|}
    \hline
    \rowcolor{headingbg}
    \textbf{Matrix} & \textbf{Rows $\times$ Columns} &
    \textbf{Matrix} & \textbf{Rows $\times$ Columns} \\
    \hline
    $\mtx{A}$ & states $\times$ states & $\hat{\mtx{x}}$ & states $\times$ 1 \\
    $\mtx{B}$ & states $\times$ inputs & $\mtx{u}$ & inputs $\times$ 1 \\
    $\mtx{C}$ & outputs $\times$ states & $\mtx{y}$ & outputs $\times$ 1 \\
    $\mtx{D}$ & outputs $\times$ inputs & $\hat{\mtx{y}}$ & outputs $\times$ 1 \\
    $\mtx{L}$ & states $\times$ outputs &  &  \\
    \hline
  \end{tabular}
  \caption{Luenberger observer matrix dimensions}
  \label{tab:luenberger_matrix_dims}
\end{table}

Using an estimator forfeits the performance guarantees from earlier, but the
responses are still generally very good.

A Luenberger observer combines the prediction and update steps of an estimator.
To run them separately, use the following equations instead.

\begin{align}
  \text{Predict step} \nonumber \\
  \hat{\mtx{x}}_{k+1}^- &= \mtx{A}\hat{\mtx{x}}_k^- + \mtx{B}\mtx{u}_k \\
  \text{Update step} \nonumber \\
  \hat{\mtx{x}}_{k+1}^+ &= \hat{\mtx{x}}_{k+1}^- + \mtx{A}^{-1}\mtx{L}
    (\mtx{y}_k - \hat{\mtx{y}}_k) \\
  \hat{\mtx{y}}_k &= \mtx{C} \hat{\mtx{x}}_k^-
\end{align}

See appendix \ref{subsec:app-luenberger-separate} for a derivation.

\section{LQE}

"LQE" stands for "Linear-Quadratic Estimator". Similar to LQR, it places the
estimator poles such that it minimizes the sum of squares of the error. The
Kalman filter is an example of one of these.

\section{Kalman filter}
\index{Kalman filter}

Read \url{http://www.bzarg.com/p/how-a-kalman-filter-works-in-pictures/} for a
graphical introduction to Kalman filters.

The following is a Kalman filter for the $k^{th}$ timestep. The current predict
step and current update step are shown.

\begin{align}
  \text{Predict step} \nonumber \\
  \hat{\mtx{x}}_{k+1}^- &= \mtx{\Phi}\hat{\mtx{x}}_k + \mtx{B} \mtx{u}_k
    \label{eq:pre1_x} \\
  \mtx{P}_{k+1}^- &= \mtx{\Phi} \mtx{P}_k^- \mtx{\Phi}^T +
    \mtx{\Gamma}\mtx{Q}\mtx{\Gamma}^T \\
  \text{Update step} \nonumber \\
  \mtx{K}_{k+1} &= \mtx{P}_{k+1}^- \mtx{H}^T (\mtx{H}\mtx{P}_{k+1}^- \mtx{H}^T +
    \mtx{R})^{-1} \\
  \hat{\mtx{x}}_{k+1}^+ &= \hat{\mtx{x}}_{k+1}^- + \mtx{K}_{k+1}(\mtx{y}_{k+1} -
    \mtx{H} \hat{\mtx{x}}_{k+1}^-) \label{eq:post1_x} \\
  \mtx{P}_{k+1}^+ &= (\mtx{I} - \mtx{K}_{k+1}\mtx{H})\mtx{P}_{k+1}^-
\end{align}

\begin{center}
  \renewcommand{\arraystretch}{1.3}
  \begin{tabulary}{\linewidth}{LLLL}
    $\mtx{\Phi}$ & system matrix & $\hat{\mtx{x}}$ & state estimate vector \\
    $\mtx{B}$ & input matrix            & $\mtx{u}$ & input vector \\
    $\mtx{H}$ & measurement matrix      & $\mtx{y}$ & output vector \\
    $\mtx{P}$ & error covariance matrix & $\mtx{Q}$ & process noise covariance
      matrix \\
    $\mtx{K}$ & Kalman gain matrix & $\mtx{R}$ & measurement noise covariance
      matrix \\
    $\mtx{\Gamma}$ & process noise intensity vector &
  \end{tabulary}
\end{center}

where a superscript of minus denotes \textit{a priori} and plus denotes
\textit{a posteriori} estimate (before and after update respectively).
$\mtx{\Phi}$ is replaced with $\mtx{A}$ for continuous systems.

\begin{table}[ht]
  \renewcommand{\arraystretch}{1.5}
  \centering
  \begin{tabular}{|ll|ll|}
    \hline
    \rowcolor{headingbg}
    \textbf{Matrix} & \textbf{Rows $\times$ Columns} &
    \textbf{Matrix} & \textbf{Rows $\times$ Columns} \\
    \hline
    $\mtx{\Phi}$ & states $\times$ states & $\hat{\mtx{x}}$ & states $\times$ 1
      \\
    $\mtx{B}$ & states $\times$ inputs & $\mtx{u}$ & inputs $\times$ 1 \\
    $\mtx{H}$ & outputs $\times$ states & $\mtx{y}$ & outputs $\times$ 1 \\
    $\mtx{P}$ & states $\times$ states & $\mtx{Q}$ & states $\times$ states \\
    $\mtx{K}$ & states $\times$ outputs & $\mtx{R}$ & outputs $\times$ outputs
      \\
    $\mtx{\Gamma}$ & states $\times$ 1 &  &  \\
    \hline
  \end{tabular}
  \caption{Kalman filter matrix dimensions}
  \label{tab:kf_matrix_dims}
\end{table}

Unknown states in a Kalman filter are generally represented by a Wiener
(pronounced VEE-ner) process. This process has the property that its variance
increases linearly with time $t$. We'll skip all the probability derivations
here, but given two data points with associated variances represented by
Gaussian distribution, the information can be optimally combined into a third
Gaussian distribution with a mean value and variance. The expected value of $x$
given a measurement $z_1$ is

\begin{equation}
  E[x|z_1] = \mu = \frac{\sigma_0^2}{\sigma_0^2 + \sigma^2}z_1 +
    \frac{\sigma^2}{\sigma_0^2 + \sigma^2}x_0
\end{equation}

The variance of $x$ given $z_1$ is

\begin{equation}
  E[(x - \mu)^2|z_1] = \frac{\sigma^2 \sigma_0^2}{\sigma_0^2 + \sigma^2}
\end{equation}

The expected value, which is also the maximum likelihood value, is the linear
combination of the prior expected (maximum likelihood) value and the
measurement. The expected value is a reasonable estimator of $x$.

\begin{align}
  \hat{x} &= E[x|z_1] = \frac{\sigma_0^2}{\sigma_0^2 + \sigma^2}z_1 +
    \frac{\sigma^2}{\sigma_0^2 + \sigma^2}x_0 \\
  \hat{x} &= w_1 z_1 + w_2 x_0 \nonumber
\end{align}

Note that the weights $w_1$ and $w_2$ sum to 1. When the prior (i.e., prior
knowledge of state) is uninformative (a large variance)

\begin{align}
  w_1 &= \lim_{\sigma_0^2 \to 0} \frac{\sigma_0^2}{\sigma_0^2 + \sigma^2} = 0 \\
  w_2 &= \lim_{\sigma_0^2 \to 0} \frac{\sigma^2}{\sigma_0^2 + \sigma^2} = 1
\end{align}

and $\hat{x} = z_1$. That is, the weight is on the observations and the estimate
is equal to the measurement.

Let us assume we have a \gls{model} providing an almost exact prior for $x$. In
that case, $\sigma_0^2$ approaches 0 and

\begin{align}
  w_1 &= \lim_{\sigma_0^2 \to 0} \frac{\sigma_0^2}{\sigma_0^2 + \sigma^2} = 1 \\
  w_2 &= \lim_{\sigma_0^2 \to 0} \frac{\sigma^2}{\sigma_0^2 + \sigma^2} = 0
\end{align}

The Kalman filter uses this optimal fusion as the basis for its operation. See
the Wikipedia page on Kalman filters \cite{bib:kalman_filter} for a more
thorough explanation of the math involved and a derivation of the equations.

\subsection{Equations to model}

The following example system will be used to describe how to define and
initialize the matrices for a Kalman filter.

A robot is between two parallel walls. It starts driving from one wall to the
other at a velocity of $0.8 cm/s$ and uses ultrasonic sensors to provide noisy
measurements of the distances to the walls in front of and behind it. To
estimate the distance between the walls, we will define three states: robot
position, robot velocity, and distance between the walls.

\begin{align}
  x_{k+1} &= x_k + v_k \Delta T \\
  v_{k+1} &= v_k \\
  x_{k+1}^w &= x_k^w
\end{align}

This can be converted to the following state-space \gls{model}.

\begin{equation}
  \mtx{x}_k = \left[
  \begin{array}{c}
    x_k \\
    v_k \\
    x_k^w
  \end{array} \right]
\end{equation}

\begin{equation}
  \mtx{x}_{k+1} = \left[
  \begin{array}{ccc}
    1 & 1 & 0 \\
    0 & 0 & 0 \\
    0 & 0 & 1
  \end{array} \right] \mtx{x}_k + \left[
  \begin{array}{c}
    0 \\
    0.8 \\
    0
  \end{array} \right] + \left[
  \begin{array}{c}
    0 \\
    0.1 \\
    0
  \end{array} \right] w_k
\end{equation}

where the Gaussian random variable $w_k$ has zero mean and variance 1. The
observation \gls{model} is

\begin{equation}
  \mtx{y}_k = \left[
  \begin{array}{ccc}
    1 & 0 & 0 \\
    -1 & 0 & 1
  \end{array} \right] \mtx{x}_k + \theta_k
\end{equation}

where the covariance matrix of Gaussian measurement noise $\theta$ is a
$2 \times 2$ matrix with both diagonals $10 cm^2$.

The state vector is usually initialized using the first measurement or two. The
covariance matrix entries are assigned by calculating the covariance of the
expressions used when assigning the state vector. Let $k = 2$.

\begin{align}
  \mtx{Q} &= \left[1\right] \\
  \mtx{R} &= \left[
  \begin{array}{cc}
    10 & 0 \\
    0 & 10
  \end{array} \right] \\
  \hat{\mtx{x}} &= \left[
  \begin{array}{c}
    \mtx{y}_{k,1} \\
    (\mtx{y}_{k,1} - \mtx{y}_{k-1,1})/dt \\
    \mtx{y}_{k,1} + \mtx{y}_{k,2}
  \end{array} \right] \\
  \mtx{P} &= \left[
  \begin{array}{ccc}
    10 & 10/dt & 10 \\
    10/dt & 20/dt^2 & 10/dt \\
    10 & 10/dt & 20
  \end{array} \right]
\end{align}

\subsection{Initial conditions}

To fill in the $\mtx{P}$ matrix, we calculate the covariance of each combination
of state variables. The resulting value is a measure of how much those variables
are correlated. Due to how the covariance calculation works out, the covariance
between two variables is the sum of the variance of matching terms which aren't
constants multiplied by any constants the two have. If no terms match, the
variables are uncorrelated and the covariance is zero.

In $\mtx{P}_{11}$, the terms in $\mtx{x}_1$ correlate with itself. Therefore,
$\mtx{P}_{11}$ is $\mtx{x}_1$'s variance, or $\mtx{P}_{11} = 10$. For
$\mtx{P}_{21}$, One term correlates between $\mtx{x}_1$ and $\mtx{x}_2$, so
$\mtx{P}_{21} = \frac{10}{dt}$. The constants from each are simply multiplied
together. For $\mtx{P}_{22}$, both measurements are correlated, so the variances
add together. Therefore, $\mtx{P}_{22} = \frac{20}{dt^2}$. It continues in this
fashion until the matrix is filled up. Order doesn't matter for correlation, so
the matrix is symmetric.

\subsection{Selection of priors}

Choosing good priors is important for a well performing filter, even if little
information is known. This applies to both the measurement noise and the noise
\gls{model}. The act of giving a state variable a large variance means you know
something about the system. Namely, you aren't sure whether your initial guess
is close to the true state. If you make a guess and specify a small variance,
you are telling the filter that you are very confident in your guess. If that
guess is incorrect, it will take the filter a long time to move away from your
guess to the true value.

\subsection{Covariance selection}

While one could assume no correlation between the state variables and set the
covariance matrix entries to zero, this may not reflect reality. The Kalman
filter is still guarenteed to converge to the steady-state covariance after an
infinite time, but it will take longer than otherwise.

\subsection{Noise model selection}

We typically use a Gaussian distribution for the noise \gls{model} because the
sum of many independent random variables produces a normal distribution by the
central limit theorem. Kalman filters only require that the noise has a zero
mean. If the true value has an equal probability of being within a certain
range, use a uniform distribution instead. Each of these communicates
information regarding what you know about a system in addition to what you do
not.

\section{Steady-state error covariance matrix}

One may have noticed that the error covariance matrix can be updated
independently of the rest of the model. The error covariance matrix tends
toward a steady-state value, and this matrix can be obtained via the discrete
algebraic Ricatti equation. This can then be used to compute a steady-state
Kalman gain.

Snippet \ref{lst:kalman} computes the steady-state matrices for a Kalman
filter.

\begin{snippet}
  \caption{Steady-state Kalman gain and error covariance matrices calculation in Python}
  \label{lst:kalman}
  \includecode[Python]{code/kalman.py}
\end{snippet}

\section{Kalman filter as Luenberger observer}

A Kalman filter can be represented as a Luenberger observer by letting
$\mtx{C} = \mtx{H}$ and $\mtx{L} = \mtx{A} \mtx{K}_k$ (see appendix
\ref{subsec:app-kalman-luenberger}). The eigenvalues of the Kalman filter are

\begin{equation}
  eig(\mtx{A}(\mtx{I} - \mtx{K}_k\mtx{H}))
\end{equation}
