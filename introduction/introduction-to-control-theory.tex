\chapterimage{introduction-to-control-theory.jpg}{Road near walking trail off of Rice Ranch Road in Santa Maria, CA}

\chapter{Introduction to control theory}

Control systems are all around us and we interact with them daily. A small list
of ones you may have seen includes heaters and air conditioners with
thermostats, cruise control and the anti-lock braking system (ABS) on
automobiles, and fan speed modulation on modern laptops. \Glspl{control system}
monitor or control the behavior of \glspl{system} like these and may consist of
humans controlling them directly (manual control), or of only machines
(automatic control).

How can we prove closed-loop \glspl{controller} on an autonomous car, for
example, will behave safely and meet the desired performance specifications in
the presence of uncertainty? Control theory is an application of algebra and
geometry used to analyze and predict the behavior of \glspl{system}, make them
respond how we want them to, and make them \glslink{robustness}{robust} to
\glspl{disturbance} and uncertainty.

Controls engineering is, put simply, the engineering process applied to control
theory. As such, it's more than just applied math. While control theory has some
beautiful math behind it, controls engineering is an engineering discipline like
any other that is filled with trade-offs. The solutions control theory gives
should always be sanity checked and informed by our performance specifications.
We don't need to be perfect; we just need to be good enough to meet our
specifications (see section \ref{sec:the_mindset_of_an_egoless_engineer} for
more on engineering).

\renewcommand*{\chapterpath}{\partpath/introduction-to-control-theory}
\section{Nomenclature}

Most resources for advanced engineering topics assume a level of knowledge well
above that which is necessary. Part of the problem is the use of jargon. While
it efficiently communicates ideas to those within the field, new people who
aren't familiar with it are lost. See the glossary for a list of words and
phrases commonly used in control theory, their origins, and their meaning. Links
to the glossary are provided for certain words throughout the book and will use
\textcolor{glscolor}{this color}.

Table \ref{tab:plant_v_controller} describes how the terms \gls{input} and
\gls{output} apply to \glspl{plant} versus \glspl{controller} and what letters
are commonly associated with each when working with them. Namely, that the terms
\gls{input} and \gls{output} are defined with respect to the \gls{plant}, not
the \gls{controller}.

\begin{booktable}
  \begin{tabular}{|l|ll|}
    \hline
    \rowcolor{headingbg}
    & \textbf{Plant} & \textbf{Controller} \\
    \hline
    Inputs & $u(t)$ & $r(t)$, $y(t)$ \\
    Outputs & $y(t)$ & $u(t)$ \\
    \hline
  \end{tabular}
  \caption{Plant versus controller nomenclature}
  \label{tab:plant_v_controller}
\end{booktable}

