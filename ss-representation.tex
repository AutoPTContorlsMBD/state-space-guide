\chapterimage{ss-representation.jpg}{OPERS field at UCSC}

\chapter{State-space representation}

\begin{remark}
  Chapters from here on use Python Control to demonstrate the concepts discussed
  and perform the complex math required. See appendix
  \ref{ch:app-installing-python-control} for how to install it.
\end{remark}

State-space representation models \glspl{system} as a set of \gls{state}, input,
and output variables related by first-order differential equations that describe
how the system's state changes over time given the current \glspl{state} and
inputs.

\section{Benefits over classical output-based control}

The state-space method provides a more convenient and compact way to model and
analyze \glspl{system} with multiple inputs and outputs. For a system with $p$
inputs and $q$ outputs, we would have to write $q \times p$ Laplace transforms
to represent it. Not only is the resulting algebra unwieldy, but it only works
for linear systems. Including nonzero initial conditions complicates the algebra
even more. State-space representation uses the time domain instead of the
Laplace domain, so it can model nonlinear systems and trivially supports nonzero
initial conditions. With that said, this book only covers analysis and control
of linear systems.

Students are still taught classical control first because it provides a
framework within which to understand the results we get from the fancy
mathematical machinery of modern control.

\section{What is a linear dynamical system?}

A dynamical system is a system whose motion varies according to a set of
differential equations. A dynamical system is considered \textit{linear} if the
differential equations describing its dynamics consist only of linear operators.
Linear operators are things like constant gain multiplications, derivatives, and
integrals. You can define reasonably accurate linear models for pretty much
everything you'll see in FRC with just those relations.

But let's say you have a DC brushed motor hooked up to a power supply and you
applied a constant voltage to it from rest. The motor approaches a steady-state
angular velocity, but the shape of the angular velocity curve over time isn't a
line. In fact, it's a decaying exponential curve akin to

\begin{equation*}
  \omega = \omega_{max}\left(1 - e^{-t}\right)
\end{equation*}

where $\omega$ is the angular velocity and $\omega_{max}$ is the maximum angular
velocity. If the motor behaves linearly, then why is this?

Remember that linearity refers to a system's equations of motion, not its
time-domain response. The equation defining the motor's change in angular
velocity over time looks like

\begin{equation*}
  \dot{\omega} = -a\omega + bV
\end{equation*}

where $\dot{\omega}$ is the derivative of $\omega$ with respect to time, $V$ is
the input voltage, and $a$ and $b$ are constants specific to the motor. This
equation, unlike the one shown before, is actually linear because it only
consists of multiplications and additions relating the input $V$ and current
state $\omega$.

Also of note is that the relation between the input voltage and the angular
velocity of the output shaft is a linear regression. You'll see why if you model
a DC brushed motor as a voltage source and generator producing back-EMF (in the
equation above, $bV$ corresponds to the voltage source and $-a\omega$
corresponds to the back-EMF). As you increase the input voltage, the back-EMF
increases linearly with the motor's angular velocity. If there was a friction
term that varied with the angular velocity squared (air resistance is one
example), the relation from input to output would be a curve. Friction that
scales with just the angular velocity would result in a lower maximum angular
velocity, but because that term can be lumped into the back-EMF term, the
response is still linear.

\section{What is state-space?}

Recall from last chapter that 2D space has two axes, $x$ and $y$. We represent
locations within this space as a pair of numbers packaged in a vector, and each
coordinate is a measure of how far to move along the corresponding axis.
State-space is a Cartesian coordinate system with an axis for each \gls{state}
variable, and we represent locations within it the same way we do for 2D space:
with a list of numbers in a vector. Each element in the vector corresponds to a
\gls{state} of the \gls{system}.

In addition to the \gls{state}, inputs and outputs are represented as vectors.
Since the mapping from the current states and inputs to the change in state is a
system of equations, it's natural to write it in matrix form.

\section{State-space notation}

Below are the continuous and discrete versions of state-space notation.

\begin{definition}[State-space notation]%
  \begin{align}
    \dot{\mtx{x}} &= \mtx{A}\mtx{x} + \mtx{B}\mtx{u} \label{eq:ss_ctrl_x} \\
    \mtx{y} &= \mtx{C}\mtx{x} + \mtx{D}\mtx{u} \label{eq:ss_ctrl_y} \\
    \nonumber \\
    \mtx{x}_{k+1} &= \mtx{A}\mtx{x}_k + \mtx{B}\mtx{u}_k \label{eq:ssz_ctrl_x} \\
    \mtx{y}_{k+1} &= \mtx{C}\mtx{x}_k + \mtx{D}\mtx{u}_k \label{eq:ssz_ctrl_y}
  \end{align}

  \begin{figurekey}
    \begin{tabulary}{\linewidth}{LLLL}
      $\mtx{A}$ & system matrix      & $\mtx{x}$ & state vector \\
      $\mtx{B}$ & input matrix       & $\mtx{u}$ & input vector \\
      $\mtx{C}$ & output matrix      & $\mtx{y}$ & output vector \\
      $\mtx{D}$ & feedthrough matrix &  &  \\
    \end{tabulary}
  \end{figurekey}
\end{definition}

\begin{booktable}
  \begin{tabular}{|ll|ll|}
    \hline
    \rowcolor{headingbg}
    \textbf{Matrix} & \textbf{Rows $\times$ Columns} &
    \textbf{Matrix} & \textbf{Rows $\times$ Columns} \\
    \hline
    $\mtx{A}$ & states $\times$ states & $\mtx{x}$ & states $\times$ 1 \\
    $\mtx{B}$ & states $\times$ inputs & $\mtx{u}$ & inputs $\times$ 1 \\
    $\mtx{C}$ & outputs $\times$ states & $\mtx{y}$ & outputs $\times$ 1 \\
    $\mtx{D}$ & outputs $\times$ inputs &  &  \\
    \hline
  \end{tabular}
  \caption{State-space matrix dimensions}
  \label{tab:ss_matrix_dims}
\end{booktable}

In the continuous case, the change in state and the output are linear
combinations of the state vector and the input vector. The $\mtx{A}$ and
$\mtx{B}$ matrices are used to map the state vector $\mtx{x}$ and the input
vector $\mtx{u}$ to a change in the state vector $\dot{\mtx{x}}$. The $\mtx{C}$
and $\mtx{D}$ matrices are used to map the state vector $\mtx{x}$ and the input
vector $\mtx{u}$ to an output vector $\mtx{y}$.

\section{Controllability}
\index{Controller design!controllability}

State controllability implies that it is possible -- by admissible inputs -- to
steer the \glspl{state} from any initial value to any final value within some
finite time window.

\begin{theorem}[Controllability]
  A continuous \gls{time-invariant} linear state-space \gls{model} is
  controllable if and only if

  \begin{equation}
    \text{rank} \left(
    \begin{bmatrix}
      \mtx{B} & \mtx{A}\mtx{B} & \mtx{A}^2\mtx{B} & \cdots &
      \mtx{A}^{n-1}\mtx{B}
    \end{bmatrix}
    \right) = n
    \label{eq:ctrl_rank}
  \end{equation}

  where rank is the number of linearly independent rows in a matrix and $n$ is
  the number of \gls{state} variables.
\end{theorem}

The matrix in equation (\ref{eq:ctrl_rank}) being rank-deficient means the
inputs cannot apply transforms along all axes in the state-space; the
transformation the matrix represents is collapsed into a lower dimension.

\section{Observability}
\index{Controller design!observability}

Observability is a measure for how well internal \glspl{state} of a \gls{system}
can be inferred by knowledge of its external outputs. The observability and
controllability of a \gls{system} are mathematical duals (i.e., as
controllability proves that an input is available that brings any initial
\gls{state} to any desired final \gls{state}, observability proves that knowing
an output trajectory provides enough information to predict the initial
\gls{state} of the \gls{system}).

\begin{theorem}[Observability]
  A continuous \gls{time-invariant} linear state-space \gls{model} is observable
  if and only if

  \begin{equation}
    \text{rank} \left(
    \begin{bmatrix}
      C \\
      CA \\
      \vdots \\
      CA^{n-1}
    \end{bmatrix}\right) = n \label{eq:obsv_rank}
  \end{equation}

  where rank is the number of linearly independent rows in a matrix and $n$ is
  the number of \gls{state} variables.
\end{theorem}

The matrix in equation (\ref{eq:obsv_rank}) being rank-deficient means the
outputs do not contain contributions from every state. That is, not all states
are mapped to a linear commbination in the output. Therefore, the outputs alone
are insufficient to estimate all the states.

\section{Eigenvalues in state-space}
\index{Stability!eigenvalues}

The eigenvalues of the system matrix can be used to determine the stability of a
\gls{system}.

We'd like to know whether the \gls{system} defined by equation
(\ref{eq:ssz_ctrl_x}) operating with the \gls{control law}
$\mtx{u}_k = \mtx{K}(\mtx{r}_k - \mtx{x}_k)$ converges to the \gls{reference}
$\mtx{r}_k$.

\begin{align}
  \mtx{x}_{k+1} &= \mtx{A}\mtx{x}_k + \mtx{B}\mtx{u}_k \nonumber \\
  \mtx{x}_{k+1} &= \mtx{A}\mtx{x}_k + \mtx{B}(\mtx{K}(\mtx{r}_k - \mtx{x}_k))
    \nonumber \\
  \mtx{x}_{k+1} &= \mtx{A}\mtx{x}_k + \mtx{B}\mtx{K}\mtx{r}_k -
    \mtx{B}\mtx{K}\mtx{x}_k \nonumber \\
  \mtx{x}_{k+1} &= \mtx{A}\mtx{x}_k - \mtx{B}\mtx{K}\mtx{x}_k +
    \mtx{B}\mtx{K}\mtx{r}_k \nonumber \\
  \mtx{x}_{k+1} &= (\mtx{A} - \mtx{B}\mtx{K})\mtx{x}_k +
    \mtx{B}\mtx{K}\mtx{r}_k \label{eq:ctrl_eig_calc}
\end{align}

For equation (\ref{eq:ctrl_eig_calc}) to have a bounded output, the eigenvalues
of $\mtx{A} - \mtx{B}\mtx{K}$ must be within the unit circle.

This derivation can be performed for a \gls{state} estimator as well to
determine whether the \gls{state} estimate converges to the true \gls{state}.
Plugging equation (\ref{eq:z_obsv_y}) into equation (\ref{eq:z_obsv_x}) gives

\begin{align*}
  \hat{\mtx{x}}_{k+1} &= \mtx{A}\hat{\mtx{x}}_k + \mtx{B}\mtx{u}_k +
    \mtx{L} (\mtx{y}_k - \hat{\mtx{y}}_k) \\
  \hat{\mtx{x}}_{k+1} &= \mtx{A}\hat{\mtx{x}}_k + \mtx{B}\mtx{u}_k +
    \mtx{L} (\mtx{y}_k - (\mtx{C}\hat{\mtx{x}}_k + \mtx{D}\mtx{u}_k)) \\
  \hat{\mtx{x}}_{k+1} &= \mtx{A}\hat{\mtx{x}}_k + \mtx{B}\mtx{u}_k +
    \mtx{L} (\mtx{y}_k - \mtx{C}\hat{\mtx{x}}_k - \mtx{D}\mtx{u}_k)
\end{align*}

Plugging in equation (\ref{eq:ssz_ctrl_y}) gives

\begin{align*}
  \hat{\mtx{x}}_{k+1} &= \mtx{A}\hat{\mtx{x}}_k + \mtx{B}\mtx{u}_k +
    \mtx{L}((\mtx{C}\mtx{x}_k + \mtx{D}\mtx{u}_k) - \mtx{C}\hat{\mtx{x}}_k -
    \mtx{D}\mtx{u}_k) \\
  \hat{\mtx{x}}_{k+1} &= \mtx{A}\hat{\mtx{x}}_k + \mtx{B}\mtx{u}_k +
    \mtx{L}(\mtx{C}\mtx{x}_k + \mtx{D}\mtx{u}_k - \mtx{C}\hat{\mtx{x}}_k -
    \mtx{D}\mtx{u}_k) \\
  \hat{\mtx{x}}_{k+1} &= \mtx{A}\hat{\mtx{x}}_k + \mtx{B}\mtx{u}_k +
    \mtx{L}(\mtx{C}\mtx{x}_k - \mtx{C}\hat{\mtx{x}}_k) \\
  \hat{\mtx{x}}_{k+1} &= \mtx{A}\hat{\mtx{x}}_k + \mtx{B}\mtx{u}_k +
    \mtx{L}\mtx{C}(\mtx{x}_k - \hat{\mtx{x}}_k)
\end{align*}

Let $E_k = \mtx{x}_k - \hat{\mtx{x}}_k$ be the error in the estimate
$\hat{\mtx{x}}_k$.

\begin{equation*}
  \hat{\mtx{x}}_{k+1} = \mtx{A}\hat{\mtx{x}}_k + \mtx{B}\mtx{u}_k +
    \mtx{L}\mtx{C}\mtx{E}_k
\end{equation*}

Subtracting this from equation (\ref{eq:ssz_ctrl_x}) gives

\begin{align}
  \mtx{x}_{k+1} - \hat{\mtx{x}}_{k+1} &= \mtx{A}\mtx{x}_k + \mtx{B}\mtx{u}_k -
    (\mtx{A}\hat{\mtx{x}}_k + \mtx{B}\mtx{u}_k +
     \mtx{L}\mtx{C}\mtx{E}_k) \nonumber \\
  \mtx{E}_{k+1} &= \mtx{A}\mtx{x}_k + \mtx{B}\mtx{u}_k -
    (\mtx{A}\hat{\mtx{x}}_k + \mtx{B}\mtx{u}_k + \mtx{L}\mtx{C}\mtx{E}_k)
    \nonumber \\
  \mtx{E}_{k+1} &= \mtx{A}\mtx{x}_k + \mtx{B}\mtx{u}_k -
    \mtx{A}\hat{\mtx{x}}_k - \mtx{B}\mtx{u}_k - \mtx{L}\mtx{C}\mtx{E}_k
    \nonumber \\
  \mtx{E}_{k+1} &= \mtx{A}\mtx{x}_k - \mtx{A}\hat{\mtx{x}}_k -
    \mtx{L}\mtx{C}\mtx{E}_k \nonumber \\
  \mtx{E}_{k+1} &= \mtx{A}(\mtx{x}_k - \hat{\mtx{x}}_k) -
    \mtx{L}\mtx{C}\mtx{E}_k \nonumber \\
  \mtx{E}_{k+1} &= \mtx{A}\mtx{E}_k - \mtx{L}\mtx{C}\mtx{E}_k \nonumber \\
  \mtx{E}_{k+1} &= (\mtx{A} - \mtx{L}\mtx{C})\mtx{E}_k \label{eq:obsv_eig_calc}
\end{align}

For equation (\ref{eq:obsv_eig_calc}) to have a bounded output, the eigenvalues
of $\mtx{A} - \mtx{L}\mtx{C}$ must be within the unit circle. These eigenvalues
represent how fast the estimator converges to the true state of the given
\gls{model}. A fast estimator converges quickly while a slow estimator avoids
amplifying noise in the measurements used to produce a state estimate.

The effect of noise can be seen if it is modeled
\glslink{stochastic process}{stochastically} as

\begin{equation*}
  \hat{\mtx{x}}_{k+1} = \mtx{A}\hat{\mtx{x}}_k + \mtx{B}\mtx{u}_k +
    \mtx{L} ((\mtx{y}_k + \mtx{\nu}_k) - \hat{\mtx{y}}_k) \\
\end{equation*}

where $\mtx{\nu}_k$ is the measurement noise. Rearranging this equation yields

\begin{align*}
  \hat{\mtx{x}}_{k+1} &= \mtx{A}\hat{\mtx{x}}_k + \mtx{B}\mtx{u}_k +
    \mtx{L} (\mtx{y}_k - \hat{\mtx{y}}_k + \mtx{\nu}_k) \\
  \hat{\mtx{x}}_{k+1} &= \mtx{A}\hat{\mtx{x}}_k + \mtx{B}\mtx{u}_k +
    \mtx{L} (\mtx{y}_k - \hat{\mtx{y}}_k) + \mtx{L}\mtx{\nu}_k
\end{align*}

As $\mtx{L}$ increases, the measurement noise is amplified.

In summary, a controller is stable if the eigenvalues of
$\mtx{A} - \mtx{B}\mtx{K}$ are within the unit circle, and an estimator is
stable if the eigenvalues of $\mtx{A} - \mtx{L}\mtx{C}$ are within the unit
circle.

As stated before, the controller and estimator are dual problems. Controller
gains can be found assuming perfect estimator (i.e., perfect knowledge of all
\glspl{state}). Estimator gains can be found assuming an accurate \gls{model}
and a controller with perfect \gls{tracking}.
