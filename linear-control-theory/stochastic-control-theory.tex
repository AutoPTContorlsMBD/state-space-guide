\chapterimage{stochastic-control-theory.jpg}{Road next to Stevenson Academic building at UCSC}

\chapter{Stochastic control theory}

Stochastic control theory is a subfield of control theory that deals with the
existence of uncertainty either in observations or in the noise that drives the
evolution of a \gls{system}. We assign probability distributions to this random
noise and aim to achieve a desired control task despite the presence of this
noise.

Stochastic optimal control is concerned with doing this with minimum cost
defined by some cost functional, like we did with LQR earlier. First, we'll
cover the basics of probability and how we represent linear stochastic
\glspl{system} in state-space representation. Then, we'll derive an optimal
estimator using this knowledge, the Kalman filter, and demonstrate creative
applications of the Kalman filter theory.

\renewcommand*{\chapterpath}{\partpath/stochastic-control-theory}
\section{Introduction to probability}
\index{Probability}

\subsection{Random variables}
\index{Probability!random variables}

A random variable is a variable whose values are the outcomes of a random
phenomenon. As such, a random variable is defined as a function that maps the
outcomes of an unpredictable process to numerical quantities. A particular
output of this function is called a sample. The sample space is the set of
possible values taken by the random variable.

\index{Probability!probability density function}
A probability density function (PDF) is a function whose value at any given
sample in the sample space is the probability of the value of the random
variable equaling that sample. The area under the function over a range gives
the probability that the sample falls within that range. Let $x$ be a random
variable, and let $p(x)$ denote the probability density function of $x$. The
probability that the value of $x$ will be in the interval
$x \in [x_1, x_1 + dx]$ is $p(x_1) \,dx_1$ (see figure \ref{fig:pdf}).

\begin{svg}{build/code/pdf}
  \caption{Probability density function}
  \label{fig:pdf}
\end{svg}

A probability of zero means that the sample will not occur and a probability of
one means that the sample will always occur. Probability density functions
require that no probabilities are negative and that the sum of all probabilities
is $1$. If the probabilities sum to $1$, that means one of those outcomes
\textit{must} happen.

\begin{equation*}
  p(x) \geq 0, \int_{-\infty}^\infty p(x) \,dx = 1
\end{equation*}

\subsection{Expected value}
\index{Probability!expected value}

Expected value or expectation is a weighted average of the values the PDF can
produce where the weight for each is the corresponding probability of that value
occurring. This can be written mathematically as

\begin{equation*}
  \overline{x} = E[x] = \int_{-\infty}^\infty x \,p(x) \,dx
\end{equation*}

The expectation can be applied to random functions as well as random variables.

\begin{equation*}
  E[f(x)] = \int_{-\infty}^\infty f(x) \,p(x) \,dx
\end{equation*}

The mean of a random variable is denoted by an overbar (e.g., $\overline{x}$).
The expectation of the difference between a random variable and its mean
converges to zero. In other words, the expectation of a random variable is its
mean.

\begin{align*}
  E[x - \overline{x}] &= \int_{-\infty}^\infty (x - \overline{x}) \,p(x) \,dx \\
  E[x - \overline{x}] &= \int_{-\infty}^\infty x \, p(x) \,dx -
    \int_{-\infty}^\infty \overline{x} \,p(x) \,dx \\
  E[x - \overline{x}] &= \int_{-\infty}^\infty x \,p(x) \,dx -
    \overline{x} \int_{-\infty}^\infty p(x) \,dx \\
  E[x - \overline{x}] &= \overline{x} - \overline{x} \cdot 1 \\
  E[x - \overline{x}] &= 0 \\
\end{align*}

\subsection{Variance}
\index{Probability!variance}

Informally, variance is a measure of how far the outcome of a random variable
deviates from its mean. Later, we will use variance to quantify how confident we
are in the estimate of a random variable. The standard deviation is the square
root of the variance.

\begin{align*}
  var(x) &= \sigma^2 = E[(x - \overline{x})^2] =
    \int_{-\infty}^{\infty} (x - \overline{x})^2 \,p(x) \,dx \\
  std[x] &= \sigma = \sqrt{var(x)}
\end{align*}

\subsection{Joint probability density functions}
\index{Probability!probability density functions}

Probability density functions can also include more than one variable. Let $x$
and $y$ are random variables. The joint probability density function $p(x, y)$
defines the probability $p(x, y) \,dx \,dy$, so that $x$ and $y$ are in the
intervals $x \in [x, x + dx], y \in [y, y + dy]$ (see figure
\ref{fig:joint_pdf} for an example of a joint PDF).

\begin{svg}{build/code/joint_pdf}
  \caption{Joint probability density function}
  \label{fig:joint_pdf}
\end{svg}

Joint probability density functions also require that no probabilities are
negative and that the sum of all probabilities is $1$.

\begin{equation*}
  p(x, y) \geq 0, \int_{-\infty}^\infty \int_{-\infty}^{\infty} p(x, y) \,dx
    \,dy = 1
\end{equation*}

The expected values for joint PDFs are as follows.

\begin{align*}
  E[x] &= \int_{-\infty}^\infty \int_{-\infty}^{\infty} x \,dx \,dy \\
  E[y] &= \int_{-\infty}^\infty \int_{-\infty}^{\infty} y \,dx \,dy \\
  E[f(x, y)] &= \int_{-\infty}^\infty \int_{-\infty}^{\infty} f(x, y) \,dx \,dy
\end{align*}

The variance of a joint PDF measures how a variable correlates with itself.

\begin{align*}
  var(x) &= \Sigma_{xx} = E[(x - \overline{x})^2] =
    \int_-\infty^{\infty} \int_{-\infty}^\infty (x - \overline{y})^2 \,p(x, y)
    \,dx \,dy \\
  var(y) &= \Sigma_{yy} = E[(y - \overline{y})^2] =
    \int_{-\infty}^\infty \int_{-\infty}^\infty (y - \overline{y})^2 \,p(x, y)
    \,dx \,dy \\
\end{align*}

\subsection{Covariance}
\index{Probability!covariance}

A covariance is a measurement of how a variable correlates with another. If they
vary in the same direction, the covariance increases. If they vary in opposite
directions, the covariance decreases.

\begin{equation*}
  cov(x, y) = \Sigma_{xy} = E[(x - \overline{x})(y - \overline{y})] =
    \int_{-\infty}^\infty \int_{-\infty}^\infty (x - \overline{y})
    (y - \overline{y}) \,p(x, y) \,dx \,dy \\
\end{equation*}

\subsection{Correlation}

Correlation is defined as

\begin{equation*}
  \rho(x, y) = \frac{\Sigma_{xy}}{\sqrt{\Sigma_{xx}\Sigma_{yy}}}, |\rho(x, y)|
    \leq 1
\end{equation*}

\subsection{Independence}

Two random variables are independent if the following relation is true.

\begin{equation*}
  p(x, y) = p(x) \,p(y)
\end{equation*}

This means that the values of $x$ do not correlate with the values of $y$. That
is, the outcome of one random variable does not affect another's outcome. If we
assume independence,

\begin{align*}
  E[xy] &= \int_{-\infty}^\infty \int_{-\infty}^\infty xy \,p(x, y) \,dx \,dy \\
  E[xy] &= \int_{-\infty}^\infty \int_{-\infty}^\infty xy \,p(x) \,p(y) \,dx
    \,dy \\
  E[xy] &= \int_{-\infty}^\infty x \,p(x) \,dx \int_{-\infty}^\infty y \,p(y)
    \,dy \\
  E[xy] &= E[x]E[y] \\
  E[xy] &= \overline{x}\,\overline{y}
\end{align*}

\begin{align*}
  cov(x, y) &= E[(x - \overline{x})(y - \overline{y})] \\
  cov(x, y) &= E[(x - \overline{x})]E[(y - \overline{y})] \\
  cov(x, y) &= 0 \cdot 0 \\
\end{align*}

Therefore, the covariance $\Sigma_{xy}$ is zero. Furthermore, $\rho(x, y) = 0$.

\subsection{Marginal probability density functions}
\index{Probability!marginal probability density functions}

Given two random variables $x$ and $y$ whose joint distribution is known, the
marginal PDF $p(x)$ expresses the probability of $x$ averaged over information
about $y$. In other words, it's the PDF of $x$ when $y$ is unknown. This is
calculated by integrating the joint PDF over $y$.

\begin{equation*}
  p(x) = \int_{-\infty}^\infty p(x, y) \,dy
\end{equation*}

\subsection{Conditional probability density functions}
\index{Probability!conditional probability density functions}

Let us assume that we know the joint PDF $p(x, y)$ and the exact value for $y$.
The conditional PDF gives the probability of $x$ in the interval $[x, x + dx]$
for the given value $y$.

If $p(x, y)$ is known, then we also know $p(x, y = y^\ast)$. However, note that
the latter is not the conditional density $p(x|y^\ast)$, instead

\begin{align*}
  C(y^\ast) &= \int_{-\infty}^\infty p(x, y = y^\ast) \,dx \\
  p(x|y^\ast) &= \frac{1}{C(y^\ast)} p(x, y = y^\ast)
\end{align*}

The scale factor $\frac{1}{C(y^\ast)}$ is used to scale the area under the PDF
to $1$.

\subsection{Bayes' rule}
\index{Probability!Bayes' rule}

Bayes' rule is used to determine the probability of an event based on prior
knowledge of conditions related to the event.

\begin{equation*}
  p(x, y) = p(x|y) \,p(y) = p(y|x) \,p(x)
\end{equation*}

If $x$ and $y$ are independent, then $p(x|y) = p(x)$, $p(y|x) = p(y)$, and
$p(x, y) = p(x) \,p(y)$.

\subsection{Conditional expectation}
\index{Probability!conditional expectation}

The concept of expectation can also be applied to conditional PDFs. This allows
us to determine what the mean of a variable is given prior knowledge of other
variables.

\begin{align*}
  E[x|y] &= \int_{-\infty}^\infty x \,p(x|y) \,dx = f(y), E[x|y] \neq E[x] \\
  E[y|x] &= \int_{-\infty}^\infty y \,p(y|x) \,dy = f(x), E[y|x] \neq E[y]
\end{align*}

\subsection{Conditional variances}
\index{Probability!conditional variances}

\begin{align*}
  var(x|y) &= E[(x - E[x|y])^2|y] \\
  var(x|y) &= \int_{-\infty}^\infty (x - E[x|y])^2 \,p(x|y) \,dx
\end{align*}

\subsection{Random vectors}

Now we will extend the probability concepts discussed so far to vectors where
each element has a PDF.

\begin{equation*}
  \mtx{x} = \begin{bmatrix}
    x_1 \\
    x_2 \\
    \ldots \\
    x_n
  \end{bmatrix}
\end{equation*}

The elements of $\mtx{x}$ are scalar variables jointly distributed with a joint
density $p(x_1, x_2, \ldots, x_n)$. The expectation is

\begin{align*}
  E[\mtx{x}] &= \meanmtx{x} = \int_{-\infty}^\infty \mtx{x} \,p(\mtx{x})
    \,d\mtx{x} \\
  E[\mtx{x}] &= \begin{bmatrix}
    E[x_1] \\
    E[x_2] \\
    \ldots \\
    E[x_n]
  \end{bmatrix} \\
  E[x_i] &= \int_{-\infty}^\infty \ldots \int_{-\infty}^\infty x_i
    \,p(x_1, x_2, \ldots, x_n) \,dx_1 \ldots dx_n \\
  E[f(\mtx{x})] &= \int_{-\infty}^\infty f(\mtx{x}) \,p(\mtx{x}) \,d\mtx{x}
\end{align*}

\subsection{Covariance matrix}
\index{Probability!covariance matrix}

The covariance matrix for a random vector $\mtx{x} \in \mathbb{R}^n$ is

\begin{align*}
  \mtx{\Sigma} &= cov(\mtx{x}, \mtx{x}) = E[(\mtx{x} - \meanmtx{x})
    (\mtx{x} - \meanmtx{x})^T] \\
  \mtx{\Sigma} &= \begin{bmatrix}
    cov(x_1, x_1) & cov(x_1, x_2) & \ldots & cov(x_1, x_n) \\
    cov(x_2, x_1) & cov(x_1, x_2) & \ldots & cov(x_1, x_n) \\
    \ldots        & \ldots        & \ldots & \ldots \\
    cov(x_n, x_1) & cov(x_n, x_2) & \ldots & cov(x_n, x_n) \\
  \end{bmatrix}
\end{align*}

This $n \times n$ matrix is symmetric and positive semidefinite. A positive
semidefinite matrix satisfies the relation that for any
$\mtx{v} \in \mathbb{R}^n$ for which $\mtx{v} \neq 0$,
$\mtx{v}^T \mtx{\Sigma} \mtx{v} \geq 0$. In other words, the eigenvalues of
$\mtx{\Sigma}$ are all greater than or equal to zero.

\subsection{Relations for independent random vectors}

First, independent vectors imply linearity from
$p(\mtx{x}, \mtx{y}) = p(\mtx{x}) \,p(\mtx{y})$.

\begin{align*}
  E[\mtx{A}\mtx{x} + \mtx{B}\mtx{y}] &= \mtx{A}E[\mtx{x}] + \mtx{B}E[\mtx{y}] \\
  E[\mtx{A}\mtx{x} + \mtx{B}\mtx{y}] &= \mtx{A}\meanmtx{x} + \mtx{B}\meanmtx{y}
\end{align*}

Second, independent vectors being uncorrelated means their covariance is zero.

\begin{align}
  \mtx{\Sigma}_{\mtx{x}\mtx{y}} &= cov(\mtx{x}, \mtx{y}) \nonumber \\
  \mtx{\Sigma}_{\mtx{x}\mtx{y}} &= E[(\mtx{x} - \meanmtx{x})
    (\mtx{y} - \meanmtx{y})^T] \nonumber \\
  \mtx{\Sigma}_{\mtx{x}\mtx{y}} &= E[\mtx{x}\mtx{y}^T] -
    E[\mtx{x}\meanmtx{y}^T] - E[\meanmtx{x}\mtx{y}^T] +
    E[\meanmtx{x}\meanmtx{y}^T] \nonumber \\
  \mtx{\Sigma}_{\mtx{x}\mtx{y}} &= E[\mtx{x}\mtx{y}^T] -
    E[\mtx{x}]\meanmtx{y}^T - \meanmtx{x}E[\mtx{y}^T] +
    \meanmtx{x}\meanmtx{y}^T \nonumber \\
  \mtx{\Sigma}_{\mtx{x}\mtx{y}} &= E[\mtx{x}\mtx{y}^T] -
    \meanmtx{x}\meanmtx{y}^T - \meanmtx{x}\meanmtx{y}^T +
    \meanmtx{x}\meanmtx{y}^T \nonumber \\
  \mtx{\Sigma}_{\mtx{x}\mtx{y}} &= E[\mtx{x}\mtx{y}^T] -
    \meanmtx{x}\meanmtx{y}^T \label{eq:prb_sigma}
\end{align}

Now, compute $E[\mtx{x}\mtx{y}^T]$.

\begin{align}
  E[\mtx{x}\mtx{y}^T] &= \int_X \int_Y \mtx{x}\mtx{y}^T \,p(\mtx{x})
    \,p(\mtx{y}) \,d\mtx{x} \,d\mtx{y}^T \nonumber \\
  E[\mtx{x}\mtx{y}^T] &= \int_X p(\mtx{x}) \,\mtx{x} \,d\mtx{x}
    \int_Y p(\mtx{y}) \,\mtx{y}^T \,d\mtx{y}^T \nonumber \\
  E[\mtx{x}\mtx{y}^T] &= \meanmtx{x}\meanmtx{y}^T \label{eq:prb_exyt}
\end{align}

Substitute equation (\ref{eq:prb_exyt}) into equation (\ref{eq:prb_sigma}).

\begin{align*}
  \mtx{\Sigma}_{\mtx{x}\mtx{y}} &= (\meanmtx{x}\meanmtx{y}^T) -
    \meanmtx{x}\meanmtx{y}^T \\
  \mtx{\Sigma}_{\mtx{x}\mtx{y}} &= 0
\end{align*}

Using these results, we can compute the covariance of
$\mtx{z} = \mtx{A}\mtx{x} + \mtx{B}\mtx{y}$ where $\Sigma_{xy} = 0$,
$\Sigma_x = cov(\mtx{x}, \mtx{x})$, and $\Sigma_y = cov(\mtx{y}, \mtx{y})$.

\begin{align*}
  \Sigma_z =~& cov(\mtx{z}, \mtx{z}) \\
  \Sigma_z =~& E[(\mtx{z} - \meanmtx{z})(\mtx{z} - \meanmtx{z})^T] \\
  \Sigma_z =~& E[(\mtx{A}\mtx{x} + \mtx{B}\mtx{y} - \mtx{A}\meanmtx{x} -
    \mtx{B}\meanmtx{y})(\mtx{A}\mtx{x} + \mtx{B}\mtx{y} -
    \mtx{A}\meanmtx{x} - \mtx{B}\meanmtx{y})^T] \\
  \Sigma_z =~& E[(\mtx{A}(\mtx{x} - \meanmtx{x}) +
    \mtx{B}(\mtx{y} - \meanmtx{y}))
    (\mtx{A}(\mtx{x} - \meanmtx{x}) +
     \mtx{B}(\mtx{y} - \meanmtx{y}))^T] \\
  \Sigma_z =~& E[(\mtx{A}(\mtx{x} - \meanmtx{x}) +
    \mtx{B}(\mtx{y} - \meanmtx{y}))
    ((\mtx{x} - \meanmtx{x})^T\mtx{A}^T +
     (\mtx{y} - \meanmtx{y})^T\mtx{B}^T)] \\
  \Sigma_z =~& E[
    \mtx{A}(\mtx{x} - \meanmtx{x})(\mtx{x} - \meanmtx{x})^T\mtx{A}^T +
    \mtx{A}(\mtx{x} - \meanmtx{x})(\mtx{y} - \meanmtx{y})^T\mtx{B}^T + \\
    &\mtx{B}(\mtx{y} - \meanmtx{y})(\mtx{x} - \meanmtx{x})^T\mtx{A}^T +
    \mtx{B}(\mtx{y} - \meanmtx{y})(\mtx{y} - \meanmtx{y})^T\mtx{B}^T]
\end{align*}

Since $\mtx{x}$ and $\mtx{y}$ are independent,

\begin{align*}
  \Sigma_z &= E[
    \mtx{A}(\mtx{x} - \meanmtx{x})(\mtx{x} - \meanmtx{x})^T\mtx{A}^T + 0 + 0 +
    \mtx{B}(\mtx{y} - \meanmtx{y})(\mtx{y} - \meanmtx{y})^T\mtx{B}^T] \\
  \Sigma_z &=
    E[\mtx{A}(\mtx{x} - \meanmtx{x})(\mtx{x} - \meanmtx{x})^T\mtx{A}^T] +
    E[\mtx{B}(\mtx{y} - \meanmtx{y})(\mtx{y} - \meanmtx{y})^T\mtx{B}^T] \\
  \Sigma_z &=
    \mtx{A}E[(\mtx{x} - \meanmtx{x})(\mtx{x} - \meanmtx{x})^T]\mtx{A}^T +
    \mtx{B}E[(\mtx{y} - \meanmtx{y})(\mtx{y} - \meanmtx{y})^T]\mtx{B}^T \\
  \Sigma_z &=\mtx{A}\Sigma_x\mtx{A}^T + \mtx{B}\Sigma_y\mtx{B}^T
\end{align*}

\subsection{Gaussian random variables}

A Gaussian random variable has the following properties:

\begin{align*}
  E[x] &= \overline{x} \\
  var(x) &= \sigma^2 \\
  p(x) &= \frac{1}{\sqrt{2\pi\sigma^2}}
    e^{-\frac{(x - \overline{x})^2}{2\sigma^2}}
\end{align*}

While we could use any random variable to represent a random process, we use the
Gaussian random variable a lot in probability theory due to the central limit
theorem.

\begin{definition}[Central limit theorem]
  When independent random variables are added, their properly normalized sum
  tends toward a normal distribution (a Gaussian distribution or ``bell curve").
\end{definition}
\index{Probability!central limit theorem}

This is the case even if the original variables themselves are not normally
distributed. The theorem is a key concept in probability theory because it
implies that probabilistic and statistical methods that work for normal
distributions can be applicable to many problems involving other types of
distributions.

For example, suppose that a sample is obtained containing a large number of
independent observations, and that the arithmetic mean of the observed values
is computed. The central limit theorem says that the computed values of the
mean will tend toward being distributed according to a normal distribution.

\section{Linear stochastic systems}

Given the following stochastic system

\begin{align*}
  \mtx{x}_{k+1} &= \mtx{\Phi}\mtx{x}_k + \mtx{B}\mtx{u}_k +
    \mtx{\Gamma}\mtx{w}_k \\
  \mtx{y}_k &= \mtx{H}\mtx{x}_k + \mtx{D}\mtx{u}_k + \mtx{v}_k
\end{align*}

where $\mtx{w}_k$ is the process noise and $\mtx{v}_k$ is the measurement noise,

\begin{align*}
  E[\mtx{w}_k] &= 0 \\
  E[\mtx{w}_k\mtx{w}_k^T] &= \mtx{Q}_k \\
  E[\mtx{v}_k] &= 0 \\
  E[\mtx{v}_k\mtx{v}_k^T] &= \mtx{R}_k
\end{align*}

where $\mtx{Q}_k$ is the process noise covariance matrix and $\mtx{R}_k$ is the
measurement noise covariance matrix. We assume the noise samples are
independent, so $E[\mtx{w}_k\mtx{w}_j^T] = 0$ and $E[\mtx{v}_k\mtx{v}_k^T] = 0$
where $k \neq j$. Furthermore, process noise samples are independent from
measurement noise samples.

We'll compute the expectation of these equations and their covariance matrices,
which we'll use later for deriving the Kalman filter.

\subsection{State vector expectation evolution}

First, we will compute how the expectation of the system state evolves.

\begin{align*}
  E[\mtx{x}_{k+1}] &= E[\mtx{\Phi}\mtx{x}_k + \mtx{B}\mtx{u}_k +
    \mtx{\Gamma}\mtx{w}_k] \\
  E[\mtx{x}_{k+1}] &= E[\mtx{\Phi}\mtx{x}_k] + E[\mtx{B}\mtx{u}_k] +
    E[\mtx{\Gamma}\mtx{w}_k] \\
  E[\mtx{x}_{k+1}] &= \mtx{\Phi}E[\mtx{x}_k] + \mtx{B}E[\mtx{u}_k] +
    \mtx{\Gamma}E[\mtx{w}_k] \\
  E[\mtx{x}_{k+1}] &= \mtx{\Phi}E[\mtx{x}_k] + \mtx{B}\mtx{u}_k + 0 \\
  \meanmtx{x}_{k+1} &= \mtx{\Phi}\meanmtx{x}_k +
    \mtx{B}\mtx{u}_k \\
\end{align*}

\subsection{State covariance matrix evolution}

Now, we will use this to compute how the state covariance matrix $\mtx{P}$
evolves.

\begin{align*}
  \mtx{x}_{k+1} - \meanmtx{x}_{k+1} &= \mtx{\Phi}\mtx{x}_k +
    \mtx{B}\mtx{u}_k + \mtx{\Gamma}\mtx{w}_k - (\mtx{\Phi}\meanmtx{x}_k -
    \mtx{B}\mtx{u}_k) \\
  \mtx{x}_{k+1} - \meanmtx{x}_{k+1} &=
    \mtx{\Phi}(\mtx{x}_k - \meanmtx{x}_k) + \mtx{\Gamma}\mtx{w}_k
\end{align*}

\begin{equation*}
  E[(\mtx{x}_{k+1} - \meanmtx{x}_{k+1})(\mtx{x}_{k+1} - \meanmtx{x}_{k+1})^T] =
    E[(\mtx{\Phi}(\mtx{x}_k - \meanmtx{x}_k) + \mtx{\Gamma}\mtx{w}_k)
      (\mtx{\Phi}(\mtx{x}_k - \meanmtx{x}_k) + \mtx{\Gamma}\mtx{w}_k)^T]
\end{equation*}

\begin{align*}
  \mtx{P}_{k+1} =~&
    E[(\mtx{\Phi}(\mtx{x}_k - \meanmtx{x}_k) + \mtx{\Gamma}\mtx{w}_k)
      (\mtx{\Phi}(\mtx{x}_k - \meanmtx{x}_k) + \mtx{\Gamma}\mtx{w}_k)^T] \\
  \mtx{P}_{k+1} =~&
    E[(\mtx{\Phi}(\mtx{x}_k - \meanmtx{x}_k)(\mtx{x}_k - \meanmtx{x}_k)^T
      \mtx{\Phi}^T] +
    E[\mtx{\Phi}(\mtx{x}_k - \meanmtx{x}_k)\mtx{w}_k^T\mtx{\Gamma}^T] + \\
    &E[\mtx{\Gamma}\mtx{w}_k(\mtx{x}_k - \meanmtx{x}_k)^T\mtx{\Phi}^T] +
    E[\mtx{\Gamma}\mtx{w}_k\mtx{w}_k^T\mtx{\Gamma}^T] \\
  \mtx{P}_{k+1} =~&
    \mtx{\Phi}E[(\mtx{x}_k - \meanmtx{x}_k)(\mtx{x}_k - \meanmtx{x}_k)^T]
    \mtx{\Phi}^T +
    \mtx{\Phi}E[(\mtx{x}_k - \meanmtx{x}_k)\mtx{w}_k^T]\mtx{\Gamma}^T + \\
    &\mtx{\Gamma} E[\mtx{w}_k(\mtx{x}_k - \meanmtx{x}_k)^T]\mtx{\Phi}^T +
    \mtx{\Gamma} E[\mtx{w}_k\mtx{w}_k^T]\mtx{\Gamma}^T \\
  \mtx{P}_{k+1} =~& \mtx{\Phi}\mtx{P}_k\mtx{\Phi}^T +
    \mtx{\Phi}E[(\mtx{x}_k - \meanmtx{x}_k)\mtx{w}_k^T]\mtx{\Gamma}^T + \\
    &\mtx{\Gamma} E[\mtx{w}_k(\mtx{x}_k - \meanmtx{x}_k)^T]\mtx{\Phi}^T +
    \mtx{\Gamma}\mtx{Q}_k\mtx{\Gamma}_k^T \\
  \mtx{P}_{k+1} =~& \mtx{\Phi}\mtx{P}_k\mtx{\Phi}^T + 0 + 0 +
    \mtx{\Gamma}\mtx{Q}_k\mtx{\Gamma}^T \\
  \mtx{P}_{k+1} =~& \mtx{\Phi}\mtx{P}_k\mtx{\Phi}^T +
    \mtx{\Gamma}\mtx{Q}_k\mtx{\Gamma}^T
\end{align*}

\subsection{Measurement vector expectation}

Next, we will compute the expectation of the output $\mtx{y}$.

\begin{align*}
  E[\mtx{y}_k] &= E[\mtx{H}\mtx{x}_k + \mtx{D}\mtx{u}_k + \mtx{v}_k] \\
  E[\mtx{y}_k] &= \mtx{H}E[\mtx{x}_k] + \mtx{D}\mtx{u}_k + 0 \\
  \meanmtx{y}_k &= \mtx{H}\meanmtx{x}_k + \mtx{D}\mtx{u}_k
\end{align*}

\subsection{Measurement covariance matrix}

Now, we will use this to compute how the measurement covariance matrix
$\mtx{S}$ evolves.

\begin{align*}
  \mtx{y}_k - \meanmtx{y}_k &= \mtx{H}\mtx{x}_k + \mtx{D}\mtx{u}_k + \mtx{v}_k -
    (\mtx{H}\meanmtx{x}_k + \mtx{D}\mtx{u}_k) \\
  \mtx{y}_k - \meanmtx{y}_k &= \mtx{H}(\mtx{x}_k - \meanmtx{x}_k) + \mtx{v}_k
\end{align*}

\begin{align*}
  E[(\mtx{y}_k - \meanmtx{y}_k)(\mtx{y}_k - \meanmtx{y}_k)^T] &=
    E[(\mtx{H}(\mtx{x}_k - \meanmtx{x}_k) + \mtx{v}_k)
      (\mtx{H}(\mtx{x}_k - \meanmtx{x}_k) + \mtx{v}_k)^T] \\
  \mtx{S}_k &= E[(\mtx{H}(\mtx{x}_k - \meanmtx{x}_k) + \mtx{v}_k)
                 (\mtx{H}(\mtx{x}_k - \meanmtx{x}_k) + \mtx{v}_k)^T] \\
  \mtx{S}_k &= E[(\mtx{H}(\mtx{x}_k - \meanmtx{x}_k)
                 (\mtx{x}_k - \meanmtx{x}_k)^T\mtx{H}^T] +
               E[\mtx{v}_k\mtx{v}_k^T] \\
  \mtx{S}_k &=
    \mtx{H}E[((\mtx{x}_k - \meanmtx{x}_k)(\mtx{x}_k - \meanmtx{x}_k)^T]
    \mtx{H}^T + \mtx{R}_k \\
  \mtx{S}_k &= \mtx{H}\mtx{P}_k\mtx{H}^T + \mtx{R}_k
\end{align*}

\section{Two-sensor problem}
\index{Stochastic!two-sensor problem}

We'll skip the probability derivations here, but given two data points with
associated variances represented by Gaussian distribution, the information can
be optimally combined into a third Gaussian distribution with a mean value and
variance. The expected value of $x$ given a measurement $z_1$ is

\begin{equation}
  E[x|z_1] = \mu = \frac{\sigma_0^2}{\sigma_0^2 + \sigma^2}z_1 +
    \frac{\sigma^2}{\sigma_0^2 + \sigma^2}x_0
\end{equation}

The variance of $x$ given $z_1$ is

\begin{equation}
  E[(x - \mu)^2|z_1] = \frac{\sigma^2 \sigma_0^2}{\sigma_0^2 + \sigma^2}
\end{equation}

The expected value, which is also the maximum likelihood value, is the linear
combination of the prior expected (maximum likelihood) value and the
measurement. The expected value is a reasonable estimator of $x$.

\begin{align}
  \hat{x} &= E[x|z_1] = \frac{\sigma_0^2}{\sigma_0^2 + \sigma^2}z_1 +
    \frac{\sigma^2}{\sigma_0^2 + \sigma^2}x_0 \\
  \hat{x} &= w_1 z_1 + w_2 x_0 \nonumber
\end{align}

Note that the weights $w_1$ and $w_2$ sum to $1$. When the prior (i.e., prior
knowledge of state) is uninformative (a large variance)

\begin{align}
  w_1 &= \lim_{\sigma_0^2 \to 0} \frac{\sigma_0^2}{\sigma_0^2 + \sigma^2} = 0 \\
  w_2 &= \lim_{\sigma_0^2 \to 0} \frac{\sigma^2}{\sigma_0^2 + \sigma^2} = 1
\end{align}

and $\hat{x} = z_1$. That is, the weight is on the observations and the estimate
is equal to the measurement.

Let us assume we have a \gls{model} providing an almost exact prior for $x$. In
that case, $\sigma_0^2$ approaches 0 and

\begin{align}
  w_1 &= \lim_{\sigma_0^2 \to 0} \frac{\sigma_0^2}{\sigma_0^2 + \sigma^2} = 1 \\
  w_2 &= \lim_{\sigma_0^2 \to 0} \frac{\sigma^2}{\sigma_0^2 + \sigma^2} = 0
\end{align}

The Kalman filter uses this optimal fusion as the basis for its operation.

\section{Kalman filter}
\index{Kalman filter}

Theorem \ref{thm:kalman_filter} shows the predict and update steps for a Kalman
filter at the $k^{th}$ timestep.

\begin{theorem}[Kalman filter]
  \begin{align}
    \text{Predict step} \nonumber \\
    \hat{\mtx{x}}_{k+1}^- &= \mtx{\Phi}\hat{\mtx{x}}_k + \mtx{B} \mtx{u}_k
      \label{eq:pre1_x} \\
    \mtx{P}_{k+1}^- &= \mtx{\Phi} \mtx{P}_k^- \mtx{\Phi}^T +
      \mtx{\Gamma}\mtx{Q}\mtx{\Gamma}^T \\
    \text{Update step} \nonumber \\
    \mtx{K}_{k+1} &=
      \mtx{P}_{k+1}^- \mtx{H}^T (\mtx{H}\mtx{P}_{k+1}^- \mtx{H}^T +
      \mtx{R})^{-1} \\
    \hat{\mtx{x}}_{k+1}^+ &=
      \hat{\mtx{x}}_{k+1}^- + \mtx{K}_{k+1}(\mtx{y}_{k+1} -
      \mtx{H} \hat{\mtx{x}}_{k+1}^-) \label{eq:post1_x} \\
    \mtx{P}_{k+1}^+ &= (\mtx{I} - \mtx{K}_{k+1}\mtx{H})\mtx{P}_{k+1}^-
  \end{align}

  \begin{figurekey}
    \begin{tabulary}{\linewidth}{LLLL}
      $\mtx{\Phi}$ & system matrix & $\hat{\mtx{x}}$ & state estimate vector \\
      $\mtx{B}$ & input matrix            & $\mtx{u}$ & input vector \\
      $\mtx{H}$ & measurement matrix      & $\mtx{y}$ & output vector \\
      $\mtx{P}$ & error covariance matrix & $\mtx{Q}$ & process noise covariance
        matrix \\
      $\mtx{K}$ & Kalman gain matrix & $\mtx{R}$ & measurement noise covariance
        matrix \\
      $\mtx{\Gamma}$ & process noise intensity vector &
    \end{tabulary}
  \end{figurekey}

  where a superscript of minus denotes \textit{a priori} and plus denotes
  \textit{a posteriori} estimate (before and after update respectively).

  \label{thm:kalman_filter}
\end{theorem}

$\mtx{\Phi}$ is replaced with $\mtx{A}$ for continuous systems.

\begin{booktable}
  \begin{tabular}{|ll|ll|}
    \hline
    \rowcolor{headingbg}
    \textbf{Matrix} & \textbf{Rows $\times$ Columns} &
    \textbf{Matrix} & \textbf{Rows $\times$ Columns} \\
    \hline
    $\mtx{\Phi}$ & states $\times$ states & $\hat{\mtx{x}}$ & states $\times$ 1
      \\
    $\mtx{B}$ & states $\times$ inputs & $\mtx{u}$ & inputs $\times$ 1 \\
    $\mtx{H}$ & outputs $\times$ states & $\mtx{y}$ & outputs $\times$ 1 \\
    $\mtx{P}$ & states $\times$ states & $\mtx{Q}$ & states $\times$ states \\
    $\mtx{K}$ & states $\times$ outputs & $\mtx{R}$ & outputs $\times$ outputs
      \\
    $\mtx{\Gamma}$ & states $\times$ 1 &  &  \\
    \hline
  \end{tabular}
  \caption{Kalman filter matrix dimensions}
  \label{tab:kf_matrix_dims}
\end{booktable}

See the Wikipedia page on Kalman filters \cite{bib:kalman_filter} for
derivations of the update step equations.

Unknown states in a Kalman filter are generally represented by a Wiener
(pronounced VEE-ner) process. This process has the property that its variance
increases linearly with time $t$.

\subsection{Equations to model}

The following example system will be used to describe how to define and
initialize the matrices for a Kalman filter.

A robot is between two parallel walls. It starts driving from one wall to the
other at a velocity of $0.8 cm/s$ and uses ultrasonic sensors to provide noisy
measurements of the distances to the walls in front of and behind it. To
estimate the distance between the walls, we will define three states: robot
position, robot velocity, and distance between the walls.

\begin{align}
  x_{k+1} &= x_k + v_k \Delta T \\
  v_{k+1} &= v_k \\
  x_{k+1}^w &= x_k^w
\end{align}

This can be converted to the following state-space \gls{model}.

\begin{equation}
  \mtx{x}_k =
  \begin{bmatrix}
    x_k \\
    v_k \\
    x_k^w
  \end{bmatrix}
\end{equation}

\begin{equation}
  \mtx{x}_{k+1} =
  \begin{bmatrix}
    1 & 1 & 0 \\
    0 & 0 & 0 \\
    0 & 0 & 1
  \end{bmatrix} \mtx{x}_k +
  \begin{bmatrix}
    0 \\
    0.8 \\
    0
  \end{bmatrix} +
  \begin{bmatrix}
    0 \\
    0.1 \\
    0
  \end{bmatrix} w_k
\end{equation}

where the Gaussian random variable $w_k$ has zero mean and variance 1. The
observation \gls{model} is

\begin{equation}
  \mtx{y}_k =
  \begin{bmatrix}
    1 & 0 & 0 \\
    -1 & 0 & 1
  \end{bmatrix} \mtx{x}_k + \theta_k
\end{equation}

where the covariance matrix of Gaussian measurement noise $\theta$ is a
$2 \times 2$ matrix with both diagonals $10 cm^2$.

The state vector is usually initialized using the first measurement or two. The
covariance matrix entries are assigned by calculating the covariance of the
expressions used when assigning the state vector. Let $k = 2$.

\begin{align}
  \mtx{Q} &= \begin{bmatrix}1\end{bmatrix} \\
  \mtx{R} &=
  \begin{bmatrix}
    10 & 0 \\
    0 & 10
  \end{bmatrix} \\
  \hat{\mtx{x}} &=
  \begin{bmatrix}
    \mtx{y}_{k,1} \\
    (\mtx{y}_{k,1} - \mtx{y}_{k-1,1})/dt \\
    \mtx{y}_{k,1} + \mtx{y}_{k,2}
  \end{bmatrix} \\
  \mtx{P} &=
  \begin{bmatrix}
    10 & 10/dt & 10 \\
    10/dt & 20/dt^2 & 10/dt \\
    10 & 10/dt & 20
  \end{bmatrix}
\end{align}

\subsection{Initial conditions}

To fill in the $\mtx{P}$ matrix, we calculate the covariance of each combination
of state variables. The resulting value is a measure of how much those variables
are correlated. Due to how the covariance calculation works out, the covariance
between two variables is the sum of the variance of matching terms which aren't
constants multiplied by any constants the two have. If no terms match, the
variables are uncorrelated and the covariance is zero.

In $\mtx{P}_{11}$, the terms in $\mtx{x}_1$ correlate with itself. Therefore,
$\mtx{P}_{11}$ is $\mtx{x}_1$'s variance, or $\mtx{P}_{11} = 10$. For
$\mtx{P}_{21}$, One term correlates between $\mtx{x}_1$ and $\mtx{x}_2$, so
$\mtx{P}_{21} = \frac{10}{dt}$. The constants from each are simply multiplied
together. For $\mtx{P}_{22}$, both measurements are correlated, so the variances
add together. Therefore, $\mtx{P}_{22} = \frac{20}{dt^2}$. It continues in this
fashion until the matrix is filled up. Order doesn't matter for correlation, so
the matrix is symmetric.

\subsection{Selection of priors}

Choosing good priors is important for a well performing filter, even if little
information is known. This applies to both the measurement noise and the noise
\gls{model}. The act of giving a state variable a large variance means you know
something about the system. Namely, you aren't sure whether your initial guess
is close to the true state. If you make a guess and specify a small variance,
you are telling the filter that you are very confident in your guess. If that
guess is incorrect, it will take the filter a long time to move away from your
guess to the true value.

\subsection{Covariance selection}

While one could assume no correlation between the state variables and set the
covariance matrix entries to zero, this may not reflect reality. The Kalman
filter is still guarenteed to converge to the steady-state covariance after an
infinite time, but it will take longer than otherwise.

\subsection{Noise model selection}

We typically use a Gaussian distribution for the noise \gls{model} because the
sum of many independent random variables produces a normal distribution by the
central limit theorem. Kalman filters only require that the noise has a zero
mean. If the true value has an equal probability of being within a certain
range, use a uniform distribution instead. Each of these communicates
information regarding what you know about a system in addition to what you do
not.

\section{Kalman smoother}
\index{State-space observers!Kalman filter!smoother}

The Kalman filter uses the data up to the current time to produce an optimal
estimate of the system \gls{state}. If data beyond the current time is
available, it can be ran through a Kalman smoother to produce a better estimate.
This is done by recording measurements, then applying the smoother to it
offline.

The Kalman smoother does a forward pass on the available data, then a backward
pass through the system dynamics so it takes into account the data before and
after the current time. This produces \gls{state} variances that are lower than
that of a Kalman filter.

We will modify the robot model so that instead of a velocity of $0.8 cm/s$ with
random noise, the velocity is modeled as a random walk from the current
velocity.

\begin{equation}
  \mtx{x}_k =
  \begin{bmatrix}
    x_k \\
    v_k \\
    x_k^w
  \end{bmatrix}
\end{equation}

\begin{equation}
  \mtx{x}_{k+1} =
  \begin{bmatrix}
    1 & 1 & 0 \\
    0 & 1 & 0 \\
    0 & 0 & 1
  \end{bmatrix} \mtx{x}_k +
  \begin{bmatrix}
    0 \\
    0.1 \\
    0
  \end{bmatrix} w_k
\end{equation}

We will use the same observation model as before.

Using the same data from subsection \ref{subsec:filter_simulation}, figures
\ref{fig:smoother_robot_pos}, \ref{fig:smoother_robot_vel}, and
\ref{fig:smoother_wall_pos} show the improved \gls{state} estimates and figure
\ref{fig:smoother_robot_pos_variance} shows the improved robot position
covariance with a Kalman smoother.

Notice how the wall position produced by the smoother is a constant. This is
because that \gls{state} has no dynamics, so the final estimate from the Kalman
filter is already the best estimate.

\begin{svg}{build/code/kalman_smoother_robot_pos}
  \caption{Robot position with Kalman smoother}
  \label{fig:smoother_robot_pos}
\end{svg}

\begin{svg}{build/code/kalman_smoother_robot_vel}
  \caption{Robot velocity with Kalman smoother}
  \label{fig:smoother_robot_vel}
\end{svg}

\begin{svg}{build/code/kalman_smoother_wall_pos}
  \caption{Wall position with Kalman smoother}
  \label{fig:smoother_wall_pos}
\end{svg}

\begin{svg}{build/code/kalman_smoother_robot_pos_variance}
  \caption{Robot position variance with Kalman smoother}
  \label{fig:smoother_robot_pos_variance}
\end{svg}

\section{MMAE}
\index{State-space observers!Kalman filter!MMAE}

MMAE stands for Multiple Model Adaptive Estimation. MMAE runs multiple Kalman
filters with different \glspl{model} on the same data. The Kalman filter with
the lowest residual has the highest likelihood of accurately reflecting reality.
This can be used to detect certain \gls{system} \glspl{state} like an aircraft
engine failing without needing to invest in costly sensors to determine this
directly.

For example, say you have three Kalman filters: one for turning left, one for
turning right, and one for going straight. If the \gls{control input} is
attempting to fly the plane straight and the Kalman filter for going left has
the lowest residual, the aircraft's left engine probably failed.

