\section{Single-jointed arm}

\subsection{Equations of motion}

This single-jointed arm consists of a DC brushed motor attached to a pulley that
spins a straight bar in pitch.

\begin{bookfigure}
  \begin{tikzpicture}[auto, >=latex', circuit ee IEC,
                      set resistor graphic=var resistor IEC graphic]
    % \draw [help lines] (-1,-3) grid (7,4);

    % Electrical equivalent circuit
    \draw (0,2) to [voltage source={direction info'={->}, info'=$V$}] (0,0);
    \draw (0,2) to [current direction={info=$I$}] (0,3);
    \draw (0,3) -- (0.5,3);
    \draw (0.5,3) to [resistor={info={$R$}}] (2,3);

    \draw (2,3) -- (2.5,3);
    \draw (2.5,3) to [voltage source={direction info'={->}, info'=$V_{emf}$}]
      (2.5,0);
    \draw (0,0) -- (2.5,0);

    % Motor
    \begin{scope}[xshift=2.4cm,yshift=1.05cm]
      \draw[fill=black] (0,0) rectangle (0.2,0.9);
      \draw[fill=white] (0.1,0.45) ellipse (0.3 and 0.3);
    \end{scope}

    % Transmission gear one
    \begin{scope}[xshift=3.75cm,yshift=1.17cm]
      \draw[fill=black!50] (0.2,0.33) ellipse (0.08 and 0.33);
      \draw[fill=black!50, color=black!50] (0,0) rectangle (0.2,0.66);
      \draw[fill=white] (0,0.33) ellipse (0.08 and 0.33);
      \draw (0,0.66) -- (0.2,0.66);
      \draw (0,0) -- (0.2,0) node[pos=0.5,below] {$G$};
    \end{scope}

    % Output shaft of motor
    \begin{scope}[xshift=2.8cm,yshift=1.45cm]
      \draw[fill=black!50] (0,0) rectangle (0.95,0.1);
    \end{scope}

    % Angular velocity arrow of drive -> transmission
    \draw[line width=0.7pt,<-] (3.2,1) arc (-30:30:1) node[above] {$\omega_m$};

    % Transmission gear two
    \begin{scope}[xshift=3.75cm,yshift=1.83cm]
      \draw[fill=black!50] (0.2,0.68) ellipse (0.13 and 0.67);
      \draw[fill=black!50, color=black!50] (0,0) rectangle (0.2,1.35);
      \draw[fill=white] (0,0.68) ellipse (0.13 and 0.67);
      \draw (0,1.35) -- (0.2,1.35);
      \draw (0,0) -- (0.2,0);
    \end{scope}

    \begin{scope}[xshift=5.075cm,yshift=2.4cm]
      % Single-jointed arm
      \draw[fill=white] (0,0) -- (0.1,-0.05) -- (0.35,1.45) -- (0.25,1.5)
        -- cycle;
      \draw[fill=black!50] (0.1,-0.05) -- (0.3,-0.05) -- (0.55,1.45) --
        (0.35,1.45) -- cycle;
      \draw[fill=white] (0.25,1.5) -- (0.35,1.45) -- (0.55,1.45) -- (0.45,1.5)
        -- cycle;

      % Arm length arrow
      \draw[line width=0.7pt,<->] (0.55,-0.05) -- node[right] {$l$} (0.8,1.45);

      % Mass label
      \draw (-0.05,1.2) node {$m$};
    \end{scope}

    % Transmission shaft from gear two to arm
    \begin{scope}[xshift=4.09cm,yshift=2.42cm]
      \draw[fill=black!50] (0,0) rectangle (1.06,0.1);
    \end{scope}

    % Angular velocity arrow between transmission and arm
    \draw[line width=0.7pt,->] (4.54,1.97) arc (-30:30:1) node[above]
      {$\omega_{arm}$};

    % Descriptions of subsystems under graphic
    \begin{scope}[xshift=-0.5cm,yshift=-0.28cm]
      \draw[decorate,decoration={brace,amplitude=10pt}]
        (3.5,0) -- (0,0) node[midway,yshift=-20pt] {circuit};
      \draw[decorate,decoration={brace,amplitude=10pt}]
        (6.55,0) -- (3.75,0) node[midway,yshift=-20pt] {mechanics};
    \end{scope}
  \end{tikzpicture}

  \caption{Single-jointed arm system diagram}
  \label{fig:single_jointed_arm}
\end{bookfigure}

Gear ratios are written as output over input, so $G$ is greater than one in
figure \ref{fig:single_jointed_arm}.

We will start with the equation derived earlier for a DC brushed motor, equation
(\ref{eq:motor_tau_V}).

\begin{equation*}
  V = \frac{\tau_m}{K_t} R + \frac{\omega_m}{K_v}
\end{equation*}

Solve for the angular acceleration. First, we'll rearrange the terms because
from inspection, $V$ is the \gls{model} \gls{input}, $\omega_m$ is the
\gls{state}, and $\tau_m$ contains the angular acceleration.

\begin{equation*}
  V = \frac{R}{K_t} \tau_m + \frac{1}{K_v} \omega_m
\end{equation*}

Solve for $\tau_m$.

\begin{align*}
  V &= \frac{R}{K_t} \tau_m + \frac{1}{K_v} \omega_m \\
  \frac{R}{K_t} \tau_m &= V - \frac{1}{K_v} \omega_m \\
  \tau_m &= \frac{K_t}{R} V - \frac{K_t}{K_v R} \omega_m
\end{align*}

Since $\tau_m G = \tau_{arm}$ and $\omega_m = G \omega_{arm}$

\begin{align}
  \left(\frac{\tau_{arm}}{G}\right) &= \frac{K_t}{R} V -
    \frac{K_t}{K_v R} (G \omega_f) \nonumber \\
  \frac{\tau_{arm}}{G} &= \frac{K_t}{R} V - \frac{G K_t}{K_v R} \omega_{arm}
    \nonumber \\
  \tau_{arm} &= \frac{G K_t}{R} V - \frac{G^2 K_t}{K_v R} \omega_{arm}
    \label{eq:tau_arm}
\end{align}

The angular velocity of the arm is defined as

\begin{equation}
  \tau_{arm} = J \dot{\omega}_{arm} \label{eq:tau_arm_def}
\end{equation}

where $J$ is the moment of inertia of the arm and $\dot{\omega}_{arm}$ is the
angular acceleration. Substitute equation (\ref{eq:tau_arm_def}) into equation
(\ref{eq:tau_arm}).

\begin{align}
  (J \dot{\omega}_{arm}) &= \frac{G K_t}{R} V - \frac{G^2 K_t}{K_v R}
    \omega_{arm} \nonumber \\
  \dot{\omega}_{arm} &= \frac{G K_t}{RJ} V - \frac{G^2 K_t}{K_v RJ} \omega_{arm}
    \label{eq:dot_omega_arm}
\end{align}

$J$ can be approximated as the moment of inertia of a thin rod rotating around
one end. Therefore

\begin{equation}
  J = \frac{1}{3}ml^2
\end{equation}

where $m$ is the mass of the arm and $l$ is the length of the arm.

\subsection{Continuous state-space model}
\index{FRC models!single-jointed arm equations}

The position and velocity of the elevator can be written as

\begin{align}
  \dot{\theta}_{arm} &= \omega_{arm} \label{eq:arm_cont_ss_pos} \\
  \dot{\omega}_{arm} &= \dot{\omega}_{arm} \label{eq:arm_cont_ss_vel}
\end{align}

By equation (\ref{eq:dot_omega_arm})

\begin{equation*}
  \dot{\omega}_{arm} = -\frac{G^2 K_t}{K_v RJ} \omega_{arm} + \frac{G K_t}{RJ} V
\end{equation*}

\begin{theorem}[Single-jointed arm state-space model]
  \begin{align*}
    \dot{\mtx{x}} &= \mtx{A} \mtx{x} + \mtx{B} \mtx{u} \\
    \mtx{y} &= \mtx{C} \mtx{x} + \mtx{D} \mtx{u}
  \end{align*}
  \begin{equation*}
    \begin{array}{ccc}
      \mtx{x} =
      \begin{bmatrix}
        \theta_{arm} \\
        \omega_{arm}
      \end{bmatrix} &
      \mtx{y} = \theta_{arm} &
      \mtx{u} = V
    \end{array}
  \end{equation*}
  \begin{equation}
    \begin{array}{cccc}
      \mtx{A} =
      \begin{bmatrix}
        0 & 1 \\
        0 & -\frac{G^2 K_t}{K_v RJ}
      \end{bmatrix} &
      \mtx{B} =
      \begin{bmatrix}
        0 \\
        \frac{G K_t}{RJ}
      \end{bmatrix} &
      \mtx{C} =
      \begin{bmatrix}
        1 & 0
      \end{bmatrix} &
      \mtx{D} = 0
    \end{array}
  \end{equation}
\end{theorem}

\subsection{Model augmentation}

As per subsection \ref{subsec:u_error_estimation}, we will now augment the
\gls{model} so a $u_{error}$ term is added to the \gls{control input}.

The \gls{plant} and \gls{observer} augmentations should be performed before the
\gls{model} is \glslink{discretization}{discretized}. After the \gls{controller}
gain is computed with the unaugmented discrete \gls{model}, the controller may
be augmented. Therefore, the \gls{plant} and \gls{observer} augmentations assume
a continuous \gls{model} and the \gls{controller} augmentation assumes a
discrete \gls{controller}.

\begin{equation*}
  \begin{array}{ccc}
    \mtx{x}_{aug} =
    \begin{bmatrix}
      \mtx{x} \\
      u_{error}
    \end{bmatrix} &
    \mtx{y} = \theta_{arm} &
    \mtx{u} = V
  \end{array}
\end{equation*}

\begin{equation}
  \begin{array}{cccc}
    \mtx{A}_{aug} =
    \begin{bmatrix}
      \mtx{A} & \mtx{B} \\
      \mtx{0}_{1 \times 2} & 0
    \end{bmatrix} &
    \mtx{B}_{aug} =
    \begin{bmatrix}
      \mtx{B} \\
      0
    \end{bmatrix} &
    \mtx{C}_{aug} =
    \begin{bmatrix}
      \mtx{C} & 0
    \end{bmatrix} &
    \mtx{D}_{aug} = \mtx{D}
  \end{array}
\end{equation}

\begin{equation}
  \begin{array}{cc}
    \mtx{K}_{aug} = \begin{bmatrix}
      \mtx{K} & 1
    \end{bmatrix} &
    \mtx{r}_{aug} = \begin{bmatrix}
      \mtx{r} \\
      0
    \end{bmatrix}
  \end{array}
\end{equation}

This will compensate for unmodeled dynamics such as gravity or other external
loading from lifted objects. However, if only gravity compensation is desired,
a feedforward of the form $u_{ff} = V_{gravity} \cos\theta$ is preferred where
$V_{gravity}$ is the voltage required to hold the arm level with the ground and
$\theta$ is the angle of the arm with the ground.

\subsection{Simulation}

Python Control will be used to \glslink{discretization}{discretize} the
\gls{model} and simulate it. One of the frccontrol
examples\footnote{\url{https://github.com/calcmogul/frccontrol/blob/master/examples/single_jointed_arm.py}}
creates and tests a controller for it.

Figure \ref{fig:single_jointed_arm_pzmaps} shows the pole-zero maps for the
open-loop \gls{system}, closed-loop \gls{system}, and \gls{observer}. Figure
\ref{fig:single_jointed_arm_response} shows the \gls{system} response with them.

\begin{svg}{build/single_jointed_arm_pzmaps}
  \caption{Single-jointed arm pole-zero maps}
  \label{fig:single_jointed_arm_pzmaps}
\end{svg}

\begin{svg}{build/single_jointed_arm_response}
  \caption{Single-jointed arm response}
  \label{fig:single_jointed_arm_response}
\end{svg}

\subsection{Implementation}

The script linked above also generates two files: SingleJointedArmCoeffs.h and
SingleJointedArmCoeffs.cpp. These can be used with the WPILib StateSpacePlant,
StateSpaceController, and StateSpaceObserver classes in C++ and Java. A C++
implementation of this single-jointed arm controller is available
online\footnote{\url{https://github.com/calcmogul/allwpilib/tree/state-space/wpilibcExamples/src/main/cpp/examples/StateSpaceSingleJointedArm}}.
