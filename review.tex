\section{What is control theory?}

How can we prove an autonomous car will behave safely? Control theory is used to
analyze and predict \gls{system} behavior. It's a pragmatic application of
algebra, geometry, and jargon with the goal of producing a desired \gls{system}
response. Feedback control facilitates producing a desired \gls{system}
response. Control theory involves judiciously expending engineering effort to
meet performance specifications.

\section{Nomenclature}

Most resources for advanced engineering topics assume a level of knowledge well
above that which is necessary. See the glossary for a list of words and phrases
commonly used in control theory, their origins, and their meaning. \\

For example, the \glspl{state} for a DC brushed motor would include the angular
velocity $\dot{\theta}$ and current $i$. Voltage $V$ is an input. With perfect
knowledge of the angular velocity and current, all future \glspl{state} of the
\gls{system} can be predicted given an input voltage signal.

\section{Control system diagrams}

\subsection{What is gain?}

Gain is a proportional value that shows the relationship between the magnitude
of the input to the magnitude of the output signal at steady state. Many
\glspl{system} contain a method by which the gain can be altered, providing more
or less "power" to the \gls{system}. However, increasing gain or decreasing gain
beyond a particular safety zone can cause the \gls{system} to become unstable.

\subsection{Block diagrams}

When designing or analyzing a control system, it is useful to model it
graphically. Block diagrams are used for this purpose. They can be manipulated
and simplified systematically \cite{bib:block_diagrams}. Figure
\ref{fig:feedback_loop} is an example of one with a feedback configuration.

\begin{figure}[H]
  \centering

  \begin{tikzpicture}[auto, >=latex']
    % Place the blocks
    \node [name=input] {$X(s)$};
    \node [sum, right=of input] (sum) {};
    \node [block, right=of sum] (P1) {$P_1$};
    \node [right=of P1] (output) {$Y(s)$};
    \node [block, below=of P1] (P2) {$P_2$};

    % Connect the nodes
    \draw [arrow] (input) -- node[pos=0.85] {$+$} (sum);
    \draw [arrow] (sum) -- node {} (P1);
    \draw [arrow] (P1) -- node[name=y] {} (output);
    \draw [arrow] (y) |- (P2);
    \draw [arrow] (P2) -| node[pos=0.97, right] {$\mp$} (sum);
  \end{tikzpicture}

  \caption{Feedback block diagram}
  \label{fig:feedback_loop}
\end{figure}

\section{Review of PID controller mathematics}

\subsection{PID basics and theory}

Negative feedback loops drive the difference between \gls{reference} and
\gls{output} to zero. \\

\textbf{Proportional} gain compensates for current \gls{error}. \\
\textbf{Integral} gain compensates for past error (i.e.,
\gls{steady-state error}). \\
\textbf{Derivative} gain compensates for future error by slowing controller down
  if error decreases over time.

\begin{figure}[H]
  \centering

  \begin{tikzpicture}[auto, >=latex']
    \fontsize{9pt}{10pt}

    % Place the blocks
    \node [name=input] {$r(t)$};
    \node [sum, right=0.5cm of input] (errorsum) {};
    \node [coordinate, right=0.75cm of errorsum] (branch) {};
    \node [block, right=0.5cm of branch] (I) { $K_i \int_0^t e(\tau) d\tau$ };
    \node [block, above=0.5cm of I] (P) { $K_p e(t)$ };
    \node [block, below=0.5cm of I] (D) { $K_d \frac{de(t)}{dt}$ };
    \node [sum, right=0.5cm of I] (ctrlsum) {};
    \node [block, right=0.75cm of ctrlsum] (plant) {Plant};
    \node [right=0.75cm of plant] (output) {};
    \node [coordinate, below=0.5cm of D] (measurements) {};

    % Connect the nodes
    \draw [arrow] (input) -- node[pos=0.9] {$+$} (errorsum);
    \draw [-] (errorsum) -- node {$e(t)$} (branch);
    \draw [arrow] (branch) |- (P);
    \draw [arrow] (branch) -- (I);
    \draw [arrow] (branch) |- (D);
    \draw [arrow] (P) -| node[pos=0.95, left] {$+$} (ctrlsum);
    \draw [arrow] (I) -- node[pos=0.9, below] {$+$} (ctrlsum);
    \draw [arrow] (D) -| node[pos=0.95, right] {$+$} (ctrlsum);
    \draw [arrow] (ctrlsum) -- node {$u(t)$} (plant);
    \draw [arrow] (plant) -- node [name=y] {$y(t)$} (output);
    \draw [-] (y) |- (measurements);
    \draw [arrow] (measurements) -| node[pos=0.99, right] {$-$} (errorsum);
  \end{tikzpicture}

  \caption{PID controller diagram}
  \label{fig:pid_ctrl_diag}
\end{figure}

\begin{table}[ht]
  \renewcommand{\arraystretch}{1.3}
  \centering
  \begin{tabulary}{\linewidth}{LLLL}
    $r(t)$ & \gls{reference} input & $u(t)$ & control input \\
    $e(t)$ & error & $y(t)$ & \gls{output} \\
  \end{tabulary}
  \label{tab:pid_def}
\end{table}

\begin{table}[ht]
  \caption{Plant versus controller}
  \renewcommand{\arraystretch}{1.3}
  \centering
  \begin{tabular}{|l|ll|}
    \hline
    \rowcolor{lightblue}
    & \textbf{Plant} & \textbf{Controller} \\
    \hline
    Input & $u(t)$ & $r(t)$, $y(t)$ \\
    Output & $y(t)$ & $u(t)$ \\
    \hline
  \end{tabular}
  \label{tab:plant_v_controller}
\end{table}

\subsection{Types of PID controllers}

PID controller inputs of different orders of derivatives, such as position and
velocity, affect the \gls{system} response differently. The position PID
controller is defined as

\begin{equation}
  u(t) = K_p e(t) + K_i \int_0^t e(\tau) d\tau + K_d \frac{de}{dt} \\
\end{equation}
\\
If a velocity is passed instead, which is a change in position, the equation
becomes

\begin{align}
  \frac{du}{dt} &= K_p \frac{de}{dt} + K_i \int_0^t \frac{de}{d\tau} d\tau +
    K_d \frac{d^2e}{dt^2} \nonumber \\
  \frac{du}{dt} &= K_p \frac{de}{dt} + K_i e(t) + K_d \frac{d^2e}{dt^2}
    \label{eq:pid_vel}
\end{align}
\\
This shows that $K_i$ and $K_p$ from the position controller act as proportional
and derivative terms respectively in the velocity controller. $K_i$ from the
position controller has no equivalent in the velocity controller. If we were to
implement one, it would use a double integral. However, it would be of limited
use since the $K_i$ term in equation (\ref{eq:pid_vel}) also eliminates
steady-state error for step changes in \gls{reference}. Relabelling the
coefficients to match the position PID controller gives

\begin{equation}
  \frac{du}{dt} = K_p \int_0^t e(\tau) d\tau + K_d e(t) \\
\end{equation}
\\
Read \url{https://en.wikipedia.org/wiki/PID_controller} for more information.

\subsection{Limitations of PID control}

PID's heuristic method of tuning is fine when there is no knowledge of the
\gls{system}. However, controllers with much better response can be developed if
a dynamical model of the \gls{system} is known.
