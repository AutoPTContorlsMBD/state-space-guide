\section{The structure of this book}

This book consists of four parts and a collection of appendices:

\begin{itemize}
  \item Part I, ``Classical control theory," introduces the basics of control
    theory, teaches the fundamentals of PID controller design, describes what a
    transfer function is, and shows how they can be used to analyze dynamical
    systems. Emphasis is placed on the geometric intuition of this analysis
    rather than the frequency domain math.
  \item Part II, ``Modern control theory notation," provides a crash course in
    the geometric intuition behind linear algebra and covers enough of the
    mechanics of evaluating matrix algebra for the reader to follow along in
    later chapters. It covers state-space representation, controllability, and
    observability.
  \item Part III, ``Linear control theory," builds on the intuition gained in
    part I and the notation described in part II to model and control linear
    multiple-input, multiple-output (MIMO) systems. It covers discretization,
    LQR controller design, LQE observer design, feedforwards, motion profiles,
    and stochastic control theory. Then, these concepts are applied to design
    and implement controllers for real systems. It walks through several
    examples of common FRC subsystems from deriving the model using kinematics
    to implementing and testing a digital controller.
  \item Part IV, ``Nonlinear control theory," introduces the basics of nonlinear
    control system analysis with Lyapunov functions. It presents an example of a
    nonlinear controller for a unicycle-like vehicle as well as how to apply it
    to a two-wheeled vehicle. Since nonlinear control isn't the focus of this
    book, we mention other books and resources for further reading.
  \item The appendices provide further enrichment that isn't required for a
    passing understanding of the material. This includes derivations for many of
    the results presented and used in the mainmatter of the book.
\end{itemize}

The Python scripts used to generate the plots in the case studies double as
reference implementations of the techniques discussed in their respective
chapters. They are available in this book's Git repository. Its location is
listed on the copyright page.

This book is intended as both a tutorial for new students and as a reference
manual for more experienced readers who need to review a thing or two. While it
isn't comprehensive, the reader will hopefully learn enough to either implement
the concepts presented themselves or know where to look for more information.

Some parts are mathematically rigorous, but I believe in giving students a solid
theoretical foundation with emphasis on intuition so they can apply it to new
problems. To achieve deep understanding of the topics in this book, math is
unavoidable.

The sections on classical control theory are intended to provide a geometric
intuition into the mathematical machinery of modern control theory. Modern
control requires doing roughly three things: develop a kinematic model of the
system, design a controller for the system based on the model, and design an
observer to estimate hidden states of the system or account for noise. This book
covers how to do each.

Some topics have been oversimplified to make them easier to grasp. For more
detail, please see the Wikibook on control systems at
\url{https://en.wikibooks.org/wiki/Control_Systems}.
