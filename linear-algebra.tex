\chapterimage{linear-algebra.jpg}

\chapter{Linear algebra}

Modern control theory borrows concepts from linear algebra. At first, linear
algebra may appear very abstract, but there are simple geometric intuitions
underlying it. First, watch 3Blue1Brown's preview video for the
\textit{Essence of Linear Algebra} video series (5 minutes)
\cite{bib:linalg_preview}. The goal here is to provide an intuitive, geometric
understanding of linear algebra as a method of linear transformations.

While only a subset of the material from the videos will be presented here, I
highly suggest watching the whole series \cite{bib:essence_of_linalg}.

\section{Vectors}

See the corresponding \textit{Essence of Linear Algebra} video for more (5
minutes) \cite{bib:linalg_vectors}.

\section{Linear combinations, span, and basis vectors}

See the corresponding \textit{Essence of Linear Algebra} video for more (10
minutes) \cite{bib:linalg_linear_combinations}.

\section{Linear transformations and matrices}

See the corresponding \textit{Essence of Linear Algebra} video for more (11
minutes) \cite{bib:linalg_linear_transformations_and_matrices}.

\section{Matrix multiplication as composition}

See the corresponding \textit{Essence of Linear Algebra} video for more (10
minutes) \cite{bib:linalg_matrix_multiplication_as_composition}.

\section{The determinant}

See the corresponding \textit{Essence of Linear Algebra} video for more (10
minutes) \cite{bib:linalg_the_determinant}.

\section{Eigenvectors and eigenvalues}

See the corresponding \textit{Essence of Linear Algebra} video for more (17
minutes) \cite{bib:linalg_eigenvectors_and_eigenvalues}.

\section{Miscellaneous notation}

This document works with two-dimensional matrices. The dimensionality of these
matrices is specified by row first, then column. For example, a matrix with two
rows and three columns would be a two-by-three matrix. A square matrix has the
same number of rows as columns.

The matrix $\mtx{I}$ is known as the identity matrix, which is a square matrix
with ones along its diagonal and zeroes elsewhere. For example

\begin{equation*}
  \left[
  \begin{array}{ccc}
    1 & 0 & 0 \\
    0 & 1 & 0 \\
    0 & 0 & 1
  \end{array}
  \right]
\end{equation*}

The matrix denoted by $\mtx{0}_{m \times n}$ is a matrix filled with zeroes with
$m$ rows and $n$ columns.

The $^T$ in $\mtx{A}^T$ denotes transpose, which flips the matrix across its
diagonal such that the rows become columns and vice versa.
