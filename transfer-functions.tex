\section{Transfer functions}

\subsection{What is a transfer function?}

A transfer function maps an input to an output in the Laplace domain. This is
essentially a two-dimensional frequency domain on the complex plane (real
numbers on the x-axis and imaginary numbers on the y-axis). These can be
obtained by applying the Laplace transform to a differential equation and
rearranging the terms to obtain a ratio of the output variable to the input
variable. Equation \ref{eq:transfer_func} is an example of a transfer function.

\begin{equation} \label{eq:transfer_func}
  H(s) = \frac{(s-9+9i)(s-9-9i)}{s(s+10)}
\end{equation}

Roots of the numerator and denominator are called residues. Residues on the top
are called zeroes while residues on the bottom are called poles. This is due to
poles making the expression approach infinity for values of $s$ that make the
residue zero (they look like poles of a circus tent on a 3D graph). Similar
logic applies to zeroes. Imaginary roots always come in complex conjugate pairs
(e.g., $-2 + 3i$, $-2 - 3i$).

\subsection{Transfer functions in feedback}

For \glspl{controller} to do their job, they must be placed in positive or
negative feedback with the \gls{plant} (whether to use positive or negative
depends on the \gls{plant} in question). The transfer function of figure
\ref{fig:feedback_loop}, a control system diagram with feedback, from input to
output is

\begin{equation}
  G_{cl}(s) = \frac{Y(s)}{X(s)} = \frac{P_1}{1 + P_1 P_2}
\end{equation}

The numerator is the \gls{open-loop gain} and the denominator is the gain around
the feedback loop, which may include parts of the \gls{open-loop gain} (see
appendix \ref{sec:app-tf-feedback-deriv} for a derivation). Plants are generally
denoted with the letter $G$. As another example, the transfer function from the
input to the error is

\begin{equation}
  G_{cl}(s) = \frac{E(s)}{X(s)} = \frac{1}{1 + P_1 P_2}
\end{equation}

The roots of the denominator of $G_{cl}(s)$ are different from those of the
open-loop transfer function $P_1(s)$. These are called the closed-loop poles.
