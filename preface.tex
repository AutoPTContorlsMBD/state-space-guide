\chapterimage{chapter_head_1.pdf}

\chapter*{Preface}
\addcontentsline{toc}{chapter}{\textcolor{deeporange}{Preface}}

\section*{Motivation}

I am the software mentor for a FIRST Robotics Competition team. My
responsibilities for that include teaching programming, software engineering
practices, and applications of control theory. The curriculum I developed so far
(located at \url{https://csweb.frc3512.com/ci/}) teaches rookies enough to be
minimally competitive, but many of the more advanced sections are incomplete. It
provides no formal avenues of growth for veteran students. \\

This project expands my curriculum to cover control theory topics I have learned
in my graduate-level engineering classes at University of California,
Santa Cruz. It introduces state-space controllers and serves as a practical
guide for formulating and implementing them. State-space control is a way of
formulating control problems using linear algebra that has several advantages
over simpler, typically heuristic approaches like
Proportional-Integral-Derivative (PID). \\

The sections on classical control theory are intended to provide a geometric
intuition into the mathematical machinery. Some topics have been oversimplified
to make them easier to grasp. For more detail, please see the Wikibook on
control systems at \url{https://en.wikibooks.org/wiki/Control_Systems}.

\section*{Abstract}

Control theory is a discipline that deals with the behavior of dynamical
\glspl{system} with inputs and how their behavior is modified by feedback.
Topics from classical and modern control are introduced, and advice is given on
when and how to apply them.

\section*{Intended Audience}

This guide is intended to make an advanced engineering topic approachable so it
can be applied by those who aren't experts in control theory. My intended
audience is high school students who are veteran members of a FIRST Robotics
Competition team. As such, they will already be familiar with feedback control
applications like PID and have basic proficiency in programming. This guide will
build on their current knowledge of control theory and teach them enough about
state-space control and auxiliary topics to be able to implement it in the
programming language of their choice. \\

Knowledge of basic algebra, complex numbers, physics, and a bit of calculus (for
the system modeling) are assumed.

\section*{Acknowledgements}

I would like to thank my controls engineering instructors Dejan Milutinovi\'c
and Gabriel Elkaim of University of California, Santa Cruz. They taught their
classes from a pragmatic perspective focused on application and intuition that I
appreciated. I would also like to thank Dejan Milutinovi\'c for introducing me
to the field of control theory and teaching me what it means to be a controls
engineer. \\

Thanks to Austin Schuh from FRC team 971 for providing the final continuous
state-space models used in the examples section.

\cleardoublepage
