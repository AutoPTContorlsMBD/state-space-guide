\chapterimage{preface.jpg}

\chapter*{Preface}
\addcontentsline{toc}{chapter}{\textcolor{themecolor}{Preface}}

\section*{Motivation}

I am the software mentor for a FIRST Robotics Competition (FRC) team. My
responsibilities for that include teaching programming, software engineering
practices, and applications of control theory. The curriculum I developed so far
(located at \url{https://csweb.frc3512.com/ci/}) teaches rookies enough to be
minimally competitive, but many of the more advanced sections are incomplete. It
provides no formal avenues of growth for veteran students.

Also, out of a six week build season, the software team usually only gets a few
days with the completed robot due to poor build schedule management. This leads
to two problems. First, two days is only enough time to verify basic software
functionality, not test and tune feedback controllers. Second, this is also the
first time the robot's electromechanical systems have been tested after
integration, so any issues that arise consume valuable software integration time
while the team traces the problem to a mechanical, electrical, or software
cause.

This book expands my curriculum to cover control theory topics I have learned in
my graduate-level engineering classes at University of California, Santa Cruz.
It introduces state-space controllers and serves as a practical guide for
formulating and implementing them. Since state-space control utilizes a system
model, both problems mentioned earlier can be addressed. Basic software
functionality can be tested against it and feedback controllers can be tuned
automatically based on system constraints. This allows software teams to test
their work much earlier in the build season in a controlled environment as well
as save time during feedback controller design, implementation, and testing.

\section*{Intended Audience}

This guide is intended to make an advanced engineering topic approachable so it
can be applied by those who aren't experts in control theory. My intended
audience is high school students who are veteran members of a FIRST Robotics
Competition team. As such, they will already be familiar with feedback control
applications like PID and have basic proficiency in programming. This guide will
build on their current knowledge of control theory and teach them enough about
state-space control and auxiliary topics to be able to implement it in the
programming language of their choice.

Knowledge of basic algebra, complex numbers, physics, and a bit of calculus (for
the system modeling) are assumed.

\section*{Acknowledgements}

I would like to thank my controls engineering instructors Dejan Milutinovi\'c
and Gabriel Elkaim of University of California, Santa Cruz. They taught their
classes from a pragmatic perspective focused on application and intuition that I
appreciated. I would also like to thank Dejan Milutinovi\'c for introducing me
to the field of control theory and teaching me what it means to be a controls
engineer.

Thanks to Austin Schuh from FRC team 971 for providing the final continuous
state-space models used in the examples section.

\cleardoublepage
