\chapterimage{kinematics-and-dynamics.jpg}{Hills by freeway between Santa Maria and Ventura}

\chapter{Model examples}

The \glspl{model} derived here should cover most types of motion seen on an FRC
robot. Furthermore, they can be easily tweaked to describe many types of
mechanisms just by pattern-matching. There's only so many ways to hook up a mass
to a motor in FRC. The flywheel \gls{model} can be used for spinning mechanisms,
the elevator \gls{model} can be used for spinning mechanisms transformed to
linear motion, and the single-jointed arm \gls{model} can be used for rotating
servo mechanisms (it's just the flywheel \gls{model} augmented with a position
\gls{state}).

These \glspl{model} assume all motor controllers driving DC brushed motors are
set to brake mode instead of coast mode. Brake mode behaves the same as coast
mode except where the applied voltage is zero. In brake mode, the motor leads
are shorted together to prevent movement. In coast mode, the motor leads are an
open circuit.

\renewcommand*{\chapterpath}{\partpath/model-examples}
\section{DC brushed motor}

\subsection{Equations of motion}

The circuit for a DC brushed motor is shown below.

\begin{figure}[H]
  \centering

  \begin{tikzpicture}[auto, >=latex', circuit ee IEC,
                      set resistor graphic=var resistor IEC graphic]
    \node [opencircuit] (start) at (0,0) {};
    \node [] (V+) at (-0.5,0) { $+$ };
    \node [opencircuit] (end) at (0,-3.5) {};
    \node [] (V-) at (-0.5,-3.5) { $-$ };
    \node [coordinate] (topright) at (2.5,0) {};
    \node [coordinate] (bottomright) at (2.5,-3.5) {};
    \node [] at (0, -1.75) { $V$ };
    \draw (start) to (topright)
                  to [resistor={near start, info'={ $R$ }},
                      voltage source={near end, direction info'={<-},
                      info={ $V_{emf}=\frac{\omega_m}{K_v}$ }}] (bottomright)
                  to (end);
  \end{tikzpicture}

  \caption{DC brushed motor circuit}
  \label{fig:dc_motor_circuit}
\end{figure}

where $V$ is the voltage applied to the motor, $I$ is the current through the
motor in Amps, $R$ is the resistance across the motor in Ohms, $\omega_m$ is the
angular velocity of the motor in radians per second, and $K_v$ is the angular
velocity constant in radians per second per Volt. This circuit reflects the
following relation.

\begin{equation}
  V = IR + \frac{\omega_m}{K_v} \label{eq:motor_V}
\end{equation}

The mechanical relation for a DC brushed motor is

\begin{equation}
  \tau_m = K_t I \label{eq:motor_tau_m}
\end{equation}

where $\tau_m$ is the torque produced by the motor in Newton-meters and $K_t$ is
the torque constant in Newton-meters per Amp. Therefore

\begin{equation*}
  I = \frac{\tau_m}{K_t}
\end{equation*}

Substitute this into equation (\ref{eq:motor_V}).

\begin{equation}
  V = \frac{\tau_m}{K_t} R + \frac{\omega_m}{K_v} \label{eq:motor_tau_V}
\end{equation}

\subsection{Calculating constants}

A typical motor's datasheet should include graphs of the motor's measured torque
and current for different angular velocities for a given voltage applied to the
motor. To find $K_t$

\begin{align}
  \tau_m &= K_t I \nonumber \\
  K_t &= \frac{\tau_m}{I} \nonumber \\
  K_t &= \frac{\tau_{m,stall}}{I_{stall}}
\end{align}

where $\tau_{m,stall}$ is the stall torque and $I_{stall}$ is the stall current
of the motor from its datasheet.

To find $R$, recall equation (\ref{eq:motor_V}).

\begin{equation*}
  V = IR + \frac{\omega_m}{K_v}
\end{equation*}

When the motor is stalled, $\omega_m = 0$.

\begin{align}
  V &= I_{stall} R \nonumber \\
  R &= \frac{V}{I_{stall}}
\end{align}

where $I_{stall}$ is the stall current of the motor and $V$ is the voltage
applied to the motor at stall.

To find $K_v$, again recall equation (\ref{eq:motor_V}).

\begin{align*}
  V &= IR + \frac{\omega_m}{K_v} \\
  V - IR &= \frac{\omega_m}{K_v} \\
  K_v &= \frac{\omega_m}{V - IR}
\end{align*}

When the motor is spinning under no load

\begin{align}
  K_v &= \frac{\omega_{m,free}}{V - I_{free}R}
\end{align}

where $\omega_{m,free}$ is the angular velocity of the motor under no load (also
known as the free speed), and $V$ is the voltage applied to the motor when it's
spinning at $\omega_{m,free}$, and $I_{free}$ is the current drawn by the motor
under no load.

If several identical motors are being used in one gearbox for a mechanism,
multiply the stall torque, stall current, and free current by the number of
motors. This makes sense because the motor characteristics calculated above
haven't changed, but the amount of torque available and current consumed for the
same input voltage has been multiplied.

\section{Pendulum} \label{sec:implementation_steps}

A \gls{model} is a set of differential equations describing how the \gls{system}
behaves over time. There are two common approaches for developing them.

\begin{enumerate}
  \item Collecting data on the physical system's behavior and performing
    \gls{system} identification with it.
  \item Using physics to derive the \gls{system}'s model from first principles.
\end{enumerate}

We'll use the second approach in this book.

Kinematics and dynamics are a rather large topics, so for now, we'll just focus
on the basics required for working with the \glspl{model} in this book. We'll
derive the same \gls{model}, a pendulum, using three approaches: sum of forces,
sum of torques, and conservation of energy.

\begin{bookfigure}
  \begin{subfigure}{0.5\textwidth}
    \centering
    \begin{tikzpicture}
      % Save length of g-vector and theta to macros
      \pgfmathsetmacro{\Gvec}{1.5}
      \pgfmathsetmacro{\myAngle}{30}
      % Calculate lengths of vector components
      \pgfmathsetmacro{\Gcos}{\Gvec*cos(\myAngle)}
      \pgfmathsetmacro{\Gsin}{\Gvec*sin(\myAngle)}

      \coordinate (centro) at (0,0);
      \draw[dashed,gray,-] (centro) -- ++ (0,-3.5)
        node (mary) [black,below] {$ $};
      \draw[thick] (centro) -- ++(270+\myAngle:3) coordinate (bob);
      \path pic [draw,->,"$\theta$",angle eccentricity=1.5]
        {angle=mary--centro--bob};
      \draw [draw=violet,-stealth] (bob) -- ($(bob)!-\Gcos cm!(centro)$)
        coordinate (gcos)
        node[midway,above right] {$mg\cos\theta$};
      \draw [dashed,draw=red,-stealth] (bob) -- ($(bob)!2*\Gsin cm!90:(centro)$)
        coordinate node[midway,above right] {};
      \draw [draw=violet,-stealth] (bob) -- ($(bob)!\Gsin cm!90:(centro)$)
        coordinate (gsin)
        node[midway,above left] {$mg\sin\theta$};
      \draw [draw=blue,-stealth] (bob) -- ++(0,-\Gvec)
        coordinate (g)
        node[near end,left] {$mg$};
      \pic [draw,->,"$\theta$",angle eccentricity=1.5] {angle=g--bob--gcos};
      \filldraw [fill=black!40,draw=black] (bob) circle[radius=0.2];
    \end{tikzpicture}
    \caption{Force diagram of a pendulum}
    \label{subfig:force_pendulum}
  \end{subfigure}%
  \begin{subfigure}{0.5\textwidth}
    \centering
    \begin{tikzpicture}
      % Save length of g-vector and theta to macros
      \pgfmathsetmacro{\Gvec}{1.5}
      \pgfmathsetmacro{\myAngle}{30}
      % Calculate lengths of vector components
      \pgfmathsetmacro{\Gcos}{\Gvec*cos(\myAngle)}
      \pgfmathsetmacro{\Gsin}{\Gvec*sin(\myAngle)}

      \coordinate (centro) at (0,0);
      \coordinate (heightmes_lo) at (-1,0);
      \coordinate (heightmes_hi) at (-0.25,0);
      \coordinate (h) at (\Gcos/2,\Gsin);

      \draw[thick] (centro) -- ++(270+\myAngle:3) coordinate (bob_lo);
      \draw[dashed,gray,-] (centro) -- ++ (0,0 |- bob_lo)
        node (mary) [black,below]{$ $};
      \draw[dashed,gray,-] (heightmes_lo |- bob_lo) -- (bob_lo)
        node [black,below]{$ $};
      \draw[<->] (heightmes_lo) -- ++ (0,0 |- bob_lo)
        node [black,pos=0.5,left]{$y_1$};
      \pic [draw,->, "$\theta$",angle eccentricity=1.5]
        {angle=mary--centro--bob_lo};

      % Save length of g-vector and theta to macros
      \pgfmathsetmacro{\Gvec}{1.5cm}
      \pgfmathsetmacro{\myAngle}{45}
      % Calculate lengths of vector components
      \pgfmathsetmacro{\Gcos}{\Gvec*cos(\myAngle)}
      \pgfmathsetmacro{\Gsin}{\Gvec*sin(\myAngle)}

      \draw[gray,thick] (centro) -- ++(270+\myAngle:3) coordinate (bob_hi);
      \pic [draw,->,"$\theta_0$",angle eccentricity=1.5,angle radius=1cm]
        {angle=mary--centro--bob_hi};

      \draw[dashed,gray,-] (0,0 |- bob_hi) -- (bob_hi)
        node (mary) [black,below]{$ $};
      \draw[<->] (heightmes_hi) -- ++ (0,0 |- bob_hi)
        node (mary) [black,pos=0.5,left]{$y_0$};
      \draw[<->] (h |- bob_hi) -- (h |- bob_lo)
        node [black,pos=0.5,left]{$h$};

      % Path of pendulum
      \pic [draw,dashed,gray,<-,angle eccentricity=1.5,angle radius=2*\Gvec]
        {angle=mary--centro--bob_hi};

      % Pendulum balls
      \filldraw [fill=black!40,draw=black] (bob_lo) circle[radius=0.2];
      \filldraw [fill=black!20,draw=gray] (bob_hi) circle[radius=0.2];
    \end{tikzpicture}
    \caption{Trigonometry of a pendulum}
    \label{subfig:trig_pendulum}
  \end{subfigure}
  \caption{Pendulum force diagrams}
\end{bookfigure}

\subsection{Force derivation}
\index{Physics!sum of forces}

Consider figure \ref{subfig:force_pendulum}, which shows the forces acting on a
pendulum.

Note that the path of the pendulum sweeps out an arc of a circle. The angle
$\theta$ is measured in radians. The blue arrow is the gravitational force
acting on the bob, and the violet arrows are that same force resolved into
components parallel and perpendicular to the bob's instantaneous motion. The
direction of the bob's instantaneous velocity always points along the red axis,
which is considered the tangential axis because its direction is always tangent
to the circle. Consider Newton's second law

\begin{equation*}
  F = ma
\end{equation*}

where $F$ is the sum of forces on the object, $m$ is mass, and $a$ is the
acceleration. Because we are only concerned with changes in speed, and because
the bob is forced to stay in a circular path, we apply Newton's equation to the
tangential axis only. The short violet arrow represents the component of the
gravitational force in the tangential axis, and trigonometry can be used to
determine its magnitude. Therefore

\begin{align*}
  -mg\sin\theta &= ma \\
  a &= -g\sin\theta
\end{align*}

where $g$ is the acceleration due to gravity near the surface of the earth. The
negative sign on the right hand side implies that $\theta$ and a always point in
opposite directions. This makes sense because when a pendulum swings further to
the left, we would expect it to accelerate back toward the right.

This linear acceleration $a$ along the red axis can be related to the change in
angle $\theta$ by the arc length formulas; $s$ is arc length and $l$ is the
length of the pendulum.

\begin{align}
  s &= l\theta \label{eq:arc_length} \\
  v &= \frac{ds}{dt} = l\frac{d\theta}{dt} \nonumber \\
  a &= \frac{d^2s}{dt^2} = l\frac{d^2\theta}{dt^2} \nonumber
\end{align}

Therefore

\begin{align*}
  l\frac{d^2\theta}{dt^2} &= -g\sin\theta \\
  \frac{d^2\theta}{dt^2} &= -\frac{g}{l}\sin\theta \\
  \ddot{\theta} &= -\frac{g}{l}\sin\theta
\end{align*}

\subsection{Torque derivation}
\index{Physics!sum of torques}

The equation can be obtained using two definitions for torque.

\begin{equation*}
  \mtx{\tau} = \mtx{r} \times \mtx{F}
\end{equation*}

First start by defining the torque on the pendulum bob using the force due to
gravity.

\begin{equation*}
  \mtx{\tau} = \mtx{l} \times \mtx{F}_g
\end{equation*}

where $\mtx{l}$ is the length vector of the pendulum and $\mtx{F}_g$ is the
force due to gravity.

For now just consider the magnitude of the torque on the pendulum.

\begin{equation*}
  \lvert\tau\rvert = -mgl\sin\theta
\end{equation*}

where $m$ is the mass of the pendulum, $g$ is the acceleration due to gravity,
$l$ is the length of the pendulum and $\theta$ is the angle between the length
vector and the force due to gravity.

Next rewrite the angular momentum.

\begin{equation*}
  \mtx{L} = \mtx{r} \times \mtx{p} =
    m\mtx{r} \times (\mtx{\omega} \times \mtx{r})
\end{equation*}

Again just consider the magnitude of the angular momentum.

\begin{align*}
  \lvert\mtx{L}\rvert &= mr^2\omega \\
  \lvert\mtx{L}\rvert &= ml^2 \frac{d\theta}{dt} \\
  \frac{d}{dt}\lvert\mtx{L}\rvert &= ml^2 \frac{d^2\theta}{dt^2}
\end{align*}

According to $\tau = \frac{d\mtx{L}}{dt}$, we can just compare the magnitudes.

\begin{align*}
  -mgl\sin\theta &= ml^2\frac{d^2\theta}{dt^2} \\
  -\frac{g}{l}\sin\theta &= \frac{d^2\theta}{dt^2} \\
  \ddot{\theta} &= -\frac{g}{l}\sin\theta
\end{align*}

which is the same result from force analysis.

\subsection{Energy derivation}
\index{Physics!conservation of energy}

The equation can also be obtained via the conservation of mechanical energy
principle: any object falling a vertical distance $h$ would acquire kinetic
energy equal to that which it lost to the fall. In other words, gravitational
potential energy is converted into kinetic energy. Change in potential energy is
given by

\begin{equation*}
  \Delta U = mgh
\end{equation*}

The change in kinetic energy (body started from rest) is given by

\begin{equation*}
  \Delta K = \frac{1}{2}mv^2
\end{equation*}

Since no energy is lost, the gain in one must be equal to the loss in the other

\begin{equation*}
  \frac{1}{2}mv^2 = mgh
\end{equation*}

The change in velocity for a given change in height can be expressed as

\begin{equation*}
  v = \sqrt{2gh}
\end{equation*}

Using equation (\ref{eq:arc_length}), this equation can be rewritten in terms of
$\frac{d\theta}{dt}$.

\begin{align}
  v = l\frac{d\theta}{dt} &= \sqrt{2gh} \nonumber \\
  \frac{d\theta}{dt} &= \frac{2gh}{l} \label{eq:energy_dtheta}
\end{align}

where $h$ is the vertical distance the pendulum fell. Look at figure \ref{subfig:trig_pendulum}, which presents the trigonometry of a pendulum. If the pendulum
starts its swing from some initial angle $\theta_0$, then $y_0$, the vertical
distance from the pivot point, is given by

\begin{equation*}
  y_0 = l\cos\theta_0
\end{equation*}

Similarly for $y_1$, we have

\begin{equation*}
  y_1 = l\cos\theta
\end{equation*}

Then $h$ is the difference of the two

\begin{equation*}
  h = l(\cos\theta - \cos\theta_0)
\end{equation*}

Substituting this into equation (\ref{eq:energy_dtheta}) gives

\begin{equation*}
  \frac{d\theta}{dt} = \sqrt{\frac{2g}{l}(\cos\theta - \cos\theta_0)}
\end{equation*}

This equation is known as the first integral of motion. It gives the velocity in
terms of the location and includes an integration constant related to the
initial displacement ($\theta_0$). We can differentiate by applying the chain
rule with respect to time. Doing so gives the acceleration.

\begin{align*}
  \frac{d}{dt}\frac{d\theta}{dt} &=
    \frac{d}{dt}\sqrt{\frac{2g}{l}(\cos\theta - \cos\theta_0)} \\
  \frac{d^2\theta}{dt^2} &= \frac{1}{2}\frac
    {-\frac{2g}{l}\sin\theta}
    {\sqrt{\frac{2g}{l}(\cos\theta - \cos\theta_0)}}\frac{d\theta}{dt} \\
  \frac{d^2\theta}{dt^2} &= \frac{1}{2}\frac
    {-\frac{2g}{l}\sin\theta}
    {\sqrt{\frac{2g}{l}(\cos\theta - \cos\theta_0)}}
    \sqrt{\frac{2g}{l}(\cos\theta - \cos\theta_0)} \\
  \frac{d^2\theta}{dt^2} &= -\frac{g}{l}\sin\theta \\
  \ddot{\theta} &= -\frac{g}{l}\sin\theta
\end{align*}

which is the same result from force analysis.

\subsection{Elevator}

\subsubsection{Equations of motion}

This elevator consists of a DC brushed motor attached to a pulley that drives a
mass up or down.

\begin{figure}[H]
  \centering

  \begin{tikzpicture}[auto, >=latex', circuit ee IEC,
                      set resistor graphic=var resistor IEC graphic]
    % \draw [help lines] (-1,-3) grid (7,4);

    % Electrical equivalent circuit
    \draw (0,2) to [voltage source={direction info'={->}, info'=$V$}] (0,0);
    \draw (0,2) to [current direction={info=$I$}] (0,3);
    \draw (0,3) -- (0.5,3);
    \draw (0.5,3) to [resistor={info={$R$}}] (2,3);

    \draw (2,3) -- (2.5,3);
    \draw (2.5,3) to [voltage source={direction info'={->}, info'=$V_{emf}$}]
      (2.5,0);
    \draw (0,0) -- (2.5,0);

    % Motor
    \begin{scope}[xshift=2.4cm,yshift=1.05cm]
      \draw[fill=black] (0,0) rectangle (0.2,0.9);
      \draw[fill=white] (0.1,0.45) ellipse (0.3 and 0.3);
    \end{scope}

    % Transmission gear one
    \begin{scope}[xshift=3.75cm,yshift=1.17cm]
      \draw[fill=black!50] (0.2,0.33) ellipse (0.08 and 0.33);
      \draw[fill=black!50, color=black!50] (0,0) rectangle (0.2,0.66);
      \draw[fill=white] (0,0.33) ellipse (0.08 and 0.33);
      \draw (0,0.66) -- (0.2,0.66);
      \draw (0,0) -- (0.2,0) node[pos=0.5,below] {$G$};
    \end{scope}

    % Output shaft of motor
    \begin{scope}[xshift=2.8cm,yshift=1.45cm]
      \draw[fill=black!50] (0,0) rectangle (0.95,0.1);
    \end{scope}

    % Angular velocity arrow of drive -> transmission
    \draw[line width=0.7pt,<-] (3.2,1) arc (-30:30:1) node[above] {$\omega_m$};

    % Transmission gear two
    \begin{scope}[xshift=3.75cm,yshift=1.83cm]
      \draw[fill=black!50] (0.2,0.68) ellipse (0.13 and 0.67);
      \draw[fill=black!50, color=black!50] (0,0) rectangle (0.2,1.35);
      \draw[fill=white] (0,0.68) ellipse (0.13 and 0.67);
      \draw (0,1.35) -- (0.2,1.35);
      \draw (0,0) -- (0.2,0);
    \end{scope}

    % Pulley rear chain
    \begin{scope}[xshift=5.03cm,yshift=0.32cm]
      \draw[fill=black!70, color=black!70] (0.01,2.17) rectangle (0.09,0);
      \draw (0,2.17) -- (0,0);
      \draw (0.1,2.17) -- (0.1,0);
    \end{scope}

    % Upper pulley
    \begin{scope}[xshift=5.05cm,yshift=2.09cm]
      \draw[fill=black!50] (0.2,0.4) ellipse (0.13 and 0.4);
      \draw[fill=black!70] (0.15,0.4) ellipse (0.13 and 0.4);
      \draw[fill=black!50, color=black!50] (0,0) rectangle (0.1,0.8);
      \draw[fill=black!70, color=black!70] (0.1,0) rectangle (0.15,0.8);
      \draw[fill=black!50] (0.05,0.4) ellipse (0.13 and 0.4);
      \draw[fill=black!50, color=black!50] (0,0) rectangle (0.05,0.8);
      \draw[fill=white] (0,0.4) ellipse (0.13 and 0.4);
      \draw (0,0) -- (0.2,0);
      \draw (0,0.8) -- (0.2,0.8);
    \end{scope}

    % Lower pulley
    \begin{scope}[xshift=5.05cm,yshift=-0.05cm]
      \draw[fill=black!50] (0.2,0.4) ellipse (0.13 and 0.4);
      \draw[fill=black!70] (0.15,0.4) ellipse (0.13 and 0.4);
      \draw[fill=black!50, color=black!50] (0,0) rectangle (0.1,0.8);
      \draw[fill=black!70, color=black!70] (0.1,0) rectangle (0.15,0.8);
      \draw[fill=black!50] (0.05,0.4) ellipse (0.13 and 0.4);
      \draw[fill=black!50, color=black!50] (0,0) rectangle (0.05,0.8);
      \draw[fill=white] (0,0.4) ellipse (0.13 and 0.4);
      \draw (0,0) -- (0.2,0);
      \draw (0,0.8) -- (0.2,0.8);
    \end{scope}

    % Transmission shaft from gear two to pulley
    \begin{scope}[xshift=4.09cm,yshift=2.42cm]
      \draw[fill=black!50] (0,0) rectangle (0.96,0.1);
    \end{scope}

    % Angular velocity arrow between transmission and pulley
    \draw[line width=0.7pt,->] (4.54,1.97) arc (-30:30:1) node[above]
      {$\omega_p$};

    % Pulley front chain
    \begin{scope}[xshift=5.23cm,yshift=0.32cm]
      \draw[fill=black!70, color=black!70] (0.01,2.17) rectangle (0.09,0);
      \draw (0,2.17) -- (0,0);
      \draw (0.1,2.17) -- (0.1,0);
    \end{scope}

    % Pulley radius arrow
    \begin{scope}[xshift=5.54cm,yshift=2.49]
      \draw[line width=0.7pt,<->] (0,0) -- node[right] {$r$} (0,0.4);
    \end{scope}

    % Mass
    \begin{scope}[xshift=4.89cm,yshift=0.82cm]
      \fill[fill=white] (0,0.8) -- (0,0.2) -- (0.2,0) -- (0.2,0.2)
        -- (0.98,0.2) -- (0.78,0.8) -- cycle;
      \draw (0,0.8) -- (0.78,0.8);
      \draw (0,0.8) -- (0,0.2);
      \draw (0,0.2) -- (0.2,0);
      \draw (0,0.8) -- (0.2,0.6);
      \draw (0.78,0.8) -- (0.98,0.6);
      \draw[fill=white] (0.2,0.6) rectangle (0.98,0);
    \end{scope}

    % Mass velocity arrow
    \begin{scope}[xshift=6.04cm,yshift=0.95cm]
      \draw[line width=0.7pt,<-] (0,0.4) -- node {$v_m$} (0,0);
    \end{scope}

    % Descriptions inside graphic
    \draw (5.48,1.12) node {$m$};

    % Descriptions of subsystems under graphic
    \begin{scope}[xshift=-0.5cm,yshift=-0.28cm]
      \draw[decorate,decoration={brace,amplitude=10pt}]
        (3.5,0) -- (0,0) node[midway,yshift=-20pt] {circuit};
      \draw[decorate,decoration={brace,amplitude=10pt}]
        (7.05,0) -- (3.75,0) node[midway,yshift=-20pt] {mechanics};
    \end{scope}
  \end{tikzpicture}

  \caption{Elevator system diagram}
  \label{fig:elevator}
\end{figure}

Gear ratios are written as output over input, so $G$ is greater than one in
figure \ref{fig:elevator}. \\

Based on figure \ref{fig:elevator}

\begin{equation}
  \tau_m G = \tau_p \label{eq:elevator_tau_m_ratio}
\end{equation}

where $G$ is the gear ratio between the motor and the pulley and $\tau_p$ is the
torque produced by the pulley.

\begin{equation}
  rF_m = \tau_p \label{eq:elevator_torque_pulley}
\end{equation}

where $r$ is the radius of the pulley. Substitute equation
(\ref{eq:elevator_tau_m_ratio}) into equation (\ref{eq:motor_tau_V}).

\begin{align*}
  V &= \frac{\frac{\tau_p}{G}}{K_t} R + \frac{\omega_m}{K_v} \\
  V &= \frac{\tau_p}{GK_t} R + \frac{\omega_m}{K_v}
\end{align*}

Substitute in equation (\ref{eq:elevator_torque_pulley}).

\begin{equation}
  V = \frac{rF_m}{GK_t} R + \frac{\omega_m}{K_v} \label{eq:elevator_Vinter1}
\end{equation}

The angular velocity of the motor armature $\omega_m$ is

\begin{equation}
  \omega_m = G \omega_p \label{eq:elevator_omega_m_ratio} \\
\end{equation}

where $\omega_p$ is the angular velocity of the pulley. The velocity of the mass
(the elevator carriage) is

\begin{equation*}
  v_m = r \omega_p
\end{equation*}

\begin{equation}
  \omega_p = \frac{v_m}{r} \label{eq:elevator_omega_p}
\end{equation}

Substitute equation (\ref{eq:elevator_omega_p}) into equation
(\ref{eq:elevator_omega_m_ratio}).

\begin{equation}
  \omega_m = G \frac{v_m}{r} \label{eq:elevator_omega_m}
\end{equation}

Substitute equation (\ref{eq:elevator_omega_m}) into equation
(\ref{eq:elevator_Vinter1}).

\begin{align*}
  V &= \frac{rF_m}{GK_t} R + \frac{G \frac{v_m}{R}}{K_v} \\
  V &= \frac{RrF_m}{GK_t} + \frac{G}{RK_v} v_m
\end{align*}

Solve for $F_m$.

\begin{align}
  \frac{RrF_m}{GK_t} &= V - \frac{G}{RK_v} v_m \nonumber \\
  F_m &= \left(V - \frac{G}{RK_v} v_m\right) \frac{GK_t}{Rr} \nonumber \\
  F_m &= \frac{GK_t}{Rr} V - \frac{G^2K_t}{R^2 rK_v} v_m \label{eq:elevator_F_m}
\end{align}

\begin{equation}
  \sum F = ma_m \label{eq:elevator_F_ma}
\end{equation}

where $\sum F$ is the sum of forces applied to the elevator carriage, $m$ is
the mass of the elevator carriage in kilograms, and $a_m$ is the acceleration of
the elevator carriage.

\begin{equation*}
  ma_m = F_m
\end{equation*}

Note that gravity is not part of the modeled dynamics because it complicates the
state-space \gls{model} and the controller will behave well enough without it.

\begin{align}
  ma_m &= \left(\frac{GK_t}{Rr} V - \frac{G^2K_t}{R^2 rK_v} v_m\right)
    \nonumber \\
  a_m &= \frac{GK_t}{Rrm} V - \frac{G^2K_t}{R^2 rmK_v} v_m
    \label{eq:elevator_accel}
\end{align}

\subsubsection{Continuous state-space model}

The position and velocity of the elevator can be written as

\begin{align}
  \dot{x}_m &= v_m \label{eq:elevator_cont_ss_pos} \\
  \dot{v}_m &= a_m \label{eq:elevator_cont_ss_vel}
\end{align}

where by equation (\ref{eq:elevator_accel})

\begin{equation*}
  a_m = \frac{GK_t}{Rrm} V - \frac{G^2 K_t}{R^2 rm K_v} v_m
\end{equation*}

Substitute this into equation (\ref{eq:elevator_cont_ss_vel}).

\begin{align}
  \dot{v}_m &= \frac{GK_t}{Rrm} V - \frac{G^2 K_t}{R^2 rm K_v} v_m \nonumber \\
  \dot{v}_m &= -\frac{G^2 K_t}{R^2 rm K_v} v_m + \frac{GK_t}{Rrm} V
\end{align}

\begin{align*}
  \dot{\mtx{x}} &= \mtx{A} \mtx{x} + \mtx{B} \mtx{u} \\
  \mtx{y} &= \mtx{C} \mtx{x} + \mtx{D} \mtx{u}
\end{align*}

\begin{align*}
  \mtx{x} &= \left[
  \begin{array}{c}
    x \\
    v_m
  \end{array}
  \right] \\
  \mtx{y} &= x \\
  \mtx{u} &= V
\end{align*}

\begin{align}
  \mtx{A} &= \left[
  \begin{array}{cc}
    0 & 1 \\
    0 & -\frac{K_t G^2}{R^2rmK_v}
  \end{array}
  \right] \\
  \mtx{B} &= \left[
  \begin{array}{c}
    0 \\
    \frac{GK_t}{Rrm}
  \end{array}
  \right] \\
  \mtx{C} &= \left[
  \begin{array}{cc}
    1 & 0
  \end{array}
  \right] \\
  \mtx{D} &= 0
\end{align}

\subsubsection{Discrete state-space model}

The position and velocity of the elevator can be written as

\begin{align}
  x_{m,k+1} &= x_{m,k} + v_{m,k} \Delta t \label{eq:elevator_disc_ss_pos} \\
  v_{m,k+1} &= v_{m,k} + a_{m,k} \Delta t \label{eq:elevator_disc_ss_vel}
\end{align}

where by equation (\ref{eq:elevator_accel})

\begin{equation*}
  a_{m,k} = \frac{GK_t}{Rrm} V_k - \frac{G^2 K_t}{R^2 rm K_v} v_{m,k}
\end{equation*}

Substitute this into equation (\ref{eq:elevator_disc_ss_vel}).

\begin{align}
  v_{m,k+1} &= v_{m,k} + \left(\frac{GK_t}{Rrm} V_k -
    \frac{G^2 K_t}{R^2 rm K_v} v_{m,k}\right) \Delta t \nonumber \\
  v_{m,k+1} &= v_{m,k} + \frac{GK_t}{Rrm} \Delta t V_k -
    \frac{G^2 K_t}{R^2 rm K_v} \Delta t v_{m,k} \nonumber \\
  v_{m,k+1} &= v_{m,k} - \frac{G^2 K_t}{R^2 rm K_v} \Delta t v_{m,k} +
    \frac{GK_t}{Rrm} \Delta t V_k \nonumber \\
  v_{m,k+1} &= \left(1 - \frac{G^2 K_t}{R^2 rm K_v} \Delta t\right) v_{m,k} +
    \frac{GK_t}{Rrm} \Delta t V_k
\end{align}

\begin{align*}
  \mtx{x}_{k+1} &= \mtx{A} \mtx{x}_k + \mtx{B} \mtx{u}_k \\
  \mtx{y}_k &= \mtx{C} \mtx{x}_k + \mtx{D} \mtx{u}_k
\end{align*}

\begin{align*}
  \mtx{x}_k &= \left[
  \begin{array}{c}
    x_k \\
    v_{m,k}
  \end{array}
  \right] \\
  \mtx{y}_k &= x_k \\
  \mtx{u}_k &= V_k
\end{align*}

\begin{align}
  \mtx{A} &= \left[
  \begin{array}{cc}
    1 & \Delta t \\
    0 & 1 - \frac{K_t G^2}{R^2rmK_v} \Delta t
  \end{array}
  \right] \\
  \mtx{B} &= \left[
  \begin{array}{c}
    0 \\
    \frac{GK_t}{Rrm} \Delta t
  \end{array}
  \right] \\
  \mtx{C} &= \left[
  \begin{array}{cc}
    1 & 0
  \end{array}
  \right] \\
  \mtx{D} &= 0
\end{align}

\section{Flywheel}

\subsection{Equations of motion}

This flywheel consists of a DC brushed motor attached to a spinning mass of
non-negligible moment of inertia.

\begin{figure}[H]
  \centering

  \begin{tikzpicture}[auto, >=latex', circuit ee IEC,
                      set resistor graphic=var resistor IEC graphic]
    % \draw [help lines] (-1,-3) grid (7,4);

    % Electrical equivalent circuit
    \draw (0,2) to [voltage source={direction info'={->}, info'=$V$}] (0,0);
    \draw (0,2) to [current direction={info=$I$}] (0,3);
    \draw (0,3) -- (0.5,3);
    \draw (0.5,3) to [resistor={info={$R$}}] (2,3);

    \draw (2,3) -- (2.5,3);
    \draw (2.5,3) to [voltage source={direction info'={->}, info'=$V_{emf}$}]
      (2.5,0);
    \draw (0,0) -- (2.5,0);

    % Motor
    \begin{scope}[xshift=2.4cm,yshift=1.05cm]
      \draw[fill=black] (0,0) rectangle (0.2,0.9);
      \draw[fill=white] (0.1,0.45) ellipse (0.3 and 0.3);
    \end{scope}

    % Transmission gear one
    \begin{scope}[xshift=3.75cm,yshift=1.17cm]
      \draw[fill=black!50] (0.2,0.33) ellipse (0.08 and 0.33);
      \draw[fill=black!50, color=black!50] (0,0) rectangle (0.2,0.66);
      \draw[fill=white] (0,0.33) ellipse (0.08 and 0.33);
      \draw (0,0.66) -- (0.2,0.66);
      \draw (0,0) -- (0.2,0) node[pos=0.5,below] {$G$};
    \end{scope}

    % Output shaft of motor
    \begin{scope}[xshift=2.8cm,yshift=1.45cm]
      \draw[fill=black!50] (0,0) rectangle (0.95,0.1);
    \end{scope}

    % Angular velocity arrow of drive -> transmission
    \draw[line width=0.7pt,<-] (3.2,1) arc (-30:30:1) node[above] {$\omega_m$};

    % Transmission gear two
    \begin{scope}[xshift=3.75cm,yshift=1.83cm]
      \draw[fill=black!50] (0.2,0.68) ellipse (0.13 and 0.67);
      \draw[fill=black!50, color=black!50] (0,0) rectangle (0.2,1.35);
      \draw[fill=white] (0,0.68) ellipse (0.13 and 0.67);
      \draw (0,1.35) -- (0.2,1.35);
      \draw (0,0) -- (0.2,0);
    \end{scope}

    % Flywheel
    \begin{scope}[xshift=5.05cm,yshift=2.09cm]
      \draw[fill=white] (0.6,0.4) ellipse (0.13 and 0.4);
      \draw[fill=white,color=white] (0,0.8) rectangle (0.6,0);
      \draw[fill=white] (0,0.4) ellipse (0.13 and 0.4);
      \draw (0,0) -- (0.6,0);
      \draw (0,0.8) -- (0.6,0.8);
    \end{scope}

    % Transmission shaft from gear two to flywheel
    \begin{scope}[xshift=4.09cm,yshift=2.42cm]
      \draw[fill=black!50] (0,0) rectangle (0.96,0.1);
    \end{scope}

    % Angular velocity arrow between transmission and flywheel
    \draw[line width=0.7pt,->] (4.54,1.97) arc (-30:30:1) node[above]
      {$\omega_f$};

    % Descriptions inside graphic
    \draw (5.45,2.49) node {$J$};

    % Descriptions of subsystems under graphic
    \begin{scope}[xshift=-0.5cm,yshift=-0.28cm]
      \draw[decorate,decoration={brace,amplitude=10pt}]
        (3.5,0) -- (0,0) node[midway,yshift=-20pt] {circuit};
      \draw[decorate,decoration={brace,amplitude=10pt}]
        (6.55,0) -- (3.75,0) node[midway,yshift=-20pt] {mechanics};
    \end{scope}
  \end{tikzpicture}

  \caption{Flywheel system diagram}
  \label{fig:flywheel}
\end{figure}

Gear ratios are written as output over input, so $G$ is greater than one in
figure \ref{fig:flywheel}. \\

We will start with the equation derived earlier for a DC brushed motor, equation
(\ref{eq:motor_tau_V}).

\begin{equation*}
  V = \frac{\tau_m}{K_t} R + \frac{\omega_m}{K_v}
\end{equation*}

Solve for the angular acceleration. First, we'll rearrange the terms because
from inspection, $V$ is the model input, $\omega_m$ is the state, and $\tau_m$
contains the angular acceleration.

\begin{equation*}
  V = \frac{R}{K_t} \tau_m + \frac{1}{K_v} \omega_m
\end{equation*}

Solve for $\tau_m$.

\begin{align*}
  V &= \frac{R}{K_t} \tau_m + \frac{1}{K_v} \omega_m \\
  \frac{R}{K_t} \tau_m &= V - \frac{1}{K_v} \omega_m \\
  \tau_m &= \frac{K_t}{R} V - \frac{K_t}{K_v R} \omega_m
\end{align*}

Since $\tau_m G = \tau_f$ and $\omega_m = G \omega_f$

\begin{align}
  \left(\frac{\tau_f}{G}\right) &= \frac{K_t}{R} V -
    \frac{K_t}{K_v R} (G \omega_f) \nonumber \\
  \frac{\tau_f}{G} &= \frac{K_t}{R} V - \frac{G K_t}{K_v R} \omega_f \nonumber
    \\
  \tau_f &= \frac{G K_t}{R} V - \frac{G^2 K_t}{K_v R} \omega_f \label{eq:tau_f}
\end{align}

The torque applied to the flywheel is defined as

\begin{equation}
  \tau_f = J \dot{\omega}_f \label{eq:tau_f_def}
\end{equation}

where $J$ is the moment of inertia of the flywheel and $\dot{\omega}_f$ is the
angular acceleration. Substitute equation (\ref{eq:tau_f_def}) into equation
(\ref{eq:tau_f}).

\begin{align}
  (J \dot{\omega}_f) &= \frac{G K_t}{R} V - \frac{G^2 K_t}{K_v R} \omega_f
    \nonumber \\
  \dot{\omega}_f &= \frac{G K_t}{RJ} V - \frac{G^2 K_t}{K_v RJ} \omega_f
    \label{eq:dot_omega_f}
\end{align}

\subsection{Continuous state-space model}

By equation (\ref{eq:dot_omega_f})

\begin{equation*}
  \dot{\omega}_f = -\frac{G^2 K_t}{K_v RJ} \omega_f + \frac{G K_t}{RJ} V
\end{equation*}

\begin{align*}
  \dot{\mtx{x}} &= \mtx{A} \mtx{x} + \mtx{B} \mtx{u} \\
  \mtx{y} &= \mtx{C} \mtx{x} + \mtx{D} \mtx{u}
\end{align*}

\begin{align*}
  \mtx{x} &= \left[
  \begin{array}{c}
    \omega_f
  \end{array}
  \right] \\
  \mtx{y} &= \omega_f \\
  \mtx{u} &= V
\end{align*}

\begin{align}
  \mtx{A} &= \left[
  \begin{array}{c}
    -\frac{G^2 K_t}{K_v RJ}
  \end{array}
  \right] \\
  \mtx{B} &= \left[
  \begin{array}{c}
    \frac{G K_t}{RJ}
  \end{array}
  \right] \\
  \mtx{C} &= 1 \\
  \mtx{D} &= 0
\end{align}

\subsection{Simulation}

Python Control will be used to discretize the model and simulate it. The script
below also creates and tests a controller for it.

\begin{snippet}
  \caption{Flywheel simulation in Python}
  \label{lst:flywheel_sim}
  \includecode[Python]{code/flywheel.py}
\end{snippet}

\section{Drivetrain}

\subsection{Equations of motion}

This drivetrain consists of two DC brushed motors per side which are chained
together on their respective sides and drive wheels which are assumed to be
massless.

\begin{figure}[H]
  \centering

  \begin{tikzpicture}[auto, >=latex', circuit ee IEC,
                      set resistor graphic=var resistor IEC graphic]
    % \draw [help lines] (-1,-3) grid (7,4);

    % Right wheel
    \begin{scope}[xshift=5.78cm,yshift=1.83cm]
      \draw[fill=black!50] (0.2,0.68) ellipse (0.13 and 0.67);
      \draw[fill=black!50, color=black!50] (0,0) rectangle (0.2,1.35);
      \draw[fill=white] (0,0.68) ellipse (0.13 and 0.67);
      \draw (0,1.35) -- (0.2,1.35);
      \draw (0,0) -- (0.2,0);
    \end{scope}

    % Right transmission shaft
    \begin{scope}[xshift=5.32cm,yshift=2.42cm]
      \draw[fill=black!50] (0,0) rectangle (0.46,0.1);
    \end{scope}

    % Chassis
    \begin{scope}[xshift=4.44cm,yshift=2.09cm]
      \fill[fill=white] (0,0.8) -- (0,0.2) -- (0.2,0) -- (0.2,0.2)
        -- (0.98,0.2) -- (0.78,0.8) -- cycle;
      \draw (0,0.8) -- (0.78,0.8);
      \draw (0,0.8) -- (0,0.2);
      \draw (0,0.2) -- (0.2,0);
      \draw (0,0.8) -- (0.2,0.6);
      \draw (0.78,0.8) -- (0.98,0.6);
      \draw[fill=white] (0.2,0.6) rectangle (0.98,0);
    \end{scope}

    % Left transmission shaft
    \begin{scope}[xshift=4.09cm,yshift=2.42cm]
      \draw[fill=black!50] (0,0) rectangle (0.46,0.1);
    \end{scope}

    % Left wheel
    \begin{scope}[xshift=3.75cm,yshift=1.83cm]
      \draw[fill=black!50] (0.2,0.68) ellipse (0.13 and 0.67);
      \draw[fill=black!50, color=black!50] (0,0) rectangle (0.2,1.35);
      \draw[fill=white] (0,0.68) ellipse (0.13 and 0.67);
      \draw (0,1.35) -- (0.2,1.35);
      \draw (0,0) -- (0.2,0);
    \end{scope}

    % Angular velocity arrow for left wheel
    \draw[line width=0.7pt,->] (4.24,1.97) arc (-30:30:1) node[above]
      {$\omega_l$};

    % Angular velocity arrow for right wheel
    \draw[line width=0.7pt,->] (5.44,1.97) arc (-30:30:1) node[above]
      {$\omega_r$};

    % Wheel radius arrow
    \begin{scope}[xshift=3.5cm,yshift=1.83cm]
      \draw[line width=0.7pt,<->] (0,0) -- node[left] {$r$} (0,0.67);
    \end{scope}

    % Robot radius arrow
    \begin{scope}[xshift=4.65cm,yshift=1.83cm]
      \draw[line width=0.7pt,<->] (0,0) -- node[below] {$r_b$} (0.39,0);
    \end{scope}

    % Descriptions inside graphic
    \draw (4.99,2.42) node {$J$};
  \end{tikzpicture}

  \caption{Drivetrain system diagram}
  \label{fig:drivetrain}
\end{figure}

From equation (\ref{eq:tau_f}) of the flywheel model derivations

\begin{equation}
  \tau = \frac{GK_t}{R} V - \frac{G^2K_t}{K_v R} \omega
    \label{eq:drivetrain_tau}
\end{equation}

where $\tau$ is the torque applied by one wheel of the drivetrain, $G$ is the
gear ratio of the drivetrain, $K_t$ is the torque constant of the motor, $R$ is
the resistance of the motor, and $K_v$ is the angular velocity constant. Since
$\tau = rF$ and $\omega = \frac{v}{r}$ where $v$ is the velocity of a given
drivetrain side along the ground and $r$ is the drivetrain wheel radius

\begin{align*}
  (rF) = \frac{GK_t}{R} V - \frac{G^2K_t}{K_v R} \left(\frac{v}{r}\right) \\
  rF = \frac{GK_t}{R} V - \frac{G^2K_t}{K_v Rr} v \\
  F = \frac{GK_t}{Rr} V - \frac{G^2K_t}{K_v Rr^2} v \\
  F = -\frac{G^2K_t}{K_v Rr^2} v + \frac{GK_t}{Rr} V
\end{align*}

Therefore, for each side of the robot

\begin{align*}
  F_l &= -\frac{G_l^2 K_t}{K_v R r^2} v_l + \frac{G_l K_t}{Rr} V_l \\
  F_r &= -\frac{G_r^2 K_t}{K_v R r^2} v_r + \frac{G_r K_t}{Rr} V_r
\end{align*}

where the $l$ and $r$ subscripts denote the side of the robot to which each
variable corresponds.

Let $C_1 = -\frac{G_l^2 K_t}{K_v R r^2}$, $C_2 = \frac{G_l K_t}{Rr}$,
$C_3 = -\frac{G_r^2 K_t}{K_v R r^2}$, and $C_4 = \frac{G_r K_t}{Rr}$.

\begin{align}
  F_l &= C_1 v_l + C_2 V_l \label{eq:drivetrain_Fl} \\
  F_r &= C_3 v_r + C_4 V_r \label{eq:drivetrain_Fr}
\end{align}

First, find the sum of forces.

\begin{align}
  \sum F &= ma \nonumber \\
  F_l + F_r &= m \dot{v} \nonumber \\
  F_l + F_r &= m \frac{\dot{v}_l + \dot{v}_r}{2} \nonumber \\
  \frac{2}{m} (F_l + F_r) &= \dot{v}_l + \dot{v}_r \nonumber \\
  \dot{v}_l &= \frac{2}{m} (F_l + F_r) - \dot{v}_r \label{eq:drivetrain_dotv_l}
\end{align}

Next, find the sum of torques.

\begin{align*}
  \sum \tau &= J \dot{\omega} \\
  \tau_l + \tau_r &= J \left(\frac{\dot{v}_r - \dot{v}_l}{2 r_b}\right)
\end{align*}

where $r_b$ is the radius of the drivetrain.

\begin{align*}
  (-r_b F_l) + (r_b F_r) &= J \frac{\dot{v}_r - \dot{v}_l}{2 r_b} \\
  -r_b F_l + r_b F_r &= \frac{J}{2 r_b} (\dot{v}_r - \dot{v}_l) \\
  -F_l + F_r &= \frac{J}{2 r_b^2} (\dot{v}_r - \dot{v}_l) \\
  \frac{2 r_b^2}{J} (-F_l + F_r) &= \dot{v}_r - \dot{v}_l \\
  \dot{v}_r &= \dot{v}_l + \frac{2 r_b^2}{J} (-F_l + F_r)
\end{align*}

Substitute in equation (\ref{eq:drivetrain_dotv_l}) to obtain an expression for
$\dot{v}_r$.

\begin{align}
  \dot{v}_r &= \left(\frac{2}{m} (F_l + F_r) - \dot{v}_r\right) +
    \frac{2 r_b^2}{J} (-F_l + F_r) \nonumber \\
  2\dot{v}_r &= \frac{2}{m} (F_l + F_r) + \frac{2 r_b^2}{J} (-F_l + F_r)
    \nonumber \\
  \dot{v}_r &= \frac{1}{m} (F_l + F_r) + \frac{r_b^2}{J} (-F_l + F_r)
    \label{eq:drivetrain_vr_2mid} \\
  \dot{v}_r &= \frac{1}{m} F_l + \frac{1}{m} F_r - \frac{r_b^2}{J} F_l +
    \frac{r_b^2}{J} F_r \nonumber \\
  \dot{v}_r &= \left(\frac{1}{m} - \frac{r_b^2}{J}\right) F_l +
    \left(\frac{1}{m} + \frac{r_b^2}{J}\right) F_r \label{eq:drivetrain_vr_2}
\end{align}

Substitute equation (\ref{eq:drivetrain_vr_2mid}) back into equation
(\ref{eq:drivetrain_dotv_l}) to obtain an expression for $\dot{v}_l$.

\begin{align}
  \dot{v}_l &= \frac{2}{m} (F_l + F_r) - \left(\frac{1}{m} (F_l + F_r) +
    \frac{r_b^2}{J} (-F_l + F_r)\right) \nonumber \\
  \dot{v}_l &= \frac{1}{m} (F_l + F_r) - \frac{r_b^2}{J} (-F_l + F_r)
    \nonumber \\
  \dot{v}_l &= \frac{1}{m} (F_l + F_r) + \frac{r_b^2}{J} (F_l - F_r) \nonumber
    \\
  \dot{v}_l &= \frac{1}{m} F_l + \frac{1}{m} F_r + \frac{r_b^2}{J} F_l -
    \frac{r_b^2}{J} F_r \nonumber \\
  \dot{v}_l &= \left(\frac{1}{m} + \frac{r_b^2}{J}\right) F_l +
    \left(\frac{1}{m} - \frac{r_b^2}{J}\right) F_r \label{eq:drivetrain_vl_2}
\end{align}

Now, plug the expressions for $F_l$ and $F_r$ into equation
(\ref{eq:drivetrain_vr_2}).

\begin{align}
  \dot{v}_r &= \left(\frac{1}{m} - \frac{r_b^2}{J}\right) F_l +
    \left(\frac{1}{m} + \frac{r_b^2}{J}\right) F_r \nonumber \\
  \dot{v}_r &= \left(\frac{1}{m} - \frac{r_b^2}{J}\right)
    \left(C_1 v_l + C_2 V_l\right) +
    \left(\frac{1}{m} + \frac{r_b^2}{J}\right) \left(C_3 v_r + C_4 V_r\right)
    \label{eq:drivetrain_model_left}
\end{align}

Now, plug the expressions for $F_l$ and $F_r$ into equation
(\ref{eq:drivetrain_vl_2}).

\begin{align}
  \dot{v}_l &= \left(\frac{1}{m} + \frac{r_b^2}{J}\right) F_l +
    \left(\frac{1}{m} - \frac{r_b^2}{J}\right) F_r \nonumber \\
  \dot{v}_l &= \left(\frac{1}{m} + \frac{r_b^2}{J}\right)
    \left(C_1 v_l + C_2 V_l\right) +
    \left(\frac{1}{m} - \frac{r_b^2}{J}\right) \left(C_3 v_r + C_4 V_r\right)
    \label{eq:drivetrain_model_right}
\end{align}

\subsection{Continuous state-space model}

The position and velocity of each drivetrain side can be written as

\begin{align}
  \dot{x}_l &= v_l \label{eq:drivetrain_cont_ss_posl} \\
  \dot{v}_l &= \dot{v}_l \label{eq:drivetrain_cont_ss_vell} \\
  \dot{x}_r &= v_r \label{eq:drivetrain_cont_ss_posr} \\
  \dot{v}_r &= \dot{v}_r \label{eq:drivetrain_cont_ss_velr}
\end{align}

By equations (\ref{eq:drivetrain_model_left}) and
(\ref{eq:drivetrain_model_right})

\begin{align*}
  \dot{v}_r &= \left(\frac{1}{m} - \frac{r_b^2}{J}\right)
    \left(C_1 v_l + C_2 V_l\right) +
    \left(\frac{1}{m} + \frac{r_b^2}{J}\right) \left(C_3 v_r + C_4 V_r\right)
    \\
  \dot{v}_l &= \left(\frac{1}{m} + \frac{r_b^2}{J}\right)
    \left(C_1 v_l + C_2 V_l\right) +
    \left(\frac{1}{m} - \frac{r_b^2}{J}\right) \left(C_3 v_r + C_4 V_r\right)
\end{align*}

\begin{align*}
  \dot{\mtx{x}} &= \mtx{A} \mtx{x} + \mtx{B} \mtx{u} \\
  \mtx{y} &= \mtx{C} \mtx{x} + \mtx{D} \mtx{u}
\end{align*}

\begin{align*}
  \mtx{x} &= \left[
  \begin{array}{c}
    x_l \\
    v_l \\
    x_r \\
    v_r
  \end{array}
  \right] \\
  \mtx{y} &= \left[
  \begin{array}{c}
    x_l \\
    x_r
  \end{array}
  \right] \\
  \mtx{u} &= \left[
  \begin{array}{c}
    V_l \\
    V_r
  \end{array}
  \right]
\end{align*}

\begin{align}
  \mtx{A} &= \left[
  \begin{array}{cccc}
    0 & 1 & 0 & 0 \\
    0 & \left(\frac{1}{m} - \frac{r_b^2}{J}\right) C_1 & 0 & \left(\frac{1}{m} + \frac{r_b^2}{J}\right) C_3 \\
    0 & 0 & 0 & 1 \\
    0 & \left(\frac{1}{m} + \frac{r_b^2}{J}\right) C_1 & 0 & \left(\frac{1}{m} - \frac{r_b^2}{J}\right) C_3
  \end{array}
  \right] \\
  \mtx{B} &= \left[
  \begin{array}{cc}
    0 & 0 \\
    \left(\frac{1}{m} + \frac{r_b^2}{J}\right) C_2 & \left(\frac{1}{m} - \frac{r_b^2}{J}\right) C_4 \\
    0 & 0 \\
    \left(\frac{1}{m} - \frac{r_b^2}{J}\right) C_2 & \left(\frac{1}{m} + \frac{r_b^2}{J}\right) C_4
  \end{array}
  \right] \\
  \mtx{C} &= \left[
  \begin{array}{cccc}
    1 & 0 & 0 & 0 \\
    0 & 0 & 1 & 0 \\
  \end{array}
  \right] \\
  \mtx{D} &= \mtx{0}_{2 \times 2}
\end{align}

where $C_1 = -\frac{G_l^2 K_t}{K_v R r^2}$, $C_2 = \frac{G_l K_t}{Rr}$,
$C_3 = -\frac{G_r^2 K_t}{K_v R r^2}$, and $C_4 = \frac{G_r K_t}{Rr}$.

\subsection{Simulation}

\section{Single-jointed arm}

\subsection{Equations of motion}

This single-jointed arm consists of a DC brushed motor attached to a pulley that
spins a straight bar in pitch.

\begin{bookfigure}
  \begin{tikzpicture}[auto, >=latex', circuit ee IEC,
                      set resistor graphic=var resistor IEC graphic]
    % \draw [help lines] (-1,-3) grid (7,4);

    % Electrical equivalent circuit
    \draw (0,2) to [voltage source={direction info'={->}, info'=$V$}] (0,0);
    \draw (0,2) to [current direction={info=$I$}] (0,3);
    \draw (0,3) -- (0.5,3);
    \draw (0.5,3) to [resistor={info={$R$}}] (2,3);

    \draw (2,3) -- (2.5,3);
    \draw (2.5,3) to [voltage source={direction info'={->}, info'=$V_{emf}$}]
      (2.5,0);
    \draw (0,0) -- (2.5,0);

    % Motor
    \begin{scope}[xshift=2.4cm,yshift=1.05cm]
      \draw[fill=black] (0,0) rectangle (0.2,0.9);
      \draw[fill=white] (0.1,0.45) ellipse (0.3 and 0.3);
    \end{scope}

    % Transmission gear one
    \begin{scope}[xshift=3.75cm,yshift=1.17cm]
      \draw[fill=black!50] (0.2,0.33) ellipse (0.08 and 0.33);
      \draw[fill=black!50, color=black!50] (0,0) rectangle (0.2,0.66);
      \draw[fill=white] (0,0.33) ellipse (0.08 and 0.33);
      \draw (0,0.66) -- (0.2,0.66);
      \draw (0,0) -- (0.2,0) node[pos=0.5,below] {$G$};
    \end{scope}

    % Output shaft of motor
    \begin{scope}[xshift=2.8cm,yshift=1.45cm]
      \draw[fill=black!50] (0,0) rectangle (0.95,0.1);
    \end{scope}

    % Angular velocity arrow of drive -> transmission
    \draw[line width=0.7pt,<-] (3.2,1) arc (-30:30:1) node[above] {$\omega_m$};

    % Transmission gear two
    \begin{scope}[xshift=3.75cm,yshift=1.83cm]
      \draw[fill=black!50] (0.2,0.68) ellipse (0.13 and 0.67);
      \draw[fill=black!50, color=black!50] (0,0) rectangle (0.2,1.35);
      \draw[fill=white] (0,0.68) ellipse (0.13 and 0.67);
      \draw (0,1.35) -- (0.2,1.35);
      \draw (0,0) -- (0.2,0);
    \end{scope}

    \begin{scope}[xshift=5.075cm,yshift=2.4cm]
      % Single-jointed arm
      \draw[fill=white] (0,0) -- (0.1,-0.05) -- (0.35,1.45) -- (0.25,1.5)
        -- cycle;
      \draw[fill=black!50] (0.1,-0.05) -- (0.3,-0.05) -- (0.55,1.45) --
        (0.35,1.45) -- cycle;
      \draw[fill=white] (0.25,1.5) -- (0.35,1.45) -- (0.55,1.45) -- (0.45,1.5)
        -- cycle;

      % Arm length arrow
      \draw[line width=0.7pt,<->] (0.55,-0.05) -- node[right] {$l$} (0.8,1.45);

      % Mass label
      \draw (-0.05,1.2) node {$m$};
    \end{scope}

    % Transmission shaft from gear two to arm
    \begin{scope}[xshift=4.09cm,yshift=2.42cm]
      \draw[fill=black!50] (0,0) rectangle (1.06,0.1);
    \end{scope}

    % Angular velocity arrow between transmission and arm
    \draw[line width=0.7pt,->] (4.54,1.97) arc (-30:30:1) node[above]
      {$\omega_{arm}$};

    % Descriptions of subsystems under graphic
    \begin{scope}[xshift=-0.5cm,yshift=-0.28cm]
      \draw[decorate,decoration={brace,amplitude=10pt}]
        (3.5,0) -- (0,0) node[midway,yshift=-20pt] {circuit};
      \draw[decorate,decoration={brace,amplitude=10pt}]
        (6.55,0) -- (3.75,0) node[midway,yshift=-20pt] {mechanics};
    \end{scope}
  \end{tikzpicture}

  \caption{Single-jointed arm system diagram}
  \label{fig:single_jointed_arm}
\end{bookfigure}

Gear ratios are written as output over input, so $G$ is greater than one in
figure \ref{fig:single_jointed_arm}.

We will start with the equation derived earlier for a DC brushed motor, equation
(\ref{eq:motor_tau_V}).

\begin{equation*}
  V = \frac{\tau_m}{K_t} R + \frac{\omega_m}{K_v}
\end{equation*}

Solve for the angular acceleration. First, we'll rearrange the terms because
from inspection, $V$ is the \gls{model} \gls{input}, $\omega_m$ is the
\gls{state}, and $\tau_m$ contains the angular acceleration.

\begin{equation*}
  V = \frac{R}{K_t} \tau_m + \frac{1}{K_v} \omega_m
\end{equation*}

Solve for $\tau_m$.

\begin{align*}
  V &= \frac{R}{K_t} \tau_m + \frac{1}{K_v} \omega_m \\
  \frac{R}{K_t} \tau_m &= V - \frac{1}{K_v} \omega_m \\
  \tau_m &= \frac{K_t}{R} V - \frac{K_t}{K_v R} \omega_m
\end{align*}

Since $\tau_m G = \tau_{arm}$ and $\omega_m = G \omega_{arm}$

\begin{align}
  \left(\frac{\tau_{arm}}{G}\right) &= \frac{K_t}{R} V -
    \frac{K_t}{K_v R} (G \omega_f) \nonumber \\
  \frac{\tau_{arm}}{G} &= \frac{K_t}{R} V - \frac{G K_t}{K_v R} \omega_{arm}
    \nonumber \\
  \tau_{arm} &= \frac{G K_t}{R} V - \frac{G^2 K_t}{K_v R} \omega_{arm}
    \label{eq:tau_arm}
\end{align}

The angular velocity of the arm is defined as

\begin{equation}
  \tau_{arm} = J \dot{\omega}_{arm} \label{eq:tau_arm_def}
\end{equation}

where $J$ is the moment of inertia of the arm and $\dot{\omega}_{arm}$ is the
angular acceleration. Substitute equation (\ref{eq:tau_arm_def}) into equation
(\ref{eq:tau_arm}).

\begin{align}
  (J \dot{\omega}_{arm}) &= \frac{G K_t}{R} V - \frac{G^2 K_t}{K_v R}
    \omega_{arm} \nonumber \\
  \dot{\omega}_{arm} &= \frac{G K_t}{RJ} V - \frac{G^2 K_t}{K_v RJ} \omega_{arm}
    \label{eq:dot_omega_arm}
\end{align}

$J$ can be approximated as the moment of inertia of a thin rod rotating around
one end. Therefore

\begin{equation}
  J = \frac{1}{3}ml^2
\end{equation}

where $m$ is the mass of the arm and $l$ is the length of the arm.

\subsection{Continuous state-space model}
\index{FRC models!single-jointed arm equations}

The position and velocity of the elevator can be written as

\begin{align}
  \dot{\theta}_{arm} &= \omega_{arm} \label{eq:arm_cont_ss_pos} \\
  \dot{\omega}_{arm} &= \dot{\omega}_{arm} \label{eq:arm_cont_ss_vel}
\end{align}

By equation (\ref{eq:dot_omega_arm})

\begin{equation*}
  \dot{\omega}_{arm} = -\frac{G^2 K_t}{K_v RJ} \omega_{arm} + \frac{G K_t}{RJ} V
\end{equation*}

\begin{theorem}[Single-jointed arm state-space model]
  \begin{align*}
    \dot{\mtx{x}} &= \mtx{A} \mtx{x} + \mtx{B} \mtx{u} \\
    \mtx{y} &= \mtx{C} \mtx{x} + \mtx{D} \mtx{u}
  \end{align*}
  \begin{equation*}
    \begin{array}{ccc}
      \mtx{x} =
      \begin{bmatrix}
        \theta_{arm} \\
        \omega_{arm}
      \end{bmatrix} &
      \mtx{y} = \theta_{arm} &
      \mtx{u} = V
    \end{array}
  \end{equation*}
  \begin{equation}
    \begin{array}{cccc}
      \mtx{A} =
      \begin{bmatrix}
        0 & 1 \\
        0 & -\frac{G^2 K_t}{K_v RJ}
      \end{bmatrix} &
      \mtx{B} =
      \begin{bmatrix}
        0 \\
        \frac{G K_t}{RJ}
      \end{bmatrix} &
      \mtx{C} =
      \begin{bmatrix}
        1 & 0
      \end{bmatrix} &
      \mtx{D} = 0
    \end{array}
  \end{equation}
\end{theorem}

\subsection{Model augmentation}

As per subsection \ref{subsec:u_error_estimation}, we will now augment the
\gls{model} so a $u_{error}$ term is added to the \gls{control input}.

The \gls{plant} and \gls{observer} augmentations should be performed before the
\gls{model} is \glslink{discretization}{discretized}. After the \gls{controller}
gain is computed with the unaugmented discrete \gls{model}, the controller may
be augmented. Therefore, the \gls{plant} and \gls{observer} augmentations assume
a continuous \gls{model} and the \gls{controller} augmentation assumes a
discrete \gls{controller}.

\begin{equation*}
  \begin{array}{ccc}
    \mtx{x}_{aug} =
    \begin{bmatrix}
      \mtx{x} \\
      u_{error}
    \end{bmatrix} &
    \mtx{y} = \theta_{arm} &
    \mtx{u} = V
  \end{array}
\end{equation*}

\begin{equation}
  \begin{array}{cccc}
    \mtx{A}_{aug} =
    \begin{bmatrix}
      \mtx{A} & \mtx{B} \\
      \mtx{0}_{1 \times 2} & 0
    \end{bmatrix} &
    \mtx{B}_{aug} =
    \begin{bmatrix}
      \mtx{B} \\
      0
    \end{bmatrix} &
    \mtx{C}_{aug} =
    \begin{bmatrix}
      \mtx{C} & 0
    \end{bmatrix} &
    \mtx{D}_{aug} = \mtx{D}
  \end{array}
\end{equation}

\begin{equation}
  \begin{array}{cc}
    \mtx{K}_{aug} = \begin{bmatrix}
      \mtx{K} & 1
    \end{bmatrix} &
    \mtx{r}_{aug} = \begin{bmatrix}
      \mtx{r} \\
      0
    \end{bmatrix}
  \end{array}
\end{equation}

This will compensate for unmodeled dynamics such as gravity or other external
loading from lifted objects. However, if only gravity compensation is desired,
a feedforward of the form $u_{ff} = V_{gravity} \cos\theta$ is preferred where
$V_{gravity}$ is the voltage required to hold the arm level with the ground and
$\theta$ is the angle of the arm with the ground.

\subsection{Simulation}

Python Control will be used to \glslink{discretization}{discretize} the
\gls{model} and simulate it. One of the frccontrol
examples\footnote{\url{https://github.com/calcmogul/frccontrol/blob/master/examples/single_jointed_arm.py}}
creates and tests a controller for it.

Figure \ref{fig:single_jointed_arm_pzmaps} shows the pole-zero maps for the
open-loop \gls{system}, closed-loop \gls{system}, and \gls{observer}. Figure
\ref{fig:single_jointed_arm_response} shows the \gls{system} response with them.

\begin{svg}{build/frccontrol/examples/single_jointed_arm_pzmaps}
  \caption{Single-jointed arm pole-zero maps}
  \label{fig:single_jointed_arm_pzmaps}
\end{svg}

\begin{svg}{build/frccontrol/examples/single_jointed_arm_response}
  \caption{Single-jointed arm response}
  \label{fig:single_jointed_arm_response}
\end{svg}

\subsection{Implementation}

The script linked above also generates two files: SingleJointedArmCoeffs.h and
SingleJointedArmCoeffs.cpp. These can be used with the WPILib StateSpacePlant,
StateSpaceController, and StateSpaceObserver classes in C++ and Java. A C++
implementation of this single-jointed arm controller is available
online\footnote{\url{https://github.com/calcmogul/allwpilib/tree/state-space/wpilibcExamples/src/main/cpp/examples/StateSpaceSingleJointedArm}}.

\subsection{Rotating claw}

\subsubsection{Equations of motion}

This claw consists of independent upper and lower jaw pieces each driven by its
own DC brushed motor.

\subsubsection{Continuous state-space model}

\subsubsection{Discrete state-space model}

