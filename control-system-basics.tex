\chapterimage{control-system-basics.jpg}{On trail between McHenry Library and Media Theater at UCSC}

\chapter{Control system basics}

\section{What is gain?}
\index{Gain}

Gain is a proportional value that shows the relationship between the magnitude
of the input to the magnitude of the output signal at steady state. Many
\glspl{system} contain a method by which the gain can be altered, providing more
or less ``power" to the \gls{system}.

Figure \ref{fig:input_output_gain} shows a system with a hypothetical input and
output. Since the output is twice the amplitude of the input, the system has a
gain of $2$.

\begin{bookfigure}
  \begin{tikzpicture}[auto, >=latex']
    % \draw [help lines] (-4,-2) grid (4,2);

    % Input
    \drawtimeplot{-2.5cm}{0cm}{0.125cm}{0.44375cm}{0.6 * cos(40 * deg(\x))}
    \draw (-2.5,1) node {\small Input};

    \node [block] (sys) {K};
    \draw (0,1) node {\small System};

    % Output
    \drawtimeplot{2.5cm}{0cm}{0.125cm}{0.44375cm}{1.2 * cos(40* deg(\x))}
    \draw (2.5,1) node {\small Output};

    % Arrows between input/output and system
    \draw[->] (-2,0) -- (sys);
    \draw[->] (sys) -- (2,0);
  \end{tikzpicture}

  \caption{Demonstration of system with a gain of $K = 2$.}
  \label{fig:input_output_gain}
\end{bookfigure}

\section{Block diagrams}
\index{Block diagrams}

When designing or analyzing a control system, it is useful to model it
graphically. Block diagrams are used for this purpose. They can be manipulated
and simplified systematically \cite{bib:block_diagrams}. Figure
\ref{fig:gain_nomenclature} is an example of one.

\begin{bookfigure}
  \begin{tikzpicture}[auto, >=latex']
    % Place the blocks
    \node [name=input] {input};
    \node [sum, right=of input] (sum) {};
    \node [block, right=of sum] (P1) {open loop};
    \node [right=of P1] (output) {output};
    \node [block, below=of P1] (P2) {feedback};

    % Connect the nodes
    \draw [arrow] (input) -- node[pos=0.85] {$+$} (sum);
    \draw [arrow] (sum) -- node {} (P1);
    \draw [arrow] (P1) -- node[name=y] {} (output);
    \draw [arrow] (y) |- (P2);
    \draw [arrow] (P2) -| node[pos=0.97, right] {$\mp$} (sum);
  \end{tikzpicture}

  \caption{Block diagram with nomenclature}
  \label{fig:gain_nomenclature}
\end{bookfigure}

The open-loop gain is the total gain from the sum node at the input to the
output branch. The feedback gain is the total gain from the output back to the
input sum node. The circle produces the sum of its inputs as its output.

Figure \ref{fig:feedback_block_diagram} is a block diagram with more formal
notation in a feedback configuration.

\begin{bookfigure}
  \begin{tikzpicture}[auto, >=latex']
    % Place the blocks
    \node [name=input] {$X(s)$};
    \node [sum, right=of input] (sum) {};
    \node [block, right=of sum] (P1) {$P_1$};
    \node [right=of P1] (output) {$Y(s)$};
    \node [block, below=of P1] (P2) {$P_2$};

    % Connect the nodes
    \draw [arrow] (input) -- node[pos=0.85] {$+$} (sum);
    \draw [arrow] (sum) -- node {} (P1);
    \draw [arrow] (P1) -- node[name=y] {} (output);
    \draw [arrow] (y) |- (P2);
    \draw [arrow] (P2) -| node[pos=0.97, right] {$\mp$} (sum);
  \end{tikzpicture}

  \caption{Feedback block diagram}
  \label{fig:feedback_block_diagram}
\end{bookfigure}

\begin{theorem}[Closed-loop gain for a feedback loop]
  \begin{equation}
    \frac{Y(s)}{X(s)} = \frac{P_1}{1 \mp P_1 P_2}
  \end{equation}
\end{theorem}

See appendix \ref{ch:app-tf-feedback-deriv} for a derivation.

\section{Why feedback control?}

Let's say we are controlling a DC brushed motor. With just a mathematical
\glslink{model}{mathematical model} and knowledge of all current \glspl{state}
of the \gls{system} (i.e., angular velocity), we can predict all future
\glspl{state} given the future voltage \glspl{input}. Why then do we need
feedback control? If the system is \glslink{disturbance}{disturbed} in any way
that isn't modeled by our equations, like a load was applied to the armature, or
voltage sag in the rest of the circuit caused the commanded voltage to not match
the actual applied voltage, the angular velocity of the motor will deviate from
the \gls{model} over time.

To combat this, we can take measurements of the system and the environment to
detect this deviation and account for it. For example, we could measure the
current position and estimate an angular velocity from it. We can then give the
motor corrective commands as well as steer our \gls{model} back to reality. This
feedback allows us to account for and be \glslink{robustness}{robust} in the
face of uncertainty.
