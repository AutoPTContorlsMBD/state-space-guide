\chapterimage{notes-to-the-reader.jpg}{Trees by Baskin Engineering building at UCSC}

\setcounter{chapter}{-1}
\chapter{Notes to the reader}

\section{Prerequisites}

Knowledge of basic algebra and complex numbers is assumed. Some introductory
physics and calculus will be taught as necessary.

\section{The structure of this book}

This book consists of three parts and a collection of appendices:

\begin{itemize}
  \item Part I, ``Classical control," introduces the basics of control theory,
    teaches the fundamentals of PID controller design, describes what a transfer
    function is, and shows how they can be used to analyze dynamical systems.
    Emphasis is placed on the geometric intuition of this analysis rather than
    the frequency domain math.
  \item Part II, ``Modern control," builds on the intuition gained in part I to
    describe and control multiple-input, multiple-output (MIMO) systems. It
    provides a crash course in the geometric intuition behind linear algebra and
    covers enough of the mechanics of evaluating matrix algebra for the reader
    to follow along in later chapters. State-space representation,
    controllability and observability, discretization, LQR controller design,
    LQE observer design, and stochastic control theory are covered.
  \item Part III, ``Controls design/implementation," describes how to apply the
    concepts learned in the earlier parts to design and implement controllers
    for real systems. It walks through several examples of common FRC subsystems
    from deriving the model using kinematics to implementing and testing a
    digital controller. Finally, Kalman filters are implemented and the models
    are augmented with $u_{error}$ estimators to help handle model uncertainty.
  \item The appendices provide further enrichment that isn't required for a
    passing understanding of the material. This includes derivations for many of
    the results presented and used in the main matter of the book.
\end{itemize}

Source code examples are available in this book's Git repository. Its location
is listed on the copyright page (page 2).

This book is intended as both a tutorial for new students and as a reference
manual for more experienced readers who need to review a thing or two. While it
isn't comprehensive, the reader will hopefully learn enough to either implement
the concepts presented themselves or know where to look for more information.

Some parts are mathematically rigorous, but I believe in giving students a solid
theoretical foundation with emphasis on intuition so they can apply it to new
problems. To achieve deep understanding of the topics in this book, math is
unavoidable.

The sections on classical control theory are intended to provide a geometric
intuition into the mathematical machinery of modern control theory. Modern
control requires doing roughly three things: develop a kinematic model of the
system, design a controller for the system based on the model, and design an
observer to estimate hidden states of the system or account for noise. This book
covers how to do each.

Some topics have been oversimplified to make them easier to grasp. For more
detail, please see the Wikibook on control systems at
\url{https://en.wikibooks.org/wiki/Control_Systems}.

\section{The mindset of an egoless engineer}

The following maxim summarizes what I hope to teach my robotics students (with
examples drawn from controls engineering).

\begin{quote}
  ``Engineer based on requirements, not an ideology."
\end{quote}

Engineering is filled with trade-offs. The tools should fit the job, and not
every problem is a nail waiting to be struck by a hammer. Instead, assess the
minimum requirements (min specs) for a solution to the task at hand and do only
enough work to satisfy them; exceeding your specifications is a waste of time
and money. If you require performance or maintainability above the min specs,
your min specs were chosen incorrectly by definition.

Controls engineering is pragmatic in a similar respect:
\textit{solve. the. problem}. For control of nonlinear systems,
\href{https://faculty.washington.edu/devasia/Inversion.html}{plant inversion}
is elegant on paper but doesn't work with an inaccurate model, yet using a
theoretically incorrect solution like linear approximations of the nonlinear
system works well enough to be used industry-wide. There are more sophisticated
controllers than PID, but we use PID anyway for its versatility and simplicity.
Sometimes the inferior solutions are more effective or have a more desirable
cost-benefit ratio than what the control system designer considers ideal or
clean. Choose the tool that is most effective.

Solutions need to be good enough, but do not need to be perfect. We want to
avoid integrators as they introduce instability, but we use them anyway because
they work well for meeting tracking specifications. One should not blindly
defend a design or follow an ideology, because there is always a case where its
antithesis is a better option. The engineer should be able to determine when
this is the case, set aside their ego, and do what will meet the specifications
of their client (e.g., system response characteristics, maintainability,
usability). Preferring one solution over another for pragmatic or technical
reasons is fine, but the engineer should not care on a personal level which
sufficient solution is chosen.

\section{Request for feedback}

While we have tried to write a book that makes the topics of control theory
approachable, it still may be dense or fast-paced for some readers (it covers
three classes of feedback control, two of which are for graduate students, in
one short book). Please send us feedback, corrections, or suggestions through
the GitHub link listed on the copyright page. New examples that demonstrate key
concepts and make them more accessible are also appreciated.
