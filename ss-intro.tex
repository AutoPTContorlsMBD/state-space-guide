\section{Linear algebra}

Modern control theory borrows concepts from linear algebra. Watch 3Blue1Brown's
\textit{Essence of Linear Algebra} video series \cite{bib:essence_of_linalg}. It
offers an intuitive, geometric understanding of linear algebra as a method of
linear transformations. \\

\textit{Note}: The $^T$ in $\mtx{A}^T$ denotes transpose, which flips the matrix
across its diagonal such that the rows become columns and vice versa.

\section{State-space representation}

A state-space representation models \glspl{system} as a set of \gls{state},
input, and output variables related by first-order differential equations.
"State space" refers to the Euclidean space in which the \gls{state} variables
are on the axes. The \gls{state} of the \gls{system} can be represented as a
vector within that space. \\

To abstract from the number of \glspl{state}, inputs, and outputs, these
variables are expressed as vectors. Additionally, if the dynamical \gls{system}
is linear, time-invariant, and finite-dimensional, then the differential and
algebraic equations may be written in matrix form.

\subsection{Benefits over classical output-based control}

The state-space method provides a more convenient and compact way to model and
analyze \glspl{system} with multiple inputs and outputs. For a system with $p$
inputs and $q$ outputs, we would have to write $q \times p$ Laplace transforms
to represent it. Not only is the resulting algebra unwieldy, but it only works
for linear systems with zero initial conditions. State-space representation uses
the time domain instead of the frequency domain, so it doesn't have this
problem. \\

Students are still taught classical control first because it provides a
framework within which to understand the results we get from the fancy
mathematical machinery of modern control.

\section{State-space notation}

\begin{align}
  \dot{\mtx{x}} &= \mtx{A}\mtx{x} + \mtx{B}\mtx{u} \label{eq:ss_ctrl_x} \\
  \mtx{y} &= \mtx{C}\mtx{x} + \mtx{D}\mtx{u} \label{eq:ss_ctrl_y}
\end{align}

\begin{align}
  \mtx{x}_{k+1} &= \mtx{A}\mtx{x}_k + \mtx{B}\mtx{u}_k \label{eq:ssz_ctrl_x} \\
  \mtx{y}_{k+1} &= \mtx{C}\mtx{x}_k + \mtx{D}\mtx{u}_k \label{eq:ssz_ctrl_y}
\end{align}

\begin{table}[ht]
  \renewcommand{\arraystretch}{1.3}
  \centering
  \begin{tabulary}{\linewidth}{LLLL}
    $\mtx{A}$ & system matrix      & $\mtx{x}$ & state vector \\
    $\mtx{B}$ & input matrix       & $\mtx{u}$ & input vector \\
    $\mtx{C}$ & output matrix      & $\mtx{y}$ & output vector \\
    $\mtx{D}$ & feedthrough matrix &  &  \\
  \end{tabulary}
  \label{tab:ss_def}
\end{table}

\begin{table}[ht]
  \caption{State-space matrix dimensions}
  \renewcommand{\arraystretch}{1.5}
  \centering
  \begin{tabular}{|ll|ll|}
    \hline
    \rowcolor{lightblue}
    \textbf{Matrix} & \textbf{Rows $\times$ Columns} &
    \textbf{Matrix} & \textbf{Rows $\times$ Columns} \\
    \hline
    $\mtx{A}$ & states $\times$ states & $\mtx{x}$ & states $\times$ 1 \\
    $\mtx{B}$ & states $\times$ inputs & $\mtx{u}$ & inputs $\times$ 1 \\
    $\mtx{C}$ & outputs $\times$ states & $\mtx{y}$ & outputs $\times$ 1 \\
    $\mtx{D}$ & outputs $\times$ inputs &  &  \\
    \hline
  \end{tabular}
  \label{tab:ss_matrix_dims}
\end{table}

\section{Canonical forms}

There are two canonical forms used to represent state-space \glspl{model}:
controllable canonical form and observable canonical form. They are used to
provide controllability and observability of a system respectively, which are
mathematical duals of each other. That is, the controller and estimator (state
observer) are complementary problems.

\subsection{Controllable canonical form} \label{subsec:ctrl_canon}

State controllability implies that it is possible -- by admissible inputs -- to
steer the \glspl{state} from any initial value to any final value within some
finite time window. A continuous \gls{time-invariant} linear state-space
\gls{model} is controllable if and only if

\begin{equation}
  rank \left[
  \begin{array}{ccccc}
    B & AB & A^2B & \cdots & A^{n-1}B
  \end{array}
  \right] = n
  \label{eq:ctrl_rank}
\end{equation}

where rank is the number of linearly independent rows in a matrix and $n$ is the
number of \gls{state} variables. \\

Given a \gls{system} of the form

\begin{equation} \label{eq:ctrl_obsv_tf}
  G(s) = \frac{n_1 s^3 + n_2 s^2 + n_3 s + n_4}
    {s^4 + d_1 s^3 + d_2 s^2 + d_3 s + d_4} \\
\end{equation}

The canonical realization of it that satisfies equation (\ref{eq:ctrl_rank}) is

\begin{align}
  \dot{\mtx{x}}(t) &= \left[
  \begin{array}{cccc}
    0 & 1 & 0 & 0 \\
    0 & 0 & 1 & 0 \\
    0 & 0 & 0 & 1 \\
    -d_4 & -d_3 & -d_2 & -d_1
  \end{array}
  \right] \mtx{x}(t) + \left[
  \begin{array}{c}
    0 \\
    0 \\
    0 \\
    1
  \end{array}
  \right] \mtx{u}(t) \\
  \mtx{y}(t) &= \left[
  \begin{array}{cccc}
    n_4 & n_3 & n_2 & n_1
  \end{array}
  \right] \mtx{x}(t)
\end{align}

\subsection{Observable canonical form} \label{subsec:obsv_canon}

Observability is a measure for how well internal \glspl{state} of a \gls{system}
can be inferred by knowledge of its external outputs. The observability and
controllability of a \gls{system} are mathematical duals (i.e., as
controllability proves that an input is available that brings any initial
\gls{state} to any desired final \gls{state}, observability proves that knowing
an output trajectory provides enough information to predict the initial
\gls{state} of the \gls{system}). \\

A continuous \gls{time-invariant} linear state-space \gls{model} is observable
if and only if

\begin{equation} \label{eq:obsv_rank}
  rank \left[
  \begin{array}{c}
    C \\
    CA \\
    \vdots \\
    CA^{n-1}
  \end{array}
  \right] = n \\
\end{equation}

The canonical realization of the \gls{system} in equation
(\ref{eq:ctrl_obsv_tf}) that satisfies equation (\ref{eq:obsv_rank}) is

\begin{align}
  \dot{\mtx{x}}(t) &= \left[
  \begin{array}{cccc}
    0 & 0 & 0 & -d_4 \\
    1 & 0 & 0 & -d_3 \\
    0 & 1 & 0 & -d_2 \\
    0 & 0 & 1 & -d_1
  \end{array}
  \right] \mtx{x}(t) + \left[
  \begin{array}{c}
    n_4 \\
    n_3 \\
    n_2 \\
    n_1
  \end{array}
  \right] \mtx{u}(t) \\
  \mtx{y}(t) &= \left[
  \begin{array}{cccc}
    0 & 0 & 0 & 1
  \end{array}
  \right] \mtx{x}(t)
\end{align}

\section{Eigenvalues in state-space}

Eigenvalues can be used to determine the stability of a \gls{system}. \\

We'd like to know whether the \gls{system} defined by equation
(\ref{eq:ssz_ctrl_x}) operating with the \gls{control law}
$\mtx{u}_k = \mtx{K}(\mtx{r}_k - \mtx{x}_k)$ converges to the \gls{reference}
$\mtx{r}_k$.

\begin{align}
  \mtx{x}_{k+1} &= \mtx{A}\mtx{x}_k + \mtx{B}\mtx{u}_k \nonumber \\
  \mtx{x}_{k+1} &= \mtx{A}\mtx{x}_k + \mtx{B}(\mtx{K}(\mtx{r}_k - \mtx{x}_k))
    \nonumber \\
  \mtx{x}_{k+1} &= \mtx{A}\mtx{x}_k + \mtx{B}\mtx{K}\mtx{r}_k -
    \mtx{B}\mtx{K}\mtx{x}_k \nonumber \\
  \mtx{x}_{k+1} &= \mtx{A}\mtx{x}_k - \mtx{B}\mtx{K}\mtx{x}_k +
    \mtx{B}\mtx{K}\mtx{r}_k \nonumber \\
  \mtx{x}_{k+1} &= (\mtx{A} - \mtx{B}\mtx{K})\mtx{x}_k +
    \mtx{B}\mtx{K}\mtx{r}_k \label{eq:ctrl_eig_calc}
\end{align}

For equation (\ref{eq:ctrl_eig_calc}) to have a bounded output, the eigenvalues
of $\mtx{A} - \mtx{B}\mtx{K}$ must be within the unit circle. \\

This derivation can be performed for a \gls{state} estimator as well to
determine whether the \gls{state} estimate converges to the true \gls{state}.
Plugging equation (\ref{eq:z_obsv_y}) into equation (\ref{eq:z_obsv_x}) gives

\begin{align*}
  \hat{\mtx{x}}_{k+1} &= \mtx{A}\hat{\mtx{x}}_k + \mtx{B}\mtx{u}_k +
    \mtx{L} (\mtx{y}_k - \hat{\mtx{y}}_k) \\
  \hat{\mtx{x}}_{k+1} &= \mtx{A}\hat{\mtx{x}}_k + \mtx{B}\mtx{u}_k +
    \mtx{L} (\mtx{y}_k - (\mtx{C}\hat{\mtx{x}}_k + \mtx{D}\mtx{u}_k)) \\
  \hat{\mtx{x}}_{k+1} &= \mtx{A}\hat{\mtx{x}}_k + \mtx{B}\mtx{u}_k +
    \mtx{L} (\mtx{y}_k - \mtx{C}\hat{\mtx{x}}_k - \mtx{D}\mtx{u}_k) \\
\end{align*}

Plugging in equation (\ref{eq:ssz_ctrl_y}) gives

\begin{align*}
  \hat{\mtx{x}}_{k+1} &= \mtx{A}\hat{\mtx{x}}_k + \mtx{B}\mtx{u}_k +
    \mtx{L}((\mtx{C}\mtx{x}_k + \mtx{D}\mtx{u}_k) - \mtx{C}\hat{\mtx{x}}_k -
    \mtx{D}\mtx{u}_k) \\
  \hat{\mtx{x}}_{k+1} &= \mtx{A}\hat{\mtx{x}}_k + \mtx{B}\mtx{u}_k +
    \mtx{L}(\mtx{C}\mtx{x}_k + \mtx{D}\mtx{u}_k - \mtx{C}\hat{\mtx{x}}_k -
    \mtx{D}\mtx{u}_k) \\
  \hat{\mtx{x}}_{k+1} &= \mtx{A}\hat{\mtx{x}}_k + \mtx{B}\mtx{u}_k +
    \mtx{L}(\mtx{C}\mtx{x}_k - \mtx{C}\hat{\mtx{x}}_k) \\
  \hat{\mtx{x}}_{k+1} &= \mtx{A}\hat{\mtx{x}}_k + \mtx{B}\mtx{u}_k +
    \mtx{L}\mtx{C}(\mtx{x}_k - \hat{\mtx{x}}_k) \\
\end{align*}

Let $E_k = \mtx{x}_k - \hat{\mtx{x}}_k$ be the error in the estimate
$\hat{\mtx{x}}_k$.

\begin{equation*}
  \hat{\mtx{x}}_{k+1} = \mtx{A}\hat{\mtx{x}}_k + \mtx{B}\mtx{u}_k +
    \mtx{L}\mtx{C}\mtx{E}_k \\
\end{equation*}

Subtracting this from equation (\ref{eq:ssz_ctrl_x}) gives

\begin{align}
  \mtx{x}_{k+1} - \hat{\mtx{x}}_{k+1} &= \mtx{A}\mtx{x}_k + \mtx{B}\mtx{u}_k -
    (\mtx{A}\hat{\mtx{x}}_k + \mtx{B}\mtx{u}_k +
     \mtx{L}\mtx{C}\mtx{E}_k) \nonumber \\
  \mtx{E}_{k+1} &= \mtx{A}\mtx{x}_k + \mtx{B}\mtx{u}_k -
    (\mtx{A}\hat{\mtx{x}}_k + \mtx{B}\mtx{u}_k + \mtx{L}\mtx{C}\mtx{E}_k)
    \nonumber \\
  \mtx{E}_{k+1} &= \mtx{A}\mtx{x}_k + \mtx{B}\mtx{u}_k -
    \mtx{A}\hat{\mtx{x}}_k - \mtx{B}\mtx{u}_k - \mtx{L}\mtx{C}\mtx{E}_k
    \nonumber \\
  \mtx{E}_{k+1} &= \mtx{A}\mtx{x}_k - \mtx{A}\hat{\mtx{x}}_k -
    \mtx{L}\mtx{C}\mtx{E}_k \nonumber \\
  \mtx{E}_{k+1} &= \mtx{A}(\mtx{x}_k - \hat{\mtx{x}}_k) -
    \mtx{L}\mtx{C}\mtx{E}_k \nonumber \\
  \mtx{E}_{k+1} &= \mtx{A}\mtx{E}_k - \mtx{L}\mtx{C}\mtx{E}_k \nonumber \\
  \mtx{E}_{k+1} &= (\mtx{A} - \mtx{L}\mtx{C})\mtx{E}_k \label{eq:obsv_eig_calc}
\end{align}

For equation (\ref{eq:obsv_eig_calc}) to have a bounded output, the eigenvalues
of $\mtx{A} - \mtx{L}\mtx{C}$ must be within the unit circle. These eigenvalues
represent how fast the estimator converges to the true state of the given
\gls{model}. A fast estimator converges quickly while a slow estimator avoids
amplifying noise in the measurements used to produce a state estimate. \\

The effect of noise can be seen if it is modeled
\glslink{stochastic process}{stochastically} as

\begin{equation*}
  \hat{\mtx{x}}_{k+1} = \mtx{A}\hat{\mtx{x}}_k + \mtx{B}\mtx{u}_k +
    \mtx{L} (\mtx{y}_k - \hat{\mtx{y}}_k) + \mtx{L}\mtx{\nu}_k \\
\end{equation*}

where $\mtx{\nu}_k$ is the measurement noise. As $\mtx{L}$ increases, the
measurement noise is amplified. \\

In summary, a controller is stable if the eigenvalues of
$\mtx{A} - \mtx{B}\mtx{K}$ are within the unit circle, and an estimator is
stable if the eigenvalues of $\mtx{A} - \mtx{L}\mtx{C}$ are within the unit
circle. \\

As stated before, the controller and estimator are dual problems. Controller
gains can be found assuming perfect estimator (i.e., perfect knowledge of all
\glspl{state}). Estimator gains can be found assuming an accurate \gls{model}
and a controller with perfect \gls{reference tracking}.
