\documentclass[11pt]{book}

\usepackage{amsmath,amssymb,amsthm}
\usepackage[english]{babel} % English language/hyphenation
\usepackage{bm}
\usepackage{booktabs} % For nicer horizontal rules in tables
\usepackage{calc} % For simpler calculation in spacing index letter headings
\usepackage{colortbl} % For colored table rows
\usepackage{datetime} % For the generated date
\usepackage{fancyhdr} % For the headers
\usepackage{gensymb} % For degree symbol
\usepackage[top=3cm,bottom=3cm,left=3cm,right=3cm,headsep=10pt,a4paper]{geometry}
\usepackage{graphicx} % For including pictures
\graphicspath{{figs/}} % Specifies the directory where pictures are stored
\usepackage{listings}
\usepackage{mathrsfs}
\DeclareMathAlphabet{\mathscrbf}{OMS}{cmsy}{m}{n}
\usepackage{makeidx}
\usepackage{parskip}
\usepackage{placeins} % For FloatBarrier command
\usepackage{subcaption}
\usepackage{tabulary}
\usepackage{tikz}
\usepackage{titletoc}

\usepackage{enumitem} % Customize lists
\setlist{nolistsep} % Reduce spacing between bullet points and numbered lists

% Font setup
\usepackage{microtype} % Slightly tweak font spacing for aesthetics
\usepackage[T1]{fontenc} % Use 8-bit encoding that has 256 glyphs

% Set main fonts
\usepackage[no-math]{fontspec}
\setmainfont[Path=./fonts/,
             BoldFont=FreeSerifBold.ttf,
             ItalicFont=FreeSerifItalic.ttf,
             BoldItalicFont=FreeSerifBoldItalic.ttf]{FreeSerif.ttf}
\setsansfont[Path=./fonts/,
             BoldFont=FreeSansBold.ttf,
             ItalicFont=FreeSansOblique.ttf,
             BoldItalicFont=FreeSansBoldOblique.ttf]{FreeSans.ttf}
\renewcommand{\familydefault}{\rmdefault}

% Bibliography setup
\usepackage[style=numeric,citestyle=numeric,sorting=nyt,sortcites=true,autopunct=true,autolang=hyphen,hyperref=true,abbreviate=false,backref=true,backend=biber]{biblatex}
\addbibresource{state-space-guide.bib}
\defbibheading{bibempty}{}

% Index setup
\makeindex

\usepackage{hyperref}
\hypersetup{hidelinks,colorlinks=false,breaklinks=true,bookmarksopen=false,
  pdftitle={Practical Guide to State-space Control},pdfauthor={Tyler Veness}}

% Load after hyperref
\usepackage[nopostdot,style=index,nonumberlist]{glossaries}

\usepackage{xcolor}  % Required for specifying colors by name
\definecolor{deepblue}{rgb}{0,0,0.5}
\definecolor{deepgreen}{rgb}{0,0.5,0}
\definecolor{deeporange}{RGB}{243,102,25}
\definecolor{deepred}{rgb}{0.6,0,0}

\colorlet{themecolor}{deeporange}
\colorlet{headingbg}{black!5}

% MAIN TABLE OF CONTENTS

\contentsmargin{0cm} % Removes the default margin

% Part text styling
\titlecontents{part}[0cm]{\addvspace{20pt}\centering\large\bfseries}{}{}{}

% Chapter text styling
\titlecontents{chapter}[1.25cm] % Indentation
  {\addvspace{12pt}\large\sffamily\bfseries} % Spacing and font options
  {\color{themecolor!60}\contentslabel[\Large\thecontentslabel]{1.25cm}%
   \color{themecolor}} % Chapter number
  {\color{themecolor}}
  {\color{themecolor!60}\normalsize\;%
   \titlerule*[.5pc]{.}\;\thecontentspage} % Page number

% Section text styling
\titlecontents{section}[1.25cm] % Indentation
  {\addvspace{3pt}\sffamily\bfseries} % Spacing and font options
  {\contentslabel[\thecontentslabel]{1.25cm}} % Section number
  {}
  {\hfill\color{black}\thecontentspage} % Page number
  []

% Subsection text styling
\titlecontents{subsection}[1.25cm] % Indentation
  {\addvspace{1pt}\sffamily\small} % Spacing and font options
  {\contentslabel[\thecontentslabel]{1.25cm}} % Subsection number
  {}
  {\ \titlerule*[.5pc]{.}\;\thecontentspage} % Page number
  []

% List of figures
\titlecontents{figure}[0em]
  {\addvspace{-5pt}\sffamily}
  {\thecontentslabel\hspace*{1em}}
  {}
  {\ \titlerule*[.5pc]{.}\;\thecontentspage}
  []

% List of tables
\titlecontents{table}[0em]
  {\addvspace{-5pt}\sffamily}
  {\thecontentslabel\hspace*{1em}}
  {}
  {\ \titlerule*[.5pc]{.}\;\thecontentspage}
  []

% List of snippets
\titlecontents{snippet}[0em]
  {\addvspace{-5pt}\sffamily}
  {\thecontentslabel\hspace*{1em}}
  {}
  {\ \titlerule*[.5pc]{.}\;\thecontentspage}
  []

% MINI TABLE OF CONTENTS IN PART HEADS

% Chapter text styling
\titlecontents{lchapter}[0em] % Indentation
  {\addvspace{15pt}\large\sffamily\bfseries} % Spacing and font options
  {\color{themecolor}\contentslabel[\Large\thecontentslabel]{1.25cm}
   \color{themecolor}} % Chapter number
  {\color{themecolor}}
  {\color{themecolor}\normalsize\sffamily\bfseries\;%
   \titlerule*[.5pc]{.}\;\thecontentspage} % Page number

% Section text styling
\titlecontents{lsection}[0em] % Indentation
  {\sffamily\small} % Spacing and font options
  {\contentslabel[\thecontentslabel]{1.25cm}} % Section number
  {}
  {}

% Subsection text styling
\titlecontents{lsubsection}[.5em] % Indentation
  {\normalfont\footnotesize\sffamily} % Font settings
  {}
  {}
  {}

% THEOREM STYLES

\makeatletter
\newcommand{\intoo}[2]{\mathopen{]}#1\,;#2\mathclose{[}}
\newcommand{\ud}{\mathop{\mathrm{{}d}}\mathopen{}}
\newcommand{\intff}[2]{\mathopen{[}#1\,;#2\mathclose{]}}
\newtheorem{notation}{Notation}[chapter]

% Boxed/framed environments
\newtheoremstyle{themecolornumbox} % Theorem style name
  {5pt} % Space above
  {5pt} % Space below
  {\normalfont} % Body font
  {} % Indent amount
  {\small\bf\sffamily\color{themecolor}} % Theorem head font
  {\;} % Punctuation after theorem head
  {0.25em} % Space after theorem head
  {\small\sffamily\color{themecolor}\thmname{#1}\nobreakspace%
   \thmnumber{\@ifnotempty{#1}{}\@upn{#2}} % Theorem text (e.g. Theorem 2.1)
   \thmnote{\nobreakspace\the\thm@notefont\sffamily\bfseries%
            \color{black}---\nobreakspace#3.}
  } % Optional theorem note
\renewcommand{\qedsymbol}{$\blacksquare$} % Optional qed square

\newtheoremstyle{blacknumex} % Theorem style name
  {5pt} % Space above
  {5pt} % Space below
  {\normalfont} % Body font
  {} % Indent amount
  {\small\bf\sffamily} % Theorem head font
  {\;} % Punctuation after theorem head
  {0.25em} % Space after theorem head
  {\small\sffamily{\tiny\ensuremath{\blacksquare}}\nobreakspace%
   \thmname{#1}\nobreakspace%
   \thmnumber{\@ifnotempty{#1}{}\@upn{#2}} % Theorem text (e.g. Theorem 2.1)
   \thmnote{\nobreakspace\the\thm@notefont%
            \sffamily\bfseries---\nobreakspace#3.}} % Optional theorem note

\newtheoremstyle{blacknumbox} % Theorem style name
  {5pt} % Space above
  {5pt} % Space below
  {\normalfont} % Body font
  {} % Indent amount
  {\small\bf\sffamily} % Theorem head font
  {\;} % Punctuation after theorem head
  {0.25em} % Space after theorem head
  {\small\sffamily\thmname{#1}\nobreakspace%
   \thmnumber{\@ifnotempty{#1}{}\@upn{#2}} % Theorem text (e.g. Theorem 2.1)
   \thmnote{\nobreakspace\the\thm@notefont%
            \sffamily\bfseries---\nobreakspace#3.}} % Optional theorem note

% Non-boxed/non-framed environments
\newtheoremstyle{themecolornum} % Theorem style name
  {5pt} % Space above
  {5pt} % Space below
  {\normalfont} % Body font
  {} % Indent amount
  {\small\bf\sffamily\color{themecolor}} % Theorem head font
  {\;} % Punctuation after theorem head
  {0.25em} % Space after theorem head
  {\small\sffamily\color{themecolor}\thmname{#1}\nobreakspace%
   \thmnumber{\@ifnotempty{#1}{}\@upn{#2}} % Theorem text (e.g. Theorem 2.1)
  \thmnote{\nobreakspace\the\thm@notefont\sffamily\bfseries%
           \color{black}---\nobreakspace#3.}} % Optional theorem note
\renewcommand{\qedsymbol}{$\blacksquare$} % Optional qed square
\makeatother

% Defines the theorem text style for each type of theorem to one of the three
% styles above
\newcounter{dummy}
\numberwithin{dummy}{section}
\theoremstyle{themecolornumbox}
\newtheorem{theoremeT}[dummy]{Theorem}
\newtheorem{problem}{Problem}[chapter]
\newtheorem{exerciseT}{Exercise}[chapter]
\theoremstyle{blacknumex}
\newtheorem{exampleT}{Example}[chapter]
\theoremstyle{blacknumbox}
\newtheorem{vocabulary}{Vocabulary}[chapter]
\newtheorem{definitionT}{Definition}[section]
\newtheorem{corollaryT}[dummy]{Corollary}
\theoremstyle{themecolornum}
\newtheorem{proposition}[dummy]{Proposition}

% DEFINITION OF COLORED BOXES

% For creating the theorem, definition, exercise, and corollary boxes
\RequirePackage[framemethod=default]{mdframed}

% Theorem box
\newmdenv[skipabove=7pt,
          skipbelow=7pt,
          backgroundcolor=black!5,
          linecolor=themecolor,
          innerleftmargin=5pt,
          innerrightmargin=5pt,
          innertopmargin=5pt,
          leftmargin=0cm,
          rightmargin=0cm,
          innerbottommargin=5pt]{tBox}

% Exercise box
\newmdenv[skipabove=7pt,
          skipbelow=7pt,
          rightline=false,
          leftline=true,
          topline=false,
          bottomline=false,
          backgroundcolor=themecolor!10,
          linecolor=themecolor,
          innerleftmargin=5pt,
          innerrightmargin=5pt,
          innertopmargin=5pt,
          innerbottommargin=5pt,
          leftmargin=0cm,
          rightmargin=0cm,
          linewidth=4pt]{eBox}

% Definition box
\newmdenv[skipabove=7pt,
          skipbelow=7pt,
          rightline=false,
          leftline=true,
          topline=false,
          bottomline=false,
          linecolor=themecolor,
          innerleftmargin=5pt,
          innerrightmargin=5pt,
          innertopmargin=2.5pt,
          leftmargin=0cm,
          rightmargin=0cm,
          linewidth=4pt,
          innerbottommargin=2.5pt]{dBox}

% Corollary box
\newmdenv[skipabove=7pt,
          skipbelow=7pt,
          rightline=false,
          leftline=true,
          topline=false,
          bottomline=false,
          linecolor=gray,
          backgroundcolor=black!5,
          innerleftmargin=5pt,
          innerrightmargin=5pt,
          innertopmargin=5pt,
          leftmargin=0cm,
          rightmargin=0cm,
          linewidth=4pt,
          innerbottommargin=5pt]{cBox}

% Creates an environment for each type of theorem and assigns it a theorem text
% style from the "Theorem Styles" section above and a colored box from above
\newenvironment{theorem}{\begin{tBox}\begin{theoremeT}}%
  {\end{theoremeT}\end{tBox}}
\newenvironment{exercise}{\begin{eBox}\begin{exerciseT}}%
  {\hfill{\color{themecolor}\tiny\ensuremath{\blacksquare}}%
   \end{exerciseT}\end{eBox}}
\newenvironment{definition}{\begin{dBox}\begin{definitionT}}%
  {\end{definitionT}\end{dBox}}
\newenvironment{example}{\begin{exampleT}}%
  {\hfill{\tiny\ensuremath{\blacksquare}}\end{exampleT}}
\newenvironment{corollary}{\begin{cBox}\begin{corollaryT}}%
  {\end{corollaryT}\end{cBox}}

% REMARK ENVIRONMENT

\newenvironment{remark}{\par\vspace{10pt}\small % Vertical white space above the remark and smaller font size
  \begin{list}{}{
    \leftmargin=35pt % Indentation on the left
    \rightmargin=25pt}\item\ignorespaces % Indentation on the right
    \makebox[-2.5pt]{%
      \begin{tikzpicture}[overlay]
        \node[draw=themecolor!60,line width=1pt,circle,fill=themecolor!25,font=\sffamily\bfseries,inner sep=2pt,outer sep=0pt] at (-15pt,0pt)%
          {\textcolor{themecolor}{R}}; % Orange R in a circle
      \end{tikzpicture}%
    }
    \advance\baselineskip -1pt}%
  {\end{list}\vskip5pt} % Tighter line spacing and white space after remark

% SECTION NUMBERING IN THE MARGIN

\makeatletter
\renewcommand{\@seccntformat}[1]{\llap{\textcolor{themecolor}%
  {\csname the#1\endcsname}\hspace{1em}}}
\renewcommand{\section}{\@startsection{section}{1}{\z@}
  {-4ex \@plus -1ex \@minus -.4ex}
  {1ex \@plus.2ex}
  {\normalfont\large\sffamily\bfseries}}
\renewcommand{\subsection}{\@startsection {subsection}{2}{\z@}
  {-3ex \@plus -0.1ex \@minus -.4ex}
  {0.5ex \@plus.2ex}
  {\normalfont\sffamily\bfseries}}
\renewcommand{\subsubsection}{\@startsection {subsubsection}{3}{\z@}
  {-2ex \@plus -0.1ex \@minus -.2ex}
  {.2ex \@plus.2ex}
  {\normalfont\small\sffamily\bfseries}}
\renewcommand\paragraph{\@startsection{paragraph}{4}{\z@}
  {-2ex \@plus-.2ex \@minus .2ex}
  {.1ex}
  {\normalfont\small\sffamily\bfseries}}

% PART HEADINGS

% numbered part in the table of contents
\newcommand{\@mypartnumtocformat}[2]{%
  \setlength\fboxsep{0pt}%
  \noindent\colorbox{themecolor!20}{%
    \strut\parbox[c][.7cm]{\ecart}{%
      \color{themecolor!70}\Large\sffamily\bfseries\centering#1}}%
  \hskip\esp\colorbox{themecolor!40}{%
    \strut\parbox[c][.7cm]{\linewidth-\ecart-\esp}{%
      \Large\sffamily\centering#2}}}%
%%%%%%%%%%%%%%%%%%%%%%%%%%%%%%%%%%
% unnumbered part in the table of contents
\newcommand{\@myparttocformat}[1]{%
  \setlength\fboxsep{0pt}%
  \noindent\colorbox{themecolor!40}{%
    \strut\parbox[c][.7cm]{\linewidth}{%
      \Large\sffamily\centering#1}}}%
%%%%%%%%%%%%%%%%%%%%%%%%%%%%%%%%%%
\newlength\esp
\setlength\esp{4pt}
\newlength\ecart
\setlength\ecart{1.2cm-\esp}
\newcommand{\thepartimage}{}%
\newcommand{\partimage}[1]{\renewcommand{\thepartimage}{#1}}%
\def\@part[#1]#2{%
  \ifnum \c@secnumdepth >-2\relax%
    \refstepcounter{part}%
    \addcontentsline{toc}{part}{%
      \texorpdfstring{\protect\@mypartnumtocformat{\thepart}{#1}}{%
        \partname~\thepart\ ---\ #1}}
  \else%
    \addcontentsline{toc}{part}{%
      \texorpdfstring{\protect\@myparttocformat{#1}}{#1}}%
  \fi%
  \startcontents%
  \markboth{}{}%
  {%
    \thispagestyle{empty}%
    \begin{tikzpicture}[remember picture,overlay]%
      \node at (current page.north west){%
        \begin{tikzpicture}[remember picture,overlay]%
          \fill[themecolor!20](0cm,0cm) rectangle (\paperwidth,-\paperheight);
          \node[anchor=north] at (4cm,-3.25cm){%
            \color{themecolor!40}\fontsize{220}{100}\sffamily\bfseries\thepart};
          \node[anchor=south east] at (\paperwidth-1cm,-\paperheight+1cm){%
            \parbox[t][][t]{8.5cm}{%
              \printcontents{l}{0}{%
                \setcounter{tocdepth}{1}%
              }%
            }%
          };
          \node[anchor=north east] at (\paperwidth-1.5cm,-3.25cm){%
            \parbox[t][][t]{15cm}{%
              \strut\raggedleft\color{white}\fontsize{30}{30}\sffamily\bfseries#2}};
        \end{tikzpicture}
      };
    \end{tikzpicture}%
  }%
\@endpart}
\def\@spart#1{%
  \startcontents%
  \phantomsection
  {%
  \thispagestyle{empty}%
  \begin{tikzpicture}[remember picture,overlay]%
    \node at (current page.north west){%
      \begin{tikzpicture}[remember picture,overlay]%
        \fill[themecolor!20](0cm,0cm) rectangle (\paperwidth,-\paperheight);
        \node[anchor=north east] at (\paperwidth-1.5cm,-3.25cm){%
          \parbox[t][][t]{15cm}{%
            \strut\raggedleft\color{white}\fontsize{30}{30}\sffamily\bfseries#1}
        };
      \end{tikzpicture}
    };
  \end{tikzpicture}%
  }
  \addcontentsline{toc}{part}{\texorpdfstring{%
    \setlength\fboxsep{0pt}%
    \noindent\protect\colorbox{themecolor!40}{%
      \strut\protect\parbox[c][.7cm]{\linewidth}{%
        \Large\sffamily\protect\centering #1\quad\mbox{}}}}{#1}
  }%
\@endpart}
\def\@endpart{%
  \vfil\newpage
  \if@twoside
    \if@openright
      \vspace*{\stretch{1}}
      \begin{center}
        \textit{This page intentionally left blank}
      \end{center}
      \vspace*{\stretch{4}}
      \null
      \thispagestyle{empty}%
      \newpage
    \fi
  \fi
  \if@tempswa
    \twocolumn
  \fi%
}

% CHAPTER HEADINGS

% A switch to conditionally include a picture, implemented by  Christian Hupfer
\newif\ifusechapterimage
\usechapterimagetrue
\newcommand{\thechapterimage}{}%
\newcommand{\chapterimage}[1]{%
  \ifusechapterimage\renewcommand{\thechapterimage}{#1}\fi}%
\newcommand{\autodot}{.}
\def\@makechapterhead#1{{%
  \parindent \z@ \raggedright \normalfont
  \ifnum \c@secnumdepth >\m@ne
    \if@mainmatter
      \begin{tikzpicture}[remember picture,overlay]
        \node at (current page.north west){%
          \begin{tikzpicture}[remember picture,overlay]
            \node[anchor=north west,inner sep=0pt] at (0,0)%
              {\ifusechapterimage\includegraphics[width=\paperwidth]%
               {\thechapterimage}\fi};
            \draw[anchor=west] (\Gm@lmargin,-9cm) node%
              [line width=2pt,rounded corners=15pt,draw=themecolor,fill=white,%
               fill opacity=0.5,inner sep=15pt]{\strut\makebox[22cm]{}};
            \draw[anchor=west] (\Gm@lmargin+.3cm,-9cm) node%
              {\huge\sffamily\bfseries\color{black}\thechapter\autodot~#1\strut};
          \end{tikzpicture}%
        };
      \end{tikzpicture}
    \else
      \begin{tikzpicture}[remember picture,overlay]
        \node at (current page.north west){%
          \begin{tikzpicture}[remember picture,overlay]
            \node[anchor=north west,inner sep=0pt] at (0,0)%
              {\ifusechapterimage\includegraphics[width=\paperwidth]{%
               \thechapterimage}\fi};
            \draw[anchor=west] (\Gm@lmargin,-9cm) node%
              [line width=2pt,rounded corners=15pt,draw=themecolor,fill=white,%
               fill opacity=0.5,inner sep=15pt]{\strut\makebox[22cm]{}};
            \draw[anchor=west] (\Gm@lmargin+.3cm,-9cm) node%
              {\huge\sffamily\bfseries\color{black}#1\strut};
          \end{tikzpicture}};
      \end{tikzpicture}
    \fi
  \fi\par\vspace*{270\p@}}}

%-------------------------------------------

\def\@makeschapterhead#1{%
  \begin{tikzpicture}[remember picture,overlay]
    \node at (current page.north west){%
      \begin{tikzpicture}[remember picture,overlay]
        \node[anchor=north west,inner sep=0pt] at (0,0){%
          \ifusechapterimage
            \includegraphics[width=\paperwidth]{\thechapterimage}
          \fi};
        \draw[anchor=west] (\Gm@lmargin,-9cm) node%
          [line width=2pt,rounded corners=15pt,draw=themecolor,fill=white,%
           fill opacity=0.5,inner sep=15pt]{\strut\makebox[22cm]{}};
        \draw[anchor=west] (\Gm@lmargin+.3cm,-9cm) node%
          {\huge\sffamily\bfseries\color{black}#1\strut};
      \end{tikzpicture}%
    };
  \end{tikzpicture}
  \par\vspace*{270\p@}}
\makeatother

\usepackage{bookmark}
\bookmarksetup{open,numbered,addtohook={%
  \ifnum\bookmarkget{level}=0 % chapter
    \bookmarksetup{bold}%
  \fi
  \ifnum\bookmarkget{level}=-1 % part
    \bookmarksetup{color=themecolor,bold}%
  \fi
}
}

% Set styles for TikZ objects
\usetikzlibrary{arrows, circuits.ee.IEC, decorations.markings,
    decorations.pathreplacing, positioning, shapes}
\tikzstyle{block} = [draw, fill=headingbg, rectangle, minimum height=3em,
    minimum width=4em]
\tikzstyle{sum} = [draw, circle, node distance=1cm]
\tikzstyle{arrow} = [arrows=->, black, align=right]
\tikzstyle{branch} = [circle, inner sep=0pt, minimum size=1mm, fill=black,
    draw=black]
\tikzstyle{opencircuit} = [circle, draw=black, fill=white, minimum size=3pt,
    inner sep=0pt]

\lstdefinestyle{customMatlab}{
  language=Matlab,
  breaklines=true,
  xleftmargin=0.125in,
  basicstyle=\footnotesize\ttfamily,
  keywordstyle=\color{blue},
  morekeywords=[2]{1}, keywordstyle=[2]{\color{black}},
  identifierstyle=\color{black},
  stringstyle=\color[RGB]{170, 55, 241},
  commentstyle=\color[RGB]{28, 172, 0},
  showstringspaces=false,
  emph=[1]{for,end,break},emphstyle=[1]\color{red},
}

\lstdefinestyle{customPython}{
  language=Python,
  breaklines=true,
  xleftmargin=0.125in,
  basicstyle=\footnotesize\ttfamily,
  otherkeywords={self,True,False},
  keywordstyle=\color{deepblue},
  emph={MyClass,__init__},
  emphstyle=\color{deepred},
  stringstyle=\color{deepgreen},
  showstringspaces=false
}

\lstdefinestyle{customcpp}{
  belowcaptionskip=1\baselineskip,
  breaklines=true,
  xleftmargin=0.125in,
  language=C++,
  showstringspaces=false,
  basicstyle=\footnotesize\ttfamily,
  keywordstyle=\color[RGB]{128, 128, 0},
  commentstyle=\color[RGB]{0, 128, 0},
  identifierstyle=\color{black},
  stringstyle=\color[RGB]{0, 128, 0},
}

\newenvironment{code}[2]%
  {\FloatBarrier
   \begin{snippet}
     \lstinputlisting[escapechar=, style=custom#1]{#2}}%
 {\end{snippet}
   \FloatBarrier}

\newenvironment{bookfigure}%
  {\begin{figure}%
    \centering%
  }%
  {\end{figure}}

% #1: column arguments to tabulary
\newenvironment{figurekey}%
  {\begin{center}%
    \renewcommand{\arraystretch}{1.3}}
  {\end{center}}

\newenvironment{booktable}%
  {\begin{table}%
    \renewcommand{\arraystretch}{1.5}%
    \centering%
  }%
  {\end{table}}

% #1: file name without extension
\newenvironment{svg}[1]
  {\begin{figure}%
    \def\svgwidth{\linewidth}%
    \input{#1.pdf_tex}}
  {\end{figure}}

% #1: file name without extension
\newenvironment{minisvg}[1]
  {\begin{minipage}{0.475\linewidth}%
    \def\svgwidth{\textwidth}%
    \input{#1.pdf_tex}}
  {\end{minipage}}

\newcommand{\mtx}[1] {\mathbf{#1}}
\newcommand{\meanmtx}[1] {\overline{\mathbf{#1}}}

% Set up snippet environment
\newcounter{snippet}[chapter]
\renewcommand{\thesnippet}{\thechapter.\arabic{snippet}}
\newenvironment{snippet}{
  \renewcommand{\caption}[1]{
    \refstepcounter{snippet}
    \begin{center}
      \par\noindent{\footnotesize Snippet \thesnippet. ##1}
    \end{center}
  }
}
{\par\noindent}

% #1: xshift
% #2: yshift
% #3: xcoordshift
% #4: ycoordshift
% #5: func to plot
\newcommand{\drawtimeplot}[5] {
  \begin{scope}[xshift=#1-0.5cm,yshift=#2-0.5cm]
    \def\xcoordshift{#3,0}
    \def\ycoordshift{0,#4}

    \draw[fill=headingbg] (0,0) rectangle (1,1);

    \begin{scope}[shift=(\xcoordshift)]
      % Draw Y axis
      \draw[->] (0cm,0.05cm) -- (0cm,0.9375cm) node {};

      \begin{scope}[shift=(\ycoordshift)]
        % Draw X axis
        \draw[->,font=\tiny] (-0.075cm,0cm) -- (0.8125cm,0cm)
          node[label={[below]90:$t$}] {};

        \draw[xscale=1.2,yscale=0.3,domain=0:0.5,smooth,variable=\x,line width=0.5]
          plot ({\x},{#5});
      \end{scope}
    \end{scope}
  \end{scope}}

% #1: xshift
% #2: yshift
\newcommand{\drawpole}[2] {
  \begin{scope}[xshift=#1,yshift=#2]
    \draw[themecolor,rotate=45] (-0.1,-0.0025) rectangle (0.1,0.0025);
    \draw[themecolor,rotate=-45] (-0.1,-0.0025) rectangle (0.1,0.0025);
  \end{scope}}

% Set custom \cleardoublepage
\makeatletter
\newcommand{\setcleardoublepage}{%
  % Removes the header from odd empty pages at the end of chapters
  \renewcommand{\cleardoublepage}{%
    \clearpage
    \ifodd
      \c@page
    \else
      \vspace*{\stretch{1}}
      \begin{center}
        \textit{This page intentionally left blank}
      \end{center}
      \vspace*{\stretch{4}}
      \thispagestyle{empty}
      \newpage
    \fi
  }
}
\makeatother

% Make \cleardoublepage do nothing
\newcommand{\unsetcleardoublepage}{%
  \renewcommand{\cleardoublepage}{}
}

\newdateformat{monthdayyeardate}{%
  \monthname[\THEMONTH]~\THEDAY, \THEYEAR}

% Disable automatic indent and provide \indent command
\newlength\tindent
\setlength{\tindent}{\parindent}
\setlength{\parindent}{0pt}
\renewcommand{\indent}{\hspace*{\tindent}}

% Paper header
\fancypagestyle{plain}{%
  \fancyhf{}
  \renewcommand{\headrulewidth}{0.1pt}
  \fancyhead[L]{Practical Guide to State-space Control}
  \fancyhead[R]{\thepage}
}
\pagestyle{plain}

\allowdisplaybreaks

\newglossaryentry{controller}{
  name={controller},
  description={Used in positive or negative feedback with a plant to bring about
    a desired system state by driving the difference between a reference signal
    and the output to zero.}}
\newglossaryentry{control law}{
  name={control law},
  description={Also known as control policy, is a mathematical formula used by
  the controller to determine the input u that is sent to the plant. This
  control law is designed to drive the system from its current state to some
  other desired state.}}
\newglossaryentry{discretization}{
  name={discretization},
  description={The process by which a continuous (e.g., analog) system or
  controller design is converted to discrete (e.g., digital).}}
\newglossaryentry{disturbance}{
  name={disturbance},
  description={An external force acting on a system that isn't included in the
    system's model.}}
\newglossaryentry{disturbance rejection}{
  name={disturbance rejection},
  description={The quality of a feedback control system to compensate for
    external forces to reach a desired reference.}}
\newglossaryentry{error}{
  name={error},
  description={Reference minus input.}}
\newglossaryentry{gain margin}{
  name={gain margin},
  description={See section \ref{sec:gain-phase-margin} on gain and phase
    margin.}}
\newglossaryentry{input}{
  name={input},
  description={An input to the plant (hence the name) that can be used to change
  the plant's state.}}
\newglossaryentry{model}{
  name={model},
  description={A set of mathematical equations that reflects some aspect of a
    physical system's behavior.}}
\newglossaryentry{noise immunity}{
  name={noise immunity},
  description={The quality of a system to have its performance or stability
    unaffected by noise in the outputs (see also: \gls{robustness}).}}
\newglossaryentry{open-loop gain}{
  name={open-loop gain},
  description={The gain directly from the input to the output, ignoring loops.}}
\newglossaryentry{output}{
  name={output},
  description={Measurements from sensors.}}
\newglossaryentry{output-based control}{
  name={output-based control},
  description={Controls the system's state via the outputs.}}
\newglossaryentry{phase margin}{
  name={phase margin},
  description={See section \ref{sec:gain-phase-margin} on gain and phase
    margin.}}
\newglossaryentry{plant}{
  name={plant},
  description={The system or collection of actuators being controlled.}}
\newglossaryentry{realization}{
  name={realization},
  description={In systems theory, this is an implementation of a given
  input-output behavior as a state-space model.}}
\newglossaryentry{reference}{
  name={reference},
  description={The desired state.}}
\newglossaryentry{regulator}{
  name={regulator},
  description={A controller that attempts to minimize the error from a constant
    reference in the presence of disturbances.}}
\newglossaryentry{robustness}{
  name={robustness},
  description={The quality of a feedback control system to remain stable in
    response to disturbances and uncertainty.}}
\newglossaryentry{state}{
  name={state},
  description={A characteristic of a system (e.g., velocity) that can be used to
    determine the system's future behavior.}}
\newglossaryentry{state feedback}{
  name={state feedback},
  description={Uses state instead of output in feedback.}}
\newglossaryentry{steady-state error}{
  name={steady-state error},
  description={Error after system reaches equilibrium.}}
\newglossaryentry{stochastic process}{
  name={stochastic process},
  description={A process whose model is partially or completely defined by
    random variables.}}
\newglossaryentry{system}{
  name={system},
  description={Maps inputs to outputs through linear combination of states.}}
\newglossaryentry{time-invariant}{
  name={time-invariant},
  description={The system's fundamental response does not change over time.}}
\newglossaryentry{tracking}{
  name={tracking},
  description={In the context of control theory, the process of making the
    output of a control system follow the reference input.}}
\newglossaryentry{unity feedback}{
  name={unity feedback},
  description={A feedback network in a control system diagram with a feedback
    gain of 1.}}

\makeglossaries

\begin{document}
% \usechapterimagefalse
\frontmatter
% Title page
\begingroup
\thispagestyle{empty}
\begin{tikzpicture}[remember picture,overlay]
  \node[inner sep=0pt] (background) at (current page.center) {%
    \includegraphics[width=\paperwidth,height=\paperheight]{front.jpg}};
  \draw (current page.center) node%
    [fill=themecolor!30!white,fill opacity=0.6,text opacity=1,inner sep=1cm]%
    {\Huge\centering\bfseries\sffamily\parbox[c][][t]{\paperwidth}%
    {\centering Practical Guide to State-space Control\\[15pt] % Book title
  {\Large Graduate-level control theory for high schoolers}\\[20pt] % Subtitle
  {\huge Tyler Veness}}}; % Author name
\end{tikzpicture}
\vfill
\endgroup

% Copyright page
\newpage
~\vfill
\thispagestyle{empty}

% Copyright notice
Copyright \copyright\ 2017 Tyler Veness

Generated on \monthdayyeardate\today.

% URL
\textsc{\url{https://github.com/calcmogul/state-space-guide}}

% License information
Licensed under the Creative Commons Attribution-ShareAlike 4.0 Unported License
(the ``License''). You may not use this file except in compliance with the
License. You may obtain a copy of the License at
\url{http://creativecommons.org/licenses/by-sa/4.0}. Unless required by
applicable law or agreed to in writing, software distributed under the License
is distributed on an \textsc{``as is'' basis, without warranties or conditions
of any kind}, either express or implied. See the License for the specific
language governing permissions and limitations under the License.

% Table of contents
\chapterimage{toc.jpg}
\pagestyle{empty} % No headers
\tableofcontents % Print the table of contents itself
\cleardoublepage
\pagestyle{fancy} % Print headers again

\chapterimage{preface.jpg}

\chapter*{Preface}
\addcontentsline{toc}{chapter}{\textcolor{themecolor}{Preface}}

\section*{Motivation}

I am the software mentor for a FIRST Robotics Competition (FRC) team. My
responsibilities for that include teaching programming, software engineering
practices, and applications of control theory. The curriculum I developed so far
(located at \url{https://csweb.frc3512.com/ci/}) teaches rookies enough to be
minimally competitive, but many of the more advanced sections are incomplete. It
provides no formal avenues of growth for veteran students.

Also, out of a six week build season, the software team usually only gets a few
days with the completed robot due to poor build schedule management. This leads
to two problems. First, two days is only enough time to verify basic software
functionality, not test and tune feedback controllers. Second, this is also the
first time the robot's electromechanical systems have been tested after
integration, so any issues that arise consume valuable software integration time
while the team traces the problem to a mechanical, electrical, or software
cause.

This book expands my curriculum to cover control theory topics I have learned in
my graduate-level engineering classes at University of California, Santa Cruz.
It introduces state-space controllers and serves as a practical guide for
formulating and implementing them. Since state-space control utilizes a system
model, both problems mentioned earlier can be addressed. Basic software
functionality can be tested against it and feedback controllers can be tuned
automatically based on system constraints. This allows software teams to test
their work much earlier in the build season in a controlled environment as well
as save time during feedback controller design, implementation, and testing.

\section*{Intended Audience}

This guide is intended to make an advanced engineering topic approachable so it
can be applied by those who aren't experts in control theory. My intended
audience is high school students who are veteran members of a FIRST Robotics
Competition team. As such, they will already be familiar with feedback control
applications like PID and have basic proficiency in programming. This guide will
build on their current knowledge of control theory and teach them enough about
state-space control and auxiliary topics to be able to implement it in the
programming language of their choice.

Knowledge of basic algebra, complex numbers, physics, and a bit of calculus (for
the system modeling) are assumed.

\section*{Acknowledgements}

I would like to thank my controls engineering instructors Dejan Milutinovi\'c
and Gabriel Elkaim of University of California, Santa Cruz. They taught their
classes from a pragmatic perspective focused on application and intuition that I
appreciated. I would also like to thank Dejan Milutinovi\'c for introducing me
to the field of control theory and teaching me what it means to be a controls
engineer.

Thanks to Austin Schuh from FRC team 971 for providing the final continuous
state-space models used in the examples section.

\cleardoublepage


\mainmatter
\chapterimage{notes-to-the-reader.jpg}{Trees by Baskin Engineering building at UCSC}

\chapter{Notes to the reader}

\renewcommand*{\chapterpath}{\partpath/notes-to-the-reader}
\section{Prerequisites}

Knowledge of basic algebra and complex numbers is assumed. Some introductory
physics and calculus will be taught as necessary.

\section{The structure of this book}

This book consists of five parts and a collection of appendices that address the
four tasks a controls engineer carries out: derive a model of the system
(kinematics), design a controller for the model (control theory), design an
observer to estimate the current state of the model (localization), and plan how
the controller is going to drive the model to a desired state (motion planning).

Part I, ``Kinematics and dynamics," introduces the basic calculus and physics
concepts required to derive the models used in the later chapters. It walks
through the derivations for several common FRC subsystems.

Part II, ``Classical control theory," introduces the basics of control theory,
teaches the fundamentals of PID controller design, describes what a transfer
function is, and shows how they can be used to analyze dynamical systems.
Emphasis is placed on the geometric intuition of this analysis rather than the
frequency domain math.

Part III, ``Modern control theory," first provides a crash course in the
geometric intuition behind linear algebra and covers enough of the mechanics of
evaluating matrix algebra for the reader to follow along in later chapters. It
covers state-space representation, controllability, and observability. The
intuition gained in part II and the notation of linear algebra are used to model
and control linear multiple-input, multiple-output (MIMO) systems and covers
discretization, LQR controller design, LQE observer design, and feedforwards.
Then, these concepts are applied to design and implement controllers for real
systems. The examples from part I are converted to state-space representation,
implemented, and tested with a digital controller.

Part III also introduces the basics of nonlinear control system analysis with
Lyapunov functions. It presents an example of a nonlinear controller for a
unicycle-like vehicle as well as how to apply it to a two-wheeled vehicle. Since
nonlinear control isn't the focus of this book, we mention other books and
resources for further reading.

Part IV, ``Estimation and localization," introduces the field of stochastic
control theory. The Luenberger observer and the probability theory behind the
Kalman filter is taught with several examples of creative applications of Kalman
filter theory.

Part V, ``Motion planning," covers planning how the robot will get from its
current state to some desired state in a manner achievable by its dynamics.
Motion profiles are introduced.

The appendices provide further enrichment that isn't required for a passing
understanding of the material. This includes derivations for many of the results
presented and used in the mainmatter of the book.

The Python scripts used to generate the plots in the case studies double as
reference implementations of the techniques discussed in their respective
chapters. They are available in this book's Git repository. Its location is
listed on the copyright page.

\section{The mindset of an egoless engineer}
\label{sec:the_mindset_of_an_egoless_engineer}

The following maxim summarizes what I hope to teach my robotics students (with
examples drawn from controls engineering).

\begin{quote}
  ``Engineer based on requirements, not an ideology."
\end{quote}

Engineering is filled with trade-offs. The tools should fit the job, and not
every problem is a nail waiting to be struck by a hammer. Instead, assess the
minimum requirements (min specs) for a solution to the task at hand and do only
enough work to satisfy them; exceeding your specifications is a waste of time
and money. If you require performance or maintainability above the min specs,
your min specs were chosen incorrectly by definition.

Controls engineering is pragmatic in a similar respect:
\textit{solve. the. problem}. For control of nonlinear systems,
\href{https://faculty.washington.edu/devasia/Inversion.html}{plant inversion}
is elegant on paper but doesn't work with an inaccurate model, yet using a
theoretically incorrect solution like linear approximations of the nonlinear
system works well enough to be used industry-wide. There are more sophisticated
controllers than PID, but we use PID anyway for its versatility and simplicity.
Sometimes the inferior solutions are more effective or have a more desirable
cost-benefit ratio than what the control system designer considers ideal or
clean. Choose the tool that is most effective.

Solutions need to be good enough, but do not need to be perfect. We want to
avoid integrators as they introduce instability, but we use them anyway because
they work well for meeting tracking specifications. One should not blindly
defend a design or follow an ideology, because there is always a case where its
antithesis is a better option. The engineer should be able to determine when
this is the case, set aside their ego, and do what will meet the specifications
of their client (e.g., system response characteristics, maintainability,
usability). Preferring one solution over another for pragmatic or technical
reasons is fine, but the engineer should not care on a personal level which
sufficient solution is chosen.

\section{Request for feedback}

While we have tried to write a book that makes the topics of control theory
approachable, it still may be dense or fast-paced for some readers (it covers
three classes of feedback control, two of which are for graduate students, in
one short book). Please send us feedback, corrections, or suggestions through
the GitHub link listed on the copyright page. New examples that demonstrate key
concepts and make them more accessible are also appreciated.


\chapterimage{introduction.jpg}{Forest between Baskin Engineering and Thimann Labs at UCSC}

\chapter{Introduction}

\section{What is control theory?}

How can we prove an autonomous car will behave safely and meet certain
performance specifications in the presence of uncertainty? Control theory is a
pragmatic application of algebra and geometry that is used to analyze and
predict the behavior of \glspl{system} such as these, make them respond how we
want them to, and make them \glslink{robustness}{robust} to \glspl{disturbance}
and uncertainty.

But what sets control theory apart from, say, applied math? While control theory
does have some beautiful math behind it, controls engineering is an engineering
discipline like any other filled with trade-offs. The solutions control theory
gives should always be sanity checked and informed by our performance
specifications. We don't need to be perfect, just good enough to meet our
specifications.

\section{Nomenclature}

Most resources for advanced engineering topics assume a level of knowledge well
above that which is necessary. Part of the problem is the use of jargon. While
it efficiently communicates ideas to those within the field, new people who
aren't familiar with it are lost. See the glossary for a list of words and
phrases commonly used in control theory, their origins, and their meaning. Below
is a table describing how the terms \textit{input} and \textit{output} apply to
\glspl{plant} versus \glspl{controller} and what letters are commonly associated
with each when working with them. Namely, that the terms input and output are
defined with respect to the \gls{plant}, not the \gls{controller}.

\begin{table}
  \renewcommand{\arraystretch}{1.3}
  \centering
  \begin{tabular}{|l|ll|}
    \hline
    \rowcolor{headingbg}
    & \textbf{Plant} & \textbf{Controller} \\
    \hline
    Input & $u(t)$ & $r(t)$, $y(t)$ \\
    Output & $y(t)$ & $u(t)$ \\
    \hline
  \end{tabular}
  \caption{Plant versus controller nomenclature}
  \label{tab:plant_v_controller}
\end{table}

\part{Classical control}
\chapterimage{control-system-basics.jpg}{On trail between McHenry Library and Media Theater at UCSC}

\chapter{Control system basics}

\section{What is gain?}
\index{Gain}

Gain is a proportional value that shows the relationship between the magnitude
of the input to the magnitude of the output signal at steady state. Many
\glspl{system} contain a method by which the gain can be altered, providing more
or less ``power" to the \gls{system}.

Figure \ref{fig:input_output_gain} shows a \gls{system} with a hypothetical
input and output. Since the output is twice the amplitude of the input, the
\gls{system} has a gain of $2$.

\begin{bookfigure}
  \begin{tikzpicture}[auto, >=latex']
    % \draw [help lines] (-4,-2) grid (4,2);

    % Input
    \drawtimeplot{-2.5cm}{0cm}{0.125cm}{0.44375cm}{0.6 * cos(40 * deg(\x))}
    \draw (-2.5,1) node {\small Input};

    \node [block] (sys) {K};
    \draw (0,1) node {\small System};

    % Output
    \drawtimeplot{2.5cm}{0cm}{0.125cm}{0.44375cm}{1.2 * cos(40* deg(\x))}
    \draw (2.5,1) node {\small Output};

    % Arrows between input/output and system
    \draw[->] (-2,0) -- (sys);
    \draw[->] (sys) -- (2,0);
  \end{tikzpicture}

  \caption{Demonstration of system with a gain of $K = 2$}
  \label{fig:input_output_gain}
\end{bookfigure}

\section{Block diagrams}
\index{Block diagrams}

When designing or analyzing a \gls{control system}, it is useful to model it
graphically. Block diagrams are used for this purpose. They can be manipulated
and simplified systematically (see appendix
\ref{ch:simplifying_block_diagrams}). Figure \ref{fig:gain_nomenclature} is an
example of one.

\begin{bookfigure}
  \begin{tikzpicture}[auto, >=latex']
    % Place the blocks
    \node [name=input] {input};
    \node [sum, right=of input] (sum) {};
    \node [block, right=of sum] (P1) {open-loop};
    \node [right=of P1] (output) {output};
    \node [block, below=of P1] (P2) {feedback};

    % Connect the nodes
    \draw [arrow] (input) -- node[pos=0.85] {$+$} (sum);
    \draw [arrow] (sum) -- node {} (P1);
    \draw [arrow] (P1) -- node[name=y] {} (output);
    \draw [arrow] (y) |- (P2);
    \draw [arrow] (P2) -| node[pos=0.97, right] {$\mp$} (sum);
  \end{tikzpicture}

  \caption{Block diagram with nomenclature}
  \label{fig:gain_nomenclature}
\end{bookfigure}

The \gls{open-loop gain} is the total gain from the sum node at the input to the
output branch. The \gls{feedback gain} is the total gain from the output back to
the input sum node. The circle's output is the sum of its inputs.

Figure \ref{fig:feedback_block_diagram} is a block diagram with more formal
notation in a feedback configuration.

\begin{bookfigure}
  \begin{tikzpicture}[auto, >=latex']
    % Place the blocks
    \node [name=input] {$X(s)$};
    \node [sum, right=of input] (sum) {};
    \node [block, right=of sum] (P1) {$P_1$};
    \node [right=of P1] (output) {$Y(s)$};
    \node [block, below=of P1] (P2) {$P_2$};

    % Connect the nodes
    \draw [arrow] (input) -- node[pos=0.85] {$+$} (sum);
    \draw [arrow] (sum) -- node {} (P1);
    \draw [arrow] (P1) -- node[name=y] {} (output);
    \draw [arrow] (y) |- (P2);
    \draw [arrow] (P2) -| node[pos=0.97, right] {$\mp$} (sum);
  \end{tikzpicture}

  \caption{Feedback block diagram}
  \label{fig:feedback_block_diagram}
\end{bookfigure}

\begin{theorem}[Closed-loop gain for a feedback loop]
  \begin{equation}
    \frac{Y(s)}{X(s)} = \frac{P_1}{1 \mp P_1 P_2}
  \end{equation}
\end{theorem}

See appendix \ref{sec:deriv_tf_feedback} for a derivation.

\section{Why feedback control?}

Let's say we are controlling a DC brushed motor. With just a
\glslink{model}{mathematical model} and knowledge of all current \glspl{state}
of the \gls{system} (i.e., angular velocity), we can predict all future
\glspl{state} given the future voltage \glspl{input}. Why then do we need
feedback control? If the \gls{system} is \glslink{disturbance}{disturbed} in any
way that isn't modeled by our equations, like a load was applied to the
armature, or voltage sag in the rest of the circuit caused the commanded voltage
to not match the actual applied voltage, the angular velocity of the motor will
deviate from the \gls{model} over time.

To combat this, we can take measurements of the \gls{system} and the environment
to detect this deviation and account for it. For example, we could measure the
current position and estimate an angular velocity from it. We can then give the
motor corrective commands as well as steer our \gls{model} back to reality. This
feedback allows us to account for uncertainty and be
\glslink{robustness}{robust} to it.


\chapterimage{review-of-pid.jpg}{Treeline by Crown/Merril bus stop at UCSC}

\chapter{Review of PID controller mathematics}

\section{PID basics and theory}

Negative feedback loops drive the difference between the \gls{reference} and
\gls{output} to zero.

\textbf{Proportional} gain compensates for current \gls{error}. \\
\textbf{Integral} gain compensates for past error (i.e.,
\gls{steady-state error}). \\
\textbf{Derivative} gain compensates for future error by slowing controller down
  if error decreases over time.

\begin{figure}[H]
  \centering

  \begin{tikzpicture}[auto, >=latex']
    \fontsize{9pt}{10pt}

    % Place the blocks
    \node [name=input] {$r(t)$};
    \node [sum, right=0.5cm of input] (errorsum) {};
    \node [coordinate, right=0.75cm of errorsum] (branch) {};
    \node [block, right=0.5cm of branch] (I) { $K_i \int_0^t e(\tau) d\tau$ };
    \node [block, above=0.5cm of I] (P) { $K_p e(t)$ };
    \node [block, below=0.5cm of I] (D) { $K_d \frac{de(t)}{dt}$ };
    \node [sum, right=0.5cm of I] (ctrlsum) {};
    \node [block, right=0.75cm of ctrlsum] (plant) {Plant};
    \node [right=0.75cm of plant] (output) {};
    \node [coordinate, below=0.5cm of D] (measurements) {};

    % Connect the nodes
    \draw [arrow] (input) -- node[pos=0.9] {$+$} (errorsum);
    \draw [-] (errorsum) -- node {$e(t)$} (branch);
    \draw [arrow] (branch) |- (P);
    \draw [arrow] (branch) -- (I);
    \draw [arrow] (branch) |- (D);
    \draw [arrow] (P) -| node[pos=0.95, left] {$+$} (ctrlsum);
    \draw [arrow] (I) -- node[pos=0.9, below] {$+$} (ctrlsum);
    \draw [arrow] (D) -| node[pos=0.95, right] {$+$} (ctrlsum);
    \draw [arrow] (ctrlsum) -- node {$u(t)$} (plant);
    \draw [arrow] (plant) -- node [name=y] {$y(t)$} (output);
    \draw [-] (y) |- (measurements);
    \draw [arrow] (measurements) -| node[pos=0.99, right] {$-$} (errorsum);
  \end{tikzpicture}

  \caption{PID controller diagram}
  \label{fig:pid_ctrl_diag}
\end{figure}

\begin{center}
  \renewcommand{\arraystretch}{1.3}
  \begin{tabulary}{\linewidth}{LLLL}
    $r(t)$ & \gls{reference} input & $u(t)$ & control input \\
    $e(t)$ & error & $y(t)$ & \gls{output} \\
  \end{tabulary}
\end{center}

\section{Types of PID controllers}

PID controller inputs of different orders of derivatives, such as position and
velocity, affect the \gls{system} response differently. Below is the standard
form for a position PID controller.

\begin{definition}[Position PID controller]
  \begin{equation}
    u(t) = K_p e(t) + K_i \int_0^t e(\tau) d\tau + K_d \frac{de}{dt}
    \label{def:pos_pid}
  \end{equation}
\end{definition}

If the controller is measuring and controlling velocity instead, the control
input $u(t)$ becomes $\frac{du}{dt}$ and the error $e(t)$ becomes
$\frac{de}{dt}$. Substituting these into equation (\ref{def:pos_pid}) yields

\begin{align}
  \frac{du}{dt} &= K_p \frac{de}{dt} + K_i \int_0^t \frac{de}{d\tau} d\tau +
    K_d \frac{d^2e}{dt^2} \nonumber \\
  \frac{du}{dt} &= K_p \frac{de}{dt} + K_i e(t) + K_d \frac{d^2e}{dt^2}
\end{align}

This shows that the proportional action ($K_p$) of a velocity controller is
caused by the integral action ($K_i$) of a position controller. The derivative
action ($K_i$) of a velocity controller is caused by the proportional action
($K_p$) of a position controller. Integral action ($K_i$) of a velocity
controller has no equivalent in the position formulation. If we were to
implement one, it would use a double integral.

Since for the velocity controller, the proportional term is controlled by an
integral action and the derivative term is controlled by a proportional action,
the coefficients can be relabelled as follows.

\begin{theorem}[Velocity PID controller]
  \begin{equation}
    u = K_p \int_0^t e(\tau) d\tau + K_d e(t)
  \end{equation}
\end{theorem}

Integral control for the velocity is analogous to the throttle pedal on a car.
One must hold the throttle pedal (the control input) at a nonzero value to keep
the car traveling at the reference velocity.

Read \url{https://en.wikipedia.org/wiki/PID_controller} for more information on
PID control theory.

\section{PID control in terms of general control theory}

PID control defines \textit{setpoint} as the desired position and
\textit{process value} as the measured position. Control theory has more general
terms for these: \gls{reference} and \gls{output} respectively.

The derivative term is commonly used to "slow down" the system if it's already
heading toward the \gls{reference}. We will explore what $K_p$ and $K_d$ are
really doing for a two-state system (position and velocity) and why $K_d$ acts
that way.

First, we will rearrange the equation for a PD controller.

\begin{equation*}
  u = K_p e_k + K_d \frac{e_k - e_{k-1}}{dt}
\end{equation*}

where $u$ is the control input and $e_k$ is the error at timestep $k$. $e_k$ is
defined as $e_k = r_k - x_k$ where $r_k$ is the reference and $x_k$ is the
current state at timestep $k$.

\begin{align*}
  u &= K_p (r_k - x_k) + K_d \frac{(r_k - x_k) - (r_{k-1} - x_{k-1})}{dt} \\
  u &= K_p (r_k - x_k) + K_d \frac{r_k - x_k - r_{k-1} + x_{k-1}}{dt} \\
  u &= K_p (r_k - x_k) + K_d \frac{r_k - r_{k-1} - x_k + x_{k-1}}{dt} \\
  u &= K_p (r_k - x_k) + K_d \frac{(r_k - r_{k-1}) - (x_k - x_{k-1})}{dt} \\
  u &= K_p (r_k - x_k) + K_d \left(\frac{r_k - r_{k-1}}{dt} -
    \frac{x_k - x_{k-1}}{dt}\right)
\end{align*}

Notice how $\frac{r_k - r_{k-1}}{dt}$ is the velocity of the reference. By the
same reasoning, $\frac{x_k - x_{k-1}}{dt}$ is the system's velocity at a given
timestep. That means the $K_d$ term is driving the estimated velocity to the
reference velocity. If the reference is constant, that means the $K_d$ term is
trying to drive the velocity of the system to zero, but it can't because $K_p$
is trying to make the system move and $K_p$ and $K_d$ are controlling the same
actuator. If $K_p$ is larger than $K_d$, one is in effect slowing down the
response of the controller during transients with the hope of decreasing
overshoot and settling time. If one makes $K_d$ much larger than $K_p$, $K_d$
overpowers $K_p$ to bring the system to a stop. However, when the velocity is
low enough, $K_p$ is stronger and starts accelerating the system again. This
oscillatory behavior in the velocity repeats as the system moves toward the
reference.

\section{Limitations of PID control}

PID's heuristic method of tuning is fine when there is no knowledge of the
\gls{system}. However, controllers with much better response can be developed if
a \glslink{model}{dynamical model} of the \gls{system} is known.

\chapterimage{transfer-functions.jpg}{Road next to Stevenson Academic building at UCSC}

\chapter{Transfer functions}

\section{What is a transfer function?}

The Laplace domain is a two-dimensional coordinate system whose coordinates are
represented by a complex number $s = \sigma + j\omega$. The real part $\sigma$
cooresponds to the x-axis and the imaginary part $j\omega$ cooresponds to the
y-axis.

A transfer function maps an input coordinate to an output coordinate in the
Laplace domain. These can be obtained by applying the Laplace transform to a
differential equation and rearranging the terms to obtain a ratio of the output
variable to the input variable. Equation (\ref{eq:transfer_func}) is an example
of a transfer function.

\begin{equation} \label{eq:transfer_func}
  H(s) = \frac{(s-9+9i)(s-9-9i)}{s(s+10)}
\end{equation}

The factors of the numerator and denominator of a transfer function are called
residues. The roots of residues in the numerator are called zeroes while the
roots of residues in the denominator are called poles. We call them zeroes
because they make the residue approach zero, which makes the transfer function
also approach zero. Likewise, poles are called such because they make the
residue approach zero, which makes the transfer function approach infinity. On a
3D graph, they look like the poles of a circus tent.

Imaginary poles and zeroes always come in complex conjugate pairs (e.g.,
$-2 + 3i$, $-2 - 3i$).

\section{Transfer functions in feedback}

For \glspl{controller} to \glslink{regulator}{regulate} a system or
\glslink{tracking}{track} a reference, they must be placed in positive or
negative feedback with the \gls{plant} (whether to use positive or negative
depends on the \gls{plant} in question).

\begin{bookfigure}
  \begin{tikzpicture}[auto, >=latex']
    % Place the blocks
    \node [name=input] {$X(s)$};
    \node [sum, right=of input] (sum) {};
    \node [block, right=of sum] (K) {$K$};
    \node [block, right=of K] (G) {$G$};
    \node [right=of G] (output) {$Y(s)$};
    \node [block, below=of $(K)!0.5!(G)$] (H) {$H$};

    % Connect the nodes
    \draw [arrow] (input) -- node[pos=0.85] {$+$} (sum);
    \draw [arrow] (sum) -- node {} (K);
    \draw [arrow] (K) -- node {} (G);
    \draw [arrow] (G) -- node[name=y] {} (output);
    \draw [arrow] (y) |- (H);
    \draw [arrow] (H) -| node[pos=0.97, right] {$-$} (sum);
  \end{tikzpicture}

  \caption{Feedback controller block diagram}
  \label{fig:feedback_controller_block_diagram}
\end{bookfigure}

\begin{figurekey}
  \begin{tabulary}{\linewidth}{LLLL}
    $X(s)$ & input & $H$ & measurement transfer function \\
    $K$ & controller gain & $Y(s)$ & output \\
    $G$ & plant transfer function & & \\
  \end{tabulary}
\end{figurekey}

The transfer function of figure \ref{fig:feedback_controller_block_diagram}, a
control system diagram with feedback, from input to output is

\begin{equation}
  G_{cl}(s) = \frac{Y(s)}{X(s)} = \frac{KG}{1 + KGH}
\end{equation}

The numerator is the \gls{open-loop gain} and the denominator is one plus the
gain around the feedback loop, which may include parts of the
\gls{open-loop gain} (see appendix \ref{ch:app-tf-feedback-deriv} for a
derivation). As another example, the transfer function from the input to the
error is

\begin{equation}
  G_{cl}(s) = \frac{E(s)}{X(s)} = \frac{1}{1 + KGH}
\end{equation}

The roots of the denominator of $G_{cl}(s)$ are different from those of the
open-loop transfer function $KG(s)$. These are called the closed-loop poles.

\chapterimage{laplace-domain-analysis.jpg}{Grass clearing by Interdisciplinary Sciences building and Thimann Labs at UCSC}

\chapter{Laplace domain analysis}

This chapter briefly discusses what transfer functions are, how the locations of
poles and zeroes affects \gls{system response} and stability, and how
controllers affect pole locations. The case studies cover various aspects of PID
control using the algebraic approach of transfer functions.

\renewcommand*{\chapterpath}{\partpath/laplace-domain-analysis}
\section{The Fourier transform}

The Fourier transform decomposes a function of time into its component
frequencies. Each of these frequencies is part of what's called a
\textit{basis}. These waveforms can be added together in linear combinations to
produce the original signal. That is, we create a sum of waveforms that are
multiplied by their respective contribution amount.

Think of an F major 4 chord which has the notes $F_4$ ($349.23\,Hz$), $A_4$
($440\,Hz$), and $C_4$ ($261.63\,Hz$). The waveform over time looks like figure
\ref{fig:fourier_chord}.

\begin{svg}{build/code/fourier_chord}
  \caption{Frequency decomposition of Fmajor4 chord}
  \label{fig:fourier_chord}
\end{svg}

Notice how this complex waveform can be represented just by three frequencies.
They show up as Dirac delta functions\footnote{The Dirac delta function is zero
everywhere except near the origin. The nonzero region has an infinitesimal width
and a height such that the area within that region is $1$.} in the frequency
domain with the area underneath them equal to their contribution (see figure
\ref{fig:fourier_chord_fft}).

\begin{svg}{build/code/fourier_chord_fft}
  \caption{Fourier transform of Fmajor4 chord}
  \label{fig:fourier_chord_fft}
\end{svg}

In summary, the Fourier transform provides a way for us to determine, given some
signal, what frequencies can we add together and in what amounts to produce the
original signal.

\section{The Laplace transform}

The Laplace domain is a generalization of the frequency domain that has the
frequency ($j\omega$) on the imaginary y-axis and a real number on the x-axis,
yielding a two-dimensional coordinate system. We represent coordinates in this
space as a complex number $s = \sigma + j\omega$. The real part $\sigma$
cooresponds to the x-axis and the imaginary part $j\omega$ cooresponds to the
y-axis (see figure \ref{fig:laplace_domain}).

\begin{bookfigure}
  \begin{tikzpicture}[auto, >=latex']
    %\draw [help lines] (-4,-2) grid (4,4);

    % Draw main axes
    \draw[<->] (-4.2,1) -- (4.2,1) node[below] {\small Re($\sigma$)};
    \draw[<->] (0,-2) -- (0,4.2) node[right] {\small Im($j\omega$)};
  \end{tikzpicture}

  \caption{Laplace domain}
  \label{fig:laplace_domain}
\end{bookfigure}

To extend our analogy of each coordinate being represented by some basis, we now
have the y coordinate representing the oscillation frequency of the
\gls{system response} (the frequency domain) and also the x coordinate
representing the speed at which that oscillation decays and the \gls{system}
converges to zero. Figure \ref{fig:impulse_response_poles} shows this for
various points.

Note that this explanation as a basis isn't exact because the Laplace basis
isn't orthogonal (that is, the x and y coordinates affect each other and have
cross-talk). In the frequency domain, we had a basis of sine waves that we
represented as delta functions in the frequency domain. Each frequency
contribution was independent of the others. In the Laplace domain, this is not
the case; a pure exponential is $\frac{1}{s -a}$ (a rational function) instead
of a delta function. This function is nonzero at points that aren't actually
frequencies present in the time domain.

The Laplace transform of a function $f(t)$ is defined as

\begin{equation*}
  \mathcal{L}\{f(t)\} = F(s) = \int_0^\infty f(t) e^{-st} \,dt
\end{equation*}

We won't be computing any Laplace transforms by hand using this formula
(everyone in the real world looks these up in a table anyway). Common Laplace
transforms (assuming zero initial conditions) are shown in table
\ref{tab:common_laplace_transforms}. Of particular note are the Laplace
transforms for the derivative, unit step\footnote{The unit step $u(t)$ is
defined as $0$ for $t < 0$ and $1$ for $t \ge 0$.}, and exponential decay. We
can see that a derivative is equivalent to multiplying by $s$, and an integral
is equivalent to multiplying by $\frac{1}{s}$. We'll discuss the decaying
exponential shortly.

\begin{booktable}
  \begin{tabular}{|ccc|}
    \hline
    \rowcolor{headingbg}
    & \textbf{Time domain} & \textbf{Laplace domain} \\
    \hline
    Linearity & $a\,f(t) + b\,g(t)$ & $a\,F(s) + b\,G(s)$ \\
    Convolution & $(f * g)(t)$ & $F(s) \,G(s)$ \\
    Derivative & $f'(t)$ & $s \,F(s)$ \\
    $n^{th}$ derivative & $f^{(n)}(t)$ & $s^n \,F(s)$ \\
    Unit step & $u(t)$ & $\frac{1}{s}$ \\
    Ramp & $t \,u(t)$ & $\frac{1}{s^2}$ \\
    Exponential decay & $e^{-\alpha t} u(t)$ & $\frac{1}{s + \alpha}$ \\
    \hline
  \end{tabular}
  \caption{Common Laplace transforms and Laplace transform properties with zero
    initial conditions}
  \label{tab:common_laplace_transforms}
\end{booktable}

\chapterimage{transfer-functions.jpg}{Road next to Stevenson Academic building at UCSC}

\chapter{Transfer functions}

\section{What is a transfer function?}

The Laplace domain is a two-dimensional coordinate system whose coordinates are
represented by a complex number $s = \sigma + j\omega$. The real part $\sigma$
cooresponds to the x-axis and the imaginary part $j\omega$ cooresponds to the
y-axis.

A transfer function maps an input coordinate to an output coordinate in the
Laplace domain. These can be obtained by applying the Laplace transform to a
differential equation and rearranging the terms to obtain a ratio of the output
variable to the input variable. Equation (\ref{eq:transfer_func}) is an example
of a transfer function.

\begin{equation} \label{eq:transfer_func}
  H(s) = \frac{(s-9+9i)(s-9-9i)}{s(s+10)}
\end{equation}

The factors of the numerator and denominator of a transfer function are called
residues. The roots of residues in the numerator are called zeroes while the
roots of residues in the denominator are called poles. We call them zeroes
because they make the residue approach zero, which makes the transfer function
also approach zero. Likewise, poles are called such because they make the
residue approach zero, which makes the transfer function approach infinity. On a
3D graph, they look like the poles of a circus tent.

Imaginary poles and zeroes always come in complex conjugate pairs (e.g.,
$-2 + 3i$, $-2 - 3i$).

\section{Transfer functions in feedback}

For \glspl{controller} to \glslink{regulator}{regulate} a system or
\glslink{tracking}{track} a reference, they must be placed in positive or
negative feedback with the \gls{plant} (whether to use positive or negative
depends on the \gls{plant} in question).

\begin{bookfigure}
  \begin{tikzpicture}[auto, >=latex']
    % Place the blocks
    \node [name=input] {$X(s)$};
    \node [sum, right=of input] (sum) {};
    \node [block, right=of sum] (K) {$K$};
    \node [block, right=of K] (G) {$G$};
    \node [right=of G] (output) {$Y(s)$};
    \node [block, below=of $(K)!0.5!(G)$] (H) {$H$};

    % Connect the nodes
    \draw [arrow] (input) -- node[pos=0.85] {$+$} (sum);
    \draw [arrow] (sum) -- node {} (K);
    \draw [arrow] (K) -- node {} (G);
    \draw [arrow] (G) -- node[name=y] {} (output);
    \draw [arrow] (y) |- (H);
    \draw [arrow] (H) -| node[pos=0.97, right] {$-$} (sum);
  \end{tikzpicture}

  \caption{Feedback controller block diagram}
  \label{fig:feedback_controller_block_diagram}
\end{bookfigure}

\begin{figurekey}
  \begin{tabulary}{\linewidth}{LLLL}
    $X(s)$ & input & $H$ & measurement transfer function \\
    $K$ & controller gain & $Y(s)$ & output \\
    $G$ & plant transfer function & & \\
  \end{tabulary}
\end{figurekey}

The transfer function of figure \ref{fig:feedback_controller_block_diagram}, a
control system diagram with feedback, from input to output is

\begin{equation}
  G_{cl}(s) = \frac{Y(s)}{X(s)} = \frac{KG}{1 + KGH}
\end{equation}

The numerator is the \gls{open-loop gain} and the denominator is one plus the
gain around the feedback loop, which may include parts of the
\gls{open-loop gain} (see appendix \ref{ch:app-tf-feedback-deriv} for a
derivation). As another example, the transfer function from the input to the
error is

\begin{equation}
  G_{cl}(s) = \frac{E(s)}{X(s)} = \frac{1}{1 + KGH}
\end{equation}

The roots of the denominator of $G_{cl}(s)$ are different from those of the
open-loop transfer function $KG(s)$. These are called the closed-loop poles.

\section{Root locus} \label{sec:root_locus}
\index{Stability!root locus}

In closed-loop, the poles can be moved around by adjusting the controller gain,
but the zeroes stay put. The root locus shows where the poles will go as the
gain for a P controller is increased and tells us for what range of gains the
controller will be stable. As the controller gain is increased, poles can move
toward negative infinity (figure \ref{fig:rlocus_infty}), move toward each other
then split toward asymptotes (figure \ref{fig:rlocus_asymptotes}), or move
toward zeroes (figure \ref{fig:rlocus_zeroes}). The \gls{system} in figure
\ref{fig:rlocus_zeroes} becomes unstable as the gain is increased.

\begin{bookfigure}
  \begin{minisvg}{build/code/rlocus_infty}
    \caption{Root locus showing pole moving toward negative infinity}
    \label{fig:rlocus_infty}
  \end{minisvg}
  \hfill
  \begin{minisvg}{build/code/rlocus_asymptotes}
    \caption{Root locus showing poles moving toward asymptotes}
    \label{fig:rlocus_asymptotes}
  \end{minisvg}
  \hfill
  \begin{minisvg}{build/code/rlocus_zeroes}
    \caption{Root locus of equation (\ref{eq:transfer_func}) showing poles
      moving toward zeroes.}
    \label{fig:rlocus_zeroes}
  \end{minisvg}
\end{bookfigure}

We won't be using root locus plots for any of our control systems analysis
later, but it does help provide an intuition for what \glspl{controller}
actually do to a \gls{system}.

If poles are much farther left in the LHP than the typical \gls{system} dynamics
exhibit, they can be considered negligible. Every \gls{system} has some form of
unmodeled high frequency, nonlinear dynamics, but they can be safely ignored
depending on the operating regime.

To demonstrate this, consider the transfer function for a second-order DC
brushed motor from voltage to position

\begin{equation*}
  G(s) = \frac{K}{s((Js + b)(Ls + R) + K^2)}
\end{equation*}

where $J = 3.2284 \times 10^{-6}$ $kg$-$m^2$, $b = 3.5077 \times 10^{-6}$
$N$-$m$-$s$, $K_e = K_t = 0.0274 \,V/rad/s$, $R = 4 \,\Omega$, and
$L = 2.75 \times 10^{-6} \,H$.

This \gls{plant} has the root locus shown in figure
\ref{fig:highfreq_unstable_rlocus}. In proportional feedback, the \gls{plant} is
unstable for large values of $K$. However, if we remove the unstable pole by
setting $L$ in the transfer function to zero, we get the root locus in figure
\ref{fig:highfreq_stable_rlocus}. For small values of $K$, both \glspl{system}
are stable and have nearly indistinguishable \glspl{step response} due to the
exceedingly small contribution from the fast pole (see figures
\ref{fig:highfreq_unstable_step} and \ref{fig:highfreq_stable_step}). The high
frequency dynamics only cause instability for large values of $K$ that induce
fast \glspl{system response}. In other words, the \glspl{system response} of the
second-order model and its first-order approximation are similar for low
frequency operating regimes.

\begin{bookfigure}
  \begin{minisvg}{build/code/highfreq_unstable_rlocus}
    \caption{Root locus of second-order DC brushed motor plant}
    \label{fig:highfreq_unstable_rlocus}
  \end{minisvg}
  \hfill
  \begin{minisvg}{build/code/highfreq_stable_rlocus}
    \caption{Root locus of first-order DC brushed motor plant}
    \label{fig:highfreq_stable_rlocus}
  \end{minisvg}
\end{bookfigure}

\begin{bookfigure}
  \begin{minisvg}{build/code/highfreq_unstable_step}
    \caption{Step response of second-order DC brushed motor plant}
    \label{fig:highfreq_unstable_step}
  \end{minisvg}
  \hfill
  \begin{minisvg}{build/code/highfreq_stable_step}
    \caption{Step response of first-order DC brushed motor plant}
    \label{fig:highfreq_stable_step}
  \end{minisvg}
\end{bookfigure}

Why can't unstable poles close to the origin be ignored in the same way? The
response of high frequency stable poles decays rapidly. Unstable poles, on the
other hand, represent unstable dynamics which cause the \gls{system}
\gls{output} to grow to infinity. Regardless of how slow these unstable dynamics
are, they will eventually dominate the response.

\section{Gain margin and phase margin}
\label{sec:gain_phase_margin}
\index{Stability!gain margin}
\index{Stability!phase margin}

One generally needs to learn about Bode plots and Nyquist plots to truly
understand gain and phase margin and their origins, but those plots are large
topics unto themselves. Since we won't be using either of these plots for
controller design, we'll just cover what gain and phase margin are in a general
sense and how they are used. We'll be discussing how \gls{discretization}
affects phase margin in section \ref{sec:phase_loss}.

Gain margin and phase margin are two metrics for measuring a \gls{system}'s
relative stability. Gain and phase margin are the amounts by which the
closed-loop gain and phase can be varied respectively before the \gls{system}
becomes unstable. In a sense, they are safety margins for when unmodeled
dynamics affect the \gls{system response}.

For a more thorough explanation of gain and phase margin, watch Brian Douglas's
video on them \cite{bib:gain_phase_margin}. He has other videos too on classical
control methods like Bode and Nyquist plots that we recommend.

\section{Case studies of Laplace domain analysis}

We'll be using equation (\ref{eq:pid_tf}), the transfer function for a PID
controller, in the case studies.

\begin{equation}
  K(s) = K_p + \frac{K_i}{s} + K_ds \label{eq:pid_tf}
\end{equation}

Remember, multiplication by $\frac{1}{s}$ corresponds to an integral in the
Laplace domain and multiplication by $s$ corresponds to a derivative.

\subsection{Flywheel PID control} \label{subsec:flywheel-pid-control}
\index{PID control}

PID controllers typically control voltage to a motor in FRC independent of the
equations of motion of that motor. For position PID control, large values of
$K_p$ can lead to overshoot and $K_d$ is commonly used to reduce overshoots.
Let's consider a flywheel controlled with a standard PID controller. Why
wouldn't $K_d$ provide damping for velocity overshoots in this case?

PID control is designed to control second-order and first-order \glspl{system}
well. It can be used to control a lot of things, but struggles when given higher
order \glspl{system}. It has three degrees of freedom. Two are used to place the
two poles of the \gls{system}, and the third is used to remove steady-state
error. With higher order \glspl{system} like a one input, seven \gls{state}
\gls{system}, there aren't enough degrees of freedom to place the \gls{system}'s
poles in desired locations. This will result in poor control.

The math for PID doesn't assume voltage, a motor, etc. It defines an output
based on derivatives and integrals of its input. We happen to use it for motors
because it actually works pretty well for it because motors are second-order
\glspl{system}.

The following math will be in continuous time, but the same ideas apply to
discrete time. This is all assuming a velocity controller.

Our simple motor model hooked up to a mass is

\begin{align}
  V &= IR + \frac{\omega}{K_v} \label{eq:cs_flywheel_1} \\
  \tau &= I K_t \label{eq:cs_flywheel_2} \\
  \tau &= J \frac{d\omega}{dt} \label{eq:cs_flywheel_3}
\end{align}

For an explanation of where these equations come from, read section
\ref{sec:dc-brushed-motor}.

First, we'll solve for $\frac{d\omega}{dt}$ in terms of $V$.

Substitute equation (\ref{eq:cs_flywheel_2}) into equation
(\ref{eq:cs_flywheel_1}).

\begin{align*}
  V &= IR + \frac{\omega}{K_v} \\
  V &= \left(\frac{\tau}{K_t}\right) R + \frac{\omega}{K_v}
\end{align*}

Substitute in equation (\ref{eq:cs_flywheel_3}).

\begin{align*}
  V &= \frac{\left(J \frac{d\omega}{dt}\right)}{K_t} R + \frac{\omega}{K_v} \\
\end{align*}

Solve for $\frac{d\omega}{dt}$.

\begin{align*}
  V &= \frac{J \frac{d\omega}{dt}}{K_t} R + \frac{\omega}{K_v} \\
  V - \frac{\omega}{K_v} &= \frac{J \frac{d\omega}{dt}}{K_t} R \\
  \frac{d\omega}{dt} &= \frac{K_t}{JR} \left(V - \frac{\omega}{K_v}\right) \\
  \frac{d\omega}{dt} &= -\frac{K_t}{JRK_v} \omega + \frac{K_t}{JR} V
\end{align*}

Now take the Laplace transform.

\begin{equation}
  s \omega = -\frac{K_t}{JRK_v} \omega + \frac{K_t}{JR} V
  \label{eq:cs_motor_tf}
\end{equation}

Solve for the transfer function $H(s) = \frac{\omega}{V}$.

\begin{align*}
  s \omega &= -\frac{K_t}{JRK_v} \omega + \frac{K_t}{JR} V \\
  \left(s + \frac{K_t}{JRK_v}\right) \omega &= \frac{K_t}{JR} V \\
  \frac{\omega}{V} &= \frac{\frac{K_t}{JR}}{s + \frac{K_t}{JRK_v}} \\
\end{align*}

That gives us a pole at $-\frac{K_t}{JRK_v}$, which is actually stable. Notice
that there is only one pole.

First, we'll use a simple P loop.

\begin{equation*}
  V = K_p (\omega_{goal} - \omega)
\end{equation*}

Substitute this controller into equation (\ref{eq:cs_motor_tf}).

\begin{equation*}
  s \omega = -\frac{K_t}{JRK_v} \omega + \frac{K_t}{JR} K_p (\omega_{goal} -
    \omega)
\end{equation*}

Solve for the transfer function $H(s) = \frac{\omega}{\omega_{goal}}$.

\begin{align*}
  s \omega &= -\frac{K_t}{JRK_v} \omega + \frac{K_t K_p}{JR} (\omega_{goal} -
    \omega) \\
  s \omega &= -\frac{K_t}{JRK_v} \omega + \frac{K_t K_p}{JR} \omega_{goal} -
    \frac{K_t K_p}{JR} \omega \\
  \left(s + \frac{K_t}{JRK_v} + \frac{K_t K_p}{JR}\right) \omega &=
    \frac{K_t K_p}{JR} \omega_{goal} \\
  \frac{\omega}{\omega_{goal}} &= \frac{\frac{K_t K_p}{JR}}
    {\left(s + \frac{K_t}{JRK_v} + \frac{K_t K_p}{JR}\right)} \\
\end{align*}

This has a pole at $-\left(\frac{K_t}{JRK_v} + \frac{K_t K_p}{JR}\right)$.
Assuming that that quantity is negative (i.e., we are stable), that pole
corresponds to a time constant of
$\frac{1}{\frac{K_t}{JRK_v} + \frac{K_t K_p}{JR}}$.

As can be seen above, a flywheel has a single pole. It therefore only needs a
single pole controller to place all of its poles anywhere.

\begin{remark}
  This analysis assumes that the motor is well coupled to the mass and that the
  time constant of the inductor is small enough that it doesn't factor into the
  motor equations. In Austin Schuh's experience with 971's robots, these are
  pretty good assumptions.
\end{remark}

Next, we'll try a PD loop. (This will use a perfect derivative, but anyone
following along closely already knows that we can't really take a derivative
here, so the math will need to be updated at some point. We could switch to
discrete time and pick a differentiation method, or pick some other way of
modeling the derivative.)

\begin{equation*}
  V = K_p (\omega_{goal} - \omega) + K_d s (\omega_{goal} - \omega)
\end{equation*}

Substitute this controller into equation (\ref{eq:cs_motor_tf}).

\begin{align*}
  s \omega &= -\frac{K_t}{JRK_v} \omega + \frac{K_t}{JR}
    \left(K_p (\omega_{goal} - \omega) + K_d s (\omega_{goal} - \omega)\right)
    \\
  s \omega &= -\frac{K_t}{JRK_v} \omega + \frac{K_t K_p}{JR}
    (\omega_{goal} - \omega) + \frac{K_t K_d s}{JR} (\omega_{goal} - \omega) \\
  s \omega &= -\frac{K_t}{JRK_v} \omega + \frac{K_t K_p}{JR} \omega_{goal} -
    \frac{K_t K_p}{JR} \omega + \frac{K_t K_d s}{JR} \omega_{goal} -
    \frac{K_t K_d s}{JR} \omega \\
\end{align*}

Collect the common terms on separate sides and refactor.

\begin{align*}
  s \omega + \frac{K_t K_d s}{JR} \omega + \frac{K_t}{JRK_v} \omega +
    \frac{K_t K_p}{JR} \omega &= \frac{K_t K_p}{JR} \omega_{goal} +
    \frac{K_t K_d s}{JR} \omega_{goal} \\
  \left(s \left(1 + \frac{K_t K_d}{JR}\right) + \frac{K_t}{JRK_v} +
    \frac{K_t K_p}{JR}\right) \omega &= \frac{K_t}{JR}
    \left(K_p + K_d s\right) \omega_{goal} \\
  \frac{\omega}{\omega_{goal}} &= \frac{\frac{K_t}{JR}
    \left(K_p + K_d s\right)}{\left(s \left(1 + \frac{K_t K_d}{JR}\right) +
    \frac{K_t}{JRK_v} + \frac{K_t K_p}{JR}\right)} \\
\end{align*}

So, we added a zero at $-\frac{K_p}{K_d}$ and moved our pole to
$-\frac{\frac{K_t}{JRK_v} + \frac{K_t K_p}{JR}}{1 + \frac{K_t K_d}{JR}}$. This
isn't progress. We've added more complexity to our \gls{system} and, practically
speaking, gotten nothing good out of it. Zeroes should be avoided if at all
possible because they amplify unwanted high frequency modes of the \gls{system}.
At least this is a stable zero, but it's still undesirable.

In summary, derivative doesn't help on a flywheel. $K_d$ may help if the real
\gls{system} isn't ideal, but we don't suggest relying on that.

\subsection{Steady-state error}
\index{Steady-state error}

To demonstrate the problem of \gls{steady-state error}, we will use a DC brushed
motor controlled by a velocity PID controller. A DC brushed motor has a transfer
function from voltage ($V$) to angular velocity ($\dot{\theta}$) of

\begin{equation}
  G(s) = \frac{\dot{\Theta}(s)}{V(s)} = \frac{K}{(Js+b)(Ls+R)+K^2}
\end{equation}

First, we'll try controlling it with a P controller defined as

\begin{equation*}
  K(s) = K_p
\end{equation*}

When these are in unity feedback, the transfer function from the input voltage
to the error is

\begin{align*}
  \frac{E(s)}{V(s)} &= \frac{1}{1 + K(s)G(s)} \\
  E(s) &= \frac{1}{1 + K(s)G(s)} V(s) \\
  E(s) &= \frac{1}{1 + (K_p) \left(\frac{K}{(Js+b)(Ls+R)+K^2}\right)} V(s) \\
  E(s) &= \frac{1}{1 + \frac{K_p K}{(Js+b)(Ls+R)+K^2}} V(s)
\end{align*}

The steady-state of a transfer function can be found via

\begin{equation}
  \lim_{s\to0} sH(s)
\end{equation}

\begin{align}
  e_{ss} &= \lim_{s\to0} sE(s) \nonumber \\
  e_{ss} &= \lim_{s\to0} s \frac{1}{1 + \frac{K_p K}{(Js+b)(Ls+R)+K^2}} V(s)
    \nonumber \\
  e_{ss} &= \lim_{s\to0} s \frac{1}{1 + \frac{K_p K}{(Js+b)(Ls+R)+K^2}}
    \frac{1}{s} \nonumber \\
  e_{ss} &= \lim_{s\to0} \frac{1}{1 + \frac{K_p K}{(Js+b)(Ls+R)+K^2}}
    \nonumber \\
  e_{ss} &= \frac{1}{1 + \frac{K_p K}{(J(0)+b)(L(0)+R)+K^2}} \nonumber \\
  e_{ss} &= \frac{1}{1 + \frac{K_p K}{bR+K^2}} \label{eq:ss_nonzero}
\end{align}

Notice that the \gls{steady-state error} is nonzero. To fix this, an integrator
must be included in the controller.

\begin{equation*}
  K(s) = K_p + \frac{K_i}{s}
\end{equation*}

The same steady-state calculations are performed as before with the new
controller.

\begin{align*}
  \frac{E(s)}{V(s)} &= \frac{1}{1 + K(s)G(s)} \\
  E(s) &= \frac{1}{1 + K(s)G(s)} V(s) \\
  E(s) &= \frac{1}{1 + \left(K_p + \frac{K_i}{s}\right)
    \left(\frac{K}{(Js+b)(Ls+R)+K^2}\right)} \left(\frac{1}{s}\right) \\
  e_{ss} &= \lim_{s\to0} s \frac{1}{1 + \left(K_p + \frac{K_i}{s}\right)
    \left(\frac{K}{(Js+b)(Ls+R)+K^2}\right)} \left(\frac{1}{s}\right) \\
  e_{ss} &= \lim_{s\to0} \frac{1}{1 + \left(K_p + \frac{K_i}{s}\right)
    \left(\frac{K}{(Js+b)(Ls+R)+K^2}\right)} \\
  e_{ss} &= \lim_{s\to0} \frac{1}{1 + \left(K_p + \frac{K_i}{s}\right)
    \left(\frac{K}{(Js+b)(Ls+R)+K^2}\right)} \frac{s}{s} \\
  e_{ss} &= \lim_{s\to0} \frac{s}{s + \left(K_p s + K_i\right)
    \left(\frac{K}{(Js+b)(Ls+R)+K^2}\right)} \\
  e_{ss} &= \frac{0}{0 + (K_p (0) + K_i)
    \left(\frac{K}{(J(0)+b)(L(0)+R)+K^2}\right)} \\
  e_{ss} &= \frac{0}{K_i \frac{K}{bR+K^2}}
\end{align*}

The denominator is nonzero, so $e_{ss} = 0$. Therefore, an integrator is
required to eliminate \gls{steady-state error} in all cases for this
\gls{model}.

It should be noted that $e_{ss}$ in equation (\ref{eq:ss_nonzero}) approaches
zero for $K_p = \infty$. This is known as a bang-bang controller. In practice,
an infinite switching frequency cannot be achieved, but it may be close enough
for some performance specifications.

\subsection{Actuator saturation}
\index{Controller design!actuator saturation}

Recall that a controller calculates its output based on the error between the
\gls{reference} and the current \gls{state}. \Gls{plant} in the real world don't
have unlimited control authority available for the controller to apply. When the
actuator limits are reached, the controller acts as if the gain has been
temporarily reduced.

We'll try to explain this through a bit of math. Let's say we have a controller
$u = k(r - x)$ where $u$ is the \gls{control effort}, $k$ is the gain, $r$ is
the \gls{reference}, and $x$ is the current \gls{state}. Let $u_{max}$ be the
limit of the actuator's output which is less than the uncapped value of $u$ and
$k_{max}$ be the associated maximum gain. We will now compare the capped and
uncapped controllers for the same \gls{reference} and current \gls{state}.

\begin{align*}
  u_{max} &< u \\
  k_{max}(r - x) &< k(r - x) \\
  k_{max} &< k
\end{align*}

For the inequality to hold, $k_{max}$ must be less than the original value for
$k$. This reduced gain is evident in a \gls{system response} when there is a
linear change in state instead of an exponential one as it approaches the
\gls{reference}. This is due to the \gls{control effort} no longer following a
decaying exponential plot. Once the \gls{system} is closer to the
\gls{reference}, the controller will stop saturating and produce realistic
controller values again.


\part{Modern control}
\chapterimage{kinematics-and-dynamics.jpg}{Hills by freeway between Santa Maria and Ventura}

\chapter{Linear algebra}

Modern control theory borrows concepts from linear algebra. At first, linear
algebra may appear very abstract, but there are simple geometric intuitions
underlying it. First, watch 3Blue1Brown's preview video for the
\textit{Essence of linear algebra} video series (5 minutes)
\cite{bib:linalg_preview}. The goal here is to provide an intuitive, geometric
understanding of linear algebra as a method of linear transformations.

While only a subset of the material from the videos will be presented here that
we think is relevant to this book, we highly suggest watching the whole series
\cite{bib:essence_of_linalg}.

\begin{remark}
  The following sections are essentially transcripts of the video content for
  those who don't want to watch an hour of YouTube videos. However, we suggest
  watching the videos instead because animations are better at conveying the
  geometric intuition involved than text.
\end{remark}

\renewcommand*{\chapterpath}{\partpath/linear-algebra}
\section{Vectors}

\subsection{What is a vector?}

The fundamental building block for linear algebra is the vector. Broadly
speaking, there are three distinct but related ideas about vectors: the physics
student perspective, the computer science student perspective, and the
mathmetician's perspective.

The physics student perspective is that vectors are arrows pointing in space. A
given vector is defined by its length and direction, but as long as those two
facts are the same, you can move it around and it's still the same vector.
Vectors in the flat plane are two-dimensional, and those sitting in broader
space that we live in are three-dimensional.

The computer science perspective is that vectors are ordered lists of numbers.
For example, let's say you were doing some analytics about house prices and the
only features you cared about where square footage and price. You might model
each house with a pair of numbers where the first indicates square footage and
the second indicates price. Notice the order matters here. In the lingo, you'd
be modeling houses as two-dimensional vectors where, in this context, vector is
a synonym for list, and what makes it two-dimensional is that the length of that
list is two.

The mathematician, on the other hand, seeks to generalize both these views by
saying that a vector can be anything where there's a sensible notion of adding
two vectors and multiplying a vector by a number (operations that we'll talk
about later on). The details of this view are rather abstract and won't be
needed for this book as we'll favor a more concrete setting. We bring it up here
because it hints at the fact that the ideas of vector addition and
multiplication by numbers will play an important role throughout linear algebra.

\subsection{Geometric interpretation of vectors}

Before we talk about vector addition and multiplication by numbers, let's settle
on a specific thought to have in mind when we say the word ``vector". Given the
geometric focus that we're intending here, whenever we introduce a new topic
involving vectors, we want you to first think about an arrow, and specifically,
think about that arrow inside a coordinate system, like the x-y plane with its
tail sitting at the origin. This is slightly different from the physics student
perspective where vectors can freely sit anywhere they want in space. In linear
algebra, it's almost always the case that your vector will be rooted at the
origin. Then, once you understand a new concept in the concept of arrows in
space, we'll translate it over to the ``list of numbers" point of view, which we
can do by considering the coordinates of the vector.

While you may already be familiar with this coordinate system, it's worth
walking through explicitly since this is where all of the important
back-and-forth happens between the two perspectives of linear algebra. Focusing
your attention on two dimensions for the moment, you have a horizontal line
called the x-axis and a vertical line called the y-axis. The point at which they
intersect is called the origin, which you should think of as the center of
space and the root of all vectors. After choosing an arbitrary length to
represent one, you make tick marks on each axis to represent this distance. The
coordinates of a vector is a pair of numbers that essentially gives instructions
for getting from the tail of that vector at the origin to its tip. The first
number is how far to move along the x-axis (positive numbers indicating
rightward motion and negative numbers indicating leftward motion), and the
second number is how far to move parallel to the y-axis after that (positive
numbers indicating upward motion and negative numbers indicating downward
motion). To distinguish vectors from points, the convention is to write this
pair of numbers vertically with square brackets around them. For example:

\begin{equation*}
  \begin{bmatrix}
    3 \\
    -1
  \end{bmatrix}
\end{equation*}

Every pair of numbers represents one and only one vector, and every vector is
associated with one and only one pair of numbers. In three dimensions, there is
a third axis called the z-axis which is perpendicular to both the x and y axes.
In this case, each vector is associated with an ordered triplet of numbers. The
first is how far to move along the x-axis, the second is how far to move
parallel to the y-axis, and the third is how far to then move parallel to this
new z-axis. For example:

\begin{equation*}
  \begin{bmatrix}
    2 \\
    1 \\
    3
  \end{bmatrix}
\end{equation*}

Every triplet of numbers represents one unique vector in space, and every vector
in space represents exactly one triplet of numbers.

\subsection{Vector addition}

Back to vector addition and multiplication by numbers. Afterall, every topic in
linear algebra is going to center around these two operations. Luckily, each
one is straightforward to define. Let's say we have two vectors, one pointing up
and a little to the right, and the other one pointing right and down a bit. To
add these two vectors, move the second one so that its tail sits at the tip of
the first one. Then, if you draw a new vector from the tail of the first one to
where the tip of the second one now sits, that new vector is their sum.

This definition of addition, by the way, is one of the only times in linear
algebra where we let vectors stray away from the origin, but why is this a
reasonable thing to do? Why this definition of addition and not some other one?
Each vector represents a sort of movement, a step with a certain distance and
direction in space. If you take a step along the first vector, then take a step
in the direction and distance described by the second vector, the overall effect
is just the same as if you moved along the sum of those two vectors to start
with.

You could think about this as an extension of how we think about adding numbers
on a number line. One way that we teach students to think about this, say with
$2 + 5$, is to think of moving two steps to the right, followed by another 5
steps to the right. The overall effect is the same as if you just took 7 steps
to the right. In fact, let's see how vector addition looks numerically. The
first vector here has coordinates $(1, 2)$ and the second has coordinates
$(3, -1)$.

\begin{equation*}
  \begin{bmatrix}
    1 \\
    2
  \end{bmatrix} + \begin{bmatrix}
    3 \\
    -1
  \end{bmatrix}
\end{equation*}

When you take the vector sum using this tip-to-tail method, you can think of a
four-step path from the origin to the tip of the second vector: ``walk 1 to the
right, then 2 up, then 3 to the right, then 1 down." Reorganizing these steps so
that you first do all of the rightward motion, then do all of the vertical
motion, you can read it as saying, ``first move $1 + 3$ to the right, then move
$2 + (-1)$ up," so the new vector has coordinates $1 + 3$ and $2 + (-1)$.

\begin{equation*}
  \begin{bmatrix}
    1 \\
    2
  \end{bmatrix} + \begin{bmatrix}
    3 \\
    -1
  \end{bmatrix} = \begin{bmatrix}
    1 + 3 \\
    2 + (-1)
  \end{bmatrix}
\end{equation*}

In general, vector addition in this list-of-numbers conception looks like
matching up their terms, and adding each one together.

\begin{equation*}
  \begin{bmatrix}
    x_1 \\
    y_1
  \end{bmatrix} + \begin{bmatrix}
    x_2 \\
    y_2
  \end{bmatrix} = \begin{bmatrix}
    x_1 + x_2 \\
    y_1 + y_2
  \end{bmatrix}
\end{equation*}

\subsection{Scalar-vector multiplication}

The other fundamental vector operation is multiplication by a number. Now this
is best understood just by looking at a few examples. If you take the number
$2$, and multiply it by a given vector, you stretch out that vector so that it's
two times as long as when you started. If you multiply that vector by, say,
$\frac{1}{3}$, you compress it down so that it's $\frac{1}{3}$ of the original
length. When you multiply it by a negative number, like $-1.8$, then the vector
is first flipped around, then stretched out by that factor of $1.8$.

This process of stretching, compressing, or reversing the direction of a vector
is called ``scaling", and whenver a number like $2$ or $\frac{1}{3}$ or $-1.8$
acting like this--scaling some vector--we call it a ``scalar". In fact,
throughout linear algebra, one of the main things numbers do is scale vectors,
so it's common to use the word ``scalar" interchangeably with the word
``number". Numerically, stretching out a vector by a factor of, say, $2$,
corresponds to multiplying each of its components by that factor, $2$.

\begin{equation*}
  2 \cdot \begin{bmatrix}
    3 \\
    1
  \end{bmatrix} = \begin{bmatrix}
    6 \\
    2
  \end{bmatrix}
\end{equation*}

So in the conception of vectors as lists of numbers, multiplying a given vector
by a scalar means multiplying each one of those components by that scalar.

\begin{equation*}
  2 \cdot \begin{bmatrix}
    x \\
    y
  \end{bmatrix} = \begin{bmatrix}
    2x \\
    2y
  \end{bmatrix}
\end{equation*}

\begin{remark}
  See the corresponding \textit{Essence of Linear Algebra} video for a more
  visual presentation (5 minutes) \cite{bib:linalg_vectors}.
\end{remark}

\section{Linear combinations, span, and basis vectors}

Vector coordinates were probably already familiar to you, but there's another
interesting way to think about these coordinates which is central to linear
algebra. When given a pair of numbers that's meant to describe a vector, like
$(3, 2)$, we want you to think about each coordinate as a scalar, meaning, think
about how each one stretches or compresses vectors.

\subsection{Basis vectors}
\index{Linear algebra!basis vectors}

In the xy-coordinate system, there are two special vectors: the one pointing to
the right with length $1$, commonly called ``i-hat", or the unit vector in the
x-direction ($\hat{i}$), and the one pointing straight up, with length $1$,
commonly called ``j-hat", or the unit vector in the y-direction ($\hat{j}$).

Now think of the x-coordinate of our vector as a scalar that scales $\hat{i}$,
stretching it by a factor of $3$, and the y-coordinate as a scalar that scales
$\hat{j}$, flipping it and stretching it by a factor of $2$. In this sense, the
vectors that these coordinates describe is the sum of two scaled vectors
$(3)\hat{i} + (-2)\hat{j}$. This idea of adding together two scaled vectors is a
surprisingly important concept. Those two vectors, $\hat{i}$ and $\hat{j}$, have
a special name, by the way. Together they're called the \textit{basis} of a
coordinate system ($\hat{i}$ and $\hat{j}$ are the ``basis vectors" of the
xy-coordinate system). When you think about coordinates as scalars, the basis
vectors are what those scalars actually scale.

By framing our coordinate system in terms of these two special basis vectors, it
raises an interesting and subtle point: we could have chosen different basis
vectors and had a completely reasonable, new coordinate system system. For
example, take some vector pointing up and to the right, along with some other
vector pointing down and to the right, in some way. Take a moment to think about
all the different vectors that you can get by choosing two scalars, using each
one to scale one of the vectors, then adding together what you get. Which
two-dimensional vectors can you reach by altering the choices of scalars? The
answer is that you can reach every possible two-dimensional vector. A new pair
of basis vectors like this still gives us a valid way to go back and forth
between pairs of numbers and two-dimensional vectors, but the association is
definitely different from the one that you get using the more standard basis of
$\hat{i}$ and $\hat{j}$.

\subsection{Linear combination}
\index{Linear algebra!linear combination}

Any time we describe vectors numerically, it depends on an implicit choice of
what basis vectors we're using. So any time that you're scaling two vectors and
adding them like this, it's called a \textit{linear combination} of those two
vectors. Below is a linear combination of vectors $\vec{v}$ and $\vec{w}$ with
scalars $a$ and $b$.

\begin{equation*}
  a \vec{v} + b \vec{w}
\end{equation*}

Where does this word ``linear" come from? Why does this have anything to do with
lines? This isn't the etymology, but if you fix one of those scalars and let the
other one change its value freely, the tip of the resulting vector draws a
straight line.

\subsection{Span}

Now, if you let both scalars range freely and consider every possible resultant
vector, there are three things that can happen. For most pairs of vectors,
you'll be able to reach every possible point in the plane; every two-dimensional
vector is within your grasp. However, in the unlucky case where your two original
vectors happen to line up, the tip of the resulting vector is limited to just a
single line passing through the origin. The vectors could also both be zero, in
which case the resultant vector is just at the origin.

The set of all possible vectors that you can reach with a linear combination of
a given pair of vectors is called the \textit{span} of those two vectors. So,
restating what we just discussed in this lingo, the span of most pairs of 2D
vectors is all vectors in 2D space, but when they line up, their span is all
vectors whose tip sits on a certain line.

Remember how we said that linear algebra revolves around vector addition and
scalar multiplication? The span of two vectors is a way of asking, ``What are
all the possible vectors one can reach using only these two fundamental
operations, vector addition and scalar multiplication?"

Thinking about a whole collection of vectors sitting on a line gets crowded, and
even more so to think about all two-dimensional vectors at once filling up the
plane. When dealing with collections of vectors like this, it's common to
represent each one with just a point in space where the tip of the vector was.
This way, if you want to think about every possible vector whose tip sits on a
certain line, just think about the line itself.

Likewise, to think about all possible two-dimensional vectors at once,
conceptualize each one as the point where its tip sits. In effect, you're
thinking about the infinite, flat sheet of two-dimensional space itself, leaving
the arrows out of it.

In general, if you're thinking about a vector on its own, think of it as an
arrow, and if you're dealing with a collection of vectors, it's convenient to
think of them all as points. Therefore, for our span example, the span of most
pairs of vectors ends up being the entire infinite sheet of two-dimensional
space, but if they line up, their span is just a line.

The idea of span gets more interesting if we start thinking about vectors in
three-dimensional space. For example, given two vectors in 3D space that are not
pointing in the same direction, what does it mean to take their span? Their span
is the collection of all possible linear combinations of those two vectors,
meaning all possible vectors you get by scaling each vector in some way, then
adding them together.

You can imagine turning two different knobs to change the two scalars defining
the linear combination, adding the scaled vectors and following the tip of the
resulting vector. That tip will trace out a flat sheet cutting through the
origin of three-dimensional space. This flat sheet is the span of the two
vectors. More precisely, the span of the two vectors is the set of all possible
vectors whose tips sit on that flat sheet.

So what happens if we add a third vector and consider the span of all three? A
linear combination of three vectors is defined similarly as it is for two;
you'll choose three different scalars, scale each of those vectors, then add
them all together. The linear combination of $\vec{v}$, $\vec{w}$, and $\vec{u}$
looks like

\begin{equation*}
  a\vec{v} + b\vec{w} + c\vec{u}
\end{equation*}

where $a$, $b$, and $c$ are allowed to vary. Again, the span of these vectors is
the set of all possible linear combinations.

Two different things could happen here. If your third vector happens to be
sitting on the span of the first two, then the span doesn't change; you're
trapped on that same flat sheet. In other words, adding a scaled version of that
third vector to the linear combination doesn't give you access to any new
vectors. However, if you just randomly choose a third vector, it's almost
certainly not sitting on the span of those first two. Then, since it's pointing
in a separate direction, it unlocks access to every possible three-dimensional
vector. As you scale that new third vector, it moves around that span sheet of
the first two, sweeping it through all of space.

Another way to think about it is that you're making full use of the three,
freely-changing scalars that you have at your disposal to acces the full three
dimensions of space.

\subsection{Linear dependence and independence}

In the case where the third vector was already sitting on the span of the first
two, or the case where two vectors happen to line up, we want some terminology
to describe the fact that at least one of these vectors is redundant--not adding
anything to our span. When there are multiple vectors and one could be removed
without reducing the span, the relevant terminology is to say that they are
\textit{linearly dependent}.

In other words, one of the vectors can be expressed as a linear combination of
the others since it's already in the span of the others.

\begin{equation*}
  \vec{u} = a\vec{v} + b\vec{w} \text{ for some values of $a$ and $b$}
\end{equation*}

On the other hand, if each vector really does add another dimension to the span,
they're said to be \textit{linearly independent}.

\begin{equation*}
  \vec{w} \neq a\vec{v} \text{ for all values of $a$}
\end{equation*}

Now with all that terminology, and hopefully some good mental images to go with
it, the technical definition of a basis of a space is as follows.

\begin{definition}[Basis of a vector space]
  The \textit{basis} of a vector space is a set of \textit{linearly independent}
  vectors that \textit{span} the full space.
\end{definition}

\begin{remark}
  See the corresponding \textit{Essence of linear algebra} video for a more
  visual presentation (10 minutes) \cite{bib:linalg_linear_combinations}.
\end{remark}

\section{Linear transformations and matrices}

This section focuses on what linear transformations look like in the case of two
dimensions and how they relate to the idea of a matrix-vector multiplication.

\begin{equation*}
  \begin{bmatrix}
    \textcolor{red}{1} & \textcolor{orange}{-3} \\
    \textcolor{green}{2} & \textcolor{cyan}{4}
  \end{bmatrix}
  \begin{bmatrix}
    \textcolor{blue}{5} \\
    \textcolor{purple}{7}
  \end{bmatrix} = \begin{bmatrix}
    (\textcolor{red}{1})(\textcolor{blue}{5}) +
      (\textcolor{orange}{-3})(\textcolor{purple}{7}) \\
    (\textcolor{green}{2})(\textcolor{blue}{5}) +
      (\textcolor{cyan}{4})(\textcolor{purple}{7})
  \end{bmatrix}
\end{equation*}

In particular, we want to show you a way to think about matrix-vector
multiplication that doesn't rely on memorization of the procedure shown above.

\subsection{What is a linear transformation?}

To start, let's just parse this term ``linear transformation". ``Transformation"
is essentially another name for ``function". It's something that takes in inputs
and returns an output for each one. Specifically in the context of linear
algebra, we consider transformations that take in some vector and spit out
another vector.

\begin{figure}[H]
  \centering

  \begin{tikzpicture}[auto, >=latex']
    % Place the nodes
    \node [name=input] {
      $\begin{bmatrix}
        5 \\
        7
      \end{bmatrix}$
    };
    \node [name=inputlabel, below=of input] {Vector input};
    \node [name=func, right=of input] {$L(\vec{v})$};
    \node [name=output, right=of func] {
      $\begin{bmatrix}
        2 \\
        -3
      \end{bmatrix}$
    };
    \node [name=outputlabel, below=of output] {Vector output};

    % Connect the nodes
    \draw [arrow] (input) -- node {} (func);
    \draw [arrow] (func) -- node {} (output);
  \end{tikzpicture}
\end{figure}

So why use the word ``transformation" instead of ``function" if they mean the
same thing? It's to be suggestive of a certain way to visualize this
input-output relation. You see, a great way to understand functions of vectors
is to use movement. If a transformation takes some input vector to some output
vector, we imagine that input vector moving over to the output vector. Then to
understand the transformation as a whole, we might imagine watching every
possible input vector move over to its corresponding output vector. it gets
really crowded to think about all the vectors all at once, where each one is an
arrow. Therefore, as we mentioned in the previous section, it's useful to
conceptualize each vector as a single point where its tip sits rather than an
arrow. To think about a transformation taking every possible input vector to
some output vector, we watch every point in space moving to some other point.

The effect of various transformations moving around all of the points in space
gives the feeling of compressing and morphing space itself. As you can imagine
though, arbitrary transformations can look complicated. Luckily, linear algebra
limits itself to a special type of transformation, ones that are easier to
understand, called ``linear" transformations. Visually speaking, a
transformation is linear if it has two properties: all straight lines must
remain as such, and the origin must remain fixed in place. In general, you
should think of linear transformations as keeping grid lines parallel and evenly
spaced.

\subsection{Describing transformations numerically}

Some transformations are simple to think about, like rotations about the origin.
Others are more difficult to describe with words. So how could one describe
these transformations numerically? If you were, say, programming some animations
to make a video teaching the topic, what formula could you give the computer so
that if you give it the coordinates of a vector, it would return the coordinates
of where that vectors lands?

\begin{figure}[H]
  \centering

  \begin{tikzpicture}[auto, >=latex']
    % Place the nodes
    \node [name=input] {
      $\begin{bmatrix}
        x_{in} \\
        y_{in}
      \end{bmatrix}$
    };
    \node [name=func, right=of input] {$????$};
    \node [name=output, right=of func] {
      $\begin{bmatrix}
        x_{out} \\
        y_{out}
      \end{bmatrix}$
    };

    % Connect the nodes
    \draw [arrow] (input) -- node {} (func);
    \draw [arrow] (func) -- node {} (output);
  \end{tikzpicture}
\end{figure}

You only need to record where the two basis vectors, $\hat{i}$ and $\hat{j}$,
each land, and everything else will follow from that. For example, consider the
vector $v$ with coordinates $(-1, 2)$, meaning $\vec{v} = -1\hat{i} + 2\hat{j}$.
If we play some transformation and follow where all three of these vectors go,
the property that grid lines remain parallel and evenly spaced has a really
important consequence: the place where $\vec{v}$ lands will be $-1$ times the
vector where $\hat{i}$ landed plus $2$ times the vector where $\hat{j}$ landed.
In other words, it started off as a certain linear combination of $\hat{i}$ and
$\hat{j}$ and it ends up as that same linear combination of where those two
vectors landed. This means you can deduce where $\vec{v}$ must go based only on
where $\hat{i}$ and $\hat{j}$ each land. For this transformation, $\hat{i}$
lands on the coordinates $(1, -2)$ and $\hat{j}$ lands on the x-axis at the
coordinates $(3, 0)$.

\begin{align*}
  \text{Transformed } \vec{v} &= -1(\text{Transformed } \hat{i}) +
    2(\text{Transformed } \hat{j}) \\
  \text{Transformed } \vec{v} &= -1\begin{bmatrix}
    1 \\
    -2
  \end{bmatrix} + 2\begin{bmatrix}
    3 \\
    0
  \end{bmatrix}
\end{align*}

Adding that all together, you can deduce that $\vec{v}$ has to land on the
vector $(5, 2)$.

\begin{align*}
  \text{Transformed } \vec{v} &= \begin{bmatrix}
    -1(1) + 2(3) \\
    -1(-2) + 2(0)
  \end{bmatrix} \\
  \text{Transformed } \vec{v} &= \begin{bmatrix}
    5 \\
    2
  \end{bmatrix}
\end{align*}

This is a good point to pause and ponder, because it's pretty important. This
gives us a technique to deduce where any vectors land, so long as we have a
record of where $\hat{i}$ and $\hat{j}$ each land, without needing to watch the
transformation itself.

Given a vector with more general coordinates $x$ and $y$, it will land on $x$
times the vector where $\hat{i}$ lands $(1, -2)$, plus $y$ times the vector
where $\hat{j}$ lands $(3, 0)$. Carrying out that sum, you see that it lands at
$(1x + 3y, -2x + 0y)$.

\begin{equation*}
  \begin{array}{cc}
    \hat{i} \rightarrow \begin{bmatrix}
      1 \\
      -2
    \end{bmatrix} &
    \hat{j} \rightarrow \begin{bmatrix}
      3 \\
      0
    \end{bmatrix}
  \end{array}
\end{equation*}

\begin{equation*}
  \begin{bmatrix}
    x \\
    y
  \end{bmatrix} \rightarrow x\begin{bmatrix}
    1 \\
    -2
  \end{bmatrix} + y\begin{bmatrix}
    3 \\
    0
  \end{bmatrix} = \begin{bmatrix}
    1x + 3y \\
    -2x + 0y
  \end{bmatrix}
\end{equation*}

Given any vector, this formula will describe where that vector lands.

What all of this is saying is that a two dimensional linear transformation is
completely described by just four numbers: the two coordinates for where
$\hat{i}$ lands and the two coordinates for where $\hat{j}$ lands. It's common
to package these coordinates into a two-by-two grid of numbers, called a
two-by-two matrix, where you can interpret the columns as the two special
vectors where $\hat{i}$ and $\hat{j}$ each land. If $\hat{i}$ lands on the
vector $(3, -2)$ and $\hat{j}$ lands on the vector $(2, 1)$, this two-by-two
matrix would be

\begin{equation*}
  \begin{bmatrix}
    3 & 2 \\
    -2 & 1
  \end{bmatrix}
\end{equation*}

If you're given a two-by-two matrix describing a linear transformation and some
specific vector, say $(5, 7)$, and you want to know where that linear
transformation takes that vector, you can multiply the coordinates of the vector
by the corresponding columns of the matrix, then add together the result.

\begin{equation*}
  \begin{bmatrix}
    3 & 2 \\
    -2 & 1
  \end{bmatrix}
  \begin{bmatrix}
    5 \\
    7
  \end{bmatrix} = 5\begin{bmatrix}
    3 \\
    -2
  \end{bmatrix} + 7\begin{bmatrix}
    2 \\
    1
  \end{bmatrix}
\end{equation*}

This corresponds with the idea of adding the scaled versions of our new basis
vectors.

Let's see what this looks like in the most general case where your matrix has
entries $a$, $b$, $c$, $d$.

\begin{equation*}
  \begin{bmatrix}
    a & b \\
    c & d
  \end{bmatrix}
\end{equation*}

Remember, this matrix is just a way of packaging the information needed to
describe a linear transformation. Always remember to interpret that first
column, $(a, c)$, as the place where the first basis vector lands and that
second column, $(b, d)$, as the place where the second basis vector lands.

When we apply this transformation to some vector $(x, y)$, the result will be
$x$ times $(a, c)$ plus $y$ times $(b, d)$. Together, this gives a vector
$(ax + by, cx + dy)$.

\begin{equation*}
  \begin{bmatrix}
    a & b \\
    c & d
  \end{bmatrix} \begin{bmatrix}
    x \\
    y
  \end{bmatrix} = x\begin{bmatrix}
    a \\
    c
  \end{bmatrix} + y\begin{bmatrix}
    b \\
    d
  \end{bmatrix} = \begin{bmatrix}
    ax + by \\
    cx + dy
  \end{bmatrix}
\end{equation*}

You could even define this as matrix-vector multiplication when you put the
matrix on the left of the vector like it's a function. Then, you could make high
schoolers memorize this, without showing them the crucial part that makes it
feel intuitive (yes, that was sarcasm). Isn't it more fun to think about these
columns as the transformed versions of your basis vectors and to think about the
result as the appropriate linear combination of those vectors?

\subsection{Examples of linear transformations}

Let's practice describing a few linear transformations with matrices. For
example, if we rotate all of space $90\degree$ counterclockwise then $\hat{i}$
lands on the coordinates $(0, 1)$ and $\hat{j}$ lands on the coordinates
$(-1, 0)$. So the matrix we end up with has the columns $(0, 1)$, $(-1, 0)$.

\begin{equation*}
  \begin{bmatrix}
    0 & -1 \\
    1 & 0
  \end{bmatrix}
\end{equation*}

To ascertain what happens to any vector after a $90\degree$ rotation, you could
just multiply its coordinates by this matrix.

\begin{equation*}
  \begin{bmatrix}
    0 & -1 \\
    1 & 0
  \end{bmatrix} \begin{bmatrix}
    x \\
    y
  \end{bmatrix}
\end{equation*}

Here's a fun transformation with a special name, called a ``shear". In it,
$\hat{i}$ remains fixed so the first column of the matrix is $(1, 0)$, but
$\hat{j}$ moves over to the coordinates $(1, 1)$ which become the second column
of the matrix.

\begin{equation*}
  \begin{bmatrix}
    1 & 1 \\
    0 & 1
  \end{bmatrix}
\end{equation*}

And, at the risk of being redundant here, figuring out how shear transforms a
given vector comes down to multiplying this matrix by that vector.

\begin{equation*}
  \begin{bmatrix}
    1 & 1 \\
    0 & 1
  \end{bmatrix} \begin{bmatrix}
    x \\
    y
  \end{bmatrix}
\end{equation*}

Let's say we want to go the other way around, starting with a matrix, say with
columns $(1, 2)$ and $(3, 1)$, and we want to deduce what its transformation
looks like. Pause and take a moment to see if you can imagine it.

\begin{equation*}
  \begin{bmatrix}
    1 & 3 \\
    2 & 1
  \end{bmatrix}
\end{equation*}

One way to do this is to first move $\hat{i}$ to $(1, 2)$. Then, move $\hat{j}$
to $(3, 1)$, always moving the rest of space in such a way that that keeps grid
lines parallel and evenly spaced.

Suppose that the vectors that $\hat{i}$ and $\hat{j}$ land on are linearly
dependent as in the following matrix (that is, it has linearly dependent
columns).

\begin{equation*}
  \begin{bmatrix}
    2 & -2 \\
    1 & -1
  \end{bmatrix}
\end{equation*}

If you recall from last section, this means that one vector is a scaled version
of the other, so that linear transformation compresses all of 2D space onto the
line where those two vectors sit. This is also known as the one-dimensional span
of those two linearly dependent vectors.

To sum up, linear transformations are a way to move around space such that the
grid lines remain parallel and evenly spaced and such that the origin remains
fixed. Delightfully, these transformations can be described using only a handful
of numbers: the coordinates of where each basis vector lands. Matrices give us a
language to describe these transformations where the columns represent those
coordinates and matrix-vector multiplication is just a way to compute what that
transformation does to a given vector. The important take-away here is that
every time you see a matrix, you can interpret it as a certain transformation of
space. Once you really digest this idea, you're in a great position to
understand linear algebra deeply. Almost all of the topics coming up, from
matrix multiplication to determinants, eigenvalues, etc. will become easier to
understand once you start thinking about matrices as transformations of space.

\begin{remark}
  See the corresponding \textit{Essence of Linear Algebra} video for a more
  visual presentation (11 minutes)
  \cite{bib:linalg_linear_transformations_and_matrices}.
\end{remark}

\section{Matrix multiplication as composition}
\index{Matrices!multiplication}

Often-times you find yourself wanting to describe the effect of applying one
transformation and then another. For example, you may want to describe what
happens when you first rotate the plane $90\degree$ counterclockwise then apply
a shear. The overall effect here, from start to finish, is another linear
transformation distinct from the rotation and the shear. This new linear
transformation is commonly called the ``composition" of the two separate
transformations we applied, and like any linear transformation, it can be
described with a matrix all its own by following $\hat{i}$ and $\hat{j}$. In
this example, the ultimate landing spot for $\hat{i}$ after both transformations
is $(1, 1)$, so that's the first column of the matrix. Likewise, $\hat{j}$
ultimately ends up at the location $(-1, 0)$, so we make that the second column
of the matrix.

\begin{equation*}
  \begin{bmatrix}
    1 & -1 \\
    1 & -0
  \end{bmatrix}
\end{equation*}

This new matrix captures the overall effect of applying a rotation then a shear
but as one single action rather than two successive ones.

Here's one way to think about that new matrix: if you were to feed some vector
through the rotation then the shear, the long way to compute where it ends up is
to, first, multiply it on the left by the rotation matrix; then, take whatever
you get and multiply that on the left by the shear matrix.

\begin{equation*}
  \begin{bmatrix}
    1 & 1 \\
    0 & 1
  \end{bmatrix}\left(
  \begin{bmatrix}
    0 & -1 \\
    1 & 0
  \end{bmatrix}
  \begin{bmatrix}
    x \\
    y
  \end{bmatrix}\right)
\end{equation*}

This is, numerically speaking, what it means to apply a rotation then a shear to
a given vector, but the result should be the same as just applying this new
composition matrix we found to that same vector. This applies to any vector
because this new matrix is supposed to capture the same overall effect as the
rotation-then-shear action.

\begin{equation*}
  \begin{bmatrix}
    1 & 1 \\
    0 & 1
  \end{bmatrix}\left(
  \begin{bmatrix}
    0 & -1 \\
    1 & 0
  \end{bmatrix}
  \begin{bmatrix}
    x \\
    y
  \end{bmatrix}\right) =
  \begin{bmatrix}
    1 & -1 \\
    1 & 0
  \end{bmatrix} \begin{bmatrix}
    x \\
    y
  \end{bmatrix}
\end{equation*}

Based on how things are written down here, it's reasonable to call this new
matrix the ``product" of the original two matrices.

\begin{equation*}
  \begin{bmatrix}
    1 & 1 \\
    0 & 1
  \end{bmatrix}
  \begin{bmatrix}
    0 & -1 \\
    1 & 0
  \end{bmatrix} =
  \begin{bmatrix}
    1 & -1 \\
    1 & 0
  \end{bmatrix}
\end{equation*}

We can think about how to compute that product more generally in just a moment,
but it's way too easy to get lost in the forest of numbers. Always remember
that multiplying two matrices like this has the geometric meaning of applying
one transformation then another.

One oddity here is that we are reading the transformations from right to left;
you first apply the transformation represented by the matrix on the right, then
you apply the transformation represented by the matrix on the left. This stems
from function notation, since we write functions on the left of variables, so
every time you compose two functions, you always have to read it right to left.

Let's look at another example. Take the matrix with columns $(1, 1)$ and
$(-2, 0)$.

\begin{equation*}
  M_1 = \begin{bmatrix}
    1 & -2 \\
    1 & 0
  \end{bmatrix}
\end{equation*}

Next, take the matrix with columns $(0, 1)$ and $(2, 0)$.

\begin{equation*}
  M_2 = \begin{bmatrix}
    0 & 2 \\
    1 & 0
  \end{bmatrix}
\end{equation*}

The total effect of applying $M_1$ then $M_2$ gives us a new transformation, so
let's find its matrix. First, we need to determine where $\hat{i}$ goes. After
applying $M_1$, the new coordinates of $\hat{i}$, by definition, are given by
that first column of $M_1$, namely, $(1, 1)$. To see what happens after applying
$M_2$, multiply the matrix for $M_2$ by that vector $(1, 1)$. Working it out the
way described in the last section, you'll get the vector $(2, 1)$.

\begin{equation*}
  \begin{bmatrix}
    0 & 2 \\
    1 & 0
  \end{bmatrix}
  \begin{bmatrix}
    1 \\
    1
  \end{bmatrix} = 1
  \begin{bmatrix}
    0 \\
    1
  \end{bmatrix} + 1
  \begin{bmatrix}
    2 \\
    0
  \end{bmatrix} =
  \begin{bmatrix}
    2 \\
    1
  \end{bmatrix}
\end{equation*}

This will be the first column of the composition matrix. Likewise, to follow
$\hat{j}$, the second column of $M_1$ tells us that it first lands on $(-2, 0)$.
Then, when we apply $M_2$ to that vector, you can work out the matrix-vector
product to get $(0, -2)$.

\begin{equation*}
  \begin{bmatrix}
    0 & 2 \\
    1 & 0
  \end{bmatrix}
  \begin{bmatrix}
    -2 \\
    0
  \end{bmatrix} = -2
  \begin{bmatrix}
    0 \\
    1
  \end{bmatrix} + 0
  \begin{bmatrix}
    2 \\
    0
  \end{bmatrix} =
  \begin{bmatrix}
    0 \\
    -2
  \end{bmatrix}
\end{equation*}

This will be the second column of the composition matrix.

\begin{equation*}
  \begin{bmatrix}
    0 & 2 \\
    1 & 0
  \end{bmatrix}
  \begin{bmatrix}
    1 & -2 \\
    1 & 0
  \end{bmatrix} =
  \begin{bmatrix}
    2 & 0 \\
    1 & -2
  \end{bmatrix}
\end{equation*}

\subsection{General matrix multiplication}

Let's go through that same process again, but this time, we'll use variable
entries in each matrix, just to show that the same line of reasoning works for
any matrices. This is more symbol-heavy, but it should be pretty satisfying for
anyone who has previously been taught matrix multiplication the more rote way.

\begin{equation*}
  \begin{bmatrix}
    a & b \\
    c & d
  \end{bmatrix}
  \begin{bmatrix}
    e & f \\
    g & h
  \end{bmatrix} =
  \begin{bmatrix}
    ? & ? \\
    ? & ?
  \end{bmatrix}
\end{equation*}

To follow where $\hat{i}$ goes, start by looking at the first column of the
matrix on the right, since this is where $\hat{i}$ initially lands. Multiplying
that column by the matrix on the left is how you can tell where the intermediate
version of $\hat{i}$ ends up after applying the second transformation.

\begin{equation*}
  \begin{bmatrix}
    a & b \\
    c & d
  \end{bmatrix}
  \begin{bmatrix}
    e \\
    g
  \end{bmatrix} = e
  \begin{bmatrix}
    a \\
    c
  \end{bmatrix} + g
  \begin{bmatrix}
    b \\
    d
  \end{bmatrix} =
  \begin{bmatrix}
    ae + bg \\
    ce + dg
  \end{bmatrix}
\end{equation*}

So the first column of the composition matrix will always equal the left matrix
times the first column of the right matrix. Likewise, $\hat{j}$ will always
initially land on the second column of the right matrix, so multiplying by this
second column will give its final location, and hence, that's the second column
of the composition matrix.

\begin{equation*}
  \begin{bmatrix}
    a & b \\
    c & d
  \end{bmatrix}
  \begin{bmatrix}
    f \\
    h
  \end{bmatrix} = f
  \begin{bmatrix}
    a \\
    c
  \end{bmatrix} + h
  \begin{bmatrix}
    b \\
    d
  \end{bmatrix} =
  \begin{bmatrix}
    af + bh \\
    cf + dh
  \end{bmatrix}
\end{equation*}

So the complete composition matrix is

\begin{equation*}
  \begin{bmatrix}
    a & b \\
    c & d
  \end{bmatrix}
  \begin{bmatrix}
    e & f \\
    g & h
  \end{bmatrix} =
  \begin{bmatrix}
    ae + bg & af + bh \\
    ce + dg & cf + dh
  \end{bmatrix}
\end{equation*}

Notice there's a lot of symbols here, and it's common to be taught this formula
as something to memorize along with a certain algorithmic process to help
remember it. Before memorizing that process, you should get in the habit of
thinking about what matrix multiplication really represents: applying one
transformation after another. This will give you a much better conceptual
framework that makes the properties of matrix multiplication much easier to
understand.

\subsection{Matrix multiplication associativity}

For example, here's a question: does it matter what order we put the two
matrices in when we multiply them? Let's think through a simple example. Take a
shear which fixes $\hat{i}$ and moves $\hat{j}$ over to the right, and a
$90\degree$ rotation. If you first do the shear then rotate, we can see that
$\hat{i}$ ends up at $(0, 1)$ and $\hat{j}$ ends up at $(-1, 1)$. Both are
generally pointing close together. If you first rotate then do the shear,
$\hat{i}$ ends up over at $(1, 1)$ and $\hat{j}$ is off on a different direction
at $(-1, 0)$ and they're pointing farther apart. The overall effect here is
clearly different, so evidently, order totally does matter. Notice by thinking
in terms of transformations, that's the kind of thing that you can do in your
head by visualizing. No matrix multiplication necessary.

Let's consider trying to prove that matrix multiplication is associative. This
means that if you have three matrices $A$, $B$, and $C$, and you multiply them
all together, it shouldn't matter if you first compute $A$ times $B$ then
multiply the result by $C$, or if you first multiply $B$ times $C$ then multiply
that result by $A$ on the left. In other words, it doesn't matter where you put
the parenthesis.

\begin{equation*}
  (AB)C \stackrel{?}{=} A(BC)
\end{equation*}

If you try to work through this numerically, it's horrible, and unenligthening
for that matter. However, when you think about matrix multiplication as applying
one transformation after another, this property is just trivial. Can you see
why? What it's saying is that if you first apply $C$ then $B$, then $A$, it's
the same as applying $C$, then $B$ then $A$. There's nothing to prove, you're
just applying the same three things one after the other all in the same order.
This might feel like cheating, but it's not. This is a valid proof that matrix
multiplication is associative, and even better than that, it's a good
explanation for why that property should be true.

\begin{remark}
  See the corresponding \textit{Essence of Linear Algebra} video for a more
  visual presentation (10 minutes)
  \cite{bib:linalg_matrix_multiplication_as_composition}.
\end{remark}

\section{The determinant}

So, moving forward, we will be assuming you have a visual understanding of
linear transformations and how they're represented with matrices.

\subsection{Scaling areas}

If you think about a couple linear transformations, you might notice how some of
them seem to stretch space out while others compress it. It's useful for
understanding these transformations to measure exactly how much it stretches or
compresses things (more specifically, to measure the factor by which areas are
scaled). For example, look at the matrix with columns $(3, 0)$, and $(0, 2)$.

\begin{equation*}
  \begin{bmatrix}
    3 & 0 \\
    0 & 2
  \end{bmatrix}
\end{equation*}

It scales $\hat{i}$ by a factor of $3$ and scales $\hat{j}$ by a factor of $2$.
Now, if we focus our attention on the one-by-one square whose bottom sits on
$\hat{i}$ and whose left side sits on $\hat{j}$, after the transformation, this
turns into a $2$ by $3$ rectangle. Since this region started out with area $1$
and ended up with area $6$, we can say the linear transformation has scaled its
area by a factor of $6$. Compare that to a shear whose matrix has columns
$(1, 0)$ and $(1, 1)$ meaning $\hat{i}$ stays in place and $\hat{j}$ moves over
to $(1, 1)$.

\begin{equation*}
  \begin{bmatrix}
    1 & 1 \\
    0 & 1
  \end{bmatrix}
\end{equation*}

That same unit square determined by $\hat{i}$ and $\hat{j}$ gets slanted and
turned into a parallelogram, but the area of that parallelogram is still $1$
since its base and height each continue to each have length $1$. Even though
this transformation pushes things about, it seems to leave areas unchanged (at
least in the case of that one uint square).

Actually though, if you know how much the area of that one single unit square
changes, you can know how the area of any possible region in space changes.
First off, notice that whatever happens to one square in the grid has to happen
to any other square in the grid no matter the size. This follows from the fact
that grid lines remain parallel and evenly spaced. Then, any shape that's not a
grid square can be approximated by grid squares pretty well with arbitrarily
good approximations if you use small enough grid squares. So, since the areas of
all those tiny grid squares are being scaled by some single amount, the area of
the shape as a whole will also be scaled by that same single amount.

\subsection{Exploring the determinant}

This special scaling factor, the factor by which a linear transformation changes
any area, is called the \textit{determinant} of that transformation. We'll show
how to compute the determinant of a transformation using its matrix later on,
but understanding what it represents is much more important than the
computation. For example, the determinant of a transformation would be $3$ if
that transformation increases the area of the region by a factor of $3$.

\begin{equation*}
  det\left(\begin{bmatrix}
    0 & 2 \\
    -1.5 & 1
  \end{bmatrix}\right) = 3
\end{equation*}

The determinant of a matrix is commonly denoted by vertical bars instead of
square brackets.

\begin{equation*}
  \begin{vmatrix}
    0 & 2 \\
    -1.5 & 1
  \end{vmatrix} = 3
\end{equation*}

The determinant of a transformation would be $\frac{1}{2}$ if it compresses all
areas by a factor of $\frac{1}{2}$.

\begin{equation*}
  \begin{vmatrix}
    0.5 & 0.5 \\
    -0.5 & 0.5
  \end{vmatrix} = 0.5
\end{equation*}

The determinant of a 2D transformation is zero if it compresses all of space
onto a line or even onto a single point since then, the area of any region would
become zero.

\begin{equation*}
  \begin{vmatrix}
    4 & 2 \\
    2 & 1
  \end{vmatrix} = 0
\end{equation*}

That last example proved to be pretty important. It means checking if the
determinant of a given matrix is zero will give a way of computing whether the
transformation associated with that matrix compresses everything into a smaller
dimension.

This analogy so far isn't quite right. The full concept of a determinant allows
for negative values.

\begin{equation*}
  \begin{vmatrix}
    1 & 2 \\
    3 & 4
  \end{vmatrix} = -2
\end{equation*}

What would scaling an area by a negative amount even mean? This has to do with
the idea of orientation. A 2D transformation with a negative determinant
essentially flips space over. Any transformations that do this are said to
"invert the orientation of space". Another way to think about it is in terms of
$\hat{i}$ and $\hat{j}$. In their starting positions, $\hat{j}$ is to the left
of $\hat{i}$. If, after a transformation, $\hat{j}$ is now on the right side of
$\hat{i}$, the orientation of space has been inverted. Whenever this happens,
the determinant will be negative. The absolute value of the determinant still
tells you the factor by which areas have been scaled.

For example, the matrix with columns $(1, 1)$ and $(2, -1)$ encodes a
transformation that has determinant $-3$.

\begin{equation*}
  \begin{vmatrix}
    1 & 2 \\
    1 & -1
  \end{vmatrix} = -3
\end{equation*}

This means that space gets flipped over and areas are scaled by a factor of $3$.

Why would this idea of a negative area scaling factor be a natural way to
describe orientation-flipping? Think about the series of transformations you get
by slowly letting $\hat{i}$ rotate closer and closer to $\hat{j}$. As $\hat{i}$
gets closer, all the areas in space are getting compressed more and more meaning
the determinant approaches zero. Once $\hat{i}$ lines up perfectly with
$\hat{j}$, the determinant is zero. Then, if $\hat{i}$ continues, doesn't it
feel natural for the determinant to keep decreasing into negative numbers?

\subsection{The determinant in 3D}

That's the understanding of determinants in two dimensions. What should it mean
for three dimensions? The determinant of a $3 \times 3$ matrix tells you how
much volumes get scaled. A determinant of zero would mean that all of space is
compressed onto something with zero volume meaning either a flat plane, a line,
or in the most extreme case, a single point. This means that the columns of the
matrix are linearly dependent.

What should negative determinants mean for three dimensions? One way to describe
orientation in 3D is with the right-hand rule. Point the forefinger of your
right hand in the direction if $\hat{i}$, stick out your middle finger in the
direction of $\hat{j}$, and notice how when you point your thumb up, it is in
the direction of $\hat{k}$. If you can still do that after the transformation,
orientation has not changed and the determinant is positive. Otherwise, if after
the transformation it only makes sense to do that with your left hand,
orientation has been flipped and the determinant is negative.

\subsection{Computing the determinant}

How do you actually compute the determinant? For a $2 \times 2$ matrix with
entries $a$, $b$, $c$, $d$, the formula is as follows.

\begin{equation*}
  \begin{vmatrix}
    a & b \\
    c & d
  \end{vmatrix} = ad - bc
\end{equation*}

Here's part of an intution for where this formula comes from. Let's say that the
terms $b$ and $c$ were both zero. Then, the term $a$ tells you how much
$\hat{i}$ is stretched in the x-direction and the term $d$ tells you how much
$\hat{j}$ is stretched in the y-direction. Since those other terms are zero, it
should make sense that $ad$ gives the area of the rectangle that the unit square
turns into. Even if only one of $b$ or $c$ are zero, you'll have a parallelogram
with a base of $a$ and a height $d$, so the area should still be $ad$. Loosely
speaking, if both $b$ and $c$ are nonzero, then that $bc$ term tells you how
much this parallelogram is stretched or compressed in the diagonal direction.

If you feel like computing determinants by hand is something that you need to
know (you won't for this book), the only way to get it down is to just practice
it with a few. This is all triply true for 3D determinants. There is a formula,
and if you feel like that's something you need to know, you should practice with
a few matrices.

\begin{equation*}
  \begin{vmatrix}
    a & b & c \\
    d & e & f \\
    g & h & i
  \end{vmatrix} =
  a \begin{vmatrix}
    e & f \\
    h & i
  \end{vmatrix}
  - b \begin{vmatrix}
    d & f \\
    g & i
  \end{vmatrix}
  + c \begin{vmatrix}
    d & e \\
    g & h
  \end{vmatrix}
\end{equation*}

We don't think those computations fall within the essence of linear algebra, but
understanding what the determinant represents falls within that essence.

\begin{remark}
  See the corresponding \textit{Essence of Linear Algebra} video for a more
  visual presentation (10 minutes) \cite{bib:linalg_the_determinant}.
\end{remark}

\section{Inverse matrices, column space, and null space}

As you can probably tell by now, the bulk of this chapter is on understanding
matrix and vector operations through that more visual lens of linear
transformations. This section is no exception, describing the concepts of
inverse matrices, columns space, rank, and null space through that lens. A fair
warning though: we're not going to talk about the methods for actually computing
these things, and some would argue that that's pretty important. There are a lot
of very good resources for learning those methods outside of this chapter.
Keywords: "Gaussian elimination" and "row echelon form". Most of the value that
we actually have to add here is on the intuition half. Plus, in practice, we
usually use software to compute these things for us anyway.

\subsection{Linear systems of equations}

First, a few words on the usefulness of linear algebra. By now, you already have
a hint for how it's used in describing the manipulation of space, which is
useful for computer graphics and robotics. However, one of the main reasons that
linear algebra is more broadly applicable, and required for just about any
technical discipline, is that it lets us solve certain systems of equations.
When we say "system of equations", we mean there is a list of variables, things
you don't know, and a list of equations relating them. For example,

\begin{align*}
  6x - 3y + 2z &= 7 \\
  x + 2y + 5z &= 0 \\
  2x - 8y - z &= -2
\end{align*}

is a system of equations with the unknowns $x$, $y$, and $z$.

In a lot of situations, those equations can get very complicated, but, if you're
lucky, they might take on a certain special form. Within each equation, the only
thing happening to each variable is that it's scaled by some constant, and the
only thing happening to each of those scaled variables is that they're added to
each other, so no exponents or fancy functions, or multiplying two variables
together.

The typical way to organize this sort of special system of equations is to throw
all the variables on the left and put any lingering constants on the right. It's
also nice to vertically line up the common variables, and to do that, you might
need to throw in some zero coefficients whenever the variable doesn't show up in
one of the equations.

\begin{align*}
  2x + 5y + 3z &= -3 \\
  4x + 0y + 8z &= 0 \\
  1x + 3y + 0z &= 2
\end{align*}

This is called a "linear system of equations". You might notice that this looks
a lot like matrix-vector multiplication. In fact, you can package all of the
equations together into a single vector equation, where you have the matrix
containing all the constant coefficients, a vector containing all the constant
coefficients, and a vector containing all the variables. Their matrix-vector
product equals some different constant vector.

\begin{equation*}
  \begin{bmatrix}
    2 & 5 & 3 \\
    4 & 0 & 8 \\
    1 & 3 & 0
  \end{bmatrix}
  \begin{bmatrix}
    x \\
    y \\
    z
  \end{bmatrix} =
  \begin{bmatrix}
    -3 \\
    0 \\
    2
  \end{bmatrix}
\end{equation*}

Let's name that constant matrix $\mtx{A}$, denote the vector holding the
variables with $\mtx{x}$, and call the constant vector on the right-hand side
$\mtx{v}$. This is more than just a notational trick to get our system of
equations written on one line. It sheds light on a pretty cool geometric
interpretation for the problem.

\begin{equation*}
  \mtx{A}\mtx{x} = \mtx{v}
\end{equation*}

The matrix $\mtx{A}$ corresponds with some linear transformation, so solving
$\mtx{A}\mtx{x} = \mtx{v}$ means we're looking for a vector $\mtx{x}$ which,
after applying the transformation $\mtx{A}$, lands on $\mtx{v}$.

Think about what's happening here for a moment. You can hold in your head this
really complicated idea of multiple variables all intermingling with each other
just by thinking about compressing or morphing space and trying to determine
which vector lands on another.

To start simple, let's say you have a system with two equations and two
unknowns. This means the matrix $\mtx{A}$ is a $2 \times 2$ matrix, and
$\mtx{v}$ and $\mtx{x}$ are each two-dimensional vectors.

\begin{align*}
  2x + 2y &= -4 \\
  1x + 3y &= -1 \\
  \begin{bmatrix}
    2 & 2 \\
    1 & 3
  \end{bmatrix}
  \begin{bmatrix}
    x \\
    y
  \end{bmatrix} &=
  \begin{bmatrix}
    -4 \\
    -1
  \end{bmatrix}
\end{align*}

\subsection{Inverse}

How we think about the solutions to this equation depends on whether the
transformation associated with $\mtx{A}$ compresses all of space into a lower
dimension, like a line or a point, or if it leaves everything spanning the full
two dimensions where it started. In the language of the last section, we
subdivide into the case where $\mtx{A}$ has zero determinant and the case where
$\mtx{A}$ has nonzero determinant.

Let's start with the most likely case where the determinant is nonzero, meaning
space does not get compressed into a zero area region. In this case, there will
always be one and only one vector that lands on $\mtx{v}$, and you can find it
by playing the transformation in reverse. Following where $\mtx{v}$ goes as we
undo the transformation, you'll find the vector $\mtx{x}$ such that $\mtx{A}$
times $\mtx{x}$ equals $\mtx{v}$.

When you play the transformation in reverse, it actually corresponds to a
separate linear transformation, commonly called "the inverse of $\mtx{A}$"
denoted $\mtx{A}^{-1}$.

\begin{equation*}
  \mtx{A}^{-1} =
  \begin{bmatrix}
    3 & 1 \\
    0 & 2
  \end{bmatrix}^{-1}
\end{equation*}

For example, if $\mtx{A}$ was a counterclockwise rotation by $90 \degree$, then
the inverse of $\mtx{A}$ would be a clockwise rotation by $90 \degree$.

\begin{equation*}
  \begin{array}{cc}
  \mtx{A} =
  \begin{bmatrix}
    0 & -1 \\
    1 & 0
  \end{bmatrix} &
  \mtx{A}^{-1} =
  \begin{bmatrix}
    0 & 1 \\
    -1 & 0
  \end{bmatrix}
  \end{array}
\end{equation*}

If $\mtx{A}$ was a rightward shear that pushes $\hat{j}$ one unit to the right,
the inverse of $\mtx{A}$ would be a leftward shear that pushes $\hat{j}$ one
unit to the left.

In general, $\mtx{A}^{-1}$ is the unique transformation with the property that
if you first apply $\mtx{A}$, then follow it with the transformation
$\mtx{A}^{-1}$, you end up back where you started. Applying one transformation
after another is captured algebraically with matrix multiplication, so the core
property of this transformation $\mtx{A}^{-1}$ is that $\mtx{A}^{-1}\mtx{A}$
equals the matrix that corresponds to doing nothing.

The transformation that does nothing is called the "identity transformation". It
leaves $\hat{i}$ and $\hat{j}$ each where they are, unmoved, so its columns are
$(1, 0)$ and $(0, 1)$.

\begin{equation*}
  \mtx{A}^{-1}\mtx{A} =
  \begin{bmatrix}
    1 & 0 \\
    0 & 1
  \end{bmatrix}
\end{equation*}

Once you find this inverse, which in practice, you do with a computer, you can
solve your equation by multipling this inverse matrix by $\mtx{v}$.

\begin{align*}
  \mtx{A}\mtx{x} &= \mtx{v} \\
  \mtx{A}^{-1}\mtx{A}\mtx{x} &= \mtx{A}^{-1}\mtx{v} \\
  \mtx{x} &= \mtx{A}^{-1}\mtx{v}
\end{align*}

Again, what this means geometrically is that you're playing the transformation
in reverse and following $\mtx{v}$. This nonzero determinant case, which for a
random choice of matrix is by far the most likely one, corresponds with the idea
that if you have two unknowns and two equations, it's almost certainly the case
that there's a single, unique solution.

This idea also makes sense in higher dimensions when the number of equations
equals the number of unknowns. Again, the system of equations can be translated
to the geometric interpretation where you have some transformation, $\mtx{A}$,
some vector $\mtx{v}$, and you're looking for the vector $\mtx{x}$ that lands on
$\mtx{v}$. As long as the transformation $\mtx{A}$ doesn't compress all of
space into a lower dimension, meaning, its determinant is nonzero, there will be
an inverse transformation, $\mtx{A}^{-1}$ , with the property that if you first
do $\mtx{A}$, then you do $\mtx{A}^{-1}$, it's the same as doing nothing. To
solve your equation, you just have to multiply that reverse transformation
matrix by the vector $\mtx{v}$.

When the determinant is zero and the transformation associated with this system
of equations compresses space into a smaller dimension, there is no inverse. You
cannot uncompress a line to turn it into a plane. At least, that's not something
that a function can do. That would require transforming each individual vector
into a whole line full of vectors, but functions can only take a single input to
a single output.

Similarly, for three equations and three unknowns, there will be no inverse if
the corresponding transformation compresses 3D space onto the plane, or even if
it compresses it onto a line, or a point. Those all correspond to a determinant
of zero since any region is compressed into something with zero volume.

It's still possible that a solution exists even when there is no inverse. It's
just that when your transformation compresses space onto, say, a line, you have
to be lucky enough that the vector $\mtx{v}$ exists somewhere on that line.

\subsection{Rank and column space}

You might notice that some of these zero determinant cases feel a lot more
restrictive than others. Given a $3 \times 3$ matrix, for example, it seems a
lot harder for a solution to exist when it compresses space onto a line compared
to when it compresses space onto a plane even though both of those have zero
determinant. We have some language that's more specific than just saying "zero
determinant". When the output of a transformation is a line, meaning it's
one-dimensional, we say the transformation has a \textit{rank} of one. If all
the vectors land on some two-dimensional plane, we say the transformation has a
rank of two. The word "rank" means the number of dimensions in the output of a
transformation.

For instance, in the case of $2 \times 2$ matrices, the highest possible rank is
$2$. It means the basis vectors continue to span the full two dimensions of
space, and the determinant is nonzero. For $3 \times 3$ matrices, rank $2$ means
that we've collapsed, but not as much as we would have collapsed for a rank 1
situation. If a 3D transformation has a nonzero determinant and its output fills
all of 3D space, it has a rank of $3$.

This set of all possible outputs for your matrix, whether it's a line, a plane,
3D space, whatever, is called the \textit{column space} of your matrix. You can
probably guess where that name comes from. The columns of your matrix tell you
where the basis vectors land, and the span of those transformed basis vectors
gives you all possible outputs. In other words, the column space is the span of
the columns of your matrix, so a more precise definition of rank would be that
it's the number of dimensions in the column space. When this rank is as high as
it can be, meaning it equals the number of columns, we call the matrix "full
rank".

\subsection{Null space}

Notice, the zero vector will always be included in the column space since linear
transformations must keep the origin fixed in place. For a full rank
transformation, the only vector that lands at the origin is the zero vector
itself, but for matrices that aren't full rank, which compress to a smaller
dimension, you can have a whole bunch of vectors that land on zero. If a 2D
transformation compresses space onto a line, for example, there is a separate
line in a different direction full of vectors that get compressed onto the
origin. If a 3D transformation compresses space onto a plane, there's also a
full line of vectors that land on the origin. If a 3D transformation compresses
all the space onto a line, then there's a whole plane full of vectors that land
on the origin.

This set of vectors that lands on the origin is called the \textit{null space}
or the \textit{kernel} of your matrix. It's the space of all vectors that become
null in the sense that they land on the zero vector. In terms of the linear
system of equations $\mtx{A}\mtx{x} = \mtx{v}$, when $\mtx{v}$ happens to be the
zero vector, the null space gives you all the possible solutions to the
equation.

\subsection{Closing remarks}

That's a high-level overview of how to think about linear systems of equations
geometrically. Each system has some kind of linear transformation associated
with it, and when that transformation has an inverse, you can use that inverse
to solve your system. Otherwise, the idea of column space lets us understand
when a solution even exists, and the idea of a null space helps us understand
what the set of all possible solutions can look like.

Again, there's a lot not covered here, most notably how to compute these things.
We also had to limit the scope to examples where the number of equations equals
the number of unknowns. The goal here is not to try to teach everything: it's
that you come away with a strong intuition for inverse matrices, column space,
and null space, and that those intuitions make any future learning that you do
more fruitful.

\begin{remark}
  See the corresponding \textit{Essence of Linear Algebra} video for a more
  visual presentation (12 minutes)
  \cite{bib:linalg_inverse_matrices_column_space_and_null_space}.
\end{remark}

\section{Nonsquare matrices as transformations between dimensions}

When we've talked about linear transformations so far, we've only really talked
about transformations from 2D vectors to other 2D vectors, represented with
$2 \times 2$ matrices; or from 3D vectors to other 3D vectors, represented with
$3 \times 3$ matrices. What about nonsquare matrices? We'll take a moment to
discuss what those mean geometrically.

By now, you have most of the background you need to start pondering a question
like this on your own, but we'll start talking through it, just to give a little
mental momentum.

It's perfectly reasonable to talk about transformations between dimensions, such
as one that takes 2D vectors to 3D vectors. Again, what makes one of these
linear is that grid lines remain parallel and evenly spaced, and that the origin
maps to the origin.

Encoding one of these transformations with a matrix the same as what we've done
before. You look at where each basis vector lands and write the coordinates of
the landing spots as the coordinates of the landing spots as the columns of a
matrix. For example, the following is a transformation that takes $\hat{i}$ to
the coordinates $(2, -1, -2)$ and $\hat{j}$ to the coordinates $(0, 1, 1)$.

\begin{equation*}
  \begin{bmatrix}
    2 & 0 \\
    -1 & 1 \\
    -2 & 1
  \end{bmatrix}
\end{equation*}

Notice, this means the matrix encoding our transformation has three rows and two
columns, which, to use standard terminology, makes it a $3 \times 2$ matrix. In
the language of last section, the column space of this matrix, the place where
all the vectors land, is a 2D plane slicing through the origin of 3D space. The
matrix is still full rank since the number of dimensions in this column space is
the same as the number of dimensions of the input space.

If you see a $3 \times 2$ matrix out in the wild, you can know that it has the
geometric interpretation of mapping two dimensions to three dimensions since the
two columns indicate that the input space has two basis vectors, and the three
rows indicate that the landing spots for each of those basis vectors is
described with three separate coordinates.

For a $2 \times 3$ matrix, the three columns indicate a starting space that has
three basis vectors, so it starts in three dimensions; and the two rows indicate
that the landing spot for each of those three basis vectors is described with
only two coordinates, so they must be landing in two dimensions. It's a
transformation from 3D space onto the 2D plane.

You could also have a transformation from two dimensions to one dimension.
One-dimensional space is really just the number line, so a transformation like
this takes in 2D vectors and returns numbers. Thinking about gridlines remaining
parallel and evenly spaced is messy due to all the compression happening here,
so in this case, the visual understanding for what linearity means is that if
you have a line of evenly spaced dots, it would remain evenly spaced once
they're mapped onto the number line.

One of these transformations is encoded with a $1 \times 2$ matrix, each of
whose two columns has just a single entry. The two columns represent where the
basis vectors land, and each one of those columns requires just one number, the
number that that basis vector landed on.

\begin{remark}
  See the corresponding \textit{Essence of linear algebra} video for a more
  visual presentation (4 minutes)
  \cite{bib:linalg_nonsquare_matrices_as_transformations_between_dimensions}.
\end{remark}

\section{Eigenvectors and eigenvalues}

\subsection{What is an eigenvector?}

To start, consider some linear transformation in two dimensions that moves the
basis vector $\hat{i}$ to the coordinates $(3, 0)$ and $\hat{j}$ to $(1, 2)$, so
it's represented with a matrix whose columns are $(3, 0)$ and $(1, 2)$.

\begin{equation*}
  \begin{bmatrix}
    3 & 1 \\
    0 & 2
  \end{bmatrix}
\end{equation*}

Focus in on what it does to one particular vector and think about the span of
that vector, the line passing through its origin and its tip. Most vectors are
going to get knocked off their span during the transformation, but some special
vectors do remain on their own span meaning the effect that the matrix has on
such a vector is just to stretch it or compress it like a scalar.

For this specific example, the basis vector $\hat{i}$ is one such special
vector. The span of $\hat{i}$ is the x-axis, and from the first column of the
matrix, we can see that $\hat{i}$ moves over to three times itself still on that
x-axis. What's more, due to the way linear transformations work, any other
vector on the x-axis is also just stretched by a factor of $3$, and hence,
remains on its own span.

A slightly sneakier vector that remains on its own span during this
transformation is $(-1, 1)$. It ends up getting stretched by a factor of $2$.
Again, linearity is going to imply that any other vector on the diagonal line
spanned by this vector is just going to get stretched out by a factor of $2$.

For this transformation, those are all the vectors with this special property of
staying on their span. Those on the x-axis get stretched out by a factor of $3$
and those on the diagonal line get stretched out by a factor of $2$. Any other
vector is going to get rotated somewhat during the transformation and knocked
off the line that it spans. As you might have guessed by now, these special
vectors are called the \textit{eigenvectors} of the transformation, and each
eigenvector has associated with it an \textit{eigenvalue}, which is just the
factor by which it's stretched or compressed during the transformation.

Of course, there's nothing special about stretching vs compressing or the fact
that these eigenvalues happen to be positive. In another example, you could have
an eigenvector with eigenvalue $-\frac{1}{2}$, meaning that the vector gets
flipped and compressed by a factor of $\frac{1}{2}$.

\begin{equation*}
  \begin{bmatrix}
    0.5 & -1 \\
    -1 & 0.5
  \end{bmatrix}
\end{equation*}

The important part here is that it stays on the line that it spans out without
getting rotated off of it.

\subsection{Eigenvectors in 3D rotation}

For a glimpse of why this might be a useful thing to think about, consider some
three-dimensional rotation. If you can find an eigenvector for that rotation, a
vector that remains on its own span, you have found the axis of rotation. It's
much easier to think about a 3D rotation in terms of some axis of rotation and
an angle by which it's rotating rather than thinking about the full $3 \times 3$
matrix associated with that transformation. In this case, by the way, the
corresponding eigenvalue would have to be $1$ since rotations never stretch or
compress anything, so the length of the vector would remain the same.

\subsection{Finding eigenvalues}
\index{Matrices!eigenvalues}

The following pattern shows up a lot in linear algebra. With any linear
transformation described by a matrix, you could understand what it's doing by
reading off the columns of this matrix as the landing spots for basis vectors,
but often a better way to get at the heart of what the linear transformation
actually does, less dependent on your particular coordinate system, is to find
the eigenvectors and eigenvalues.

I won't cover the full details on methods for computing eigenvectors and
eigenvalues here, but I'll try to give an overview of the computational ideas
that are most important for a conceptual understanding. Symbolically, an
eigenvector look like the following

\begin{equation*}
  \mtx{A}\mtx{v} = \lambda \mtx{v}
\end{equation*}

$\mtx{A}$ is the matrix representing some transformation, $\mtx{v}$ is the
eigenvector, and $\lambda$ is a number, namely the corresponding eigenvalue.
This expression is saying that the matrix-vector product $\mtx{A}\mtx{v}$ gives
the same result as just scaling the eigenvector $\mtx{v}$ by some value
$\lambda$. Finding the eigenvectors and their eigenvalues of the matrix
$\mtx{A}$ involves finding the values of $\mtx{v}$ and $\lambda$ that make this
expression true. It's awkward to work with at first because that left-hand side
represents matrix-vector multiplication, but the right-hand side is
scalar-vector multiplication. Let's rewrite the right-hand side as some kind of
matrix-vector multiplication using a matrix which has the effect of scaling any
vector by a factor of $\lambda$. The columns of such a matrix will represent
what happens to each basis vector, and each basis vector is simply multiplied by
$\lambda$, so this matrix will have the number $\lambda$ down the diagonal and
zeroes everywhere else.

\begin{equation*}
  \begin{bmatrix}
    \lambda & 0 & 0 \\
    0 & \lambda & 0 \\
    0 & 0 & \lambda \\
  \end{bmatrix}
\end{equation*}

The common way to write this is to factor out $\lambda$ and write it as
$\lambda\mtx{I}$ where $\mtx{I}$ is the identity matrix with ones down the
diagonal.

\begin{equation*}
  \mtx{A}\mtx{v} = (\lambda\mtx{I}) \mtx{v}
\end{equation*}

With both sides looking like matrix-vector multiplication, we can subtract off
that right-hand side and factor out $\mtx{v}$.

\begin{align*}
  \mtx{A}\mtx{v} - (\lambda\mtx{I}) \mtx{v} &= \mtx{0} \\
  (\mtx{A} - \lambda\mtx{I})\mtx{v} &= \mtx{0}
\end{align*}

We now have a new matrix $\mtx{A} - \lambda\mtx{I}$, and we're looking for a
vector $\mtx{v}$ such that this new matrix times $\mtx{v}$ gives the zero
vector. This will always be true if $\mtx{v}$ itself is the zero vector, but
that's boring. We want a nonzero eigenvector. The only way it's possible for the
product of a matrix with a nonzero vector to become zero is if the
transformation associated with that matrix compresses space into a lower
dimension. That compression corresponds to a zero determinant for the matrix.

\begin{equation*}
  det(\mtx{A} - \lambda\mtx{I}) = 0
\end{equation*}

To be concrete, let's say your matrix $\mtx{A}$ has columns $(2, 1)$ and
$(2, 3)$, and think about subtracting off a variable amount $\lambda$.

\begin{equation*}
  det\left(\begin{bmatrix}
    2 - \lambda & 2 \\
    1 & 3 - \lambda
  \end{bmatrix}\right) = 0
\end{equation*}

The goal is to find a value of $\lambda$ that will make this determinant zero
meaning the tweaked transformation compresses space into a lower dimension. In
this case, that value is $\lambda = 1$. Of course, if we had chosen some other
matrix, the eigenvalue might not necessarily be $1$.

This is kind of a lot, but let's unravel what this is saying. When
$\lambda = 1$, the matrix $\mtx{A} - \lambda\mtx{I}$ compresses space onto a
line. That means there's a nonzero vector $\mtx{v}$ such that
$(\mtx{A} - \lambda\mtx{I})\mtx{v}$ equals the zero vector.

\begin{equation*}
  (\mtx{A} - \lambda\mtx{I})\mtx{v} = \mtx{0}
\end{equation*}

Remember, we care about that because it means
$\mtx{A}\mtx{v} = \lambda\mtx{v}$, which you can read off as saying that the
vector $\mtx{v}$ is an eigenvector of $\mtx{A}$ staying on its own span during
the transformation $\mtx{A}$. For the following example

\begin{equation*}
  \begin{bmatrix}
    2 & 2 \\
    1 & 3
  \end{bmatrix}
  \mtx{v} = 1\mtx{v}
\end{equation*}

the corresponding eigenvalue is $1$, so $\mtx{v}$ would actually just stay fixed
in place.

To summarize our line of reasoning:

\begin{align*}
  \mtx{A}\mtx{v} &= \lambda\mtx{v} \\
  \mtx{A}\mtx{v} - \lambda\mtx{I}\mtx{v} &= \mtx{0} \\
  (\mtx{A} - \lambda\mtx{I})\mtx{v} &= \mtx{0} \\
  det(\mtx{A} - \lambda\mtx{I}) &= \mtx{0}
\end{align*}

To see this in action, let's visit the example from the start with a matrix
whose columns are $(3, 0)$ and $(1, 2)$. To determine if a value $\lambda$ is an
eigenvalue, subtract it from the diagonals of this matrix and compute the
determinant.

\begin{align*}
  \begin{bmatrix}
    3 & 1 \\
    0 & 2
  \end{bmatrix} & \\
  \begin{bmatrix}
    3 - \lambda & 1 \\
    0 & 2 - \lambda
  \end{bmatrix} & \\
  det\left(\begin{bmatrix}
    3 - \lambda & 1 \\
    0 & 2 - \lambda
  \end{bmatrix}\right) &=
  (3 - \lambda)(2 - \lambda) - 1 \cdot 0 \\
  &= (3 - \lambda)(2 - \lambda)
\end{align*}

We get a certain quadratic polynomial in $\lambda$. Since $\lambda$ can only be
an eigenvalue if this determinant happens to be zero, you can conclude that the
only possible eigenvalues are $\lambda = 2$ and $\lambda = 3$.

To determine what the eigenvectors are that actually have one of these
eigenvalues, say $\lambda = 2$, plug in that value of $\lambda$ to the matrix
and then solve for which vectors this diagonally altered matrix sends to zero.

\begin{equation*}
  \begin{bmatrix}
    3 - 2 & 1 \\
    0 & 2 - 2
  \end{bmatrix}
  \begin{bmatrix}
    x \\
    y
  \end{bmatrix} =
  \begin{bmatrix}
    0 \\
    0
  \end{bmatrix}
\end{equation*}

If you computed this the way you would any other linear system, you'd see that
the solutions are all the vectors on the diagonal line spanned by $(-1, 1)$.
This corresponds to the fact that the unaltered matrix has the effect of
stretching all those vectors by a factor of $2$.

\subsection{Transformations with no eigenvectors}

A 2D transformation doesn't have to have eigenvectors. For example, consider a
rotation by $90 \degree$.

\begin{equation*}
  \begin{bmatrix}
    0 & -1 \\
    1 & 0
  \end{bmatrix}
\end{equation*}

This doesn't have any eigenvectors since it rotates every vector off its own
span. If you actually tried computing the eigenvalues of a rotation like this,
notice what happens.

\begin{align*}
  \begin{bmatrix}
    0 & -1 \\
    1 & 0
  \end{bmatrix} & \\
  \begin{bmatrix}
    - \lambda & -1 \\
    1 & -\lambda
  \end{bmatrix} & \\
  det\left(\begin{bmatrix}
    -\lambda & -1 \\
    1 & -\lambda
  \end{bmatrix}\right) &=
  (-\lambda)(-\lambda) - (-1)(1) \\
  &= \lambda^2 + 1 = 0
\end{align*}

The only roots of that polynomial are the imaginary numbers $i$ and $-i$. The
fact that there are no real number solutions indicates that there are no
eigenvectors.

\begin{remark}
  Interestingly though, the fact that multiplication by $i$ in the complex plane
  looks like a $90 \degree$ rotation is related to the fact that $i$ is an
  eigenvalue of this transformation of 2D real vectors. The specifics of this
  are out of scope, but note that eigenvalues which are complex numbers
  generally correspond to some kind of rotation in the transformation.
\end{remark}

\subsection{Repeated eigenvalues}

Another interesting example is a shear which fixes $\hat{i}$ in place and moves
$\hat{j}$ over by $1$.

\begin{equation*}
  \begin{bmatrix}
    1 & 1 \\
    0 & 1
  \end{bmatrix}
\end{equation*}

All the vectors on the x-axis are eigenvectors with eigenvalue $1$. In fact,
these are the only eigenvectors.

\begin{align*}
  det\left(\begin{bmatrix}
    1 - \lambda & 1 \\
    0 & 1 - \lambda
  \end{bmatrix}\right) &= (1 - \lambda)(1 - \lambda) = 0 \\
  (1 - \lambda)^2 &= 0
\end{align*}

The only root of this expression is $\lambda = 1$.

\subsection{Transformations with larger eigenvector spans}

Keep in mind it's also possible to have just one eigenvalue but with more than
just a line full of eigenvectors. A simple example is a matrix that scales
everything by $2$.

\begin{equation*}
  \begin{bmatrix}
    2 & 0 \\
    0 & 2
  \end{bmatrix}
\end{equation*}

The only eigenvalue is $2$, but every vector in the plane gets to be an
eigenvector with that eigenvalue.

\begin{remark}
  See the corresponding \textit{Essence of linear algebra} video for a more
  visual presentation (17 minutes)
  \cite{bib:linalg_eigenvectors_and_eigenvalues}.
\end{remark}

\section{Miscellaneous notation}

This book works with two-dimensional matrices in the sense that they only have
rows and columns. The dimensionality of these matrices is specified by row
first, then column. For example, a matrix with two rows and three columns would
be a two-by-three matrix. A square matrix has the same number of rows as
columns. Matrices commonly use capital letters while vectors use lowercase
letters.

The matrix $\mtx{I}$ is known as the identity matrix, which is a square matrix
with ones along its diagonal and zeroes elsewhere. For example

\begin{equation*}
  \begin{bmatrix}
    1 & 0 & 0 \\
    0 & 1 & 0 \\
    0 & 0 & 1
  \end{bmatrix}
\end{equation*}

The matrix denoted by $\mtx{0}_{m \times n}$ is a matrix filled with zeroes with
$m$ rows and $n$ columns.

The $^T$ in $\mtx{A}^T$ denotes transpose, which flips the matrix across its
diagonal such that the rows become columns and vice versa.

The $^\dagger$ in $\mtx{B}^\dagger$ denotes the Moore-Penrose pseudoinverse
given by $\mtx{B}^\dagger = (\mtx{B}^T\mtx{B})^{-1}\mtx{B}^T$. The pseudoinverse
is used when the matrix is nonsquare and thus not invertible to produce a close
approximation of an inverse.


\chapterimage{ss-representation.jpg}{OPERS field at UCSC}

\chapter{State-space representation}

\begin{remark}
  Chapters from here on use Python Control to demonstrate the concepts discussed
  and perform the complex math required. See appendix
  \ref{ch:app-installing-python-control} for how to install it.
\end{remark}

State-space representation models \glspl{system} as a set of \gls{state}, input,
and output variables related by first-order differential equations that describe
how the system's state changes over time given the current \glspl{state} and
inputs.

\section{Benefits over classical output-based control}

The state-space method provides a more convenient and compact way to model and
analyze \glspl{system} with multiple inputs and outputs. For a system with $p$
inputs and $q$ outputs, we would have to write $q \times p$ Laplace transforms
to represent it. Not only is the resulting algebra unwieldy, but it only works
for linear systems. Including nonzero initial conditions complicates the algebra
even more. State-space representation uses the time domain instead of the
Laplace domain, so it can model nonlinear systems\footnote{This book focuses on
analysis and control of linear systems. See appendix \ref{ch:nonlinear-control}
for more on nonlinear control.} and trivially supports nonzero initial
conditions.

Students are still taught classical control first because it provides a
framework within which to understand the results we get from the fancy
mathematical machinery of modern control.

\section{What is a linear dynamical system?}

A dynamical system is a system whose motion varies according to a set of
differential equations. A dynamical system is considered \textit{linear} if the
differential equations describing its dynamics consist only of linear operators.
Linear operators are things like constant gain multiplications, derivatives, and
integrals. You can define reasonably accurate linear models for pretty much
everything you'll see in FRC with just those relations.

But let's say you have a DC brushed motor hooked up to a power supply and you
applied a constant voltage to it from rest. The motor approaches a steady-state
angular velocity, but the shape of the angular velocity curve over time isn't a
line. In fact, it's a decaying exponential curve akin to

\begin{equation*}
  \omega = \omega_{max}\left(1 - e^{-t}\right)
\end{equation*}

where $\omega$ is the angular velocity and $\omega_{max}$ is the maximum angular
velocity. If the motor behaves linearly, then why is this?

Remember that linearity refers to a system's equations of motion, not its
time-domain response. The equation defining the motor's change in angular
velocity over time looks like

\begin{equation*}
  \dot{\omega} = -a\omega + bV
\end{equation*}

where $\dot{\omega}$ is the derivative of $\omega$ with respect to time, $V$ is
the input voltage, and $a$ and $b$ are constants specific to the motor. This
equation, unlike the one shown before, is actually linear because it only
consists of multiplications and additions relating the input $V$ and current
state $\omega$.

Also of note is that the relation between the input voltage and the angular
velocity of the output shaft is a linear regression. You'll see why if you model
a DC brushed motor as a voltage source and generator producing back-EMF (in the
equation above, $bV$ corresponds to the voltage source and $-a\omega$
corresponds to the back-EMF). As you increase the input voltage, the back-EMF
increases linearly with the motor's angular velocity. If there was a friction
term that varied with the angular velocity squared (air resistance is one
example), the relation from input to output would be a curve. Friction that
scales with just the angular velocity would result in a lower maximum angular
velocity, but because that term can be lumped into the back-EMF term, the
response is still linear.

\section{What is state-space?}

Recall from last chapter that 2D space has two axes, $x$ and $y$. We represent
locations within this space as a pair of numbers packaged in a vector, and each
coordinate is a measure of how far to move along the corresponding axis.
State-space is a Cartesian coordinate system with an axis for each \gls{state}
variable, and we represent locations within it the same way we do for 2D space:
with a list of numbers in a vector. Each element in the vector corresponds to a
\gls{state} of the \gls{system}.

In addition to the \gls{state}, inputs and outputs are represented as vectors.
Since the mapping from the current states and inputs to the change in state is a
system of equations, it's natural to write it in matrix form.

\section{State-space notation}

Below are the continuous and discrete versions of state-space notation.

\begin{definition}[State-space notation]%
  \begin{align}
    \dot{\mtx{x}} &= \mtx{A}\mtx{x} + \mtx{B}\mtx{u} \label{eq:ss_ctrl_x} \\
    \mtx{y} &= \mtx{C}\mtx{x} + \mtx{D}\mtx{u} \label{eq:ss_ctrl_y} \\
    \nonumber \\
    \mtx{x}_{k+1} &= \mtx{A}\mtx{x}_k + \mtx{B}\mtx{u}_k \label{eq:ssz_ctrl_x} \\
    \mtx{y}_{k+1} &= \mtx{C}\mtx{x}_k + \mtx{D}\mtx{u}_k \label{eq:ssz_ctrl_y}
  \end{align}

  \begin{figurekey}
    \begin{tabulary}{\linewidth}{LLLL}
      $\mtx{A}$ & system matrix      & $\mtx{x}$ & state vector \\
      $\mtx{B}$ & input matrix       & $\mtx{u}$ & input vector \\
      $\mtx{C}$ & output matrix      & $\mtx{y}$ & output vector \\
      $\mtx{D}$ & feedthrough matrix &  &  \\
    \end{tabulary}
  \end{figurekey}
\end{definition}

\begin{booktable}
  \begin{tabular}{|ll|ll|}
    \hline
    \rowcolor{headingbg}
    \textbf{Matrix} & \textbf{Rows $\times$ Columns} &
    \textbf{Matrix} & \textbf{Rows $\times$ Columns} \\
    \hline
    $\mtx{A}$ & states $\times$ states & $\mtx{x}$ & states $\times$ 1 \\
    $\mtx{B}$ & states $\times$ inputs & $\mtx{u}$ & inputs $\times$ 1 \\
    $\mtx{C}$ & outputs $\times$ states & $\mtx{y}$ & outputs $\times$ 1 \\
    $\mtx{D}$ & outputs $\times$ inputs &  &  \\
    \hline
  \end{tabular}
  \caption{State-space matrix dimensions}
  \label{tab:ss_matrix_dims}
\end{booktable}

In the continuous case, the change in state and the output are linear
combinations of the state vector and the input vector. The $\mtx{A}$ and
$\mtx{B}$ matrices are used to map the state vector $\mtx{x}$ and the input
vector $\mtx{u}$ to a change in the state vector $\dot{\mtx{x}}$. The $\mtx{C}$
and $\mtx{D}$ matrices are used to map the state vector $\mtx{x}$ and the input
vector $\mtx{u}$ to an output vector $\mtx{y}$.

\section{Controllability}
\index{Controller design!controllability}

State controllability implies that it is possible -- by admissible inputs -- to
steer the \glspl{state} from any initial value to any final value within some
finite time window.

\begin{theorem}[Controllability]
  A continuous \gls{time-invariant} linear state-space \gls{model} is
  controllable if and only if

  \begin{equation}
    \text{rank} \left(
    \begin{bmatrix}
      \mtx{B} & \mtx{A}\mtx{B} & \mtx{A}^2\mtx{B} & \cdots &
      \mtx{A}^{n-1}\mtx{B}
    \end{bmatrix}
    \right) = n
    \label{eq:ctrl_rank}
  \end{equation}

  where rank is the number of linearly independent rows in a matrix and $n$ is
  the number of \gls{state} variables.
\end{theorem}

The matrix in equation (\ref{eq:ctrl_rank}) being rank-deficient means the
inputs cannot apply transforms along all axes in the state-space; the
transformation the matrix represents is collapsed into a lower dimension.

\section{Observability}
\index{Controller design!observability}

Observability is a measure for how well internal \glspl{state} of a \gls{system}
can be inferred by knowledge of its external outputs. The observability and
controllability of a \gls{system} are mathematical duals (i.e., as
controllability proves that an input is available that brings any initial
\gls{state} to any desired final \gls{state}, observability proves that knowing
an output trajectory provides enough information to predict the initial
\gls{state} of the \gls{system}).

\begin{theorem}[Observability]
  A continuous \gls{time-invariant} linear state-space \gls{model} is observable
  if and only if

  \begin{equation}
    \text{rank} \left(
    \begin{bmatrix}
      C \\
      CA \\
      \vdots \\
      CA^{n-1}
    \end{bmatrix}\right) = n \label{eq:obsv_rank}
  \end{equation}

  where rank is the number of linearly independent rows in a matrix and $n$ is
  the number of \gls{state} variables.
\end{theorem}

The matrix in equation (\ref{eq:obsv_rank}) being rank-deficient means the
outputs do not contain contributions from every state. That is, not all states
are mapped to a linear commbination in the output. Therefore, the outputs alone
are insufficient to estimate all the states.

\section{Eigenvalues in state-space}
\index{Stability!eigenvalues}

The eigenvalues of the system matrix can be used to determine the stability of a
\gls{system}.

We'd like to know whether the \gls{system} defined by equation
(\ref{eq:ssz_ctrl_x}) operating with the \gls{control law}
$\mtx{u}_k = \mtx{K}(\mtx{r}_k - \mtx{x}_k)$ converges to the \gls{reference}
$\mtx{r}_k$.

\begin{align}
  \mtx{x}_{k+1} &= \mtx{A}\mtx{x}_k + \mtx{B}\mtx{u}_k \nonumber \\
  \mtx{x}_{k+1} &= \mtx{A}\mtx{x}_k + \mtx{B}(\mtx{K}(\mtx{r}_k - \mtx{x}_k))
    \nonumber \\
  \mtx{x}_{k+1} &= \mtx{A}\mtx{x}_k + \mtx{B}\mtx{K}\mtx{r}_k -
    \mtx{B}\mtx{K}\mtx{x}_k \nonumber \\
  \mtx{x}_{k+1} &= \mtx{A}\mtx{x}_k - \mtx{B}\mtx{K}\mtx{x}_k +
    \mtx{B}\mtx{K}\mtx{r}_k \nonumber \\
  \mtx{x}_{k+1} &= (\mtx{A} - \mtx{B}\mtx{K})\mtx{x}_k +
    \mtx{B}\mtx{K}\mtx{r}_k \label{eq:ctrl_eig_calc}
\end{align}

For equation (\ref{eq:ctrl_eig_calc}) to have a bounded output, the eigenvalues
of $\mtx{A} - \mtx{B}\mtx{K}$ must be within the unit circle.

This derivation can be performed for a \gls{state} estimator as well to
determine whether the \gls{state} estimate converges to the true \gls{state}.
Plugging equation (\ref{eq:z_obsv_y}) into equation (\ref{eq:z_obsv_x}) gives

\begin{align*}
  \hat{\mtx{x}}_{k+1} &= \mtx{A}\hat{\mtx{x}}_k + \mtx{B}\mtx{u}_k +
    \mtx{L} (\mtx{y}_k - \hat{\mtx{y}}_k) \\
  \hat{\mtx{x}}_{k+1} &= \mtx{A}\hat{\mtx{x}}_k + \mtx{B}\mtx{u}_k +
    \mtx{L} (\mtx{y}_k - (\mtx{C}\hat{\mtx{x}}_k + \mtx{D}\mtx{u}_k)) \\
  \hat{\mtx{x}}_{k+1} &= \mtx{A}\hat{\mtx{x}}_k + \mtx{B}\mtx{u}_k +
    \mtx{L} (\mtx{y}_k - \mtx{C}\hat{\mtx{x}}_k - \mtx{D}\mtx{u}_k)
\end{align*}

Plugging in equation (\ref{eq:ssz_ctrl_y}) gives

\begin{align*}
  \hat{\mtx{x}}_{k+1} &= \mtx{A}\hat{\mtx{x}}_k + \mtx{B}\mtx{u}_k +
    \mtx{L}((\mtx{C}\mtx{x}_k + \mtx{D}\mtx{u}_k) - \mtx{C}\hat{\mtx{x}}_k -
    \mtx{D}\mtx{u}_k) \\
  \hat{\mtx{x}}_{k+1} &= \mtx{A}\hat{\mtx{x}}_k + \mtx{B}\mtx{u}_k +
    \mtx{L}(\mtx{C}\mtx{x}_k + \mtx{D}\mtx{u}_k - \mtx{C}\hat{\mtx{x}}_k -
    \mtx{D}\mtx{u}_k) \\
  \hat{\mtx{x}}_{k+1} &= \mtx{A}\hat{\mtx{x}}_k + \mtx{B}\mtx{u}_k +
    \mtx{L}(\mtx{C}\mtx{x}_k - \mtx{C}\hat{\mtx{x}}_k) \\
  \hat{\mtx{x}}_{k+1} &= \mtx{A}\hat{\mtx{x}}_k + \mtx{B}\mtx{u}_k +
    \mtx{L}\mtx{C}(\mtx{x}_k - \hat{\mtx{x}}_k)
\end{align*}

Let $E_k = \mtx{x}_k - \hat{\mtx{x}}_k$ be the error in the estimate
$\hat{\mtx{x}}_k$.

\begin{equation*}
  \hat{\mtx{x}}_{k+1} = \mtx{A}\hat{\mtx{x}}_k + \mtx{B}\mtx{u}_k +
    \mtx{L}\mtx{C}\mtx{E}_k
\end{equation*}

Subtracting this from equation (\ref{eq:ssz_ctrl_x}) gives

\begin{align}
  \mtx{x}_{k+1} - \hat{\mtx{x}}_{k+1} &= \mtx{A}\mtx{x}_k + \mtx{B}\mtx{u}_k -
    (\mtx{A}\hat{\mtx{x}}_k + \mtx{B}\mtx{u}_k +
     \mtx{L}\mtx{C}\mtx{E}_k) \nonumber \\
  \mtx{E}_{k+1} &= \mtx{A}\mtx{x}_k + \mtx{B}\mtx{u}_k -
    (\mtx{A}\hat{\mtx{x}}_k + \mtx{B}\mtx{u}_k + \mtx{L}\mtx{C}\mtx{E}_k)
    \nonumber \\
  \mtx{E}_{k+1} &= \mtx{A}\mtx{x}_k + \mtx{B}\mtx{u}_k -
    \mtx{A}\hat{\mtx{x}}_k - \mtx{B}\mtx{u}_k - \mtx{L}\mtx{C}\mtx{E}_k
    \nonumber \\
  \mtx{E}_{k+1} &= \mtx{A}\mtx{x}_k - \mtx{A}\hat{\mtx{x}}_k -
    \mtx{L}\mtx{C}\mtx{E}_k \nonumber \\
  \mtx{E}_{k+1} &= \mtx{A}(\mtx{x}_k - \hat{\mtx{x}}_k) -
    \mtx{L}\mtx{C}\mtx{E}_k \nonumber \\
  \mtx{E}_{k+1} &= \mtx{A}\mtx{E}_k - \mtx{L}\mtx{C}\mtx{E}_k \nonumber \\
  \mtx{E}_{k+1} &= (\mtx{A} - \mtx{L}\mtx{C})\mtx{E}_k \label{eq:obsv_eig_calc}
\end{align}

For equation (\ref{eq:obsv_eig_calc}) to have a bounded output, the eigenvalues
of $\mtx{A} - \mtx{L}\mtx{C}$ must be within the unit circle. These eigenvalues
represent how fast the estimator converges to the true state of the given
\gls{model}. A fast estimator converges quickly while a slow estimator avoids
amplifying noise in the measurements used to produce a state estimate.

The effect of noise can be seen if it is modeled
\glslink{stochastic process}{stochastically} as

\begin{equation*}
  \hat{\mtx{x}}_{k+1} = \mtx{A}\hat{\mtx{x}}_k + \mtx{B}\mtx{u}_k +
    \mtx{L} ((\mtx{y}_k + \mtx{\nu}_k) - \hat{\mtx{y}}_k) \\
\end{equation*}

where $\mtx{\nu}_k$ is the measurement noise. Rearranging this equation yields

\begin{align*}
  \hat{\mtx{x}}_{k+1} &= \mtx{A}\hat{\mtx{x}}_k + \mtx{B}\mtx{u}_k +
    \mtx{L} (\mtx{y}_k - \hat{\mtx{y}}_k + \mtx{\nu}_k) \\
  \hat{\mtx{x}}_{k+1} &= \mtx{A}\hat{\mtx{x}}_k + \mtx{B}\mtx{u}_k +
    \mtx{L} (\mtx{y}_k - \hat{\mtx{y}}_k) + \mtx{L}\mtx{\nu}_k
\end{align*}

As $\mtx{L}$ increases, the measurement noise is amplified.

In summary, a controller is stable if the eigenvalues of
$\mtx{A} - \mtx{B}\mtx{K}$ are within the unit circle, and an estimator is
stable if the eigenvalues of $\mtx{A} - \mtx{L}\mtx{C}$ are within the unit
circle.

As stated before, the controller and estimator are dual problems. Controller
gains can be found assuming perfect estimator (i.e., perfect knowledge of all
\glspl{state}). Estimator gains can be found assuming an accurate \gls{model}
and a controller with perfect \gls{tracking}.

\chapterimage{going-digital.jpg}

\chapter{Going digital}

The complex plane discussed so far deals with continuous \glspl{system}. In
decades past, \glspl{plant} and controllers were implemented using analog
electronics, which are continuous in nature. Nowadays, microprocessors can be
used to achieve cheaper, less complex controller designs.
\glslink{discretization}{Discretization} converts the continuous model we've
worked with so far from a set of differential equations like

\begin{equation}
  \dot{x} = x - 3 \label{eq:differential_equ_example}
\end{equation}

to a set of difference equations like

\begin{equation}
  x_{k+1} = x_k + (x_k - 3) \Delta T \label{eq:difference_equ_example}
\end{equation}

where the difference equation is run a some update rate denoted by $T$,
$\Delta T$, or sometimes sloppily as $dt$.

While higher order terms of a differential equation are derivatives of the state
variable (e.g., $\ddot{x}$ in relation to equation
(\ref{eq:differential_equ_example})), higher order terms of a difference
equation are delayed copies of the state variable (e.g., $x_{k-1}$ with respect
to $x_k$ in equation (\ref{eq:difference_equ_example})).

\section{Phase loss}

However, \gls{discretization} has drawbacks. Since a microcontroller performs
discrete steps, there is a sample delay that introduces phase loss in the
controller. Phase loss is the reduction of phase margin (see section
\ref{sec:gain-phase-margin}) that occurs in digital implementations of feedback
controllers from sampling the continuous system at discrete time intervals. As
the sample rate of the controller decreases, the phase margin decreases rapidly
and will lead to instability if the phase margin reaches zero. Large amounts of
phase loss can make a stable controller in the continuous domain become unstable
in the discrete domain. Here are a few ways to combat this.

\begin{itemize}
  \item Run the controller with a high sample rate.
  \item Designing the controller in the analog domain with enough phase margin
    to compensate for any phase loss that occurs as part of discretization.
  \item Convert the \gls{plant} to the digital domain and design the controller
    completely in the digital domain.
\end{itemize}

\section{s-plane to z-plane}

Transfer functions are converted to impulse responses using the Z-transform. The
s-plane's LHP maps to the inside of a unit circle in the z-plane. Table
\ref{tab:s-plane2z-plane} contains a few common points.

\begin{table}
  \renewcommand{\arraystretch}{1.3}
  \centering
  \begin{tabular}{|cc|}
    \hline
    \rowcolor{headingbg}
    \textbf{s-plane} & \textbf{z-plane} \\
    \hline
    $(0, 0)$ & $(0, 1)$ \\
    imaginary axis & edge of unit circle \\
    $(0, -\infty)$ & $(0, 0)$ \\
    \hline
  \end{tabular}
  \caption{Mapping from s-plane to z-plane}
  \label{tab:s-plane2z-plane}
\end{table}

You may notice that poles can be placed at $(0, 0)$ in the z-plane. This is
known as a deadbeat controller. An $\rm N^{th}$ order deadbeat controller decays
to the \gls{reference} in N timesteps. While this sounds great, there are other
considerations like actuation effort, \gls{robustness}, and
\gls{noise immunity}. These will be discussed in detail with LQR and LQE.

\section{Discretization methods}

Discretization is done using a zero-order hold. That is, the system state is
only updated at discrete intervals and the value is held constant between
samples. The exact method of applying this uses the matrix exponential, but this
can be computationaly expensive. Instead, approximations such as the following
are used.

\begin{enumerate}
  \item Forward Euler method. This is defined as
    $y_{n+1} = y_n + f(t_n, y_n) \Delta t$.
  \item Backward Euler method. This is defined as
    $y_{n+1} = y_n + f(t_{n+1}, y_{n+1}) \Delta t$.
  \item Bilinear transform. The first-order bilinear approximation is
    $s = \frac{2}{T} \frac{1 - z^{-1}}{1 + z^{-1}}$.
\end{enumerate}

where $T$ is the sample period for the discrete system. Since these are
approximations, there is distortion between the real discrete system's poles and
the approximate poles. For fast-changing systems, this distortion can quickly
lead to instability.

The exact method of computing the zero-order hold for state-space is shown
below.

\begin{theorem}[Zero-order hold for state-space]
  \begin{align}
    \mtx{A}_d &= e^{\mtx{A}_c T} \\
    \mtx{B}_d &= \int_0^T e^{\mtx{A}_c \tau} d\tau \mtx{B}_c \\
    \mtx{C}_d &= \mtx{C}_c \\
    \mtx{D}_d &= \mtx{D}_c
  \end{align}

  where a subscript of $d$ denotes discrete, a subscript of $c$ denotes
  continuous, $T$ is the sample period for the discrete system, and
  $e^{\mtx{A}_c T}$ is the matrix exponential of $\mtx{A}_c$.
\end{theorem}

See the Wikipedia page on discretization for more \cite{bib:discretization}.

\subsection{Computing the matrix exponential}

The matrix exponential is typically solved with a computer. We will use Python
Control's \texttt{sample\_system()} function for this purpose later.

\begin{definition}[Matrix exponential]
  Let $\mtx{X}$ be an $n \times n$ matrix. The exponential of $\mtx{X}$ denoted
  by $e^{\mtx{X}}$ is the $n \times n$ matrix given by the following power
  series.

  \begin{equation}
    e^{\mtx{X}} = \sum_{k=0}^\infty \frac{1}{k!} \mtx{X}^k
  \end{equation}

  where $\mtx{X}^0$ is defined to be the identity matrix $\mtx{I}$ with the same
  dimensions as $\mtx{X}$.
\end{definition}

\chapterimage{ss-controllers.jpg}{OPERS field at UCSC}

\chapter{State-space controllers}

When we want to command a \gls{system} to a set of states, we design a
controller with certain \glspl{control law} to do it. PID controllers use the
system outputs with proportional, integral, and derivative \glspl{control law}.
In state-space, we also have knowledge of the system states so we can do better.

\section{From PID to model-based control}
\index{PID control}

As mentioned before, controls engineers have a more general framework to
describe control theory than just PID control. PID controller designers are
focused on fiddling with controller parameters relating to the current, past,
and future error rather than the underlying system states. Integral control is a
commonly used tool, and some people use integral action as the majority of the
control action. While this approach works in a lot of situations, it is an
incomplete view of the world.

Model-based control has a completely different mindset. Controls designers using
model-based control care about developing an accurate model of the system, then
driving the states they care about to zero (or to a \gls{reference}). Integral
control is added with $u_{error}$ estimation if needed to handle model
uncertainty, but we prefer not to use it because its response is hard to tune
and some of its destabilizing dynamics aren't visible during simulation.

\section{Closed-loop controller}

With the \gls{control law} $\mtx{u} = \mtx{K}(\mtx{r} - \mtx{x})$, we can derive
the closed-loop state-space equations. We'll discuss where this control law
comes from in subsection \ref{subsec:pareto_optimal_curve}.

First is the state update equation. Substitute the control law into equation
(\ref{eq:ss_ctrl_x}).

\begin{align}
  \dot{\mtx{x}} &= \mtx{A}\mtx{x} + \mtx{B}\mtx{K}(\mtx{r} - \mtx{x}) \nonumber
    \\
  \dot{\mtx{x}} &= \mtx{A}\mtx{x} + \mtx{B}\mtx{K}\mtx{r} -
    \mtx{B}\mtx{K}\mtx{x} \nonumber \\
  \dot{\mtx{x}} &= (\mtx{A} - \mtx{B}\mtx{K})\mtx{x} + \mtx{B}\mtx{K}\mtx{r}
\end{align}

Now for the output equation. Substitute the control law into equation
(\ref{eq:ss_ctrl_y}).

\begin{align}
  \mtx{y} &= \mtx{C}\mtx{x} + \mtx{D}(\mtx{K}(\mtx{r} - \mtx{x})) \nonumber \\
  \mtx{y} &= \mtx{C}\mtx{x} + \mtx{D}\mtx{K}\mtx{r} - \mtx{D}\mtx{K}\mtx{x}
    \nonumber \\
  \mtx{y} &= (\mtx{C} - \mtx{D}\mtx{K})\mtx{x} + \mtx{D}\mtx{K}\mtx{r}
\end{align}

\begin{theorem}[Closed-loop state-space controller]
  \begin{align}
    \dot{\mtx{x}} &= (\mtx{A} - \mtx{B}\mtx{K})\mtx{x} + \mtx{B}\mtx{K}\mtx{r}
      \label{eq:s_ref_ctrl_x} \\
    \mtx{y} &= (\mtx{C} - \mtx{D}\mtx{K})\mtx{x} + \mtx{D}\mtx{K}\mtx{r}
      \label{eq:s_ref_ctrl_y}
  \end{align}

  \begin{figurekey}
    \begin{tabulary}{\linewidth}{LLLL}
      $\mtx{A}$ & system matrix      & $\mtx{x}$ & state vector \\
      $\mtx{B}$ & input matrix       & $\mtx{u}$ & input vector \\
      $\mtx{C}$ & output matrix      & $\mtx{y}$ & output vector \\
      $\mtx{D}$ & feedthrough matrix & $\mtx{r}$ & \gls{reference} vector \\
      $\mtx{K}$ & controller gain matrix &  &  \\
    \end{tabulary}
  \end{figurekey}
\end{theorem}

\begin{booktable}
  \begin{tabular}{|ll|ll|}
    \hline
    \rowcolor{headingbg}
    \textbf{Matrix} & \textbf{Rows $\times$ Columns} &
    \textbf{Matrix} & \textbf{Rows $\times$ Columns} \\
    \hline
    $\mtx{A}$ & states $\times$ states & $\mtx{x}$ & states $\times$ 1 \\
    $\mtx{B}$ & states $\times$ inputs & $\mtx{u}$ & inputs $\times$ 1 \\
    $\mtx{C}$ & outputs $\times$ states & $\mtx{y}$ & outputs $\times$ 1 \\
    $\mtx{D}$ & outputs $\times$ inputs & $\mtx{r}$ & states $\times$ 1 \\
    $\mtx{K}$ & inputs $\times$ states &  &  \\
    \hline
  \end{tabular}
  \caption{Controller matrix dimensions}
  \label{tab:ctrl_matrix_dims}
\end{booktable}

Instead of commanding the system to a state using the vector $\mtx{u}$ directly,
we can now specify a vector of desired states through $\mtx{r}$ and the system
will choose values of $\mtx{u}$ for us over time to make the system converge to
the reference. The rate of convergence and stability of the closed-loop system
can be changed by moving the poles via the eigenvalues of $\mtx{A} -
\mtx{B}\mtx{K}$. $\mtx{A}$ and $\mtx{B}$ are inherent to the system, but
$\mtx{K}$ can be chosen arbitrarily by the controller designer.

\section{Pole placement}
\index{Controller design!pole placement}

This is the practice of manually placing the poles of the closed-loop system to
produce a desired response. This is typically done with controllable canonical
form (see section \ref{sec:ctrl-canon}). This can be done with state observers
as well with observable canonical form (see section \ref{sec:obsv-canon}).

\section{LQR}
\index{Controller design!LQR}
\index{Optimal control!LQR}

While we could place the poles for our controller manually, we can do better by
using math to select them for us. ``LQR" stands for ``Linear-Quadratic
\glslink{regulator}{Regulator}". This method of controller design uses a
quadratic function for the cost-to-go defined as the sum of the error and
control effort over time for the linear system. LQR places the poles of the
controller such that this cost function is minimized.

The minimum of LQR's cost function is found by setting the derivative of the
cost function to zero and solving for the control law $\mtx{u}$. However, matrix
calculus is used instead of normal calculus to take the derivative.

Next, we'll define the cost function used by LQR and offer some advice for
tuning its parameters.

\subsection{Pareto optimal curve} \label{subsec:pareto_optimal_curve}

A system $\dot{\mtx{x}} = \mtx{A}\mtx{x} + \mtx{B}\mtx{u}$ has the cost function

\begin{equation*}
  J = \int\limits_0^\infty \left(\mtx{x}^T\mtx{Q}\mtx{x} +
    \mtx{u}^T\mtx{R}\mtx{u}\right) dt
\end{equation*}

where $J$ represents a tradeoff between \gls{state} excursion and control effort
with the weighting factors $\mtx{Q}$ and $\mtx{R}$. $\mtx{Q}$ and $\mtx{R}$
slide the cost along a Pareto optimal curve between state tracking and control
effort (see figure \ref{fig:pareto_curve}). This means that an improvement in
state tracking cannot be obtained without using more control effort to do so.
Also, a reduction in control effort cannot be obtained without sacrificing state
tracking performance. Pole placement, on the other hand, will have a cost
anywhere on or above/to the right of the Pareto optimal curve (no cost can be
inside the curve).

\begin{svg}{build/code/pareto_curve}
  \caption{Pareto optimal curve for LQR}
  \label{fig:pareto_curve}
\end{svg}

The feedback \gls{control law} that minimizes $J$, which we'll call the
``optimal control law", is shown in theorem \ref{thm:optimal_control_law}.

\begin{theorem}[Optimal control law]
  \begin{equation}
    \mtx{u} = -\mtx{K}\mtx{x}
  \end{equation}
  \label{thm:optimal_control_law}
\end{theorem}
\index{Optimal control!LQR!optimal control law}

This means that optimal control can be achieved with simply a set of
proportional gains on all the states. This \gls{control law} will make all
states converge to zero assuming the system is controllable. To converge to
nonzero states, a reference vector $\mtx{r}$ can be added to the state
$\mtx{x}$.

\begin{theorem}[Optimal control law with nonzero reference]
  \begin{equation}
    \mtx{u} = \mtx{K}(\mtx{r} - \mtx{x})
  \end{equation}
\end{theorem}

To use the control law, we need knowledge of the full state of the system. That
means we either have to measure all our states directly or estimate those we do
not measure.

See appendix \ref{ch:app-optimal-control-law-deriv} for how $\mtx{K}$ is
calculated in Python. If the result is finite, the controller is guaranteed to
be stable and \glslink{robustness}{robust} with a \gls{phase margin} of 60
degrees \cite{bib:lqr-phase-margin}.

\subsection{Bryson's rule}
\index{Optimal control!LQR!Bryson's rule}

The next obvious question is what values to choose for $\mtx{Q}$ and $\mtx{R}$.
With Bryson's rule, the $\mtx{Q}$ and $\mtx{R}$ matrices are chosen based on the
maximum acceptable value for each \gls{state} and actuator. The balance between
$\mtx{Q}$ and $\mtx{R}$ can be slid along the Pareto optimal curve using a
weighting factor $\rho$.

\begin{equation*}
  J = \int\limits_0^\infty \left(\rho \left[
    \left(\frac{x_1}{x_{1,max}}\right)^2 + \ldots +
    \left(\frac{x_n}{x_{n,max}}\right)^2\right] + \left[
    \left(\frac{u_1}{u_{1,max}}\right)^2 + \ldots +
    \left(\frac{u_n}{u_{n,max}}\right)^2\right]\right) dt
\end{equation*}

Small values of $\rho$ penalize control effort while large values of $\rho$
penalize \gls{state} excursions. Large values would be chosen in applications
like fighter jets where performance is necessary. Spacecrafts would use small
values to conserve their limited fuel supply.

\chapterimage{ss-observers.jpg}

\chapter{State-space observers}

State-space observers are used to estimate unmeasured \glspl{state}.

\section{Luenberger observer}

\begin{theorem}[Luenberger observer]
  \begin{align}
    \dot{\hat{\mtx{x}}} &= \mtx{A}\hat{\mtx{x}} + \mtx{B}\mtx{u} +
      \mtx{L} (\mtx{y} - \hat{\mtx{y}}) \label{eq:s_obsv_x} \\
    \hat{\mtx{y}} &= \mtx{C}\hat{\mtx{x}} + \mtx{D}\mtx{u} \label{eq:s_obsv_y}
  \end{align}

  \begin{align}
    \hat{\mtx{x}}_{k+1} &= \mtx{A}\hat{\mtx{x}}_k + \mtx{B}\mtx{u}_k +
      \mtx{L}(\mtx{y}_k - \hat{\mtx{y}}_k) \label{eq:z_obsv_x} \\
    \hat{\mtx{y}}_k &= \mtx{C}\hat{\mtx{x}}_k + \mtx{D}\mtx{u}_k
      \label{eq:z_obsv_y} \\ \nonumber
  \end{align}

  \begin{center}
    \renewcommand{\arraystretch}{1.3}
    \begin{tabulary}{\linewidth}{LLLL}
      $\mtx{A}$ & system matrix      & $\hat{\mtx{x}}$ & state estimate vector \\
      $\mtx{B}$ & input matrix       & $\mtx{u}$ & input vector \\
      $\mtx{C}$ & output matrix      & $\mtx{y}$ & output vector \\
      $\mtx{D}$ & feedthrough matrix & $\hat{\mtx{y}}$ & output estimate vector \\
      $\mtx{L}$ & estimator gain matrix & & \\
    \end{tabulary}
  \end{center}
\end{theorem}

\begin{table}[ht]
  \renewcommand{\arraystretch}{1.5}
  \centering
  \begin{tabular}{|ll|ll|}
    \hline
    \rowcolor{headingbg}
    \textbf{Matrix} & \textbf{Rows $\times$ Columns} &
    \textbf{Matrix} & \textbf{Rows $\times$ Columns} \\
    \hline
    $\mtx{A}$ & states $\times$ states & $\hat{\mtx{x}}$ & states $\times$ 1 \\
    $\mtx{B}$ & states $\times$ inputs & $\mtx{u}$ & inputs $\times$ 1 \\
    $\mtx{C}$ & outputs $\times$ states & $\mtx{y}$ & outputs $\times$ 1 \\
    $\mtx{D}$ & outputs $\times$ inputs & $\hat{\mtx{y}}$ & outputs $\times$ 1 \\
    $\mtx{L}$ & states $\times$ outputs &  &  \\
    \hline
  \end{tabular}
  \caption{Luenberger observer matrix dimensions}
  \label{tab:luenberger_matrix_dims}
\end{table}

Using an estimator forfeits the performance guarantees from earlier, but the
responses are still generally very good.

A Luenberger observer combines the prediction and update steps of an estimator.
To run them separately, use the following equations instead.

\begin{theorem}[Luenberger observer with separate predict/update]
  \begin{align}
    \text{Predict step} \nonumber \\
    \hat{\mtx{x}}_{k+1}^- &= \mtx{A}\hat{\mtx{x}}_k^- + \mtx{B}\mtx{u}_k \\
    \text{Update step} \nonumber \\
    \hat{\mtx{x}}_{k+1}^+ &= \hat{\mtx{x}}_{k+1}^- + \mtx{A}^{-1}\mtx{L}
      (\mtx{y}_k - \hat{\mtx{y}}_k) \\
    \hat{\mtx{y}}_k &= \mtx{C} \hat{\mtx{x}}_k^-
  \end{align}
\end{theorem}

See appendix \ref{subsec:app-luenberger-separate} for a derivation.

\section{LQE}

"LQE" stands for "Linear-Quadratic Estimator". Similar to LQR, it places the
estimator poles such that it minimizes the sum of squares of the error. The
Kalman filter is an example of one of these.

\section{Kalman filter}
\index{Kalman filter}

Read \url{http://www.bzarg.com/p/how-a-kalman-filter-works-in-pictures/} for a
graphical introduction to Kalman filters.

The following is a Kalman filter for the $k^{th}$ timestep. The current predict
step and current update step are shown.

\begin{theorem}[Kalman filter]
  \begin{align}
    \text{Predict step} \nonumber \\
    \hat{\mtx{x}}_{k+1}^- &= \mtx{\Phi}\hat{\mtx{x}}_k + \mtx{B} \mtx{u}_k
      \label{eq:pre1_x} \\
    \mtx{P}_{k+1}^- &= \mtx{\Phi} \mtx{P}_k^- \mtx{\Phi}^T +
      \mtx{\Gamma}\mtx{Q}\mtx{\Gamma}^T \\
    \text{Update step} \nonumber \\
    \mtx{K}_{k+1} &=
      \mtx{P}_{k+1}^- \mtx{H}^T (\mtx{H}\mtx{P}_{k+1}^- \mtx{H}^T +
      \mtx{R})^{-1} \\
    \hat{\mtx{x}}_{k+1}^+ &=
      \hat{\mtx{x}}_{k+1}^- + \mtx{K}_{k+1}(\mtx{y}_{k+1} -
      \mtx{H} \hat{\mtx{x}}_{k+1}^-) \label{eq:post1_x} \\
    \mtx{P}_{k+1}^+ &= (\mtx{I} - \mtx{K}_{k+1}\mtx{H})\mtx{P}_{k+1}^-
  \end{align}

  \begin{center}
    \renewcommand{\arraystretch}{1.3}
    \begin{tabulary}{\linewidth}{LLLL}
      $\mtx{\Phi}$ & system matrix & $\hat{\mtx{x}}$ & state estimate vector \\
      $\mtx{B}$ & input matrix            & $\mtx{u}$ & input vector \\
      $\mtx{H}$ & measurement matrix      & $\mtx{y}$ & output vector \\
      $\mtx{P}$ & error covariance matrix & $\mtx{Q}$ & process noise covariance
        matrix \\
      $\mtx{K}$ & Kalman gain matrix & $\mtx{R}$ & measurement noise covariance
        matrix \\
      $\mtx{\Gamma}$ & process noise intensity vector &
    \end{tabulary}
  \end{center}

  where a superscript of minus denotes \textit{a priori} and plus denotes
  \textit{a posteriori} estimate (before and after update respectively).
\end{theorem}

$\mtx{\Phi}$ is replaced with $\mtx{A}$ for continuous systems.

\begin{table}[ht]
  \renewcommand{\arraystretch}{1.5}
  \centering
  \begin{tabular}{|ll|ll|}
    \hline
    \rowcolor{headingbg}
    \textbf{Matrix} & \textbf{Rows $\times$ Columns} &
    \textbf{Matrix} & \textbf{Rows $\times$ Columns} \\
    \hline
    $\mtx{\Phi}$ & states $\times$ states & $\hat{\mtx{x}}$ & states $\times$ 1
      \\
    $\mtx{B}$ & states $\times$ inputs & $\mtx{u}$ & inputs $\times$ 1 \\
    $\mtx{H}$ & outputs $\times$ states & $\mtx{y}$ & outputs $\times$ 1 \\
    $\mtx{P}$ & states $\times$ states & $\mtx{Q}$ & states $\times$ states \\
    $\mtx{K}$ & states $\times$ outputs & $\mtx{R}$ & outputs $\times$ outputs
      \\
    $\mtx{\Gamma}$ & states $\times$ 1 &  &  \\
    \hline
  \end{tabular}
  \caption{Kalman filter matrix dimensions}
  \label{tab:kf_matrix_dims}
\end{table}

Unknown states in a Kalman filter are generally represented by a Wiener
(pronounced VEE-ner) process. This process has the property that its variance
increases linearly with time $t$. We'll skip all the probability derivations
here, but given two data points with associated variances represented by
Gaussian distribution, the information can be optimally combined into a third
Gaussian distribution with a mean value and variance. The expected value of $x$
given a measurement $z_1$ is

\begin{equation}
  E[x|z_1] = \mu = \frac{\sigma_0^2}{\sigma_0^2 + \sigma^2}z_1 +
    \frac{\sigma^2}{\sigma_0^2 + \sigma^2}x_0
\end{equation}

The variance of $x$ given $z_1$ is

\begin{equation}
  E[(x - \mu)^2|z_1] = \frac{\sigma^2 \sigma_0^2}{\sigma_0^2 + \sigma^2}
\end{equation}

The expected value, which is also the maximum likelihood value, is the linear
combination of the prior expected (maximum likelihood) value and the
measurement. The expected value is a reasonable estimator of $x$.

\begin{align}
  \hat{x} &= E[x|z_1] = \frac{\sigma_0^2}{\sigma_0^2 + \sigma^2}z_1 +
    \frac{\sigma^2}{\sigma_0^2 + \sigma^2}x_0 \\
  \hat{x} &= w_1 z_1 + w_2 x_0 \nonumber
\end{align}

Note that the weights $w_1$ and $w_2$ sum to 1. When the prior (i.e., prior
knowledge of state) is uninformative (a large variance)

\begin{align}
  w_1 &= \lim_{\sigma_0^2 \to 0} \frac{\sigma_0^2}{\sigma_0^2 + \sigma^2} = 0 \\
  w_2 &= \lim_{\sigma_0^2 \to 0} \frac{\sigma^2}{\sigma_0^2 + \sigma^2} = 1
\end{align}

and $\hat{x} = z_1$. That is, the weight is on the observations and the estimate
is equal to the measurement.

Let us assume we have a \gls{model} providing an almost exact prior for $x$. In
that case, $\sigma_0^2$ approaches 0 and

\begin{align}
  w_1 &= \lim_{\sigma_0^2 \to 0} \frac{\sigma_0^2}{\sigma_0^2 + \sigma^2} = 1 \\
  w_2 &= \lim_{\sigma_0^2 \to 0} \frac{\sigma^2}{\sigma_0^2 + \sigma^2} = 0
\end{align}

The Kalman filter uses this optimal fusion as the basis for its operation. See
the Wikipedia page on Kalman filters \cite{bib:kalman_filter} for a more
thorough explanation of the math involved and a derivation of the equations.

\subsection{Equations to model}

The following example system will be used to describe how to define and
initialize the matrices for a Kalman filter.

A robot is between two parallel walls. It starts driving from one wall to the
other at a velocity of $0.8 cm/s$ and uses ultrasonic sensors to provide noisy
measurements of the distances to the walls in front of and behind it. To
estimate the distance between the walls, we will define three states: robot
position, robot velocity, and distance between the walls.

\begin{align}
  x_{k+1} &= x_k + v_k \Delta T \\
  v_{k+1} &= v_k \\
  x_{k+1}^w &= x_k^w
\end{align}

This can be converted to the following state-space \gls{model}.

\begin{equation}
  \mtx{x}_k = \left[
  \begin{array}{c}
    x_k \\
    v_k \\
    x_k^w
  \end{array} \right]
\end{equation}

\begin{equation}
  \mtx{x}_{k+1} = \left[
  \begin{array}{ccc}
    1 & 1 & 0 \\
    0 & 0 & 0 \\
    0 & 0 & 1
  \end{array} \right] \mtx{x}_k + \left[
  \begin{array}{c}
    0 \\
    0.8 \\
    0
  \end{array} \right] + \left[
  \begin{array}{c}
    0 \\
    0.1 \\
    0
  \end{array} \right] w_k
\end{equation}

where the Gaussian random variable $w_k$ has zero mean and variance 1. The
observation \gls{model} is

\begin{equation}
  \mtx{y}_k = \left[
  \begin{array}{ccc}
    1 & 0 & 0 \\
    -1 & 0 & 1
  \end{array} \right] \mtx{x}_k + \theta_k
\end{equation}

where the covariance matrix of Gaussian measurement noise $\theta$ is a
$2 \times 2$ matrix with both diagonals $10 cm^2$.

The state vector is usually initialized using the first measurement or two. The
covariance matrix entries are assigned by calculating the covariance of the
expressions used when assigning the state vector. Let $k = 2$.

\begin{align}
  \mtx{Q} &= \left[1\right] \\
  \mtx{R} &= \left[
  \begin{array}{cc}
    10 & 0 \\
    0 & 10
  \end{array} \right] \\
  \hat{\mtx{x}} &= \left[
  \begin{array}{c}
    \mtx{y}_{k,1} \\
    (\mtx{y}_{k,1} - \mtx{y}_{k-1,1})/dt \\
    \mtx{y}_{k,1} + \mtx{y}_{k,2}
  \end{array} \right] \\
  \mtx{P} &= \left[
  \begin{array}{ccc}
    10 & 10/dt & 10 \\
    10/dt & 20/dt^2 & 10/dt \\
    10 & 10/dt & 20
  \end{array} \right]
\end{align}

\subsection{Initial conditions}

To fill in the $\mtx{P}$ matrix, we calculate the covariance of each combination
of state variables. The resulting value is a measure of how much those variables
are correlated. Due to how the covariance calculation works out, the covariance
between two variables is the sum of the variance of matching terms which aren't
constants multiplied by any constants the two have. If no terms match, the
variables are uncorrelated and the covariance is zero.

In $\mtx{P}_{11}$, the terms in $\mtx{x}_1$ correlate with itself. Therefore,
$\mtx{P}_{11}$ is $\mtx{x}_1$'s variance, or $\mtx{P}_{11} = 10$. For
$\mtx{P}_{21}$, One term correlates between $\mtx{x}_1$ and $\mtx{x}_2$, so
$\mtx{P}_{21} = \frac{10}{dt}$. The constants from each are simply multiplied
together. For $\mtx{P}_{22}$, both measurements are correlated, so the variances
add together. Therefore, $\mtx{P}_{22} = \frac{20}{dt^2}$. It continues in this
fashion until the matrix is filled up. Order doesn't matter for correlation, so
the matrix is symmetric.

\subsection{Selection of priors}

Choosing good priors is important for a well performing filter, even if little
information is known. This applies to both the measurement noise and the noise
\gls{model}. The act of giving a state variable a large variance means you know
something about the system. Namely, you aren't sure whether your initial guess
is close to the true state. If you make a guess and specify a small variance,
you are telling the filter that you are very confident in your guess. If that
guess is incorrect, it will take the filter a long time to move away from your
guess to the true value.

\subsection{Covariance selection}

While one could assume no correlation between the state variables and set the
covariance matrix entries to zero, this may not reflect reality. The Kalman
filter is still guarenteed to converge to the steady-state covariance after an
infinite time, but it will take longer than otherwise.

\subsection{Noise model selection}

We typically use a Gaussian distribution for the noise \gls{model} because the
sum of many independent random variables produces a normal distribution by the
central limit theorem. Kalman filters only require that the noise has a zero
mean. If the true value has an equal probability of being within a certain
range, use a uniform distribution instead. Each of these communicates
information regarding what you know about a system in addition to what you do
not.

\section{Steady-state error covariance matrix}

One may have noticed that the error covariance matrix can be updated
independently of the rest of the model. The error covariance matrix tends
toward a steady-state value, and this matrix can be obtained via the discrete
algebraic Ricatti equation. This can then be used to compute a steady-state
Kalman gain.

Snippet \ref{lst:kalman} computes the steady-state matrices for a Kalman
filter.

\begin{snippet}
  \caption{Steady-state Kalman gain and error covariance matrices calculation in Python}
  \label{lst:kalman}
  \includecode[Python]{code/wpicontrol/kalman.py}
\end{snippet}

\section{Kalman filter as Luenberger observer}

A Kalman filter can be represented as a Luenberger observer by letting
$\mtx{C} = \mtx{H}$ and $\mtx{L} = \mtx{A} \mtx{K}_k$ (see appendix
\ref{subsec:app-kalman-luenberger}). The eigenvalues of the Kalman filter are

\begin{equation}
  eig(\mtx{A}(\mtx{I} - \mtx{K}_k\mtx{H}))
\end{equation}

\part{Controls design/implementation}
\chapterimage{application-notes.jpg}{Trees by Interdisciplinary Sciences building at UCSC}

\chapter{Application notes}

Up to now, we've just been teaching what tools are available. Now, we'll go into
specifics on how to apply them and provide advice on certain applications.

\section{Model augmentation}

This section will teach various tricks for manipulating state-space models with
the goal of demystifying the matrix algebra at play. We will use the
augmentation techniques discussed here in the section on integral control.

Matrix augmentation is the process of appending rows or columns to a matrix. In
state-space, there are several common types of augmentation used: plant
augmentation, controller augmentation, and observer augmentation.

\subsection{Plant augmentation}

Plant augmentation is the process of adding a state to a model's state vector
and adding a corresponding row to the $\mtx{A}$ and $\mtx{B}$ matrices.

\subsection{Controller augmentation}

Controller augmentation is the process of adding a column to a controller's
$\mtx{K}$ matrix. This is often done in combination with plant augmentation to
add controller dynamics relating to a newly added state.

\subsection{Observer augmentation}

Observer augmentation is closely related to plant augmentation. In addition to
adding entries to the observer matrix $\mtx{L}$, the observer is using this
augmented plant for estimation purposes. This is better explained with an
example.

By augmenting the plant with a bias term with no dynamics (represented by zeroes
in its rows in $\mtx{A}$ and $\mtx{B}$, the observer will attempt to estimate a
value for this bias term that makes the model best reflect the measurements
taken of the real system. Note that we're not collecting any data on this bias
term directly; it's what's known as a hidden state. Rather than our inputs and
other states affecting it directly, the observer determines a value for it based
on what is most likely given the model and current measurements. We just tell
the plant what kind of dynamics the term has and the observer will estimate it
for us.

\subsection{Output augmentation}

Output augmentation is the process of adding rows to the $\mtx{C}$ matrix. This
is done to help the controls designer visualize the behavior of internal states
or other aspects of the system in MATLAB or Python Control. $\mtx{C}$ matrix
augmentation doesn't affect state feedback, so the designer has a lot of freedom
here. Noting that the output is defined as
$\mtx{y} = \mtx{C}\mtx{x} + \mtx{D}\mtx{u}$, The following row augmentations of
$\mtx{C}$ may prove useful. Of course, $\mtx{D}$ needs to be augmented with
zeroes as well in these cases to maintain the correct matrix dimensionality.

Since $\mtx{u} = -\mtx{K}\mtx{x}$, augmenting $\mtx{C}$ with $-\mtx{K}$ makes
the observer estimate the control input $\mtx{u}$ applied.

\begin{align*}
  \mtx{y} &= \mtx{C}\mtx{x} + \mtx{D}\mtx{u} \\
  \begin{bmatrix}
    \mtx{y} \\
    \mtx{u}
  \end{bmatrix} &=
  \begin{bmatrix}
    \mtx{C} \\
    -\mtx{K}
  \end{bmatrix}
  \mtx{x} +
  \begin{bmatrix}
    \mtx{D} \\
    \mtx{0}
  \end{bmatrix}
  \mtx{u}
\end{align*}

This works because $\mtx{K}$ has the same number of columns as states.

Various states can also be produced in the output with $\mtx{I}$ matrix
augmentation.

\section{Integral control}

A common way of implementing integral control is to add an additional state that
is the integral of the error of the variable intended to have zero steady-state
error.

There are two drawbacks to this method. First, there is integral windup on a
unit step input. That is, the integrator accumulates even if the system is
tracking the model correctly. The second is demonstrated by an example from
Jared Russell of FRC team 254. Say there is a position/velocity trajectory for
some plant to follow. Without integral control, one can calculate a desired
$\mtx{K}\mtx{x}$ to use as the reference input to the controller. As a result of
using both desired position and velocity, reference tracking is good. With
integral control added, the reference is always the desired position, but there
is no way to tell the controller the desired velocity.

Consider carefully whether integral control is necessary. One can get relatively
close without integral control, and integral adds all the issues listed above.
Below, it is assumed that the controls designer has determined that integral
control will be worth the inconvenience.

There are three methods FRC team 971 has used over the years:

\begin{enumerate}
  \item Augment the plant as described earlier. For an arm, one would add an
    ``integral of position" state.
  \item Add an integrator to the output of the controller, then estimate the
    control effort being applied. 971 has called this Delta U control. The
    upside is that it doesn't have the windup issue described above; the
    integrator only acts if the system isn't behaving like the model, which was
    the original intent. The downside is working with it is very confusing.
  \item Estimate an ``error" in the observer and compensate for it. This
    quantity is the difference between what was applied and what was observed to
    happen. To use it, you simply add it to your control input and it will
    converge. This is 971's primary method.
\end{enumerate}

We'll present the first and third methods since the third is strictly better
than the second.

\subsection{Plant augmentation}

We want to augment the system with an integral term that integrates the error
$\mtx{e} = \mtx{r} - \mtx{y} = \mtx{r} - \mtx{C}\mtx{x}$.

\begin{align*}
  \mtx{x}_I &= \int \mtx{e} \,dt \\
  \dot{\mtx{x}}_I &= \mtx{e} = \mtx{r} - \mtx{C}\mtx{x}
\end{align*}

The plant is augmented as

\begin{align*}
  \dot{\begin{bmatrix}
    \mtx{x} \\
    \mtx{x}_I
  \end{bmatrix}} &=
  \begin{bmatrix}
    \mtx{A} & \mtx{0} \\
    -\mtx{C} & \mtx{0}
  \end{bmatrix}
  \begin{bmatrix}
    \mtx{x} \\
    \mtx{x}_I
  \end{bmatrix} +
  \begin{bmatrix}
    \mtx{B} \\
    \mtx{0}
  \end{bmatrix}
  \mtx{u} +
  \begin{bmatrix}
    \mtx{0} \\
    \mtx{I}
  \end{bmatrix}
  \mtx{r} \\
  \dot{\begin{bmatrix}
    \mtx{x} \\
    \mtx{x}_I
  \end{bmatrix}} &=
  \begin{bmatrix}
    \mtx{A} & \mtx{0} \\
    -\mtx{C} & \mtx{0}
  \end{bmatrix}
  \begin{bmatrix}
    \mtx{x} \\
    \mtx{x}_I
  \end{bmatrix} +
  \begin{bmatrix}
    \mtx{B} & \mtx{0} \\
    \mtx{0} & \mtx{I}
  \end{bmatrix}
  \begin{bmatrix}
    \mtx{u} \\
    \mtx{r}
  \end{bmatrix}
\end{align*}

The controller is augmented as

\begin{align*}
  \mtx{u} &= \mtx{K} (\mtx{r} - \mtx{x}) - \mtx{K}_I\mtx{x}_I \\
  \mtx{u} &=
  \begin{bmatrix}
    \mtx{K} & \mtx{K}_I
  \end{bmatrix}
  \left(\begin{bmatrix}
    \mtx{r} \\
    \mtx{0}
  \end{bmatrix} -
  \begin{bmatrix}
    \mtx{x} \\
    \mtx{x}_I
  \end{bmatrix}\right)
\end{align*}

\subsection{U error estimation}

Let $u_{error}$ be the error in a system's input. The $u_{error}$ term is then
added to the system as follows.

\begin{align*}
  \dot{\mtx{x}} &= \mtx{A}\mtx{x} + \mtx{B}\left(\mtx{u} + u_{error}\right) \\
  \dot{\mtx{x}} &= \mtx{A}\mtx{x} + \mtx{B}\mtx{u} + \mtx{B}u_{error}
\end{align*}

For a multiple-output system, this would be

\begin{equation*}
  \dot{\mtx{x}} = \mtx{A}\mtx{x} + \mtx{B}\mtx{u} + \mtx{B}_{error}u_{error}
\end{equation*}

where $\mtx{B}_{error}$ is the column vector that maps $u_{error}$ to changes in
the rest of the state the same way $\mtx{B}$ does for $\mtx{u}$.
$\mtx{B}_{error}$ is only a column of $\mtx{B}$ if $u_{error}$ corresponds to an
existing input within $\mtx{u}$.

The plant is augmented as

\begin{align*}
  \dot{\begin{bmatrix}
    \mtx{x} \\
    u_{error}
  \end{bmatrix}} &=
  \begin{bmatrix}
    \mtx{A} & \mtx{B}_{error} \\
    0 & 0
  \end{bmatrix}
  \begin{bmatrix}
    \mtx{x} \\
    u_{error}
  \end{bmatrix} +
  \begin{bmatrix}
    \mtx{B} \\
    0
  \end{bmatrix}
  \mtx{u}
\end{align*}

With this model, the observer will estimate both the state and the $u_{error}$
term. The controller is augmented similarly. $\mtx{r}$ is augmented with a zero
for the goal $u_{error}$ term.

\begin{align*}
  \mtx{u} &= \mtx{K} \left(\mtx{r} - \mtx{x}\right) - \mtx{k}_{error}u_{error}
    \\
  \mtx{u} &=
  \begin{bmatrix}
    \mtx{K} & \mtx{k}_{error}
  \end{bmatrix}
  \left(\begin{bmatrix}
    \mtx{r} \\
    0
  \end{bmatrix} -
  \begin{bmatrix}
    \mtx{x} \\
    u_{error}
  \end{bmatrix}\right)
\end{align*}

where $\mtx{k}_{error}$ is a column vector with a $1$ in a given row if
$u_{error}$ should be applied to that input or a $0$ otherwise.

This process can be repeated for an arbitrary error which can be corrected via
some linear combination of the inputs.

\section{Feedforwards}

Feedforwards are used to inject information about either the system's dynamics
(like a model does) or the intended movement into a controller. Feedforward is
generally used to handle the control actions we already know must be applied to
make a system track a reference, then let the feedback controller correct for
what we do not or cannot know about the system at runtime. We will present
two ways of implementing feedforward for state feedback.

\subsection{Steady-state feedforward}

Steady-state feedforwards apply the control effort required to keep a system at
the reference if it is no longer moving (i.e., the system is at steady-state).
The first steady-state feedforward converts a desired output to a desired state.

\begin{equation*}
  \mtx{x}_c = \mtx{N}_x\mtx{y}_c
\end{equation*}

$\mtx{N}_x$ converts a desired output $\mtx{y}_c$ to a desired state
$\mtx{x}_c$ (also known as $\mtx{r}$). For steady-state, that is

\begin{equation}
  \mtx{x}_{ss} = \mtx{N}_x\mtx{y}_{ss} \label{eq:x_ss}
\end{equation}

The second steady-state feedforward converts the desired output $\mtx{y}$ to the
control input required at steady-state.

\begin{equation*}
  \mtx{u}_c = \mtx{N}_u\mtx{y}_c
\end{equation*}

$\mtx{N}_u$ converts the desired output $\mtx{y}$ to the control input $\mtx{u}$
required at steady-state. For steady-state, that is

\begin{equation}
  \mtx{u}_{ss} = \mtx{N}_u\mtx{y}_{ss} \label{eq:u_ss}
\end{equation}

To find the control input required at steady-state, set equation
(\ref{eq:ss_ctrl_x}) to zero.

\begin{align*}
  \dot{\mtx{x}} &= \mtx{A}\mtx{x} + \mtx{B}\mtx{u} \\
  \mtx{y} &= \mtx{C}\mtx{x} + \mtx{D}\mtx{u}
\end{align*}

\begin{align*}
  \mtx{0} &= \mtx{A}\mtx{x}_{ss} + \mtx{B}\mtx{u}_{ss} \\
  \mtx{y}_{ss} &= \mtx{C}\mtx{x}_{ss} + \mtx{D}\mtx{u}_{ss}
\end{align*}

\begin{align*}
  \mtx{0} &= \mtx{A}\mtx{N}_x\mtx{y}_{ss} + \mtx{B}\mtx{N}_u\mtx{y}_{ss} \\
  \mtx{y}_{ss} &= \mtx{C}\mtx{N}_x\mtx{y}_{ss} + \mtx{D}\mtx{N}_u\mtx{y}_{ss}
\end{align*}

\begin{align*}
  \begin{bmatrix}
    \mtx{0} \\
    \mtx{y}_{ss}
  \end{bmatrix} &=
  \begin{bmatrix}
    \mtx{A}\mtx{N}_x + \mtx{B}\mtx{N}_u \\
    \mtx{C}\mtx{N}_x + \mtx{D}\mtx{N}_u
  \end{bmatrix}
  \mtx{y}_{ss} \\
  \begin{bmatrix}
    \mtx{0} \\
    \mtx{1}
  \end{bmatrix} &=
  \begin{bmatrix}
    \mtx{A}\mtx{N}_x + \mtx{B}\mtx{N}_u \\
    \mtx{C}\mtx{N}_x + \mtx{D}\mtx{N}_u
  \end{bmatrix} \\
  \begin{bmatrix}
    \mtx{0} \\
    \mtx{1}
  \end{bmatrix} &=
  \begin{bmatrix}
    \mtx{A} & \mtx{B} \\
    \mtx{C} & \mtx{D}
  \end{bmatrix}
  \begin{bmatrix}
    \mtx{N}_x \\
    \mtx{N}_u
  \end{bmatrix} \\
  \begin{bmatrix}
    \mtx{N}_x \\
    \mtx{N}_u
  \end{bmatrix} &=
  \begin{bmatrix}
    \mtx{A} & \mtx{B} \\
    \mtx{C} & \mtx{D}
  \end{bmatrix}^{\dagger}
  \begin{bmatrix}
    \mtx{0} \\
    \mtx{1}
  \end{bmatrix}
\end{align*}

where $^\dagger$ is the Moore-Penrose pseudoinverse.

Now, we'll find an expression that uses $\mtx{N}_x$ and $\mtx{N}_u$ to convert
the reference $\mtx{r}$ to a control input feedforward $\mtx{u}_{ff}$. Let's
start with equation (\ref{eq:x_ss}).

\begin{align*}
  \mtx{x}_{ss} &= \mtx{N}_x \mtx{y}_{ss} \\
  \mtx{N}_x^\dagger \mtx{x}_{ss} &= \mtx{y}_{ss}
\end{align*}

Now substitute this into equation (\ref{eq:u_ss}).

\begin{align*}
  \mtx{u}_{ss} &= \mtx{N}_u \mtx{y}_{ss} \\
  \mtx{u}_{ss} &= \mtx{N}_u (\mtx{N}_x^\dagger \mtx{x}_{ss}) \\
  \mtx{u}_{ss} &= \mtx{N}_u \mtx{N}_x^\dagger \mtx{x}_{ss}
\end{align*}

$\mtx{u}_{ss}$ and $\mtx{x}_{ss}$ are also known as $\mtx{u}_{ff}$ and $\mtx{r}$
respectively.

\begin{align*}
  \mtx{u}_{ff} = \mtx{N}_u \mtx{N}_x^\dagger \mtx{r}
\end{align*}

So all together, we get theorem \ref{thm:steady-state_ff}.

\index{Feedforward!steady-state feedforward}
\begin{theorem}[Steady-state feedforward]
  \begin{align}
    \begin{bmatrix}
      \mtx{N}_x \\
      \mtx{N}_u
    \end{bmatrix} &=
    \begin{bmatrix}
      \mtx{A} & \mtx{B} \\
      \mtx{C} & \mtx{D}
    \end{bmatrix}^{\dagger}
    \begin{bmatrix}
      \mtx{0} \\
      \mtx{1}
    \end{bmatrix} \\
    \mtx{u}_{ff} &= \mtx{N}_u \mtx{N}_x^\dagger \mtx{r}
  \end{align}

  where $^\dagger$ is the Moore-Penrose pseudoinverse.

  In the augmented matrix, $\mtx{B}$ should contain one column corresponding to
  an actuator and $\mtx{C}$ should contain one row whose output will be driven
  by that actuator. More than one actuator or output can be included in the
  computation at once, but the result won't be the same as if they were computed
  independently and summed afterward.

  After computing the feedforward for each actuator-output pair, the respective
  collections of $\mtx{N}_x$ and $\mtx{N}_u$ matrices can summed to produce the
  combined feedforward.

  \label{thm:steady-state_ff}
\end{theorem}

If the augmented matrix in theorem \ref{thm:steady-state_ff} is square (number
of inputs = number of outputs), the normal inverse can be used instead.

\subsection{Two-state feedforward}

Let's start with the equation for the reference dynamics

\begin{equation*}
  \mtx{r}_{k+1} = \mtx{A}\mtx{r}_k + \mtx{B}\mtx{u}_{ff}
\end{equation*}

where $\mtx{u}_{ff}$ is the feedforward input. Note that this feedforward
equation does not and should not take into account any feedback terms. We want
to find the optimal $\mtx{u}_{ff}$ such that we minimize the tracking error
between $\mtx{r}_{k+1}$ and $\mtx{r}_k$.

\begin{equation*}
  \mtx{r}_{k+1} - \mtx{A}\mtx{r}_k = \mtx{B}\mtx{u}_{ff}
\end{equation*}

To solve for $\mtx{u}_{ff}$, we need to take the inverse of the nonsquare matrix
$\mtx{B}$. This isn't possible, but we can find the pseudoinverse given some
constraints on the state tracking error and control effort. To find the optimal
solution for these sorts of trade-offs, one can define a cost function and
attempt to minimize it. To do this, we'll first solve the expression for
$\mtx{0}$.

\begin{equation*}
  \mtx{0} = \mtx{B}\mtx{u}_{ff} - (\mtx{r}_{k+1} - \mtx{A}\mtx{r}_k)
\end{equation*}

This expression will be the state tracking cost we use in our cost function.

Our cost function will use an $H_2$ norm with $\mtx{Q}$ as the state cost matrix
with dimensionality $states \times states$ and $\mtx{R}$ as the control input
cost matrix with dimensionality $inputs \times inputs$.

\begin{equation*}
  \mtx{J} = (\mtx{B}\mtx{u}_{ff} - (\mtx{r}_{k+1} - \mtx{A}\mtx{r}_k))^T \mtx{Q}
    (\mtx{B}\mtx{u}_{ff} - (\mtx{r}_{k+1} - \mtx{A}\mtx{r}_k)) +
    \mtx{u}_{ff}^T\mtx{R}\mtx{u}_{ff}
\end{equation*}

\begin{remark}
  $\mtx{r}_{k+1} - \mtx{A}\mtx{r}_k$ will only return a nonzero vector if the
  reference isn't following the system dynamics. If it is, the feedback
  controller already compensates for it. This feedforward compensates for any
  unmodeled dynamics reflected in how the reference is changing (or not
  changing). In the case of a constant reference, the feedforward opposes any
  system dynamics that would change the state over time.
\end{remark}

The following theorems will be needed to find the minimum of $\mtx{J}$.

\begin{theorem}
  $\frac{\partial \mtx{x}^T\mtx{A}\mtx{x}}{\partial\mtx{x}} =
    2\mtx{A}\mtx{x}$ where $\mtx{A}$ is symmetric.
  \label{thm:partial_xax}
\end{theorem}

\begin{theorem}
  $\frac{\partial (\mtx{A}\mtx{x} + \mtx{b})^T\mtx{C}
    (\mtx{D}\mtx{x} + \mtx{e})}{\partial\mtx{x}} =
    \mtx{A}^T\mtx{C}(\mtx{D}\mtx{x} + \mtx{e}) + \mtx{D}^T\mtx{C}^T
    (\mtx{A}\mtx{x} + \mtx{b})$
  \label{thm:partial_ax_b}
\end{theorem}

\begin{corollary}
  $\frac{\partial (\mtx{A}\mtx{x} + \mtx{b})^T\mtx{C}
    (\mtx{A}\mtx{x} + \mtx{b})}{\partial\mtx{x}} =
    2\mtx{A}^T\mtx{C}(\mtx{A}\mtx{x} + \mtx{b})$ where $\mtx{C}$ is symmetric.
  \label{cor:partial_ax_b}

  Proof:
  \begin{align*}
    \frac{\partial (\mtx{A}\mtx{x} + \mtx{b})^T\mtx{C}
      (\mtx{A}\mtx{x} + \mtx{b})}{\partial\mtx{x}} &=
      \mtx{A}^T\mtx{C}(\mtx{A}\mtx{x} + \mtx{b}) + \mtx{A}^T\mtx{C}^T
      (\mtx{A}\mtx{x} + \mtx{b}) \\
    \frac{\partial (\mtx{A}\mtx{x} + \mtx{b})^T\mtx{C}
      (\mtx{A}\mtx{x} + \mtx{b})}{\partial\mtx{x}} &=
      (\mtx{A}^T\mtx{C} + \mtx{A}^T\mtx{C}^T)(\mtx{A}\mtx{x} + \mtx{b})
  \end{align*}

  $\mtx{C}$ is symmetric, so

  \begin{align*}
    \frac{\partial (\mtx{A}\mtx{x} + \mtx{b})^T\mtx{C}
      (\mtx{A}\mtx{x} + \mtx{b})}{\partial\mtx{x}} &=
      (\mtx{A}^T\mtx{C} + \mtx{A}^T\mtx{C})(\mtx{A}\mtx{x} + \mtx{b}) \\
    \frac{\partial (\mtx{A}\mtx{x} + \mtx{b})^T\mtx{C}
      (\mtx{A}\mtx{x} + \mtx{b})}{\partial\mtx{x}} &=
      2\mtx{A}^T\mtx{C}(\mtx{A}\mtx{x} + \mtx{b})
  \end{align*}
\end{corollary}

Given theorem \ref{thm:partial_xax} and corollary \ref{cor:partial_ax_b}, find
the minimum of $\mtx{J}$ by taking the partial derivative with respect to
$\mtx{u}_{ff}$ and setting the result to $\mtx{0}$.

\begin{align*}
  \frac{\partial\mtx{J}}{\partial\mtx{u}_{ff}} &= 2\mtx{B}^T\mtx{Q}
    (\mtx{B}\mtx{u}_{ff} - (\mtx{r}_{k+1} - \mtx{A}\mtx{r}_k)) +
    2\mtx{R}\mtx{u}_{ff} \\
  \mtx{0} &= 2\mtx{B}^T\mtx{Q}
    (\mtx{B}\mtx{u}_{ff} - (\mtx{r}_{k+1} - \mtx{A}\mtx{r}_k)) +
    2\mtx{R}\mtx{u}_{ff} \\
  \mtx{0} &= \mtx{B}^T\mtx{Q}
    (\mtx{B}\mtx{u}_{ff} - (\mtx{r}_{k+1} - \mtx{A}\mtx{r}_k)) +
    \mtx{R}\mtx{u}_{ff} \\
  \mtx{0} &= \mtx{B}^T\mtx{Q}\mtx{B}\mtx{u}_{ff} -
    \mtx{B}^T\mtx{Q}(\mtx{r}_{k+1} - \mtx{A}\mtx{r}_k) + \mtx{R}\mtx{u}_{ff} \\
  \mtx{B}^T\mtx{Q}\mtx{B}\mtx{u}_{ff} + \mtx{R}\mtx{u}_{ff} &=
    \mtx{B}^T\mtx{Q}(\mtx{r}_{k+1} - \mtx{A}\mtx{r}_k) \\
  (\mtx{B}^T\mtx{Q}\mtx{B} + \mtx{R})\mtx{u}_{ff} &=
    \mtx{B}^T\mtx{Q}(\mtx{r}_{k+1} - \mtx{A}\mtx{r}_k) \\
  \mtx{u}_{ff} &= (\mtx{B}^T\mtx{Q}\mtx{B} + \mtx{R})^{-1}
    \mtx{B}^T\mtx{Q}(\mtx{r}_{k+1} - \mtx{A}\mtx{r}_k)
\end{align*}

\begin{theorem}[Two-state feedforward]
  \begin{align}
    &\mtx{u}_{ff} = \mtx{K}_{ff} (\mtx{r}_{k+1} - \mtx{A}\mtx{r}_k) \\
    &\text{where } \mtx{K}_{ff} =
      (\mtx{B}^T\mtx{Q}\mtx{B} + \mtx{R})^{-1}\mtx{B}^T\mtx{Q}
  \end{align}
  \label{thm:two-state_ff}
\end{theorem}
\index{Feedforward!two-state feedforward}
\index{Optimal control!two-state feedforward}

If control effort is considered inexpensive, $\mtx{R} \ll \mtx{Q}$ and
$\mtx{u}_{ff}$ approaches corollary \ref{cor:two-state_ff_no_r}.

\begin{corollary}[Two-state feedforward with inexpensive control effort]
  \begin{align}
    &\mtx{u}_{ff} = \mtx{K}_{ff} (\mtx{r}_{k+1} - \mtx{A}\mtx{r}_k) \\
    &\text{where } \mtx{K}_{ff} =
      (\mtx{B}^T\mtx{Q}\mtx{B})^{-1}\mtx{B}^T\mtx{Q}
  \end{align}
  \label{cor:two-state_ff_no_r}
\end{corollary}

\begin{remark}
  If the cost matrix $\mtx{Q}$ isn't included in the cost function (that is,
  $\mtx{Q}$ is set to the identity matrix), $\mtx{K}_{ff}$ becomes the
  Moore-Penrose pseudoinverse of $\mtx{B}$ given by
  $\mtx{B}^\dagger = (\mtx{B}^T\mtx{B})^{-1}\mtx{B}^T$.
\end{remark}

\section{Implementation steps} \label{sec:implementation-steps}

\subsection{Derive physical model}

A \gls{model} is a set of differential equations describing how the system
behaves over time. There are two common approaches for developing them.

\begin{enumerate}
  \item Collecting data on the physical system's behavior and performing system
  identification with it.
  \item Using physics to derive the system's model from first principles.
\end{enumerate}

We'll use the second approach in this book.

Kinematics and dynamics are a rather large topics, so for now, we'll just focus
on the basics required for working with the models in this book. We'll derive
the same model, a pendulum, using three approaches: sum of forces, sum of
torques, and conservation of energy.

\begin{bookfigure}
  \begin{subfigure}{0.5\textwidth}
    \centering
    \begin{tikzpicture}
      % Save length of g-vector and theta to macros
      \pgfmathsetmacro{\Gvec}{1.5}
      \pgfmathsetmacro{\myAngle}{30}
      % Calculate lengths of vector components
      \pgfmathsetmacro{\Gcos}{\Gvec*cos(\myAngle)}
      \pgfmathsetmacro{\Gsin}{\Gvec*sin(\myAngle)}

      \coordinate (centro) at (0,0);
      \draw[dashed,gray,-] (centro) -- ++ (0,-3.5)
        node (mary) [black,below] {$ $};
      \draw[thick] (centro) -- ++(270+\myAngle:3) coordinate (bob);
      \path pic [draw,->,"$\theta$",angle eccentricity=1.5]
        {angle=mary--centro--bob};
      \draw [draw=violet,-stealth] (bob) -- ($(bob)!-\Gcos cm!(centro)$)
        coordinate (gcos)
        node[midway,above right] {$mg\cos\theta$};
      \draw [dashed,draw=red,-stealth] (bob) -- ($(bob)!2*\Gsin cm!90:(centro)$)
        coordinate node[midway,above right] {};
      \draw [draw=violet,-stealth] (bob) -- ($(bob)!\Gsin cm!90:(centro)$)
        coordinate (gsin)
        node[midway,above left] {$mg\sin\theta$};
      \draw [draw=blue,-stealth] (bob) -- ++(0,-\Gvec)
        coordinate (g)
        node[near end,left] {$mg$};
      \pic [draw,->,"$\theta$",angle eccentricity=1.5] {angle=g--bob--gcos};
      \filldraw [fill=black!40,draw=black] (bob) circle[radius=0.2];
    \end{tikzpicture}
    \caption{Force diagram of a pendulum}
    \label{subfig:force_pendulum}
  \end{subfigure}%
  \begin{subfigure}{0.5\textwidth}
    \centering
    \begin{tikzpicture}
      % Save length of g-vector and theta to macros
      \pgfmathsetmacro{\Gvec}{1.5}
      \pgfmathsetmacro{\myAngle}{30}
      % Calculate lengths of vector components
      \pgfmathsetmacro{\Gcos}{\Gvec*cos(\myAngle)}
      \pgfmathsetmacro{\Gsin}{\Gvec*sin(\myAngle)}

      \coordinate (centro) at (0,0);
      \coordinate (heightmes_lo) at (-1,0);
      \coordinate (heightmes_hi) at (-0.25,0);
      \coordinate (h) at (\Gcos/2,\Gsin);

      \draw[thick] (centro) -- ++(270+\myAngle:3) coordinate (bob_lo);
      \draw[dashed,gray,-] (centro) -- ++ (0,0 |- bob_lo)
        node (mary) [black,below]{$ $};
      \draw[dashed,gray,-] (heightmes_lo |- bob_lo) -- (bob_lo)
        node [black,below]{$ $};
      \draw[<->] (heightmes_lo) -- ++ (0,0 |- bob_lo)
        node [black,pos=0.5,left]{$y_1$};
      \pic [draw,->, "$\theta$",angle eccentricity=1.5]
        {angle=mary--centro--bob_lo};

      % Save length of g-vector and theta to macros
      \pgfmathsetmacro{\Gvec}{1.5cm}
      \pgfmathsetmacro{\myAngle}{45}
      % Calculate lengths of vector components
      \pgfmathsetmacro{\Gcos}{\Gvec*cos(\myAngle)}
      \pgfmathsetmacro{\Gsin}{\Gvec*sin(\myAngle)}

      \draw[gray,thick] (centro) -- ++(270+\myAngle:3) coordinate (bob_hi);
      \pic [draw,->,"$\theta_0$",angle eccentricity=1.5,angle radius=1cm]
        {angle=mary--centro--bob_hi};

      \draw[dashed,gray,-] (0,0 |- bob_hi) -- (bob_hi)
        node (mary) [black,below]{$ $};
      \draw[<->] (heightmes_hi) -- ++ (0,0 |- bob_hi)
        node (mary) [black,pos=0.5,left]{$y_0$};
      \draw[<->] (h |- bob_hi) -- (h |- bob_lo)
        node [black,pos=0.5,left]{$h$};

      % Path of pendulum
      \pic [draw,dashed,gray,<-,angle eccentricity=1.5,angle radius=2*\Gvec]
        {angle=mary--centro--bob_hi};

      % Pendulum balls
      \filldraw [fill=black!40,draw=black] (bob_lo) circle[radius=0.2];
      \filldraw [fill=black!20,draw=gray] (bob_hi) circle[radius=0.2];
    \end{tikzpicture}
    \caption{Trigonometry of a pendulum}
    \label{subfig:trig_pendulum}
  \end{subfigure}
  \caption{Pendulum force diagrams}
\end{bookfigure}

\subsubsection{Force derivation}
\index{Physics!sum of forces}

Consider figure \ref{subfig:force_pendulum}, which shows the forces acting on a
pendulum.

Note that the path of the pendulum sweeps out an arc of a circle. The angle
$\theta$ is measured in radians. The blue arrow is the gravitational force
acting on the bob, and the violet arrows are that same force resolved into
components parallel and perpendicular to the bob's instantaneous motion. The
direction of the bob's instantaneous velocity always points along the red axis,
which is considered the tangential axis because its direction is always tangent
to the circle. Consider Newton's second law

\begin{equation*}
  F = ma
\end{equation*}

where $F$ is the sum of forces on the object, $m$ is mass, and $a$ is the
acceleration. Because we are only concerned with changes in speed, and because
the bob is forced to stay in a circular path, we apply Newton's equation to the
tangential axis only. The short violet arrow represents the component of the
gravitational force in the tangential axis, and trigonometry can be used to
determine its magnitude. Therefore

\begin{align*}
  -mg\sin\theta &= ma \\
  a &= -g\sin\theta
\end{align*}

where $g$ is the acceleration due to gravity near the surface of the earth. The
negative sign on the right hand side implies that $\theta$ and a always point in
opposite directions. This makes sense because when a pendulum swings further to
the left, we would expect it to accelerate back toward the right.

This linear acceleration $a$ along the red axis can be related to the change in
angle $\theta$ by the arc length formulas; $s$ is arc length and $l$ is the
length of the pendulum.

\begin{align}
  s &= l\theta \label{eq:arc_length} \\
  v &= \frac{ds}{dt} = l\frac{d\theta}{dt} \nonumber \\
  a &= \frac{d^2s}{dt^2} = l\frac{d^2\theta}{dt^2} \nonumber
\end{align}

Therefore

\begin{align*}
  l\frac{d^2\theta}{dt^2} &= -g\sin\theta \\
  \frac{d^2\theta}{dt^2} &= -\frac{g}{l}\sin\theta \\
  \ddot{\theta} &= -\frac{g}{l}\sin\theta
\end{align*}

\subsubsection{Torque derivation}
\index{Physics!sum of torques}

The equation can be obtained using two definitions for torque.

\begin{equation*}
  \mtx{\tau} = \mtx{r} \times \mtx{F}
\end{equation*}

First start by defining the torque on the pendulum bob using the force due to
gravity.

\begin{equation*}
  \mtx{\tau} = \mtx{l} \times \mtx{F}_g
\end{equation*}

where $\mtx{l}$ is the length vector of the pendulum and $\mtx{F}_g$ is the
force due to gravity.

For now just consider the magnitude of the torque on the pendulum.

\begin{equation*}
  \lvert\tau\rvert = -mgl\sin\theta
\end{equation*}

where $m$ is the mass of the pendulum, $g$ is the acceleration due to gravity,
$l$ is the length of the pendulum and $\theta$ is the angle between the length
vector and the force due to gravity.

Next rewrite the angular momentum.

\begin{equation*}
  \mtx{L} = \mtx{r} \times \mtx{p} =
    m\mtx{r} \times (\mtx{\omega} \times \mtx{r})
\end{equation*}

Again just consider the magnitude of the angular momentum.

\begin{align*}
  \lvert\mtx{L}\rvert &= mr^2\omega \\
  \lvert\mtx{L}\rvert &= ml^2 \frac{d\theta}{dt} \\
  \frac{d}{dt}\lvert\mtx{L}\rvert &= ml^2 \frac{d^2\theta}{dt^2}
\end{align*}

According to $\tau = \frac{d\mtx{L}}{dt}$, we can just compare the magnitudes.

\begin{align*}
  -mgl\sin\theta &= ml^2\frac{d^2\theta}{dt^2} \\
  -\frac{g}{l}\sin\theta &= \frac{d^2\theta}{dt^2} \\
  \ddot{\theta} &= -\frac{g}{l}\sin\theta
\end{align*}

which is the same result from force analysis.

\subsubsection{Energy derivation}
\index{Physics!conservation of energy}

The equation can also be obtained via the conservation of mechanical energy
principle: any object falling a vertical distance $h$ would acquire kinetic
energy equal to that which it lost to the fall. In other words, gravitational
potential energy is converted into kinetic energy. Change in potential energy is
given by

\begin{equation*}
  \Delta U = mgh
\end{equation*}

The change in kinetic energy (body started from rest) is given by

\begin{equation*}
  \Delta K = \frac{1}{1}mv^2
\end{equation*}

Since no energy is lost, the gain in one must be equal to the loss in the other

\begin{equation*}
  \frac{1}{2}mv^2 = mgh
\end{equation*}

The change in velocity for a given change in height can be expressed as

\begin{equation*}
  v = \sqrt{2gh}
\end{equation*}

Using equation (\ref{eq:arc_length}), this equation can be rewritten in terms of
$\frac{d\theta}{dt}$.

\begin{align}
  v = l\frac{d\theta}{dt} &= \sqrt{2gh} \nonumber \\
  \frac{d\theta}{dt} &= \frac{2gh}{l} \label{eq:energy_dtheta}
\end{align}

where $h$ is the vertical distance the pendulum fell. Look at figure \ref{subfig:trig_pendulum}, which presents the trigonometry of a pendulum. If the pendulum
starts its swing from some initial angle $\theta_0$, then $y_0$, the vertical
distance from the pivot point, is given by

\begin{equation*}
  y_0 = l\cos\theta_0
\end{equation*}

Similarly for $y_1$, we have

\begin{equation*}
  y_1 = l\cos\theta
\end{equation*}

Then $h$ is the difference of the two

\begin{equation*}
  h = l(\cos\theta - \cos\theta_0)
\end{equation*}

Substituting this into equation (\ref{eq:energy_dtheta}) gives

\begin{equation*}
  \frac{d\theta}{dt} = \sqrt{\frac{2g}{l}(\cos\theta - \cos\theta_0)}
\end{equation*}

This equation is known as the first integral of motion. It gives the velocity in
terms of the location and includes an integration constant related to the
initial displacement ($\theta_0$). We can differentiate by applying the chain
rule with respect to time. Doing so gives the acceleration.

\begin{align*}
  \frac{d}{dt}\frac{d\theta}{dt} &=
    \frac{d}{dt}\sqrt{\frac{2g}{l}(\cos\theta - \cos\theta_0)} \\
  \frac{d^2\theta}{dt^2} &= \frac{1}{2}\frac
    {-\frac{2g}{l}\sin\theta}
    {\sqrt{\frac{2g}{l}(\cos\theta - \cos\theta_0)}}\frac{d\theta}{dt} \\
  \frac{d^2\theta}{dt^2} &= \frac{1}{2}\frac
    {-\frac{2g}{l}\sin\theta}
    {\sqrt{\frac{2g}{l}(\cos\theta - \cos\theta_0)}}
    \sqrt{\frac{2g}{l}(\cos\theta - \cos\theta_0)} \\
  \frac{d^2\theta}{dt^2} &= -\frac{g}{l}\sin\theta \\
  \ddot{\theta} &= -\frac{g}{l}\sin\theta
\end{align*}

which is the same result from force analysis.

\subsection{Write model in state-space representation}

Below is the \gls{model} for a pendulum

\begin{equation*}
  \ddot{\theta} = -\frac{g}{l}\sin\theta
\end{equation*}

where $\theta$ is the angle of the pendulum and $l$ is the length of the
pendulum.

Since state-space representation requires that only single derivatives be used,
they should be broken up as separate states. We'll reassign $\dot{\theta}$ to be
$\omega$ so the derivatives are easier to keep straight for state-space
representation.

\begin{equation*}
  \dot{\omega} = -\frac{g}{l}\sin\theta
\end{equation*}

Now separate the states.

\begin{align*}
  \dot{\theta} &= \omega \\
  \dot{\omega} &= -\frac{g}{l} \sin\theta
\end{align*}

Since this \gls{model} is nonlinear, we should
\glslink{linearization}{linearize} \index{Nonlinear control!linearization} it.
We will use the small angle approximation ($\sin\theta = \theta$ for small
values of $\theta$).

\begin{align*}
  \dot{\theta} &= \omega \\
  \dot{\omega} &= -\frac{g}{l} \theta
\end{align*}

Now write the model in state-space representation.

\begin{align}
  \dot{
  \begin{bmatrix}
    \theta \\
    \omega
  \end{bmatrix}} =
  \begin{bmatrix}
    0 & 1 \\
    -\frac{g}{l} & 0
  \end{bmatrix}
  \begin{bmatrix}
    \theta \\
    \omega
  \end{bmatrix}
\end{align}

\subsection{Add estimator for unmeasured states}

For full state feedback, knowledge of all states is required. If not all states
are measured directly, an estimator can be used to supplement them.

For example, we may only be measuring $\theta$ in the pendulum example, not
$\dot{\theta}$, so we'll need to estimate the latter. The $\mtx{C}$ matrix the
observer would use in this case is

\begin{equation*}
  \mtx{C} = \begin{bmatrix}
    1 \\
    0
  \end{bmatrix}
\end{equation*}

\subsection{Implement controller}

Use Bryson's rule when making the performance vs actuation effort trade-off.
Optimizing for performance will get you to the reference as fast as possible
while optimizing actuation effort will get you to the reference in the most
``fuel-efficient" way possible. The latter, for example, would potentially avoid
voltage drops from motor usage on robots with a limited power supply, but the
result would be slower to reach the reference.

\subsection{Simulate model/controller}

This can be done in any platform supporting numerical computation. Common
choices are MATLAB, v-REP, or Python. Tweak the LQR gains as necessary.

If you're comfortable enough with it, you can use the controller designed by LQR
as a starting point and tweak the pole locations after that with pole placement
to produce the desired response.

\subsubsection{Simulating a closed-loop system}

Recall equation (\ref{eq:s_ref_ctrl_x}) where a closed-loop system is written as
$\dot{\mtx{x}} = (\mtx{A} - \mtx{B}\mtx{K})\mtx{x} + \mtx{B}\mtx{K}\mtx{r}$. In
the open-loop system, our control input was just $\mtx{u}$, but in the
closed-loop system, our control input $\mtx{u}$ is $\mtx{K}\mtx{r}$. To use this
form in simulation, the corresponding state-space matrices, which we'll denote
$\mtx{A}'$ and $\mtx{B}'$, would be

\begin{align*}
  \mtx{A}' &= \mtx{A} - \mtx{B}\mtx{K} \\
  \mtx{B}' &= \mtx{B}\mtx{K}
\end{align*}

\subsection{Verify pole locations}

Check the pole locations as a sanity check and to potentially gain an intuition
for the chosen pole locations.

\subsection{Unit test}

Write unit tests to test the \gls{model} performance and \gls{robustness} under
different initial conditions and command inputs. If you are using C++, we
recommend Google Test.

\subsection{Test on real system}

Try the controller on a real \gls{system} with low maximum outputs for safety.
The outputs can be increased after verifying the sensors function and mechanisms
move the correct direction.


\chapterimage{ss-model-examples.jpg}{Fake plant in hotel lobby in Ventura, CA}

\chapter{State-space model examples}

Up to now, we've just been teaching what tools are available. Now, we'll go into
specifics on how to apply them and provide advice on certain applications.

The \gls{model} examples are intended to walk the reader through the
implementation steps described in section \ref{sec:implementation-steps}. The
code shown in each example can be obtained from frccontrol's Git repository at
\url{https://github.com/calcmogul/frccontrol/tree/master/examples}.
See appendix \ref{ch:app-installing-python-control} for setup instructions.

The \glspl{model} derived here should cover most types of motion seen on an FRC
robot. Furthermore, they can be easily tweaked to describe many types of
mechanisms just by pattern-matching. There's only so many ways to hook up a mass
to a motor in FRC. The flywheel \gls{model} can be used for spinning mechanisms,
the elevator \gls{model} can be used for spinning mechanisms transformed to
linear motion, and the single-jointed arm \gls{model} can be used for rotating
servo mechanisms (it's just the flywheel \gls{model} augmented with a position
\gls{state}).

These \glspl{model} assume all motor controllers driving DC brushed motors are
set to brake mode instead of coast mode. Brake mode behaves the same as coast
mode except where the applied voltage is zero. In brake mode, the motor leads
are shorted together to prevent movement. In coast mode, the motor leads are an
open circuit.

\section{Implementation steps} \label{sec:implementation-steps}

\subsection{Derive physical model}

A \gls{model} is a set of differential equations describing how the \gls{system}
behaves over time. There are two common approaches for developing them.

\begin{enumerate}
  \item Collecting data on the physical system's behavior and performing
    \gls{system} identification with it.
  \item Using physics to derive the \gls{system}'s model from first principles.
\end{enumerate}

We'll use the second approach in this book.

Kinematics and dynamics are a rather large topics, so for now, we'll just focus
on the basics required for working with the \glspl{model} in this book. We'll
derive the same \gls{model}, a pendulum, using three approaches: sum of forces,
sum of torques, and conservation of energy.

\begin{bookfigure}
  \begin{subfigure}{0.5\textwidth}
    \centering
    \begin{tikzpicture}
      % Save length of g-vector and theta to macros
      \pgfmathsetmacro{\Gvec}{1.5}
      \pgfmathsetmacro{\myAngle}{30}
      % Calculate lengths of vector components
      \pgfmathsetmacro{\Gcos}{\Gvec*cos(\myAngle)}
      \pgfmathsetmacro{\Gsin}{\Gvec*sin(\myAngle)}

      \coordinate (centro) at (0,0);
      \draw[dashed,gray,-] (centro) -- ++ (0,-3.5)
        node (mary) [black,below] {$ $};
      \draw[thick] (centro) -- ++(270+\myAngle:3) coordinate (bob);
      \path pic [draw,->,"$\theta$",angle eccentricity=1.5]
        {angle=mary--centro--bob};
      \draw [draw=violet,-stealth] (bob) -- ($(bob)!-\Gcos cm!(centro)$)
        coordinate (gcos)
        node[midway,above right] {$mg\cos\theta$};
      \draw [dashed,draw=red,-stealth] (bob) -- ($(bob)!2*\Gsin cm!90:(centro)$)
        coordinate node[midway,above right] {};
      \draw [draw=violet,-stealth] (bob) -- ($(bob)!\Gsin cm!90:(centro)$)
        coordinate (gsin)
        node[midway,above left] {$mg\sin\theta$};
      \draw [draw=blue,-stealth] (bob) -- ++(0,-\Gvec)
        coordinate (g)
        node[near end,left] {$mg$};
      \pic [draw,->,"$\theta$",angle eccentricity=1.5] {angle=g--bob--gcos};
      \filldraw [fill=black!40,draw=black] (bob) circle[radius=0.2];
    \end{tikzpicture}
    \caption{Force diagram of a pendulum}
    \label{subfig:force_pendulum}
  \end{subfigure}%
  \begin{subfigure}{0.5\textwidth}
    \centering
    \begin{tikzpicture}
      % Save length of g-vector and theta to macros
      \pgfmathsetmacro{\Gvec}{1.5}
      \pgfmathsetmacro{\myAngle}{30}
      % Calculate lengths of vector components
      \pgfmathsetmacro{\Gcos}{\Gvec*cos(\myAngle)}
      \pgfmathsetmacro{\Gsin}{\Gvec*sin(\myAngle)}

      \coordinate (centro) at (0,0);
      \coordinate (heightmes_lo) at (-1,0);
      \coordinate (heightmes_hi) at (-0.25,0);
      \coordinate (h) at (\Gcos/2,\Gsin);

      \draw[thick] (centro) -- ++(270+\myAngle:3) coordinate (bob_lo);
      \draw[dashed,gray,-] (centro) -- ++ (0,0 |- bob_lo)
        node (mary) [black,below]{$ $};
      \draw[dashed,gray,-] (heightmes_lo |- bob_lo) -- (bob_lo)
        node [black,below]{$ $};
      \draw[<->] (heightmes_lo) -- ++ (0,0 |- bob_lo)
        node [black,pos=0.5,left]{$y_1$};
      \pic [draw,->, "$\theta$",angle eccentricity=1.5]
        {angle=mary--centro--bob_lo};

      % Save length of g-vector and theta to macros
      \pgfmathsetmacro{\Gvec}{1.5cm}
      \pgfmathsetmacro{\myAngle}{45}
      % Calculate lengths of vector components
      \pgfmathsetmacro{\Gcos}{\Gvec*cos(\myAngle)}
      \pgfmathsetmacro{\Gsin}{\Gvec*sin(\myAngle)}

      \draw[gray,thick] (centro) -- ++(270+\myAngle:3) coordinate (bob_hi);
      \pic [draw,->,"$\theta_0$",angle eccentricity=1.5,angle radius=1cm]
        {angle=mary--centro--bob_hi};

      \draw[dashed,gray,-] (0,0 |- bob_hi) -- (bob_hi)
        node (mary) [black,below]{$ $};
      \draw[<->] (heightmes_hi) -- ++ (0,0 |- bob_hi)
        node (mary) [black,pos=0.5,left]{$y_0$};
      \draw[<->] (h |- bob_hi) -- (h |- bob_lo)
        node [black,pos=0.5,left]{$h$};

      % Path of pendulum
      \pic [draw,dashed,gray,<-,angle eccentricity=1.5,angle radius=2*\Gvec]
        {angle=mary--centro--bob_hi};

      % Pendulum balls
      \filldraw [fill=black!40,draw=black] (bob_lo) circle[radius=0.2];
      \filldraw [fill=black!20,draw=gray] (bob_hi) circle[radius=0.2];
    \end{tikzpicture}
    \caption{Trigonometry of a pendulum}
    \label{subfig:trig_pendulum}
  \end{subfigure}
  \caption{Pendulum force diagrams}
\end{bookfigure}

\subsubsection{Force derivation}
\index{Physics!sum of forces}

Consider figure \ref{subfig:force_pendulum}, which shows the forces acting on a
pendulum.

Note that the path of the pendulum sweeps out an arc of a circle. The angle
$\theta$ is measured in radians. The blue arrow is the gravitational force
acting on the bob, and the violet arrows are that same force resolved into
components parallel and perpendicular to the bob's instantaneous motion. The
direction of the bob's instantaneous velocity always points along the red axis,
which is considered the tangential axis because its direction is always tangent
to the circle. Consider Newton's second law

\begin{equation*}
  F = ma
\end{equation*}

where $F$ is the sum of forces on the object, $m$ is mass, and $a$ is the
acceleration. Because we are only concerned with changes in speed, and because
the bob is forced to stay in a circular path, we apply Newton's equation to the
tangential axis only. The short violet arrow represents the component of the
gravitational force in the tangential axis, and trigonometry can be used to
determine its magnitude. Therefore

\begin{align*}
  -mg\sin\theta &= ma \\
  a &= -g\sin\theta
\end{align*}

where $g$ is the acceleration due to gravity near the surface of the earth. The
negative sign on the right hand side implies that $\theta$ and a always point in
opposite directions. This makes sense because when a pendulum swings further to
the left, we would expect it to accelerate back toward the right.

This linear acceleration $a$ along the red axis can be related to the change in
angle $\theta$ by the arc length formulas; $s$ is arc length and $l$ is the
length of the pendulum.

\begin{align}
  s &= l\theta \label{eq:arc_length} \\
  v &= \frac{ds}{dt} = l\frac{d\theta}{dt} \nonumber \\
  a &= \frac{d^2s}{dt^2} = l\frac{d^2\theta}{dt^2} \nonumber
\end{align}

Therefore

\begin{align*}
  l\frac{d^2\theta}{dt^2} &= -g\sin\theta \\
  \frac{d^2\theta}{dt^2} &= -\frac{g}{l}\sin\theta \\
  \ddot{\theta} &= -\frac{g}{l}\sin\theta
\end{align*}

\subsubsection{Torque derivation}
\index{Physics!sum of torques}

The equation can be obtained using two definitions for torque.

\begin{equation*}
  \mtx{\tau} = \mtx{r} \times \mtx{F}
\end{equation*}

First start by defining the torque on the pendulum bob using the force due to
gravity.

\begin{equation*}
  \mtx{\tau} = \mtx{l} \times \mtx{F}_g
\end{equation*}

where $\mtx{l}$ is the length vector of the pendulum and $\mtx{F}_g$ is the
force due to gravity.

For now just consider the magnitude of the torque on the pendulum.

\begin{equation*}
  \lvert\tau\rvert = -mgl\sin\theta
\end{equation*}

where $m$ is the mass of the pendulum, $g$ is the acceleration due to gravity,
$l$ is the length of the pendulum and $\theta$ is the angle between the length
vector and the force due to gravity.

Next rewrite the angular momentum.

\begin{equation*}
  \mtx{L} = \mtx{r} \times \mtx{p} =
    m\mtx{r} \times (\mtx{\omega} \times \mtx{r})
\end{equation*}

Again just consider the magnitude of the angular momentum.

\begin{align*}
  \lvert\mtx{L}\rvert &= mr^2\omega \\
  \lvert\mtx{L}\rvert &= ml^2 \frac{d\theta}{dt} \\
  \frac{d}{dt}\lvert\mtx{L}\rvert &= ml^2 \frac{d^2\theta}{dt^2}
\end{align*}

According to $\tau = \frac{d\mtx{L}}{dt}$, we can just compare the magnitudes.

\begin{align*}
  -mgl\sin\theta &= ml^2\frac{d^2\theta}{dt^2} \\
  -\frac{g}{l}\sin\theta &= \frac{d^2\theta}{dt^2} \\
  \ddot{\theta} &= -\frac{g}{l}\sin\theta
\end{align*}

which is the same result from force analysis.

\subsubsection{Energy derivation}
\index{Physics!conservation of energy}

The equation can also be obtained via the conservation of mechanical energy
principle: any object falling a vertical distance $h$ would acquire kinetic
energy equal to that which it lost to the fall. In other words, gravitational
potential energy is converted into kinetic energy. Change in potential energy is
given by

\begin{equation*}
  \Delta U = mgh
\end{equation*}

The change in kinetic energy (body started from rest) is given by

\begin{equation*}
  \Delta K = \frac{1}{1}mv^2
\end{equation*}

Since no energy is lost, the gain in one must be equal to the loss in the other

\begin{equation*}
  \frac{1}{2}mv^2 = mgh
\end{equation*}

The change in velocity for a given change in height can be expressed as

\begin{equation*}
  v = \sqrt{2gh}
\end{equation*}

Using equation (\ref{eq:arc_length}), this equation can be rewritten in terms of
$\frac{d\theta}{dt}$.

\begin{align}
  v = l\frac{d\theta}{dt} &= \sqrt{2gh} \nonumber \\
  \frac{d\theta}{dt} &= \frac{2gh}{l} \label{eq:energy_dtheta}
\end{align}

where $h$ is the vertical distance the pendulum fell. Look at figure \ref{subfig:trig_pendulum}, which presents the trigonometry of a pendulum. If the pendulum
starts its swing from some initial angle $\theta_0$, then $y_0$, the vertical
distance from the pivot point, is given by

\begin{equation*}
  y_0 = l\cos\theta_0
\end{equation*}

Similarly for $y_1$, we have

\begin{equation*}
  y_1 = l\cos\theta
\end{equation*}

Then $h$ is the difference of the two

\begin{equation*}
  h = l(\cos\theta - \cos\theta_0)
\end{equation*}

Substituting this into equation (\ref{eq:energy_dtheta}) gives

\begin{equation*}
  \frac{d\theta}{dt} = \sqrt{\frac{2g}{l}(\cos\theta - \cos\theta_0)}
\end{equation*}

This equation is known as the first integral of motion. It gives the velocity in
terms of the location and includes an integration constant related to the
initial displacement ($\theta_0$). We can differentiate by applying the chain
rule with respect to time. Doing so gives the acceleration.

\begin{align*}
  \frac{d}{dt}\frac{d\theta}{dt} &=
    \frac{d}{dt}\sqrt{\frac{2g}{l}(\cos\theta - \cos\theta_0)} \\
  \frac{d^2\theta}{dt^2} &= \frac{1}{2}\frac
    {-\frac{2g}{l}\sin\theta}
    {\sqrt{\frac{2g}{l}(\cos\theta - \cos\theta_0)}}\frac{d\theta}{dt} \\
  \frac{d^2\theta}{dt^2} &= \frac{1}{2}\frac
    {-\frac{2g}{l}\sin\theta}
    {\sqrt{\frac{2g}{l}(\cos\theta - \cos\theta_0)}}
    \sqrt{\frac{2g}{l}(\cos\theta - \cos\theta_0)} \\
  \frac{d^2\theta}{dt^2} &= -\frac{g}{l}\sin\theta \\
  \ddot{\theta} &= -\frac{g}{l}\sin\theta
\end{align*}

which is the same result from force analysis.

\subsection{Write model in state-space representation}

Below is the \gls{model} for a pendulum

\begin{equation*}
  \ddot{\theta} = -\frac{g}{l}\sin\theta
\end{equation*}

where $\theta$ is the angle of the pendulum and $l$ is the length of the
pendulum.

Since state-space representation requires that only single derivatives be used,
they should be broken up as separate \glspl{state}. We'll reassign
$\dot{\theta}$ to be $\omega$ so the derivatives are easier to keep straight for
state-space representation.

\begin{equation*}
  \dot{\omega} = -\frac{g}{l}\sin\theta
\end{equation*}

Now separate the \glspl{state}.

\begin{align*}
  \dot{\theta} &= \omega \\
  \dot{\omega} &= -\frac{g}{l} \sin\theta
\end{align*}

Since this \gls{model} is nonlinear, we should
\glslink{linearization}{linearize} \index{Nonlinear control!linearization} it.
We will use the small angle approximation ($\sin\theta = \theta$ for small
values of $\theta$).

\begin{align*}
  \dot{\theta} &= \omega \\
  \dot{\omega} &= -\frac{g}{l} \theta
\end{align*}

Now write the \gls{model} in state-space representation.

\begin{align}
  \dot{
  \begin{bmatrix}
    \theta \\
    \omega
  \end{bmatrix}} =
  \begin{bmatrix}
    0 & 1 \\
    -\frac{g}{l} & 0
  \end{bmatrix}
  \begin{bmatrix}
    \theta \\
    \omega
  \end{bmatrix}
\end{align}

\subsection{Add estimator for unmeasured states}

For full \gls{state} feedback, knowledge of all \glspl{state} is required. If
not all \glspl{state} are measured directly, an estimator can be used to
supplement them.

For example, we may only be measuring $\theta$ in the pendulum example, not
$\dot{\theta}$, so we'll need to estimate the latter. The $\mtx{C}$ matrix the
\gls{observer} would use in this case is

\begin{equation*}
  \mtx{C} = \begin{bmatrix}
    1 & 0 \\
  \end{bmatrix}
\end{equation*}

\subsection{Implement controller}

Use Bryson's rule when making the performance vs \gls{control effort} trade-off.
Optimizing for performance will get you to the \gls{reference} as fast as
possible while optimizing \gls{control effort} will get you to the
\gls{reference} in the most ``fuel-efficient" way possible. The latter, for
example, would potentially avoid voltage drops from motor usage on robots with a
limited power supply, but the result would be slower to reach the
\gls{reference}.

\subsection{Simulate model/controller}

This can be done in any platform supporting numerical computation. Common
choices are MATLAB, v-REP, or Python. Tweak the LQR gains as necessary.

If you're comfortable enough with it, you can use the controller designed by LQR
as a starting point and tweak the pole locations after that with pole placement
to produce the desired response.

\subsubsection{Simulating a closed-loop system}

Recall equation (\ref{eq:s_ref_ctrl_x}) where a closed-loop system is written as
$\dot{\mtx{x}} = (\mtx{A} - \mtx{B}\mtx{K})\mtx{x} + \mtx{B}\mtx{K}\mtx{r}$. In
the open-loop \gls{system}, our \gls{control input} $\mtx{u}$ was a vector of
\gls{system} \glspl{input}. In the closed-loop \gls{system}, our
\gls{control input} $\mtx{u}$ is now a vector of \gls{reference} \glspl{state}
$\mtx{r}$. To use this form in simulation, the corresponding state-space
matrices, which we'll denote with apostrophes, would be

\begin{align*}
  \mtx{A}' &= \mtx{A} - \mtx{B}\mtx{K} \\
  \mtx{B}' &= \mtx{B}\mtx{K} \\
  \mtx{C}' &= \mtx{C} - \mtx{D}\mtx{K} \\
  \mtx{D}' &= \mtx{D}\mtx{K}
\end{align*}

\subsection{Verify pole locations}

Check the pole locations as a sanity check and to potentially gain an intuition
for the chosen pole locations.

\subsection{Unit test}

Write unit tests to test the \gls{model} performance and \gls{robustness} with
different initial conditions and \glspl{reference}. For C++, we recommend Google
Test.

\subsection{Test on real system}

Try the controller on a real \gls{system} with low maximum \glspl{control input}
for safety. The \glspl{control input} can be increased after verifying the
sensors function and mechanisms move the correct direction.

\section{DC brushed motor}

\subsection{Equations of motion}

The circuit for a DC brushed motor is shown below.

\begin{figure}[H]
  \centering

  \begin{tikzpicture}[auto, >=latex', circuit ee IEC,
                      set resistor graphic=var resistor IEC graphic]
    \node [opencircuit] (start) at (0,0) {};
    \node [] (V+) at (-0.5,0) { $+$ };
    \node [opencircuit] (end) at (0,-3.5) {};
    \node [] (V-) at (-0.5,-3.5) { $-$ };
    \node [coordinate] (topright) at (2.5,0) {};
    \node [coordinate] (bottomright) at (2.5,-3.5) {};
    \node [] at (0, -1.75) { $V$ };
    \draw (start) to (topright)
                  to [resistor={near start, info'={ $R$ }},
                      voltage source={near end, direction info'={<-},
                      info={ $V_{emf}=\frac{\omega_m}{K_v}$ }}] (bottomright)
                  to (end);
  \end{tikzpicture}

  \caption{DC brushed motor circuit}
  \label{fig:dc_motor_circuit}
\end{figure}

where $V$ is the voltage applied to the motor, $I$ is the current through the
motor in Amps, $R$ is the resistance across the motor in Ohms, $\omega_m$ is the
angular velocity of the motor in radians per second, and $K_v$ is the angular
velocity constant in radians per second per Volt. This circuit reflects the
following relation.

\begin{equation}
  V = IR + \frac{\omega_m}{K_v} \label{eq:motor_V}
\end{equation}

The mechanical relation for a DC brushed motor is

\begin{equation}
  \tau_m = K_t I \label{eq:motor_tau_m}
\end{equation}

where $\tau_m$ is the torque produced by the motor in Newton-meters and $K_t$ is
the torque constant in Newton-meters per Amp. Therefore

\begin{equation*}
  I = \frac{\tau_m}{K_t}
\end{equation*}

Substitute this into equation (\ref{eq:motor_V}).

\begin{equation}
  V = \frac{\tau_m}{K_t} R + \frac{\omega_m}{K_v} \label{eq:motor_tau_V}
\end{equation}

\subsection{Calculating constants}

A typical motor's datasheet should include graphs of the motor's measured torque
and current for different angular velocities for a given voltage applied to the
motor. To find $K_t$

\begin{align}
  \tau_m &= K_t I \nonumber \\
  K_t &= \frac{\tau_m}{I} \nonumber \\
  K_t &= \frac{\tau_{m,stall}}{I_{stall}}
\end{align}

where $\tau_{m,stall}$ is the stall torque and $I_{stall}$ is the stall current
of the motor from its datasheet.

To find $R$, recall equation (\ref{eq:motor_V}).

\begin{equation*}
  V = IR + \frac{\omega_m}{K_v}
\end{equation*}

When the motor is stalled, $\omega_m = 0$.

\begin{align}
  V &= I_{stall} R \nonumber \\
  R &= \frac{V}{I_{stall}}
\end{align}

where $I_{stall}$ is the stall current of the motor and $V$ is the voltage
applied to the motor at stall.

To find $K_v$, again recall equation (\ref{eq:motor_V}).

\begin{align*}
  V &= IR + \frac{\omega_m}{K_v} \\
  V - IR &= \frac{\omega_m}{K_v} \\
  K_v &= \frac{\omega_m}{V - IR}
\end{align*}

When the motor is spinning under no load

\begin{align}
  K_v &= \frac{\omega_{m,free}}{V - I_{free}R}
\end{align}

where $\omega_{m,free}$ is the angular velocity of the motor under no load (also
known as the free speed), and $V$ is the voltage applied to the motor when it's
spinning at $\omega_{m,free}$, and $I_{free}$ is the current drawn by the motor
under no load.

If several identical motors are being used in one gearbox for a mechanism,
multiply the stall torque, stall current, and free current by the number of
motors. This makes sense because the motor characteristics calculated above
haven't changed, but the amount of torque available and current consumed for the
same input voltage has been multiplied.

\section{Elevator}

\subsection{Equations of motion}

This elevator consists of a DC brushed motor attached to a pulley that drives a
mass up or down.

\begin{bookfigure}
  \begin{tikzpicture}[auto, >=latex', circuit ee IEC,
                      set resistor graphic=var resistor IEC graphic]
    % \draw [help lines] (-1,-3) grid (7,4);

    % Electrical equivalent circuit
    \draw (0,2) to [voltage source={direction info'={->}, info'=$V$}] (0,0);
    \draw (0,2) to [current direction={info=$I$}] (0,3);
    \draw (0,3) -- (0.5,3);
    \draw (0.5,3) to [resistor={info={$R$}}] (2,3);

    \draw (2,3) -- (2.5,3);
    \draw (2.5,3) to [voltage source={direction info'={->}, info'=$V_{emf}$}]
      (2.5,0);
    \draw (0,0) -- (2.5,0);

    % Motor
    \begin{scope}[xshift=2.4cm,yshift=1.05cm]
      \draw[fill=black] (0,0) rectangle (0.2,0.9);
      \draw[fill=white] (0.1,0.45) ellipse (0.3 and 0.3);
    \end{scope}

    % Transmission gear one
    \begin{scope}[xshift=3.75cm,yshift=1.17cm]
      \draw[fill=black!50] (0.2,0.33) ellipse (0.08 and 0.33);
      \draw[fill=black!50, color=black!50] (0,0) rectangle (0.2,0.66);
      \draw[fill=white] (0,0.33) ellipse (0.08 and 0.33);
      \draw (0,0.66) -- (0.2,0.66);
      \draw (0,0) -- (0.2,0) node[pos=0.5,below] {$G$};
    \end{scope}

    % Output shaft of motor
    \begin{scope}[xshift=2.8cm,yshift=1.45cm]
      \draw[fill=black!50] (0,0) rectangle (0.95,0.1);
    \end{scope}

    % Angular velocity arrow of drive -> transmission
    \draw[line width=0.7pt,<-] (3.2,1) arc (-30:30:1) node[above] {$\omega_m$};

    % Transmission gear two
    \begin{scope}[xshift=3.75cm,yshift=1.83cm]
      \draw[fill=black!50] (0.2,0.68) ellipse (0.13 and 0.67);
      \draw[fill=black!50, color=black!50] (0,0) rectangle (0.2,1.35);
      \draw[fill=white] (0,0.68) ellipse (0.13 and 0.67);
      \draw (0,1.35) -- (0.2,1.35);
      \draw (0,0) -- (0.2,0);
    \end{scope}

    % Pulley rear chain
    \begin{scope}[xshift=5.03cm,yshift=0.32cm]
      \draw[fill=black!70, color=black!70] (0.01,2.17) rectangle (0.09,0);
      \draw (0,2.17) -- (0,0);
      \draw (0.1,2.17) -- (0.1,0);
    \end{scope}

    % Upper pulley
    \begin{scope}[xshift=5.05cm,yshift=2.09cm]
      \draw[fill=black!50] (0.2,0.4) ellipse (0.13 and 0.4);
      \draw[fill=black!70] (0.15,0.4) ellipse (0.13 and 0.4);
      \draw[fill=black!50, color=black!50] (0,0) rectangle (0.1,0.8);
      \draw[fill=black!70, color=black!70] (0.1,0) rectangle (0.15,0.8);
      \draw[fill=black!50] (0.05,0.4) ellipse (0.13 and 0.4);
      \draw[fill=black!50, color=black!50] (0,0) rectangle (0.05,0.8);
      \draw[fill=white] (0,0.4) ellipse (0.13 and 0.4);
      \draw (0,0) -- (0.2,0);
      \draw (0,0.8) -- (0.2,0.8);
    \end{scope}

    % Lower pulley
    \begin{scope}[xshift=5.05cm,yshift=-0.05cm]
      \draw[fill=black!50] (0.2,0.4) ellipse (0.13 and 0.4);
      \draw[fill=black!70] (0.15,0.4) ellipse (0.13 and 0.4);
      \draw[fill=black!50, color=black!50] (0,0) rectangle (0.1,0.8);
      \draw[fill=black!70, color=black!70] (0.1,0) rectangle (0.15,0.8);
      \draw[fill=black!50] (0.05,0.4) ellipse (0.13 and 0.4);
      \draw[fill=black!50, color=black!50] (0,0) rectangle (0.05,0.8);
      \draw[fill=white] (0,0.4) ellipse (0.13 and 0.4);
      \draw (0,0) -- (0.2,0);
      \draw (0,0.8) -- (0.2,0.8);
    \end{scope}

    % Transmission shaft from gear two to pulley
    \begin{scope}[xshift=4.09cm,yshift=2.42cm]
      \draw[fill=black!50] (0,0) rectangle (0.96,0.1);
    \end{scope}

    % Angular velocity arrow between transmission and pulley
    \draw[line width=0.7pt,->] (4.54,1.97) arc (-30:30:1) node[above]
      {$\omega_p$};

    % Pulley front chain
    \begin{scope}[xshift=5.23cm,yshift=0.32cm]
      \draw[fill=black!70, color=black!70] (0.01,2.17) rectangle (0.09,0);
      \draw (0,2.17) -- (0,0);
      \draw (0.1,2.17) -- (0.1,0);
    \end{scope}

    % Pulley radius arrow
    \begin{scope}[xshift=5.54cm,yshift=2.49]
      \draw[line width=0.7pt,<->] (0,0) -- node[right] {$r$} (0,0.4);
    \end{scope}

    % Mass
    \begin{scope}[xshift=4.89cm,yshift=0.82cm]
      \fill[fill=white] (0,0.8) -- (0,0.2) -- (0.2,0) -- (0.2,0.2)
        -- (0.98,0.2) -- (0.78,0.8) -- cycle;
      \draw (0,0.8) -- (0.78,0.8);
      \draw (0,0.8) -- (0,0.2);
      \draw (0,0.2) -- (0.2,0);
      \draw (0,0.8) -- (0.2,0.6);
      \draw (0.78,0.8) -- (0.98,0.6);
      \draw[fill=white] (0.2,0.6) rectangle (0.98,0);
    \end{scope}

    % Mass velocity arrow
    \begin{scope}[xshift=6.04cm,yshift=0.95cm]
      \draw[line width=0.7pt,<-] (0,0.4) -- node {$v_m$} (0,0);
    \end{scope}

    % Descriptions inside graphic
    \draw (5.48,1.12) node {$m$};

    % Descriptions of subsystems under graphic
    \begin{scope}[xshift=-0.5cm,yshift=-0.28cm]
      \draw[decorate,decoration={brace,amplitude=10pt}]
        (3.5,0) -- (0,0) node[midway,yshift=-20pt] {circuit};
      \draw[decorate,decoration={brace,amplitude=10pt}]
        (7.05,0) -- (3.75,0) node[midway,yshift=-20pt] {mechanics};
    \end{scope}
  \end{tikzpicture}

  \caption{Elevator system diagram}
  \label{fig:elevator}
\end{bookfigure}

Gear ratios are written as output over input, so $G$ is greater than one in
figure \ref{fig:elevator}.

Based on figure \ref{fig:elevator}

\begin{equation}
  \tau_m G = \tau_p \label{eq:elevator_tau_m_ratio}
\end{equation}

where $G$ is the gear ratio between the motor and the pulley and $\tau_p$ is the
torque produced by the pulley.

\begin{equation}
  rF_m = \tau_p \label{eq:elevator_torque_pulley}
\end{equation}

where $r$ is the radius of the pulley. Substitute equation
(\ref{eq:elevator_tau_m_ratio}) into equation (\ref{eq:motor_tau_V}).

\begin{align*}
  V &= \frac{\frac{\tau_p}{G}}{K_t} R + \frac{\omega_m}{K_v} \\
  V &= \frac{\tau_p}{GK_t} R + \frac{\omega_m}{K_v}
\end{align*}

Substitute in equation (\ref{eq:elevator_torque_pulley}).

\begin{equation}
  V = \frac{rF_m}{GK_t} R + \frac{\omega_m}{K_v} \label{eq:elevator_Vinter1}
\end{equation}

The angular velocity of the motor armature $\omega_m$ is

\begin{equation}
  \omega_m = G \omega_p \label{eq:elevator_omega_m_ratio}
\end{equation}

where $\omega_p$ is the angular velocity of the pulley. The velocity of the mass
(the elevator carriage) is

\begin{equation*}
  v_m = r \omega_p
\end{equation*}

\begin{equation}
  \omega_p = \frac{v_m}{r} \label{eq:elevator_omega_p}
\end{equation}

Substitute equation (\ref{eq:elevator_omega_p}) into equation
(\ref{eq:elevator_omega_m_ratio}).

\begin{equation}
  \omega_m = G \frac{v_m}{r} \label{eq:elevator_omega_m}
\end{equation}

Substitute equation (\ref{eq:elevator_omega_m}) into equation
(\ref{eq:elevator_Vinter1}).

\begin{align*}
  V &= \frac{rF_m}{GK_t} R + \frac{G \frac{v_m}{R}}{K_v} \\
  V &= \frac{RrF_m}{GK_t} + \frac{G}{RK_v} v_m
\end{align*}

Solve for $F_m$.

\begin{align}
  \frac{RrF_m}{GK_t} &= V - \frac{G}{RK_v} v_m \nonumber \\
  F_m &= \left(V - \frac{G}{RK_v} v_m\right) \frac{GK_t}{Rr} \nonumber \\
  F_m &= \frac{GK_t}{Rr} V - \frac{G^2K_t}{R^2 rK_v} v_m \label{eq:elevator_F_m}
\end{align}

\begin{equation}
  \sum F = ma_m \label{eq:elevator_F_ma}
\end{equation}

where $\sum F$ is the sum of forces applied to the elevator carriage, $m$ is the
mass of the elevator carriage in kilograms, and $a_m$ is the acceleration of the
elevator carriage.

\begin{equation*}
  ma_m = F_m
\end{equation*}

\begin{remark}
  Gravity is not part of the modeled dynamics because it complicates the
  state-space \gls{model} and the controller will behave well enough without it.
\end{remark}

\begin{align}
  ma_m &= \left(\frac{GK_t}{Rr} V - \frac{G^2K_t}{R^2 rK_v} v_m\right)
    \nonumber \\
  a_m &= \frac{GK_t}{Rrm} V - \frac{G^2K_t}{R^2 rmK_v} v_m
    \label{eq:elevator_accel}
\end{align}

\subsection{Continuous state-space model}
\index{FRC models!elevator equations}

The position and velocity of the elevator can be written as

\begin{align}
  \dot{x}_m &= v_m \label{eq:elevator_cont_ss_pos} \\
  \dot{v}_m &= a_m \label{eq:elevator_cont_ss_vel}
\end{align}

where by equation (\ref{eq:elevator_accel})

\begin{equation*}
  a_m = \frac{GK_t}{Rrm} V - \frac{G^2 K_t}{R^2 rm K_v} v_m
\end{equation*}

Substitute this into equation (\ref{eq:elevator_cont_ss_vel}).

\begin{align}
  \dot{v}_m &= \frac{GK_t}{Rrm} V - \frac{G^2 K_t}{R^2 rm K_v} v_m \nonumber \\
  \dot{v}_m &= -\frac{G^2 K_t}{R^2 rm K_v} v_m + \frac{GK_t}{Rrm} V
\end{align}

\begin{align*}
  \dot{\mtx{x}} &= \mtx{A} \mtx{x} + \mtx{B} \mtx{u} \\
  \mtx{y} &= \mtx{C} \mtx{x} + \mtx{D} \mtx{u}
\end{align*}

\begin{align*}
  \mtx{x} &=
  \begin{bmatrix}
    x \\
    v_m
  \end{bmatrix} \\
  \mtx{y} &= x \\
  \mtx{u} &= V
\end{align*}

\begin{align}
  \mtx{A} &=
  \begin{bmatrix}
    0 & 1 \\
    0 & -\frac{G^2 K_t}{R^2rmK_v}
  \end{bmatrix} \\
  \mtx{B} &=
  \begin{bmatrix}
    0 \\
    \frac{GK_t}{Rrm}
  \end{bmatrix} \\
  \mtx{C} &=
  \begin{bmatrix}
    1 & 0
  \end{bmatrix} \\
  \mtx{D} &= 0
\end{align}

\subsection{Simulation}

Python Control will be used to discretize the model and simulate it. The script
at
\url{https://github.com/calcmogul/state-space-guide/blob/master/code/frccontrol/examples/elevator.py}
creates and tests a controller for it.

Figure \ref{fig:elevator_pzmaps} shows the pole-zero maps for the open-loop
system, closed-loop system, and observer. Figure \ref{fig:elevator_response}
shows the system response with them.

\begin{svg}{build/code/frccontrol/examples/elevator_pzmaps}
  \caption{Elevator pole-zero maps}
  \label{fig:elevator_pzmaps}
\end{svg}

\begin{svg}{build/code/frccontrol/examples/elevator_response}
  \caption{Elevator response}
  \label{fig:elevator_response}
\end{svg}

\section{Flywheel}

\subsection{Equations of motion}

This flywheel consists of a DC brushed motor attached to a spinning mass of
non-negligible moment of inertia.

\begin{figure}[H]
  \centering

  \begin{tikzpicture}[auto, >=latex', circuit ee IEC,
                      set resistor graphic=var resistor IEC graphic]
    % \draw [help lines] (-1,-3) grid (7,4);

    % Electrical equivalent circuit
    \draw (0,2) to [voltage source={direction info'={->}, info'=$V$}] (0,0);
    \draw (0,2) to [current direction={info=$I$}] (0,3);
    \draw (0,3) -- (0.5,3);
    \draw (0.5,3) to [resistor={info={$R$}}] (2,3);

    \draw (2,3) -- (2.5,3);
    \draw (2.5,3) to [voltage source={direction info'={->}, info'=$V_{emf}$}]
      (2.5,0);
    \draw (0,0) -- (2.5,0);

    % Motor
    \begin{scope}[xshift=2.4cm,yshift=1.05cm]
      \draw[fill=black] (0,0) rectangle (0.2,0.9);
      \draw[fill=white] (0.1,0.45) ellipse (0.3 and 0.3);
    \end{scope}

    % Transmission gear one
    \begin{scope}[xshift=3.75cm,yshift=1.17cm]
      \draw[fill=black!50] (0.2,0.33) ellipse (0.08 and 0.33);
      \draw[fill=black!50, color=black!50] (0,0) rectangle (0.2,0.66);
      \draw[fill=white] (0,0.33) ellipse (0.08 and 0.33);
      \draw (0,0.66) -- (0.2,0.66);
      \draw (0,0) -- (0.2,0) node[pos=0.5,below] {$G$};
    \end{scope}

    % Output shaft of motor
    \begin{scope}[xshift=2.8cm,yshift=1.45cm]
      \draw[fill=black!50] (0,0) rectangle (0.95,0.1);
    \end{scope}

    % Angular velocity arrow of drive -> transmission
    \draw[line width=0.7pt,<-] (3.2,1) arc (-30:30:1) node[above] {$\omega_m$};

    % Transmission gear two
    \begin{scope}[xshift=3.75cm,yshift=1.83cm]
      \draw[fill=black!50] (0.2,0.68) ellipse (0.13 and 0.67);
      \draw[fill=black!50, color=black!50] (0,0) rectangle (0.2,1.35);
      \draw[fill=white] (0,0.68) ellipse (0.13 and 0.67);
      \draw (0,1.35) -- (0.2,1.35);
      \draw (0,0) -- (0.2,0);
    \end{scope}

    % Flywheel
    \begin{scope}[xshift=5.05cm,yshift=2.09cm]
      \draw[fill=white] (0.6,0.4) ellipse (0.13 and 0.4);
      \draw[fill=white,color=white] (0,0.8) rectangle (0.6,0);
      \draw[fill=white] (0,0.4) ellipse (0.13 and 0.4);
      \draw (0,0) -- (0.6,0);
      \draw (0,0.8) -- (0.6,0.8);
    \end{scope}

    % Transmission shaft from gear two to flywheel
    \begin{scope}[xshift=4.09cm,yshift=2.42cm]
      \draw[fill=black!50] (0,0) rectangle (0.96,0.1);
    \end{scope}

    % Angular velocity arrow between transmission and flywheel
    \draw[line width=0.7pt,->] (4.54,1.97) arc (-30:30:1) node[above]
      {$\omega_f$};

    % Descriptions inside graphic
    \draw (5.45,2.49) node {$J$};

    % Descriptions of subsystems under graphic
    \begin{scope}[xshift=-0.5cm,yshift=-0.28cm]
      \draw[decorate,decoration={brace,amplitude=10pt}]
        (3.5,0) -- (0,0) node[midway,yshift=-20pt] {circuit};
      \draw[decorate,decoration={brace,amplitude=10pt}]
        (6.55,0) -- (3.75,0) node[midway,yshift=-20pt] {mechanics};
    \end{scope}
  \end{tikzpicture}

  \caption{Flywheel system diagram}
  \label{fig:flywheel}
\end{figure}

Gear ratios are written as output over input, so $G$ is greater than one in
figure \ref{fig:flywheel}. \\

We will start with the equation derived earlier for a DC brushed motor, equation
(\ref{eq:motor_tau_V}).

\begin{equation*}
  V = \frac{\tau_m}{K_t} R + \frac{\omega_m}{K_v}
\end{equation*}

Solve for the angular acceleration. First, we'll rearrange the terms because
from inspection, $V$ is the model input, $\omega_m$ is the state, and $\tau_m$
contains the angular acceleration.

\begin{equation*}
  V = \frac{R}{K_t} \tau_m + \frac{1}{K_v} \omega_m
\end{equation*}

Solve for $\tau_m$.

\begin{align*}
  V &= \frac{R}{K_t} \tau_m + \frac{1}{K_v} \omega_m \\
  \frac{R}{K_t} \tau_m &= V - \frac{1}{K_v} \omega_m \\
  \tau_m &= \frac{K_t}{R} V - \frac{K_t}{K_v R} \omega_m
\end{align*}

Since $\tau_m G = \tau_f$ and $\omega_m = G \omega_f$

\begin{align}
  \left(\frac{\tau_f}{G}\right) &= \frac{K_t}{R} V -
    \frac{K_t}{K_v R} (G \omega_f) \nonumber \\
  \frac{\tau_f}{G} &= \frac{K_t}{R} V - \frac{G K_t}{K_v R} \omega_f \nonumber
    \\
  \tau_f &= \frac{G K_t}{R} V - \frac{G^2 K_t}{K_v R} \omega_f \label{eq:tau_f}
\end{align}

The torque applied to the flywheel is defined as

\begin{equation}
  \tau_f = J \dot{\omega}_f \label{eq:tau_f_def}
\end{equation}

where $J$ is the moment of inertia of the flywheel and $\dot{\omega}_f$ is the
angular acceleration. Substitute equation (\ref{eq:tau_f_def}) into equation
(\ref{eq:tau_f}).

\begin{align}
  (J \dot{\omega}_f) &= \frac{G K_t}{R} V - \frac{G^2 K_t}{K_v R} \omega_f
    \nonumber \\
  \dot{\omega}_f &= \frac{G K_t}{RJ} V - \frac{G^2 K_t}{K_v RJ} \omega_f
    \label{eq:dot_omega_f}
\end{align}

\subsection{Continuous state-space model}

By equation (\ref{eq:dot_omega_f})

\begin{equation*}
  \dot{\omega}_f = -\frac{G^2 K_t}{K_v RJ} \omega_f + \frac{G K_t}{RJ} V
\end{equation*}

\begin{align*}
  \dot{\mtx{x}} &= \mtx{A} \mtx{x} + \mtx{B} \mtx{u} \\
  \mtx{y} &= \mtx{C} \mtx{x} + \mtx{D} \mtx{u}
\end{align*}

\begin{align*}
  \mtx{x} &= \left[
  \begin{array}{c}
    \omega_f
  \end{array}
  \right] \\
  \mtx{y} &= \omega_f \\
  \mtx{u} &= V
\end{align*}

\begin{align}
  \mtx{A} &= \left[
  \begin{array}{c}
    -\frac{G^2 K_t}{K_v RJ}
  \end{array}
  \right] \\
  \mtx{B} &= \left[
  \begin{array}{c}
    \frac{G K_t}{RJ}
  \end{array}
  \right] \\
  \mtx{C} &= 1 \\
  \mtx{D} &= 0
\end{align}

\subsection{Simulation}

Python Control will be used to discretize the model and simulate it. The script
below also creates and tests a controller for it.

\begin{snippet}
  \caption{Flywheel simulation in Python}
  \label{lst:flywheel_sim}
  \includecode[Python]{code/flywheel.py}
\end{snippet}

\section{Drivetrain}

\subsection{Equations of motion}

This drivetrain consists of two DC brushed motors per side which are chained
together on their respective sides and drive wheels which are assumed to be
massless.

\begin{bookfigure}
  \begin{tikzpicture}[auto, >=latex', circuit ee IEC,
                      set resistor graphic=var resistor IEC graphic]
    % \draw [help lines] (-1,-3) grid (7,4);

    % Right wheel
    \begin{scope}[xshift=5.78cm,yshift=1.83cm]
      \draw[fill=black!50] (0.2,0.68) ellipse (0.13 and 0.67);
      \draw[fill=black!50, color=black!50] (0,0) rectangle (0.2,1.35);
      \draw[fill=white] (0,0.68) ellipse (0.13 and 0.67);
      \draw (0,1.35) -- (0.2,1.35);
      \draw (0,0) -- (0.2,0);
    \end{scope}

    % Right transmission shaft
    \begin{scope}[xshift=5.32cm,yshift=2.42cm]
      \draw[fill=black!50] (0,0) rectangle (0.46,0.1);
    \end{scope}

    % Chassis
    \begin{scope}[xshift=4.44cm,yshift=2.09cm]
      \fill[fill=white] (0,0.8) -- (0,0.2) -- (0.2,0) -- (0.2,0.2)
        -- (0.98,0.2) -- (0.78,0.8) -- cycle;
      \draw (0,0.8) -- (0.78,0.8);
      \draw (0,0.8) -- (0,0.2);
      \draw (0,0.2) -- (0.2,0);
      \draw (0,0.8) -- (0.2,0.6);
      \draw (0.78,0.8) -- (0.98,0.6);
      \draw[fill=white] (0.2,0.6) rectangle (0.98,0);
    \end{scope}

    % Left transmission shaft
    \begin{scope}[xshift=4.09cm,yshift=2.42cm]
      \draw[fill=black!50] (0,0) rectangle (0.46,0.1);
    \end{scope}

    % Left wheel
    \begin{scope}[xshift=3.75cm,yshift=1.83cm]
      \draw[fill=black!50] (0.2,0.68) ellipse (0.13 and 0.67);
      \draw[fill=black!50, color=black!50] (0,0) rectangle (0.2,1.35);
      \draw[fill=white] (0,0.68) ellipse (0.13 and 0.67);
      \draw (0,1.35) -- (0.2,1.35);
      \draw (0,0) -- (0.2,0);
    \end{scope}

    % Angular velocity arrow for left wheel
    \draw[line width=0.7pt,->] (4.24,1.97) arc (-30:30:1) node[above]
      {$\omega_l$};

    % Angular velocity arrow for right wheel
    \draw[line width=0.7pt,->] (5.44,1.97) arc (-30:30:1) node[above]
      {$\omega_r$};

    % Wheel radius arrow
    \begin{scope}[xshift=3.5cm,yshift=1.83cm]
      \draw[line width=0.7pt,<->] (0,0) -- node[left] {$r$} (0,0.67);
    \end{scope}

    % Robot radius arrow
    \begin{scope}[xshift=4.65cm,yshift=1.83cm]
      \draw[line width=0.7pt,<->] (0,0) -- node[below] {$r_b$} (0.39,0);
    \end{scope}

    % Descriptions inside graphic
    \draw (4.99,2.42) node {$J$};
  \end{tikzpicture}

  \caption{Drivetrain system diagram}
  \label{fig:drivetrain}
\end{bookfigure}

From equation (\ref{eq:tau_f}) of the flywheel model derivations

\begin{equation}
  \tau = \frac{GK_t}{R} V - \frac{G^2K_t}{K_v R} \omega
    \label{eq:drivetrain_tau}
\end{equation}

where $\tau$ is the torque applied by one wheel of the drivetrain, $G$ is the
gear ratio of the drivetrain, $K_t$ is the torque constant of the motor, $R$ is
the resistance of the motor, and $K_v$ is the angular velocity constant. Since
$\tau = rF$ and $\omega = \frac{v}{r}$ where $v$ is the velocity of a given
drivetrain side along the ground and $r$ is the drivetrain wheel radius

\begin{align*}
  (rF) = \frac{GK_t}{R} V - \frac{G^2K_t}{K_v R} \left(\frac{v}{r}\right) \\
  rF = \frac{GK_t}{R} V - \frac{G^2K_t}{K_v Rr} v \\
  F = \frac{GK_t}{Rr} V - \frac{G^2K_t}{K_v Rr^2} v \\
  F = -\frac{G^2K_t}{K_v Rr^2} v + \frac{GK_t}{Rr} V
\end{align*}

Therefore, for each side of the robot

\begin{align*}
  F_l &= -\frac{G_l^2 K_t}{K_v R r^2} v_l + \frac{G_l K_t}{Rr} V_l \\
  F_r &= -\frac{G_r^2 K_t}{K_v R r^2} v_r + \frac{G_r K_t}{Rr} V_r
\end{align*}

where the $l$ and $r$ subscripts denote the side of the robot to which each
variable corresponds.

Let $C_1 = -\frac{G_l^2 K_t}{K_v R r^2}$, $C_2 = \frac{G_l K_t}{Rr}$,
$C_3 = -\frac{G_r^2 K_t}{K_v R r^2}$, and $C_4 = \frac{G_r K_t}{Rr}$.

\begin{align}
  F_l &= C_1 v_l + C_2 V_l \label{eq:drivetrain_Fl} \\
  F_r &= C_3 v_r + C_4 V_r \label{eq:drivetrain_Fr}
\end{align}

First, find the sum of forces.

\begin{align}
  \sum F &= ma \nonumber \\
  F_l + F_r &= m \dot{v} \nonumber \\
  F_l + F_r &= m \frac{\dot{v}_l + \dot{v}_r}{2} \nonumber \\
  \frac{2}{m} (F_l + F_r) &= \dot{v}_l + \dot{v}_r \nonumber \\
  \dot{v}_l &= \frac{2}{m} (F_l + F_r) - \dot{v}_r \label{eq:drivetrain_dotv_l}
\end{align}

Next, find the sum of torques.

\begin{align*}
  \sum \tau &= J \dot{\omega} \\
  \tau_l + \tau_r &= J \left(\frac{\dot{v}_r - \dot{v}_l}{2 r_b}\right)
\end{align*}

where $r_b$ is the radius of the drivetrain.

\begin{align*}
  (-r_b F_l) + (r_b F_r) &= J \frac{\dot{v}_r - \dot{v}_l}{2 r_b} \\
  -r_b F_l + r_b F_r &= \frac{J}{2 r_b} (\dot{v}_r - \dot{v}_l) \\
  -F_l + F_r &= \frac{J}{2 r_b^2} (\dot{v}_r - \dot{v}_l) \\
  \frac{2 r_b^2}{J} (-F_l + F_r) &= \dot{v}_r - \dot{v}_l \\
  \dot{v}_r &= \dot{v}_l + \frac{2 r_b^2}{J} (-F_l + F_r)
\end{align*}

Substitute in equation (\ref{eq:drivetrain_dotv_l}) to obtain an expression for
$\dot{v}_r$.

\begin{align}
  \dot{v}_r &= \left(\frac{2}{m} (F_l + F_r) - \dot{v}_r\right) +
    \frac{2 r_b^2}{J} (-F_l + F_r) \nonumber \\
  2\dot{v}_r &= \frac{2}{m} (F_l + F_r) + \frac{2 r_b^2}{J} (-F_l + F_r)
    \nonumber \\
  \dot{v}_r &= \frac{1}{m} (F_l + F_r) + \frac{r_b^2}{J} (-F_l + F_r)
    \label{eq:drivetrain_vr_2mid} \\
  \dot{v}_r &= \frac{1}{m} F_l + \frac{1}{m} F_r - \frac{r_b^2}{J} F_l +
    \frac{r_b^2}{J} F_r \nonumber \\
  \dot{v}_r &= \left(\frac{1}{m} - \frac{r_b^2}{J}\right) F_l +
    \left(\frac{1}{m} + \frac{r_b^2}{J}\right) F_r \label{eq:drivetrain_vr_2}
\end{align}

Substitute equation (\ref{eq:drivetrain_vr_2mid}) back into equation
(\ref{eq:drivetrain_dotv_l}) to obtain an expression for $\dot{v}_l$.

\begin{align}
  \dot{v}_l &= \frac{2}{m} (F_l + F_r) - \left(\frac{1}{m} (F_l + F_r) +
    \frac{r_b^2}{J} (-F_l + F_r)\right) \nonumber \\
  \dot{v}_l &= \frac{1}{m} (F_l + F_r) - \frac{r_b^2}{J} (-F_l + F_r)
    \nonumber \\
  \dot{v}_l &= \frac{1}{m} (F_l + F_r) + \frac{r_b^2}{J} (F_l - F_r) \nonumber
    \\
  \dot{v}_l &= \frac{1}{m} F_l + \frac{1}{m} F_r + \frac{r_b^2}{J} F_l -
    \frac{r_b^2}{J} F_r \nonumber \\
  \dot{v}_l &= \left(\frac{1}{m} + \frac{r_b^2}{J}\right) F_l +
    \left(\frac{1}{m} - \frac{r_b^2}{J}\right) F_r \label{eq:drivetrain_vl_2}
\end{align}

Now, plug the expressions for $F_l$ and $F_r$ into equation
(\ref{eq:drivetrain_vr_2}).

\begin{align}
  \dot{v}_r &= \left(\frac{1}{m} - \frac{r_b^2}{J}\right) F_l +
    \left(\frac{1}{m} + \frac{r_b^2}{J}\right) F_r \nonumber \\
  \dot{v}_r &= \left(\frac{1}{m} - \frac{r_b^2}{J}\right)
    \left(C_1 v_l + C_2 V_l\right) +
    \left(\frac{1}{m} + \frac{r_b^2}{J}\right) \left(C_3 v_r + C_4 V_r\right)
    \label{eq:drivetrain_model_left}
\end{align}

Now, plug the expressions for $F_l$ and $F_r$ into equation
(\ref{eq:drivetrain_vl_2}).

\begin{align}
  \dot{v}_l &= \left(\frac{1}{m} + \frac{r_b^2}{J}\right) F_l +
    \left(\frac{1}{m} - \frac{r_b^2}{J}\right) F_r \nonumber \\
  \dot{v}_l &= \left(\frac{1}{m} + \frac{r_b^2}{J}\right)
    \left(C_1 v_l + C_2 V_l\right) +
    \left(\frac{1}{m} - \frac{r_b^2}{J}\right) \left(C_3 v_r + C_4 V_r\right)
    \label{eq:drivetrain_model_right}
\end{align}

\subsection{Continuous state-space model}
\index{FRC models!drivetrain equations}

The position and velocity of each drivetrain side can be written as

\begin{align}
  \dot{x}_l &= v_l \label{eq:drivetrain_cont_ss_posl} \\
  \dot{v}_l &= \dot{v}_l \label{eq:drivetrain_cont_ss_vell} \\
  \dot{x}_r &= v_r \label{eq:drivetrain_cont_ss_posr} \\
  \dot{v}_r &= \dot{v}_r \label{eq:drivetrain_cont_ss_velr}
\end{align}

By equations (\ref{eq:drivetrain_model_left}) and
(\ref{eq:drivetrain_model_right})

\begin{align*}
  \dot{v}_r &= \left(\frac{1}{m} - \frac{r_b^2}{J}\right)
    \left(C_1 v_l + C_2 V_l\right) +
    \left(\frac{1}{m} + \frac{r_b^2}{J}\right) \left(C_3 v_r + C_4 V_r\right)
    \\
  \dot{v}_l &= \left(\frac{1}{m} + \frac{r_b^2}{J}\right)
    \left(C_1 v_l + C_2 V_l\right) +
    \left(\frac{1}{m} - \frac{r_b^2}{J}\right) \left(C_3 v_r + C_4 V_r\right)
\end{align*}

\begin{align*}
  \dot{\mtx{x}} &= \mtx{A} \mtx{x} + \mtx{B} \mtx{u} \\
  \mtx{y} &= \mtx{C} \mtx{x} + \mtx{D} \mtx{u}
\end{align*}

\begin{align*}
  \mtx{x} &=
  \begin{bmatrix}
    x_l \\
    v_l \\
    x_r \\
    v_r
  \end{bmatrix} \\
  \mtx{y} &=
  \begin{bmatrix}
    x_l \\
    x_r
  \end{bmatrix} \\
  \mtx{u} &=
  \begin{bmatrix}
    V_l \\
    V_r
  \end{bmatrix}
\end{align*}

\begin{align}
  \mtx{A} &=
  \begin{bmatrix}
    0 & 1 & 0 & 0 \\
    0 & \left(\frac{1}{m} - \frac{r_b^2}{J}\right) C_1 & 0 & \left(\frac{1}{m} + \frac{r_b^2}{J}\right) C_3 \\
    0 & 0 & 0 & 1 \\
    0 & \left(\frac{1}{m} + \frac{r_b^2}{J}\right) C_1 & 0 & \left(\frac{1}{m} - \frac{r_b^2}{J}\right) C_3
  \end{bmatrix} \\
  \mtx{B} &=
  \begin{bmatrix}
    0 & 0 \\
    \left(\frac{1}{m} + \frac{r_b^2}{J}\right) C_2 & \left(\frac{1}{m} - \frac{r_b^2}{J}\right) C_4 \\
    0 & 0 \\
    \left(\frac{1}{m} - \frac{r_b^2}{J}\right) C_2 & \left(\frac{1}{m} + \frac{r_b^2}{J}\right) C_4
  \end{bmatrix} \\
  \mtx{C} &=
  \begin{bmatrix}
    1 & 0 & 0 & 0 \\
    0 & 0 & 1 & 0 \\
  \end{bmatrix} \\
  \mtx{D} &= \mtx{0}_{2 \times 2}
\end{align}

where $C_1 = -\frac{G_l^2 K_t}{K_v R r^2}$, $C_2 = \frac{G_l K_t}{Rr}$,
$C_3 = -\frac{G_r^2 K_t}{K_v R r^2}$, and $C_4 = \frac{G_r K_t}{Rr}$.

\subsection{Simulation}

Python Control will be used to discretize the model and simulate it. The script
at
\url{https://github.com/calcmogul/frccontrol/blob/master/examples/drivetrain.py}
creates and tests a controller for it.

\begin{remark}
  Python Control currently doesn't support finding the transmission zeroes of
  MIMO systems with a different number of inputs than outputs, so
  \texttt{control.pzmap()} and \texttt{frccontrol.System.plot\_pzmaps()} fail
  with an error if Slycot isn't installed.
\end{remark}

Figure \ref{fig:drivetrain_pzmaps} shows the pole-zero maps for the open-loop
system, closed-loop system, and observer. Figure \ref{fig:drivetrain_response}
shows the system response with them.

\begin{svg}{build/drivetrain_pzmaps}
  \caption{Drivetrain pole-zero maps}
  \label{fig:drivetrain_pzmaps}
\end{svg}

\begin{svg}{build/drivetrain_response}
  \caption{Drivetrain response}
  \label{fig:drivetrain_response}
\end{svg}

Given the high inertia in drivetrains, it's better to drive the reference with a
motion profile instead of a step input for reproducibility.

\subsection{Implementation}

The script linked above also generates two files: DrivetrainCoeffs.h and
DrivetrainCoeffs.cpp. These can be used with the WPILib StateSpacePlant,
StateSpaceController, and StateSpaceObserver classes in C++ and Java. A C++
implementation of this drivetrain controller is located at
\url{https://github.com/calcmogul/allwpilib/tree/state-space/wpilibcExamples/src/main/cpp/examples/StateSpaceDrivetrain}.

\section{Single-jointed arm}

\subsection{Equations of motion}

This single-jointed arm consists of a DC brushed motor attached to a pulley that
spins a straight bar in pitch.

\begin{bookfigure}
  \begin{tikzpicture}[auto, >=latex', circuit ee IEC,
                      set resistor graphic=var resistor IEC graphic]
    % \draw [help lines] (-1,-3) grid (7,4);

    % Electrical equivalent circuit
    \draw (0,2) to [voltage source={direction info'={->}, info'=$V$}] (0,0);
    \draw (0,2) to [current direction={info=$I$}] (0,3);
    \draw (0,3) -- (0.5,3);
    \draw (0.5,3) to [resistor={info={$R$}}] (2,3);

    \draw (2,3) -- (2.5,3);
    \draw (2.5,3) to [voltage source={direction info'={->}, info'=$V_{emf}$}]
      (2.5,0);
    \draw (0,0) -- (2.5,0);

    % Motor
    \begin{scope}[xshift=2.4cm,yshift=1.05cm]
      \draw[fill=black] (0,0) rectangle (0.2,0.9);
      \draw[fill=white] (0.1,0.45) ellipse (0.3 and 0.3);
    \end{scope}

    % Transmission gear one
    \begin{scope}[xshift=3.75cm,yshift=1.17cm]
      \draw[fill=black!50] (0.2,0.33) ellipse (0.08 and 0.33);
      \draw[fill=black!50, color=black!50] (0,0) rectangle (0.2,0.66);
      \draw[fill=white] (0,0.33) ellipse (0.08 and 0.33);
      \draw (0,0.66) -- (0.2,0.66);
      \draw (0,0) -- (0.2,0) node[pos=0.5,below] {$G$};
    \end{scope}

    % Output shaft of motor
    \begin{scope}[xshift=2.8cm,yshift=1.45cm]
      \draw[fill=black!50] (0,0) rectangle (0.95,0.1);
    \end{scope}

    % Angular velocity arrow of drive -> transmission
    \draw[line width=0.7pt,<-] (3.2,1) arc (-30:30:1) node[above] {$\omega_m$};

    % Transmission gear two
    \begin{scope}[xshift=3.75cm,yshift=1.83cm]
      \draw[fill=black!50] (0.2,0.68) ellipse (0.13 and 0.67);
      \draw[fill=black!50, color=black!50] (0,0) rectangle (0.2,1.35);
      \draw[fill=white] (0,0.68) ellipse (0.13 and 0.67);
      \draw (0,1.35) -- (0.2,1.35);
      \draw (0,0) -- (0.2,0);
    \end{scope}

    \begin{scope}[xshift=5.075cm,yshift=2.4cm]
      % Single-jointed arm
      \draw[fill=white] (0,0) -- (0.1,-0.05) -- (0.35,1.45) -- (0.25,1.5)
        -- cycle;
      \draw[fill=black!50] (0.1,-0.05) -- (0.3,-0.05) -- (0.55,1.45) --
        (0.35,1.45) -- cycle;
      \draw[fill=white] (0.25,1.5) -- (0.35,1.45) -- (0.55,1.45) -- (0.45,1.5)
        -- cycle;

      % Arm length arrow
      \draw[line width=0.7pt,<->] (0.55,-0.05) -- node[right] {$l$} (0.8,1.45);

      % Mass label
      \draw (-0.05,1.2) node {$m$};
    \end{scope}

    % Transmission shaft from gear two to arm
    \begin{scope}[xshift=4.09cm,yshift=2.42cm]
      \draw[fill=black!50] (0,0) rectangle (1.06,0.1);
    \end{scope}

    % Angular velocity arrow between transmission and arm
    \draw[line width=0.7pt,->] (4.54,1.97) arc (-30:30:1) node[above]
      {$\omega_{arm}$};

    % Descriptions of subsystems under graphic
    \begin{scope}[xshift=-0.5cm,yshift=-0.28cm]
      \draw[decorate,decoration={brace,amplitude=10pt}]
        (3.5,0) -- (0,0) node[midway,yshift=-20pt] {circuit};
      \draw[decorate,decoration={brace,amplitude=10pt}]
        (6.55,0) -- (3.75,0) node[midway,yshift=-20pt] {mechanics};
    \end{scope}
  \end{tikzpicture}

  \caption{Single-jointed arm system diagram}
  \label{fig:single_jointed_arm}
\end{bookfigure}

Gear ratios are written as output over input, so $G$ is greater than one in
figure \ref{fig:single_jointed_arm}.

We will start with the equation derived earlier for a DC brushed motor, equation
(\ref{eq:motor_tau_V}).

\begin{equation*}
  V = \frac{\tau_m}{K_t} R + \frac{\omega_m}{K_v}
\end{equation*}

Solve for the angular acceleration. First, we'll rearrange the terms because
from inspection, $V$ is the \gls{model} \gls{input}, $\omega_m$ is the
\gls{state}, and $\tau_m$ contains the angular acceleration.

\begin{equation*}
  V = \frac{R}{K_t} \tau_m + \frac{1}{K_v} \omega_m
\end{equation*}

Solve for $\tau_m$.

\begin{align*}
  V &= \frac{R}{K_t} \tau_m + \frac{1}{K_v} \omega_m \\
  \frac{R}{K_t} \tau_m &= V - \frac{1}{K_v} \omega_m \\
  \tau_m &= \frac{K_t}{R} V - \frac{K_t}{K_v R} \omega_m
\end{align*}

Since $\tau_m G = \tau_{arm}$ and $\omega_m = G \omega_{arm}$

\begin{align}
  \left(\frac{\tau_{arm}}{G}\right) &= \frac{K_t}{R} V -
    \frac{K_t}{K_v R} (G \omega_f) \nonumber \\
  \frac{\tau_{arm}}{G} &= \frac{K_t}{R} V - \frac{G K_t}{K_v R} \omega_{arm}
    \nonumber \\
  \tau_{arm} &= \frac{G K_t}{R} V - \frac{G^2 K_t}{K_v R} \omega_{arm}
    \label{eq:tau_arm}
\end{align}

The angular velocity of the arm is defined as

\begin{equation}
  \tau_{arm} = J \dot{\omega}_{arm} \label{eq:tau_arm_def}
\end{equation}

where $J$ is the moment of inertia of the arm and $\dot{\omega}_{arm}$ is the
angular acceleration. Substitute equation (\ref{eq:tau_arm_def}) into equation
(\ref{eq:tau_arm}).

\begin{align}
  (J \dot{\omega}_{arm}) &= \frac{G K_t}{R} V - \frac{G^2 K_t}{K_v R}
    \omega_{arm} \nonumber \\
  \dot{\omega}_{arm} &= \frac{G K_t}{RJ} V - \frac{G^2 K_t}{K_v RJ} \omega_{arm}
    \label{eq:dot_omega_arm}
\end{align}

$J$ can be approximated as the moment of inertia of a thin rod rotating around
one end. Therefore

\begin{equation}
  J = \frac{1}{3}ml^2
\end{equation}

where $m$ is the mass of the arm and $l$ is the length of the arm.

\subsection{Continuous state-space model}
\index{FRC models!single-jointed arm equations}

The position and velocity of the elevator can be written as

\begin{align}
  \dot{\theta}_{arm} &= \omega_{arm} \label{eq:arm_cont_ss_pos} \\
  \dot{\omega}_{arm} &= \dot{\omega}_{arm} \label{eq:arm_cont_ss_vel}
\end{align}

By equation (\ref{eq:dot_omega_arm})

\begin{equation*}
  \dot{\omega}_{arm} = -\frac{G^2 K_t}{K_v RJ} \omega_{arm} + \frac{G K_t}{RJ} V
\end{equation*}

\begin{theorem}[Single-jointed arm state-space model]
  \begin{align*}
    \dot{\mtx{x}} &= \mtx{A} \mtx{x} + \mtx{B} \mtx{u} \\
    \mtx{y} &= \mtx{C} \mtx{x} + \mtx{D} \mtx{u}
  \end{align*}
  \begin{equation*}
    \begin{array}{ccc}
      \mtx{x} =
      \begin{bmatrix}
        \theta_{arm} \\
        \omega_{arm}
      \end{bmatrix} &
      \mtx{y} = \theta_{arm} &
      \mtx{u} = V
    \end{array}
  \end{equation*}
  \begin{equation}
    \begin{array}{cccc}
      \mtx{A} =
      \begin{bmatrix}
        0 & 1 \\
        0 & -\frac{G^2 K_t}{K_v RJ}
      \end{bmatrix} &
      \mtx{B} =
      \begin{bmatrix}
        0 \\
        \frac{G K_t}{RJ}
      \end{bmatrix} &
      \mtx{C} =
      \begin{bmatrix}
        1 & 0
      \end{bmatrix} &
      \mtx{D} = 0
    \end{array}
  \end{equation}
\end{theorem}

\subsection{Model augmentation}

As per subsection \ref{subsec:u_error_estimation}, we will now augment the
\gls{model} so a $u_{error}$ term is added to the \gls{control input}.

The \gls{plant} and \gls{observer} augmentations should be performed before the
\gls{model} is \glslink{discretization}{discretized}. After the \gls{controller}
gain is computed with the unaugmented discrete \gls{model}, the controller may
be augmented. Therefore, the \gls{plant} and \gls{observer} augmentations assume
a continuous \gls{model} and the \gls{controller} augmentation assumes a
discrete \gls{controller}.

\begin{equation*}
  \begin{array}{ccc}
    \mtx{x}_{aug} =
    \begin{bmatrix}
      \mtx{x} \\
      u_{error}
    \end{bmatrix} &
    \mtx{y} = \theta_{arm} &
    \mtx{u} = V
  \end{array}
\end{equation*}

\begin{equation}
  \begin{array}{cccc}
    \mtx{A}_{aug} =
    \begin{bmatrix}
      \mtx{A} & \mtx{B} \\
      \mtx{0}_{1 \times 2} & 0
    \end{bmatrix} &
    \mtx{B}_{aug} =
    \begin{bmatrix}
      \mtx{B} \\
      0
    \end{bmatrix} &
    \mtx{C}_{aug} =
    \begin{bmatrix}
      \mtx{C} & 0
    \end{bmatrix} &
    \mtx{D}_{aug} = \mtx{D}
  \end{array}
\end{equation}

\begin{equation}
  \begin{array}{cc}
    \mtx{K}_{aug} = \begin{bmatrix}
      \mtx{K} & 1
    \end{bmatrix} &
    \mtx{r}_{aug} = \begin{bmatrix}
      \mtx{r} \\
      0
    \end{bmatrix}
  \end{array}
\end{equation}

This will compensate for unmodeled dynamics such as gravity or other external
loading from lifted objects. However, if only gravity compensation is desired,
a feedforward of the form $u_{ff} = V_{gravity} \cos\theta$ is preferred where
$V_{gravity}$ is the voltage required to hold the arm level with the ground and
$\theta$ is the angle of the arm with the ground.

\subsection{Simulation}

Python Control will be used to \glslink{discretization}{discretize} the
\gls{model} and simulate it. One of the frccontrol
examples\footnote{\url{https://github.com/calcmogul/frccontrol/blob/master/examples/single_jointed_arm.py}}
creates and tests a controller for it.

Figure \ref{fig:single_jointed_arm_pzmaps} shows the pole-zero maps for the
open-loop \gls{system}, closed-loop \gls{system}, and \gls{observer}. Figure
\ref{fig:single_jointed_arm_response} shows the \gls{system} response with them.

\begin{svg}{build/single_jointed_arm_pzmaps}
  \caption{Single-jointed arm pole-zero maps}
  \label{fig:single_jointed_arm_pzmaps}
\end{svg}

\begin{svg}{build/single_jointed_arm_response}
  \caption{Single-jointed arm response}
  \label{fig:single_jointed_arm_response}
\end{svg}

\subsection{Implementation}

The script linked above also generates two files: SingleJointedArmCoeffs.h and
SingleJointedArmCoeffs.cpp. These can be used with the WPILib StateSpacePlant,
StateSpaceController, and StateSpaceObserver classes in C++ and Java. A C++
implementation of this single-jointed arm controller is available
online\footnote{\url{https://github.com/calcmogul/allwpilib/tree/state-space/wpilibcExamples/src/main/cpp/examples/StateSpaceSingleJointedArm}}.

\section{Rotating claw}

\subsection{Equations of motion}

This claw consists of independent upper and lower jaw pieces each driven by its
own DC brushed motor.

\subsection{Continuous state-space model}
\index{FRC models!rotating claw equations}

\subsection{Simulation}



% Appendices
\part{Appendices}
\appendix

% Appendix text font settings
\renewcommand{\chaptermark}[1]{\markboth{\sffamily\normalsize\bfseries\appendixname\ \thechapter.\ #1}{}}
\chapterimage{appendices.jpg}{Sunset in an airplane over New Mexico}
\chapter{Transfer function feedback derivation}
\label{ch:app-tf-feedback-deriv}

Given the feedback network in figure \ref{fig:closed_loop_deriv}, find an
expression for $Y(s)$.

\begin{bookfigure}
  \begin{tikzpicture}[auto, >=latex']
    % Place the blocks
    \node [name=input] {$X(s)$};
    \node [sum, right=of input] (sum) {};
    \node [block, right=of sum] (G) {$G(s)$};
    \node [right=of G] (output) {$Y(s)$};
    \node [block, below=of G] (measurements) {$H(s)$};

    % Connect the nodes
    \draw [arrow] (input) -- node[pos=0.85] {$+$} (sum);
    \draw [arrow] (sum) -- node {$Z(s)$} (G);
    \draw [arrow] (G) -- node [name=y] {} (output);
    \draw [arrow] (y) |- (measurements);
    \draw [arrow] (measurements) -| node[pos=0.99, right] {$-$} (sum);
  \end{tikzpicture}

  \caption{Closed loop block diagram}
  \label{fig:closed_loop_deriv}
\end{bookfigure}

\begin{align}
  Y(s) &= Z(s) G(s) \nonumber \\
  Z(s) &= X(s) - Y(s) H(s) \nonumber \\
  X(s) &= Z(s) + Y(s) H(s) \nonumber \\
  X(s) &= Z(s) + Z(s) G(s) H(s) \nonumber \\
  \frac{Y(s)}{X(s)} &= \frac{Z(s) G(s)}{Z(s) + Z(s) G(s) H(s)} \nonumber \\
  \frac{Y(s)}{X(s)} &= \frac{G(s)}{1 + G(s) H(s)}
\end{align}

A more general form is

\begin{equation}
  \frac{Y(s)}{X(s)} = \frac{G(s)}{1 \mp G(s) H(s)}
\end{equation}

where positive feedback uses the top sign and negative feedback uses the bottom
sign.

\chapter{Installing Python Control} \label{ch:app-installing-python-control}

\section{Windows}

Download Python 3.5 or higher from \url{https://www.python.org/downloads/} and
install it. Run \texttt{py -3 -m pip install control} via cmd.exe or Powershell
to install Python Control and its dependencies.

\section{Linux}

Install the appropriate packages from table \ref{tab:required_system_packages}
using your system's package manager. Run \texttt{pip3 install --user control} to
install Python Control and its dependencies. Using \texttt{--user} makes
installation not require root privileges.

\begin{booktable}
  \begin{tabular}{|ll|}
    \hline
    \rowcolor{headingbg}
    \textbf{Debian/Ubuntu} & \textbf{Arch Linux} \\
    \hline
    python3 & python \\
    python3-pip & python-pip \\
    \hline
  \end{tabular}
  \caption{Required system packages}
  \label{tab:required_system_packages}
\end{booktable}

\chapter{Optimal control law derivation}
\label{ch:app-optimal-control-law-deriv}

For a continuous-time linear system described by

\begin{equation}
  \dot{\mtx{x}} = \mtx{A}\mtx{x} + \mtx{B}\mtx{u}
\end{equation}

with the cost function

\begin{equation*}
  J = \int\limits_0^\infty \left(\mtx{x}^T\mtx{Q}\mtx{x} +
    \mtx{u}^T\mtx{R}\mtx{u}\right) dt
\end{equation*}

where $J$ represents a tradeoff between \gls{state} excursion and control effort
with the weighting factors $\mtx{Q}$ and $\mtx{R}$, the feedback
\gls{control law} which minimizes $J$ is

\begin{equation*}
  \mtx{u} = -\mtx{K}\mtx{x}
\end{equation*}

where $\mtx{K}$ is given by

\begin{equation*}
  \mtx{K} = \mtx{R}^{-1} \left(\mtx{B}^T\mtx{P} + \mtx{N}^T\right)
\end{equation*}

and $\mtx{P}$ is found by solving the continuous-time algebraic Riccati equation
defined as

\begin{equation*}
  \mtx{A}^T\mtx{P} + \mtx{P}\mtx{A} - \left(\mtx{P}\mtx{B} +
    \mtx{N}\right) \mtx{R}^{-1} \left(\mtx{B}^T\mtx{P} + \mtx{N}^T\right) +
    \mtx{Q} = 0
\end{equation*}

or alternatively

\begin{equation*}
  \mathscrbf{A}^T\mtx{P} + \mtx{P}\mathscrbf{A} -
    \mtx{P}\mtx{B}\mtx{R}^{-1}\mtx{B}^T\mtx{P} + \mathscrbf{Q} = 0
\end{equation*}

with

\begin{align*}
  \mathscrbf{A} &= \mtx{A} - \mtx{B}\mtx{R}^{-1}\mtx{N}^T \\
  \mathscrbf{Q} &= \mtx{Q} - \mtx{N}\mtx{R}^{-1}\mtx{N}^T
\end{align*}

Snippet \ref{lst:dlqr} computes the optimal infinite-horizon, discrete-time
LQR controller.

\begin{snippet}
  \caption{Infinite-horizon, discrete-time LQR computation in Python}
  \label{lst:dlqr}
  \includecode[Python]{code/frccontrol/dlqr.py}
\end{snippet}

Other formulations of LQR for finite-horizon and discrete-time can be seen at
\url{https://en.wikipedia.org/wiki/Linear–quadratic_regulator}.

\chapter{Luenberger observer derivations}

\section{Kalman filter as Luenberger observer}
\label{subsec:app-kalman-luenberger}

A Luenberger observer is defined as

\begin{align}
  \hat{\mtx{x}}_{k+1}^+ &= \mtx{A}\hat{\mtx{x}}_k^- + \mtx{B}\mtx{u}_k + \mtx{L}
    (\mtx{y}_k - \hat{\mtx{y}}_k) \label{eq:luenberger1} \\
  \hat{\mtx{y}}_k &= \mtx{C} \hat{\mtx{x}}_k^- \label{eq:luenberger2}
\end{align}

where a superscript of minus denotes \textit{a priori} and plus denotes
\textit{a posteriori} estimate. Combining equation (\ref{eq:luenberger1}) and
equation (\ref{eq:luenberger2}) gives

\begin{equation} \label{eq:luenberger}
  \hat{\mtx{x}}_{k+1}^+ = \mtx{A}\hat{\mtx{x}}_k^- + \mtx{B}\mtx{u}_k + \mtx{L}
    (\mtx{y}_k - \mtx{C}\hat{\mtx{x}}_k^-)
\end{equation}

The following is a Kalman filter that considers the current update step and the
next predict step together rather than the current predict step and current
update step.

\begin{align}
  \text{Update step} \nonumber \\
  \mtx{K}_k &= \mtx{P}_k^- \mtx{H}^T (\mtx{H}\mtx{P}_k^- \mtx{H}^T +
    \mtx{R})^{-1} \\
  \hat{\mtx{x}}_k^+ &= \hat{\mtx{x}}_k^- + \mtx{K}_k(\mtx{y}_k -
    \mtx{H}\hat{\mtx{x}}_k^-) \label{eq:post2_x} \\
  \mtx{P}_k^+ &= (\mtx{I} - \mtx{K}_k\mtx{H})\mtx{P}_k^- \\
  \text{Predict step} \nonumber \\
  \hat{\mtx{x}}_{k+1}^+ &= \mtx{A}\hat{\mtx{x}}_k^+ + \mtx{B}\mtx{u}_k
    \label{eq:pre2_x} \\
  \mtx{P}_{k+1}^- &= \mtx{A} \mtx{P}_k^+ \mtx{A}^T +
    \mtx{\Gamma}\mtx{Q}\mtx{\Gamma}^T
\end{align}

Substitute equation (\ref{eq:post2_x}) into equation (\ref{eq:pre2_x}).

\begin{align*}
  \hat{\mtx{x}}_{k+1}^+ &= \mtx{A}(\hat{\mtx{x}}_k^- + \mtx{K}_k(\mtx{y}_k -
    \mtx{H}\hat{\mtx{x}}_k^-)) + \mtx{B}\mtx{u}_k \\
  \hat{\mtx{x}}_{k+1}^+ &= \mtx{A}\mtx{x}_k^- + \mtx{A}\mtx{K}_k(\mtx{y}_k -
    \mtx{H}\hat{\mtx{x}}_k^-) + \mtx{B}\mtx{u}_k \\
  \hat{\mtx{x}}_{k+1}^+ &= \mtx{A}\hat{\mtx{x}}_k^- + \mtx{B}\mtx{u}_k +
    \mtx{A}\mtx{K}_k(\mtx{y}_k - \mtx{H}\hat{\mtx{x}}_k^-)
\end{align*}

Let $\mtx{C} = \mtx{H}$ and $\mtx{L} = \mtx{A} \mtx{K}_k$.

\begin{equation} \label{eq:app_kalman_leunberger}
  \hat{\mtx{x}}_{k+1}^+ = \mtx{A}\hat{\mtx{x}}_k^- + \mtx{B}\mtx{u}_k + \mtx{L}
    (\mtx{y}_k - \mtx{C}\hat{\mtx{x}}_k^-)
\end{equation}

which matches equation (\ref{eq:luenberger}). Therefore, the eigenvalues of the
Kalman filter observer can be obtained by

\begin{align}
  &eig(\mtx{A} - \mtx{L}\mtx{C}) \nonumber \\
  &eig(\mtx{A} - (\mtx{A}\mtx{K}_k)(\mtx{H})) \nonumber \\
  &eig(\mtx{A}(\mtx{I} - \mtx{K}_k\mtx{H}))
\end{align}

\subsection{Luenberger observer with separate prediction and update}
\label{subsec:app-luenberger-separate}

To run a Luenberger observer with separate prediction and update steps,
substitute the relationship between the Luenberger observer and Kalman filter
matrices derived above into the Kalman filter equations.

Appendix \ref{subsec:app-kalman-luenberger} shows that $\mtx{C} = \mtx{H}$ and
$\mtx{L} = \mtx{A}\mtx{K}_k$. Since $\mtx{L}$ and $\mtx{A}$ are constant, one
must assume $\mtx{K}_k$ has reached steady-state. Then,
$\mtx{K} = \mtx{A}^{-1}\mtx{L}$. Substitute this and $\mtx{C} = \mtx{H}$ into
the Kalman filter update equation.

\begin{align*}
  \hat{\mtx{x}}_{k+1}^+ &= \hat{\mtx{x}}_{k+1}^- + \mtx{K}(\mtx{y}_{k+1} -
    \mtx{H}\hat{\mtx{x}}_{k+1}^-) \\
  \hat{\mtx{x}}_{k+1}^+ &= \hat{\mtx{x}}_{k+1}^- + \mtx{A}^{-1}\mtx{L}
    (\mtx{y}_{k+1} - \mtx{C}\hat{\mtx{x}}_{k+1}^-)
\end{align*}

Substitute in equation (\ref{eq:z_obsv_y}).

\begin{equation*}
  \hat{\mtx{x}}_{k+1}^+ = \hat{\mtx{x}}_{k+1}^- + \mtx{A}^{-1}\mtx{L}
    (\mtx{y}_{k+1} - \hat{\mtx{y}}_{k+1})
\end{equation*}

The predict step is the same as the Kalman filter's. Therefore, a Luenberger
observer run with prediction and update steps is written as follows.

\begin{align}
  \text{Predict step} \nonumber \\
  \hat{\mtx{x}}_{k+1}^- &= \mtx{A}\hat{\mtx{x}}_k^- + \mtx{B}\mtx{u}_k \\
  \text{Update step} \nonumber \\
  \hat{\mtx{x}}_{k+1}^+ &= \hat{\mtx{x}}_{k+1}^- + \mtx{A}^{-1}\mtx{L}
    (\mtx{y}_{k+1} - \hat{\mtx{y}}_{k+1}) \\
  \hat{\mtx{y}}_{k+1} &= \mtx{C} \hat{\mtx{x}}_{k+1}^-
\end{align}


% Back matter
\renewcommand{\chaptermark}[1]{\markboth{\sffamily\normalsize\bfseries #1}{}}
\chapterimage{back.jpg}

\glsaddall
\printglossary[title={\sffamily\bfseries Glossary},%
               toctitle={\textcolor{deeporange}{Glossary}}]

\backmatter
\chapter{Bibliography}

\phantomsection
\section*{Online}
\addcontentsline{toc}{section}{Online}
\printbibliography[heading=bibempty,type=online]

\phantomsection
\section*{Misc}
\addcontentsline{toc}{section}{Miscellaneous}
\printbibliography[heading=bibempty,type=misc]

% Index
\cleardoublepage
\phantomsection
\setlength{\columnsep}{0.75cm}
\addcontentsline{toc}{chapter}{\textcolor{deeporange}{Index}}
\printindex

\end{document}
