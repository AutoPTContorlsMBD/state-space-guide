\section{Drivetrain}

\subsection{Equations of motion}

This drivetrain consists of two DC brushed motors per side which are chained
together on their respective sides and drive wheels which are assumed to be
massless.

\begin{bookfigure}
  \begin{tikzpicture}[auto, >=latex', circuit ee IEC,
                      set resistor graphic=var resistor IEC graphic]
    % \draw [help lines] (-1,-3) grid (7,4);

    % Right wheel
    \begin{scope}[xshift=5.78cm,yshift=1.83cm]
      \draw[fill=black!50] (0.2,0.68) ellipse (0.13 and 0.67);
      \draw[fill=black!50, color=black!50] (0,0) rectangle (0.2,1.35);
      \draw[fill=white] (0,0.68) ellipse (0.13 and 0.67);
      \draw (0,1.35) -- (0.2,1.35);
      \draw (0,0) -- (0.2,0);
    \end{scope}

    % Right transmission shaft
    \begin{scope}[xshift=5.32cm,yshift=2.42cm]
      \draw[fill=black!50] (0,0) rectangle (0.46,0.1);
    \end{scope}

    % Chassis
    \begin{scope}[xshift=4.44cm,yshift=2.09cm]
      \fill[fill=white] (0,0.8) -- (0,0.2) -- (0.2,0) -- (0.2,0.2)
        -- (0.98,0.2) -- (0.78,0.8) -- cycle;
      \draw (0,0.8) -- (0.78,0.8);
      \draw (0,0.8) -- (0,0.2);
      \draw (0,0.2) -- (0.2,0);
      \draw (0,0.8) -- (0.2,0.6);
      \draw (0.78,0.8) -- (0.98,0.6);
      \draw[fill=white] (0.2,0.6) rectangle (0.98,0);
    \end{scope}

    % Left transmission shaft
    \begin{scope}[xshift=4.09cm,yshift=2.42cm]
      \draw[fill=black!50] (0,0) rectangle (0.46,0.1);
    \end{scope}

    % Left wheel
    \begin{scope}[xshift=3.75cm,yshift=1.83cm]
      \draw[fill=black!50] (0.2,0.68) ellipse (0.13 and 0.67);
      \draw[fill=black!50, color=black!50] (0,0) rectangle (0.2,1.35);
      \draw[fill=white] (0,0.68) ellipse (0.13 and 0.67);
      \draw (0,1.35) -- (0.2,1.35);
      \draw (0,0) -- (0.2,0);
    \end{scope}

    % Angular velocity arrow for left wheel
    \draw[line width=0.7pt,->] (4.24,1.97) arc (-30:30:1) node[above]
      {$\omega_l$};

    % Angular velocity arrow for right wheel
    \draw[line width=0.7pt,->] (5.44,1.97) arc (-30:30:1) node[above]
      {$\omega_r$};

    % Wheel radius arrow
    \begin{scope}[xshift=3.5cm,yshift=1.83cm]
      \draw[line width=0.7pt,<->] (0,0) -- node[left] {$r$} (0,0.67);
    \end{scope}

    % Robot radius arrow
    \begin{scope}[xshift=4.65cm,yshift=1.83cm]
      \draw[line width=0.7pt,<->] (0,0) -- node[below] {$r_b$} (0.39,0);
    \end{scope}

    % Descriptions inside graphic
    \draw (4.99,2.42) node {$J$};
  \end{tikzpicture}

  \caption{Drivetrain system diagram}
  \label{fig:drivetrain}
\end{bookfigure}

From equation (\ref{eq:tau_f}) of the flywheel model derivations

\begin{equation}
  \tau = \frac{GK_t}{R} V - \frac{G^2K_t}{K_v R} \omega
    \label{eq:drivetrain_tau}
\end{equation}

where $\tau$ is the torque applied by one wheel of the drivetrain, $G$ is the
gear ratio of the drivetrain, $K_t$ is the torque constant of the motor, $R$ is
the resistance of the motor, and $K_v$ is the angular velocity constant. Since
$\tau = rF$ and $\omega = \frac{v}{r}$ where $v$ is the velocity of a given
drivetrain side along the ground and $r$ is the drivetrain wheel radius

\begin{align*}
  (rF) = \frac{GK_t}{R} V - \frac{G^2K_t}{K_v R} \left(\frac{v}{r}\right) \\
  rF = \frac{GK_t}{R} V - \frac{G^2K_t}{K_v Rr} v \\
  F = \frac{GK_t}{Rr} V - \frac{G^2K_t}{K_v Rr^2} v \\
  F = -\frac{G^2K_t}{K_v Rr^2} v + \frac{GK_t}{Rr} V
\end{align*}

Therefore, for each side of the robot

\begin{align*}
  F_l &= -\frac{G_l^2 K_t}{K_v R r^2} v_l + \frac{G_l K_t}{Rr} V_l \\
  F_r &= -\frac{G_r^2 K_t}{K_v R r^2} v_r + \frac{G_r K_t}{Rr} V_r
\end{align*}

where the $l$ and $r$ subscripts denote the side of the robot to which each
variable corresponds.

Let $C_1 = -\frac{G_l^2 K_t}{K_v R r^2}$, $C_2 = \frac{G_l K_t}{Rr}$,
$C_3 = -\frac{G_r^2 K_t}{K_v R r^2}$, and $C_4 = \frac{G_r K_t}{Rr}$.

\begin{align}
  F_l &= C_1 v_l + C_2 V_l \label{eq:drivetrain_Fl} \\
  F_r &= C_3 v_r + C_4 V_r \label{eq:drivetrain_Fr}
\end{align}

First, find the sum of forces.

\begin{align}
  \sum F &= ma \nonumber \\
  F_l + F_r &= m \dot{v} \nonumber \\
  F_l + F_r &= m \frac{\dot{v}_l + \dot{v}_r}{2} \nonumber \\
  \frac{2}{m} (F_l + F_r) &= \dot{v}_l + \dot{v}_r \nonumber \\
  \dot{v}_l &= \frac{2}{m} (F_l + F_r) - \dot{v}_r \label{eq:drivetrain_dotv_l}
\end{align}

Next, find the sum of torques.

\begin{align*}
  \sum \tau &= J \dot{\omega} \\
  \tau_l + \tau_r &= J \left(\frac{\dot{v}_r - \dot{v}_l}{2 r_b}\right)
\end{align*}

where $r_b$ is the radius of the drivetrain.

\begin{align*}
  (-r_b F_l) + (r_b F_r) &= J \frac{\dot{v}_r - \dot{v}_l}{2 r_b} \\
  -r_b F_l + r_b F_r &= \frac{J}{2 r_b} (\dot{v}_r - \dot{v}_l) \\
  -F_l + F_r &= \frac{J}{2 r_b^2} (\dot{v}_r - \dot{v}_l) \\
  \frac{2 r_b^2}{J} (-F_l + F_r) &= \dot{v}_r - \dot{v}_l \\
  \dot{v}_r &= \dot{v}_l + \frac{2 r_b^2}{J} (-F_l + F_r)
\end{align*}

Substitute in equation (\ref{eq:drivetrain_dotv_l}) to obtain an expression for
$\dot{v}_r$.

\begin{align}
  \dot{v}_r &= \left(\frac{2}{m} (F_l + F_r) - \dot{v}_r\right) +
    \frac{2 r_b^2}{J} (-F_l + F_r) \nonumber \\
  2\dot{v}_r &= \frac{2}{m} (F_l + F_r) + \frac{2 r_b^2}{J} (-F_l + F_r)
    \nonumber \\
  \dot{v}_r &= \frac{1}{m} (F_l + F_r) + \frac{r_b^2}{J} (-F_l + F_r)
    \label{eq:drivetrain_vr_2mid} \\
  \dot{v}_r &= \frac{1}{m} F_l + \frac{1}{m} F_r - \frac{r_b^2}{J} F_l +
    \frac{r_b^2}{J} F_r \nonumber \\
  \dot{v}_r &= \left(\frac{1}{m} - \frac{r_b^2}{J}\right) F_l +
    \left(\frac{1}{m} + \frac{r_b^2}{J}\right) F_r \label{eq:drivetrain_vr_2}
\end{align}

Substitute equation (\ref{eq:drivetrain_vr_2mid}) back into equation
(\ref{eq:drivetrain_dotv_l}) to obtain an expression for $\dot{v}_l$.

\begin{align}
  \dot{v}_l &= \frac{2}{m} (F_l + F_r) - \left(\frac{1}{m} (F_l + F_r) +
    \frac{r_b^2}{J} (-F_l + F_r)\right) \nonumber \\
  \dot{v}_l &= \frac{1}{m} (F_l + F_r) - \frac{r_b^2}{J} (-F_l + F_r)
    \nonumber \\
  \dot{v}_l &= \frac{1}{m} (F_l + F_r) + \frac{r_b^2}{J} (F_l - F_r) \nonumber
    \\
  \dot{v}_l &= \frac{1}{m} F_l + \frac{1}{m} F_r + \frac{r_b^2}{J} F_l -
    \frac{r_b^2}{J} F_r \nonumber \\
  \dot{v}_l &= \left(\frac{1}{m} + \frac{r_b^2}{J}\right) F_l +
    \left(\frac{1}{m} - \frac{r_b^2}{J}\right) F_r \label{eq:drivetrain_vl_2}
\end{align}

Now, plug the expressions for $F_l$ and $F_r$ into equation
(\ref{eq:drivetrain_vr_2}).

\begin{align}
  \dot{v}_r &= \left(\frac{1}{m} - \frac{r_b^2}{J}\right) F_l +
    \left(\frac{1}{m} + \frac{r_b^2}{J}\right) F_r \nonumber \\
  \dot{v}_r &= \left(\frac{1}{m} - \frac{r_b^2}{J}\right)
    \left(C_1 v_l + C_2 V_l\right) +
    \left(\frac{1}{m} + \frac{r_b^2}{J}\right) \left(C_3 v_r + C_4 V_r\right)
    \label{eq:drivetrain_model_left}
\end{align}

Now, plug the expressions for $F_l$ and $F_r$ into equation
(\ref{eq:drivetrain_vl_2}).

\begin{align}
  \dot{v}_l &= \left(\frac{1}{m} + \frac{r_b^2}{J}\right) F_l +
    \left(\frac{1}{m} - \frac{r_b^2}{J}\right) F_r \nonumber \\
  \dot{v}_l &= \left(\frac{1}{m} + \frac{r_b^2}{J}\right)
    \left(C_1 v_l + C_2 V_l\right) +
    \left(\frac{1}{m} - \frac{r_b^2}{J}\right) \left(C_3 v_r + C_4 V_r\right)
    \label{eq:drivetrain_model_right}
\end{align}

\subsection{Continuous state-space model}
\index{FRC models!drivetrain equations}

The position and velocity of each drivetrain side can be written as

\begin{align}
  \dot{x}_l &= v_l \label{eq:drivetrain_cont_ss_posl} \\
  \dot{v}_l &= \dot{v}_l \label{eq:drivetrain_cont_ss_vell} \\
  \dot{x}_r &= v_r \label{eq:drivetrain_cont_ss_posr} \\
  \dot{v}_r &= \dot{v}_r \label{eq:drivetrain_cont_ss_velr}
\end{align}

By equations (\ref{eq:drivetrain_model_left}) and
(\ref{eq:drivetrain_model_right})

\begin{align*}
  \dot{v}_r &= \left(\frac{1}{m} - \frac{r_b^2}{J}\right)
    \left(C_1 v_l + C_2 V_l\right) +
    \left(\frac{1}{m} + \frac{r_b^2}{J}\right) \left(C_3 v_r + C_4 V_r\right)
    \\
  \dot{v}_l &= \left(\frac{1}{m} + \frac{r_b^2}{J}\right)
    \left(C_1 v_l + C_2 V_l\right) +
    \left(\frac{1}{m} - \frac{r_b^2}{J}\right) \left(C_3 v_r + C_4 V_r\right)
\end{align*}

\begin{align*}
  \dot{\mtx{x}} &= \mtx{A} \mtx{x} + \mtx{B} \mtx{u} \\
  \mtx{y} &= \mtx{C} \mtx{x} + \mtx{D} \mtx{u}
\end{align*}

\begin{align*}
  \mtx{x} &=
  \begin{bmatrix}
    x_l \\
    v_l \\
    x_r \\
    v_r
  \end{bmatrix} \\
  \mtx{y} &=
  \begin{bmatrix}
    x_l \\
    x_r
  \end{bmatrix} \\
  \mtx{u} &=
  \begin{bmatrix}
    V_l \\
    V_r
  \end{bmatrix}
\end{align*}

\begin{align}
  \mtx{A} &=
  \begin{bmatrix}
    0 & 1 & 0 & 0 \\
    0 & \left(\frac{1}{m} - \frac{r_b^2}{J}\right) C_1 & 0 & \left(\frac{1}{m} + \frac{r_b^2}{J}\right) C_3 \\
    0 & 0 & 0 & 1 \\
    0 & \left(\frac{1}{m} + \frac{r_b^2}{J}\right) C_1 & 0 & \left(\frac{1}{m} - \frac{r_b^2}{J}\right) C_3
  \end{bmatrix} \\
  \mtx{B} &=
  \begin{bmatrix}
    0 & 0 \\
    \left(\frac{1}{m} + \frac{r_b^2}{J}\right) C_2 & \left(\frac{1}{m} - \frac{r_b^2}{J}\right) C_4 \\
    0 & 0 \\
    \left(\frac{1}{m} - \frac{r_b^2}{J}\right) C_2 & \left(\frac{1}{m} + \frac{r_b^2}{J}\right) C_4
  \end{bmatrix} \\
  \mtx{C} &=
  \begin{bmatrix}
    1 & 0 & 0 & 0 \\
    0 & 0 & 1 & 0 \\
  \end{bmatrix} \\
  \mtx{D} &= \mtx{0}_{2 \times 2}
\end{align}

where $C_1 = -\frac{G_l^2 K_t}{K_v R r^2}$, $C_2 = \frac{G_l K_t}{Rr}$,
$C_3 = -\frac{G_r^2 K_t}{K_v R r^2}$, and $C_4 = \frac{G_r K_t}{Rr}$.

\subsection{Simulation}

Python Control will be used to discretize the model and simulate it. The script
at
\url{https://github.com/calcmogul/state-space-guide/blob/master/code/frccontrol/examples/drivetrain.py}
creates and tests a controller for it.

\begin{remark}
  Python Control currently doesn't support finding the transmission zeroes of
  MIMO systems with a different number of inputs than outputs, so
  \texttt{control.pzmap()} and \texttt{frccontrol.System.plot\_pzmaps()} fail
  with an error.
\end{remark}
