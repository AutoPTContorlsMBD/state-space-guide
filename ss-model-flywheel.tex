\subsection{Flywheel}

\subsubsection{Equations of motion}

This flywheel consists of a DC brushed motor attached to a spinning mass of
non-negligible moment of inertia.

\begin{figure}[H]
  \centering

  \begin{tikzpicture}[auto, >=latex', circuit ee IEC,
                      set resistor graphic=var resistor IEC graphic]
    % \draw [help lines] (-1,-3) grid (7,4);

    % Electrical equivalent circuit
    \draw (0,2) to [voltage source={direction info'={->}, info'=$V$}] (0,0);
    \draw (0,2) to [current direction={info=$I$}] (0,3);
    \draw (0,3) -- (0.5,3);
    \draw (0.5,3) to [resistor={info={$R$}}] (2,3);

    \draw (2,3) -- (2.5,3);
    \draw (2.5,3) to [voltage source={direction info'={->}, info'=$V_{emf}$}]
      (2.5,0);
    \draw (0,0) -- (2.5,0);

    % Motor
    \draw[fill=black] (2.4,1.05) rectangle (2.6,1.95);
    \draw[fill=white] (2.5,1.5) ellipse (0.3 and 0.3);

    % Transmission gear one
    \draw[fill=black!50] (3.95,1.5) ellipse (0.08 and 0.33);
    \draw[fill=black!50, color=black!50] (3.75,1.17) rectangle (3.95,1.83);
    \draw[fill=white] (3.75,1.5) ellipse (0.08 and 0.33);
    \draw (3.75,1.83) -- (3.95,1.83);
    \draw (3.75,1.17) -- (3.95,1.17);

    % Output shaft of motor
    \draw[fill=black!50] (2.8,1.45) rectangle (3.75,1.55);

    % Angular velocity arrow of drive -> transmission
    \draw[line width=0.7pt,<-] (3.2,1) arc (-30:30:1) node[above] {$\omega_m$};

    % Transmission gear two
    \draw[fill=black!50] (3.95,2.51) ellipse (0.13 and 0.67);
    \draw[fill=black!50, color=black!50] (3.75,1.83) rectangle (3.95,3.18);
    \draw[fill=white] (3.75,2.51) ellipse (0.13 and 0.67);
    \draw (3.75,3.18) -- (3.95,3.18);
    \draw (3.75,1.83) -- (3.95,1.83);

    % Flywheel
    \draw[fill=white] (5.65,2.49) ellipse (0.13 and 0.4);
    \draw[fill=white,color=white] (5.05,2.89) rectangle (5.65,2.09);
    \draw[fill=white] (5.05,2.49) ellipse (0.13 and 0.4);
    \draw (5.05,2.09) -- (5.65,2.09);
    \draw (5.05,2.89) -- (5.65,2.89);

    % Transmission shaft from gear two to pulley
    \draw[fill=black!50] (4.09,2.42) rectangle (5.05,2.52);

    % Angular velocity arrow between transmission and pulley
    \draw[line width=0.7pt,->] (4.54,1.97) arc (-30:30:1) node[above]
      {$\omega_f$};

    % Descriptions inside graphic
    \draw (3.85,0.9) node {$G$};
    \draw (5.45,2.49) node {$J$};

    % Descriptions of subsystems under graphic
    \draw[decorate,decoration={brace,amplitude=10pt}]
      (3,-0.28) -- (-0.5,-0.28) node[midway,yshift=-20pt] {circuit};
    \draw[decorate,decoration={brace,amplitude=10pt}]
      (6.05,-0.28) -- (3.25,-0.28) node[midway,yshift=-20pt] {mechanics};
  \end{tikzpicture}

  \caption{Flywheel system diagram}
  \label{fig:flywheel}
\end{figure}

We will start with the equation derived earlier for a DC brushed motor, equation
(\ref{eq:motor_tau_V}).

\begin{equation*}
  V = \frac{\tau_m}{K_t} R + \frac{\omega_m}{K_v}
\end{equation*}

Solve for the angular acceleration. First, we'll rearrange the terms because
from inspection, $V$ is the model input, $\omega_m$ is the state, and $\tau_m$
contains the angular acceleration.

\begin{equation*}
  V = \frac{R}{K_t} \tau_m + \frac{1}{K_v} \omega_m
\end{equation*}

Since $\tau_m = J \dot{\omega}_m$ where $J$ is the moment of inertia and
$\dot{\omega}_m$ is the angular acceleration

\begin{align}
  V &= \frac{R}{K_t} (J \dot{\omega}_m) + \frac{1}{K_v} \omega_m \nonumber \\
  V &= \frac{RJ}{K_t} \dot{\omega}_m + \frac{1}{K_v} \omega_m \nonumber \\
  \frac{RJ}{K_t} \dot{\omega}_m &= -\frac{1}{K_v} \omega_m + V \nonumber \\
  \dot{\omega}_m &= -\frac{K_t}{K_v RJ} \omega_m + \frac{K_t}{RJ} V
    \label{eq:dot_omega_m}
\end{align}

The angular velocity of the motor armature $\omega_m$ is

\begin{equation}
  \omega_m = G \omega_f \label{eq:omega_m_omega_f}
\end{equation}

where $G$ is the gear ratio between the motor and the flywheel and $\omega_f$ is
the angular velocity of the flywheel. \\

Substitute equation (\ref{eq:omega_m_omega_f}) into equation
(\ref{eq:dot_omega_m}).

\begin{align}
  (G\dot{\omega}_f) &= -\frac{K_t}{K_v RJ} (G\omega_f) + \frac{K_t}{RJ} V
    \nonumber \\
  G\dot{\omega}_f &= -\frac{K_t G}{K_v RJ} \omega_f + \frac{K_t}{RJ} V
    \nonumber \\
  \dot{\omega}_f &= -\frac{K_t}{K_v RJ} \omega_f + \frac{K_t}{GRJ} V
    \label{eq:dot_omega_f}
\end{align}

\subsubsection{Continuous state-space model}

By equation (\ref{eq:dot_omega_f})

\begin{equation*}
  \dot{\omega}_f = -\frac{K_t}{K_v RJ} \omega_f + \frac{K_t}{GRJ} V
\end{equation*}

\begin{align*}
  \dot{\mtx{x}} &= \mtx{A} \mtx{x} + \mtx{B} \mtx{u} \\
  \dot{\mtx{y}} &= \mtx{C} \mtx{x} + \mtx{D} \mtx{u}
\end{align*}

\begin{align*}
  \mtx{x} &= \left[
  \begin{array}{c}
    \omega_f
  \end{array}
  \right] \\
  \mtx{y} &= \omega_f \\
  \mtx{u} &= V
\end{align*}

\begin{align}
  \dot{\mtx{x}} &= \left[
  \begin{array}{c}
    -\frac{K_t}{K_v RJ}
  \end{array}
  \right] \left[
  \begin{array}{c}
    \omega_f \\
  \end{array}
  \right] + \left[
  \begin{array}{c}
    \frac{K_t}{GRJ}
  \end{array}
  \right] V \\
  \mtx{y} &= \left[
  \begin{array}{c}
    1
  \end{array}
  \right] \left[
  \begin{array}{c}
    \omega_f \\
  \end{array}
  \right] + 0 \cdot V
\end{align}

\subsubsection{Discrete state-space model}

The angular velocity of the flywheel can be written as

\begin{align}
  \omega_{f,k+1} &= \omega_{f,k} + \dot{\omega}_{f,k} \Delta t
    \label{eq:flywheel_disc_ss_vel} \\
\end{align}

where by equation (\ref{eq:dot_omega_f})

\begin{equation*}
  \dot{\omega}_{f,k} = -\frac{K_t}{K_v RJ} \omega_{f,k} + \frac{K_t}{GRJ} V_k
\end{equation*}

Substitute this into equation (\ref{eq:flywheel_disc_ss_vel}).

\begin{align}
  \omega_{f,k+1} &= \omega_{f,k} + \left(-\frac{K_t}{K_v RJ} \omega_{f,k} +
    \frac{K_t}{GRJ} V_k\right) \Delta t \nonumber \\
  \omega_{f,k+1} &= \omega_{f,k} - \frac{K_t}{K_v RJ} \omega_{f,k} \Delta t +
    \frac{K_t}{GRJ} \Delta t V_k \nonumber \\
  \omega_{f,k+1} &= \left(1 - \frac{K_t}{K_v RJ} \Delta t\right) \omega_{f,k} +
    \frac{K_t}{GRJ} \Delta t V_k \nonumber \\
\end{align}

\begin{align*}
  \mtx{x}_{k+1} &= \mtx{A} \mtx{x}_k + \mtx{B} \mtx{u}_k \\
  \mtx{y}_{k+1} &= \mtx{C} \mtx{x}_k + \mtx{D} \mtx{u}_k
\end{align*}

\begin{align*}
  \mtx{x}_k &= \left[
  \begin{array}{c}
    \omega_{f,k}
  \end{array}
  \right] \\
  \mtx{y}_k &= \omega_{f,k} \\
  \mtx{u}_k &= V_k
\end{align*}

\begin{align}
  \mtx{x}_{k+1} &= \left[
  \begin{array}{c}
    1 - \frac{K_t}{K_v RJ} \Delta t
  \end{array}
  \right] \left[
  \begin{array}{c}
    \omega_{f,k} \\
  \end{array}
  \right] + \left[
  \begin{array}{c}
    \frac{K_t}{GRJ} \Delta t
  \end{array}
  \right] V_k \\
  \mtx{y}_k &= \left[
  \begin{array}{c}
    1
  \end{array}
  \right] \left[
  \begin{array}{c}
    \omega_{f,k} \\
  \end{array}
  \right] + 0 \cdot V_k
\end{align}
