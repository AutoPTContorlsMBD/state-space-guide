\subsection{DC brushed motor}

\subsubsection{Equations of motion}

The circuit for a DC brushed motor is shown below.

\begin{figure}[H]
  \centering

  \begin{tikzpicture}[auto, >=latex', circuit ee IEC,
                      set resistor graphic=var resistor IEC graphic]
    \node [opencircuit] (start) at (0,0) {};
    \node [] (V+) at (-0.5,0) { $+$ };
    \node [opencircuit] (end) at (0,-3.5) {};
    \node [] (V-) at (-0.5,-3.5) { $-$ };
    \node [coordinate] (topright) at (2.5,0) {};
    \node [coordinate] (bottomright) at (2.5,-3.5) {};
    \node [] at (0, -1.75) { $V$ };
    \draw (start) to (topright)
                  to [resistor={near start, info'={ $R$ }},
                      voltage source={near end, direction info'={<-},
                      info={ $V_{emf}=\frac{\omega}{K_v}$ }}] (bottomright)
                  to (end);
  \end{tikzpicture}

  \caption{DC brushed motor circuit}
  \label{fig:dc_motor_circuit}
\end{figure}

where $V$ is the voltage applied to the motor, $I$ is the current through the
motor in Amps, $R$ is the resistance across the motor in Ohms, $\omega_m$ is the
angular velocity of the motor in radians per second, and $K_v$ is a
proportionality constant. This circuit reflects the following relation.

\begin{equation}
  V = IR + \frac{\omega_m}{K_v} \label{eq:motor_V}
\end{equation}

The mechanical relation for a DC brushed motor is

\begin{equation}
  \tau_m = K_t I \label{eq:motor_tau_m}
\end{equation}

where $\tau_m$ is the torque produced by the motor in Newton-meters and $K_t$ is
the torque constant in Newton-meters per Amp. Therefore

\begin{equation*}
  I = \frac{\tau_m}{K_t}
\end{equation*}

Substitute this into equation (\ref{eq:motor_V}).

\begin{equation}
  V = \frac{\tau_m}{K_t} R + \frac{\omega_m}{K_v} \label{eq:motor_tau_V}
\end{equation}

\subsubsection{Calculating constants}

A typical motor's datasheet should include graphs of the motor's measured torque
and current for different angular velocities for a given voltage applied to the
motor. To find $K_t$

\begin{align}
  \tau_m &= K_t I \nonumber \\
  K_t &= \frac{\tau_m}{I} \nonumber \\
  K_t &= \frac{\tau_{m,stall}}{I_{stall}}
\end{align}

where $\tau_{m,stall}$ is the stall torque and $I_{stall}$ is the stall current
of the motor from its datasheet. \\

To find $R$, recall equation (\ref{eq:motor_V}).

\begin{equation*}
  V = IR + \frac{\omega_m}{K_v}
\end{equation*}

When the motor is stalled, $\omega_m = 0$.

\begin{align}
  V &= I_{stall} R \nonumber \\
  R &= \frac{V}{I_{stall}}
\end{align}

where $I_{stall}$ is the stall current of the motor and $V$ is the voltage
applied to the motor at stall. \\

To find $K_v$, again recall equation (\ref{eq:motor_V}).

\begin{equation*}
  V = IR + \frac{\omega_m}{K_v}
\end{equation*}

When the motor is spinning under no load, no current is drawn so $I = 0$.

\begin{align}
  V &= \frac{\omega_m}{K_v} \nonumber \\
  K_v &= \frac{V}{\omega_{m,free}}
\end{align}

where $\omega_{m,free}$ is the angular velocity of the motor under no load (also
known as the free speed) and $V$ is the voltage applied to the motor when
it's spinning at $\omega_{m,free}$.
