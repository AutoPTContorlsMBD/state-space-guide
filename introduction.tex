\chapter{Introduction}

\section{What is control theory?}

How can we prove an autonomous car will behave safely and meet certain
performance specifications in the presence of uncertainty? Control theory is a
pragmatic application of algebra and geometry that is used to analyze and
predict the behavior of \glspl{system} such as these, make them respond how we
want them to, and make them \glslink{robustness}{robust} to \glspl{disturbance}
and uncertainty. \\

But what sets control theory apart from, say, applied math? While control theory
does have some beautiful math behind it, controls engineering is an engineering
discipline like any other filled with trade-offs. The solutions control theory
gives should always be sanity checked and informed by our performance
specifications. We don't need to be perfect, just good enough to meet our
specifications.

\section{Nomenclature}

Most resources for advanced engineering topics assume a level of knowledge well
above that which is necessary. See the glossary for a list of words and phrases
commonly used in control theory, their origins, and their meaning. Below is a
table describing how the terms \textit{input} and \textit{output} apply to
\glspl{plant} vs \glspl{controller} and what letters are commonly associated
with each when working with them. Namely, that the terms input and output are
defined with respect to the \gls{plant}, not the \gls{controller}.

\begin{table}
  \renewcommand{\arraystretch}{1.3}
  \centering
  \begin{tabular}{|l|ll|}
    \hline
    \rowcolor{headingbg}
    & \textbf{Plant} & \textbf{Controller} \\
    \hline
    Input & $u(t)$ & $r(t)$, $y(t)$ \\
    Output & $y(t)$ & $u(t)$ \\
    \hline
  \end{tabular}
  \caption{Plant versus controller nomenclature}
  \label{tab:plant_v_controller}
\end{table}
