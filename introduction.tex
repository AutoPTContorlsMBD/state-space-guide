\chapterimage{introduction.jpg}{Road near walking trail off of Rice Ranch Road in Santa Maria, CA}

\chapter{Introduction}

\section{What is control theory?}

How can we prove an autonomous car will behave safely and meet certain
performance specifications in the presence of uncertainty? Control theory is a
pragmatic application of algebra and geometry that is used to analyze and
predict the behavior of \glspl{system} such as these, make them respond how we
want them to, and make them \glslink{robustness}{robust} to \glspl{disturbance}
and uncertainty.

But what sets control theory apart from, say, applied math? While control theory
does have some beautiful math behind it, controls engineering is an engineering
discipline like any other that is filled with trade-offs. The solutions control
theory gives should always be sanity checked and informed by our performance
specifications. We don't need to be perfect; we just need to be good enough to
meet our specifications.

\section{Nomenclature}

Most resources for advanced engineering topics assume a level of knowledge well
above that which is necessary. Part of the problem is the use of jargon. While
it efficiently communicates ideas to those within the field, new people who
aren't familiar with it are lost. See the glossary for a list of words and
phrases commonly used in control theory, their origins, and their meaning. Links
to the glossary are provided for certain words throughout the book and will use
\textcolor{glscolor}{this color}.

Table \ref{tab:plant_v_controller} describes how the terms \gls{input} and
\gls{output} apply to \glspl{plant} versus \glspl{controller} and what letters
are commonly associated with each when working with them. Namely, that the terms
\gls{input} and \gls{output} are defined with respect to the \gls{plant}, not
the \gls{controller}.

\begin{booktable}
  \begin{tabular}{|l|ll|}
    \hline
    \rowcolor{headingbg}
    & \textbf{Plant} & \textbf{Controller} \\
    \hline
    Inputs & $u(t)$ & $r(t)$, $y(t)$ \\
    Outputs & $y(t)$ & $u(t)$ \\
    \hline
  \end{tabular}
  \caption{Plant versus controller nomenclature}
  \label{tab:plant_v_controller}
\end{booktable}

\section{Essence of calculus}

This book uses derivatives and integrals occasionally to represent small changes
in values and the infintessimal sum of values over time respectively. If you
aren't already familiar with these concepts, 3Blue1Brown does a fantastic job of
introducing them in his \textit{Essence of calculus} video series
\cite{bib:essence_of_calculus}. We recommend watching videos 1 through 3 and 7
through 11 from that playlist for a solid foundation. The Taylor series
(presented in video 11) will be used in chapter \ref{ch:digital_control}.
