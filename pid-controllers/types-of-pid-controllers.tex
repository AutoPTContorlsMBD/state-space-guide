\section{Types of PID controllers}

PID controller inputs that are different orders of derivatives, such as position
and velocity, affect the \gls{system response} differently. Below is the
standard form for a position PID controller.

\begin{definition}[Position PID controller]
  \begin{equation}
    u(t) = K_p e(t) + K_i \int_0^t e(\tau) \,d\tau + K_d \frac{de}{dt}
    \label{eq:pos_pid}
  \end{equation}

  where $e(t)$ is the position error at the current time $t$, $\tau$ is the
  integration variable, $K_p$ is the proportional gain, $K_i$ is the integral
  gain, and $K_d$ is the derivative gain.
\end{definition}

The integral integrates from time $0$ to the current time $t$. We use $\tau$ for
the integration because we need a variable to take on multiple values throughout
the integral, but we can't use $t$ because we already defined that as the
current time.

If the position controller formulation is measuring and controlling velocity
instead, the error $e(t)$ becomes $\frac{de}{dt}$. Substituting this into
equation (\ref{eq:pos_pid}) yields

\begin{align}
  u(t) &= K_p \frac{de}{dt} + K_i \int_0^t \frac{de}{d\tau} \,d\tau +
    K_d \frac{d^2e}{dt^2} \nonumber \\
  u(t) &= K_p \frac{de}{dt} + K_i e(t) + K_d \frac{d^2e}{dt^2}
    \label{eq:u_de_dt}
\end{align}

So the velocity $\frac{de}{dt}$ is integrated by the integral term. By the
fundamental theorem of calculus, the derivative and integral cancel because they
are inverse operations. This produces just the error $e(t)$, so the $K_i$ term
has the effect of proportional control. Furthermore, the $K_p$ term in equation
(\ref{eq:u_de_dt}) behaves like a derivative term for velocity PID control due
to the $\frac{de}{dt}$. Letting $e_v = \frac{de}{dt}$, this means the velocity
controller is analogous to theorem \ref{thm:vel_pid}.

Note this isn't strictly true because the fundamental theorem of calculus
requires that $e(t)$ is continuous, but when we actually implement the
controller, $e(t)$ is discrete. Therefore, the result here is only correct up to
the accuracy of the iterative integration method used. We'll discuss
approximations like these in section \ref{sec:discretization_methods} and how
they affect controller behavior.

\begin{theorem}[Velocity PID controller]
  \label{thm:vel_pid}

  \begin{equation}
    u(t) = K_p e_v(t) + K_i \int_0^t e_v(\tau) \,d\tau
  \end{equation}

  where $e_v(t)$ is the velocity error at the current time $t$, $\tau$ is the
  integration variable, $K_p$ is now the derivative gain, and $K_i$ is now the
  proportional gain.
\end{theorem}

Integral control for the velocity is analogous to the throttle pedal on a car.
One must hold the throttle pedal (the control input) at a nonzero value to keep
the car traveling at the reference velocity.

Read \url{https://en.wikipedia.org/wiki/PID_controller} for more information on
PID control theory.
