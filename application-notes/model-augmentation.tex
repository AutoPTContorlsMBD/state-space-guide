\section{Model augmentation}

This section will teach various tricks for manipulating state-space models with
the goal of demystifying the matrix algebra at play. We will use the
augmentation techniques discussed here in the section on integral control.

Matrix augmentation is the process of appending rows or columns to a matrix. In
state-space, there are several common types of augmentation used: plant
augmentation, controller augmentation, and observer augmentation.

\subsection{Plant augmentation}

Plant augmentation is the process of adding a state to a model's state vector
and adding a corresponding row to the $\mtx{A}$ and $\mtx{B}$ matrices.

\subsection{Controller augmentation}

Controller augmentation is the process of adding a column to a controller's
$\mtx{K}$ matrix. This is often done in combination with plant augmentation to
add controller dynamics relating to a newly added state.

\subsection{Observer augmentation}

Observer augmentation is closely related to plant augmentation. In addition to
adding entries to the observer matrix $\mtx{L}$, the observer is using this
augmented plant for estimation purposes. This is better explained with an
example.

By augmenting the plant with a bias term with no dynamics (represented by zeroes
in its rows in $\mtx{A}$ and $\mtx{B}$, the observer will attempt to estimate a
value for this bias term that makes the model best reflect the measurements
taken of the real system. Note that we're not collecting any data on this bias
term directly; it's what's known as a hidden state. Rather than our inputs and
other states affecting it directly, the observer determines a value for it based
on what is most likely given the model and current measurements. We just tell
the plant what kind of dynamics the term has and the observer will estimate it
for us.

\subsection{Output augmentation}

Output augmentation is the process of adding rows to the $\mtx{C}$ matrix. This
is done to help the controls designer visualize the behavior of internal states
or other aspects of the system in MATLAB or Python Control. $\mtx{C}$ matrix
augmentation doesn't affect state feedback, so the designer has a lot of freedom
here. Noting that the output is defined as
$\mtx{y} = \mtx{C}\mtx{x} + \mtx{D}\mtx{u}$, The following row augmentations of
$\mtx{C}$ may prove useful. Of course, $\mtx{D}$ needs to be augmented with
zeroes as well in these cases to maintain the correct matrix dimensionality.

Since $\mtx{u} = -\mtx{K}\mtx{x}$, augmenting $\mtx{C}$ with $-\mtx{K}$ makes
the observer estimate the control input $\mtx{u}$ applied.

\begin{align*}
  \mtx{y} &= \mtx{C}\mtx{x} + \mtx{D}\mtx{u} \\
  \begin{bmatrix}
    \mtx{y} \\
    \mtx{u}
  \end{bmatrix} &=
  \begin{bmatrix}
    \mtx{C} \\
    -\mtx{K}
  \end{bmatrix}
  \mtx{x} +
  \begin{bmatrix}
    \mtx{D} \\
    \mtx{0}
  \end{bmatrix}
  \mtx{u}
\end{align*}

This works because $\mtx{K}$ has the same number of columns as states.

Various states can also be produced in the output with $\mtx{I}$ matrix
augmentation.
