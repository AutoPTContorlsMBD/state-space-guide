\section{Integral control}

A common way of implementing integral control is to add an additional state that
is the integral of the error of the variable intended to have zero steady-state
error.

There are two drawbacks to this method. First, there is integral windup on a
unit step input. That is, the integrator accumulates even if the system is
tracking the model correctly. The second is demonstrated by an example from
Jared Russell of FRC team 254. Say there is a position/velocity trajectory for
some plant to follow. Without integral control, one can calculate a desired
$\mtx{K}\mtx{x}$ to use as the reference input to the controller. As a result of
using both desired position and velocity, reference tracking is good. With
integral control added, the reference is always the desired position, but there
is no way to tell the controller the desired velocity.

Consider carefully whether integral control is necessary. One can get relatively
close without integral control, and integral adds all the issues listed above.
Below, it is assumed that the controls designer has determined that integral
control will be worth the inconvenience.

There are three methods FRC team 971 has used over the years:

\begin{enumerate}
  \item Augment the plant as described earlier. For an arm, one would add an
    ``integral of position" state.
  \item Add an integrator to the output of the controller, then estimate the
    control effort being applied. 971 has called this Delta U control. The
    up-side is that it doesn't have the windup issue described above; the
    integrator only acts if the system isn't behaving like the model, which was
    the original intent. The downside is working with it is very confusing.
  \item Estimate an ``error" in the observer and compensate for it. This
    quantity is the difference between what was applied and what was observed to
    happen. To use it, you simply add it to your control output and it will
    converge. This is 971's primary method.
\end{enumerate}

\subsection{Plant augmentation}

\subsection{Delta U control}

\subsection{U error estimation}

Let $u_{error}$ be the error in a system's input. The $u_{error}$ term is then
added to the system as follows.

\begin{align*}
  \dot{\mtx{x}} &= \mtx{A}\mtx{x} + \mtx{B}\left(\mtx{u} + u_{error}\right) \\
  \dot{\mtx{x}} &= \mtx{A}\mtx{x} + \mtx{B}\mtx{u} + \mtx{B}u_{error}
\end{align*}

For a multiple-output system, this would be

\begin{equation*}
  \dot{\mtx{x}} = \mtx{A}\mtx{x} + \mtx{B}\mtx{u} + \mtx{B}_{error}u_{error}
\end{equation*}

where $\mtx{B}_{error}$ is the column vector that maps $u_{error}$ to changes in
the rest of the state the same way $\mtx{B}$ does for $\mtx{u}$.
$\mtx{B}_{error}$ is only a column of $\mtx{B}$ if $u_{error}$ corresponds to an
existing input within $\mtx{u}$.

$\mtx{x}$ is augmented as

\begin{equation*}
  \mtx{x'} =
  \begin{bmatrix}
    \mtx{x} \\
    u_{error}
  \end{bmatrix}
\end{equation*}

and the augmented model is

\begin{align*}
  \dot{\mtx{x}}' &=
  \begin{bmatrix}
    \mtx{A} & \mtx{B}_{error} \\
    0 & 0
  \end{bmatrix}
  \mtx{x} +
  \begin{bmatrix}
    \mtx{B} \\
    0
  \end{bmatrix}
  \mtx{u}
\end{align*}

With this model, the observer will estimate both the state and the $u_{error}$
term. The controller is augmented similarly.

\begin{equation*}
  \mtx{u} = \mtx{K} \left(\mtx{r} - \mtx{x}\right) -
  \begin{bmatrix}
    u_{error} \\
    \mtx{0}
  \end{bmatrix}
\end{equation*}

This can be rewritten as

\begin{equation*}
  \mtx{u} =
  \begin{bmatrix}
    \mtx{K} & \mtx{k}_{error}
  \end{bmatrix}
  \left(\mtx{r}' - \mtx{x}'\right)
\end{equation*}

where $\mtx{k}_{error}$ is a column vector with a $1$ in a given row if
$u_{error}$ should be applied to that input or a $0$ otherwise. $\mtx{r}$ is
augmented with a zero for the goal $u_{error}$ term.

\begin{equation*}
  \mtx{r}' =
  \begin{bmatrix}
    \mtx{r} \\
    0
  \end{bmatrix}
\end{equation*}

This process can be repeated for an arbitrary error which can be corrected via
some linear combination of the inputs.
