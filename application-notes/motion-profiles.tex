\section{Motion profiles}

If smooth, predictable motion of a system over time is desired, it's best to
only change a system's reference as fast as the system is able to physically
move. Motion profiles, also known as trajectories, are used for this purpose.
For multi-state systems, each state is given its own trajectory. Since these
states are usually position and velocity, they share different derivatives of
the same profile.

For one degree of freedom ($1$ DOF) point-to-point movements in FRC, the most
commonly used profiles are the trapezoidal profile (figure
\ref{fig:trapezoidal_profile}) and the S-curve profile (figure
\ref{fig:s_curve_profile}). These profiles accelerate the system to a maximum
velocity from rest, then decelerate it later such that the final acceleration
velocity, are zero at the moment the system arrives at the desired location.

\begin{svg}{build/code/trapezoid_profile}
  \caption{Trapezoidal profile}
  \label{fig:trapezoidal_profile}
\end{svg}

\begin{svg}{build/code/s_curve_profile}
  \caption{S-curve profile}
  \label{fig:s_curve_profile}
\end{svg}

These profiles are given their names based on the shape of their velocity
trajectory. The trapezoidal profile has a velocity trajectory shaped like a
trapezoid and the S-curve profile has a velocity trajectory shaped like an
S-curve.

In the context of a point-to-point move, a full S-curve consists of seven
distinct phases of motion. Phase I starts moving the system from rest at a
linearly increasing acceleration until it reaches the maximum acceleration. In
phase II, the profile accelerates at this maximum acceleration rate until it
must start decreasing as it approaches the maximum velocity. This occurs in
phase III when the acceleration linearly decreases until it reaches zero. In
phase IV, the velocity is constant until deceleration begins, at which point the
profiles decelerates in a manner symmetric to phases I, II and III.

A trapezoidal profile, on the other hand, has three phases. It is a subset of an
S-curve profile, having only the phases corresponding to phase II of the S-curve
profile (constant acceleration), phase IV (constant velocity), and phase VI
(constant deceleration). This reduced number of phases underscores the
difference between these two profiles: the S-curve profile has extra motion
phases which transition between periods of acceleration, and periods of
non-acceleration; the trapezoidal profile has instantaneous transitions between
these phases. This can be seen in the acceleration graphs of the corresponding
velocity profiles for these two profile types.

\subsection{Jerk}

The motion characteristic that defines the change in acceleration, or
transitional period, is known as "jerk". Jerk is defined as the rate of change
of acceleration with time. In a trapezoidal profile, the jerk (change in
acceleration) is infinite at the phase transitions, while in the S-curve profile
the jerk is a constant value, spreading the change in acceleration over a period
of time.

From figures \ref{fig:trapezoidal_profile} and \ref{fig:s_curve_profile}, we can
see S-curve profiles are smoother than trapezoidal profiles. Why, however, do
the S-curve profile result in less load oscillation? For a given load, the
higher the jerk, the greater the amount of unwanted vibration energy will be
generated, and the broader the frequency spectrum of the vibration's energy will
be.

This means that more rapid changes in acceleration induce more powerful
vibrations, and more vibrational modes will be excited. Because vibrational
energy is absorbed in the system mechanics, it may cause an increase in settling
time or reduced accuracy if the vibration frequency matches resonances in the
mechanical system.

\subsection{Profile selection}

Since trapezoidal profiles spend their time at full acceleration or full
deceleration, they are, from the standpoint of profile execution, faster than
S-curve profiles. However, if this ``all on"/``all off" approach causes an
increase in settling time, the advantage is lost. Often, only a small amount of
``S" (transition between acceleration and no acceleration) can substantially
reduce induced vibration. Therefore to optimize throughput, the S-curve profile
must be tuned for each a given load and given desired transfer speed.

What S-curve form is right for a given system? On an application by application
basis, the specific choice of the form of the S-curve will depend on the
mechanical nature of the system and the desired performance specifications. For
example, in medical applications which involve liquid transfers that should not
be jostled, it would be appropriate to choose a profile with no phase II and VI
segment at all. Instead the acceleration transitions would be spread out as far
as possible, thereby maximizing smoothness.

In other applications involving high speed pick and place, overall transfer
speed is most important, so a good choice might be an S-curve with transition
phases (phases I, III, V, and VII) that are five to fifteen percent of phase II
and VI. In this case, the S-curve profile will add a small amount of time to the
overall transfer time. However, the reduced load oscillation at the end of the
move considerably decreases the total effective transfer time. Trial and error
using a motion measurement system is generally the best way to determine the
right amount of ``S" because modelling high frequency dynamics is difficult to
do accurately.

Another consideration is whether that ``S" segment will actually lead to
smoother control of the system. If the high frequency dynamics at play are
negligible, one can use the simpler trapezoidal profile.

\subsection{Profile equations}

The trapezoidal profile uses the following equations.

\begin{align*}
  x(t) &= x_0 + v_0t + \frac{1}{2}at^2 \\
  v(t) &= v_0 + at
\end{align*}

where $x(t)$ is the position at time $t$, $x_0$ is the initial position, $v_0$
is the initial velocity, and $a$ is the acceleration at time $t$. The S-curve
profile equations also include jerk.

\begin{align*}
  x(t) &= x_0 + v_0t + \frac{1}{2}at^2 + \frac{1}{6}jt^3 \\
  v(t) &= v_0 + at + \frac{1}{2}jt^2 \\
  a(t) &= a_0 + jt
\end{align*}

where $j$ is the jerk at time $t$, $a(t)$ is the acceleration at time $t$, and
$a_0$ is the initial acceleration.

More derivations are required to determine when to start and stop the different
profile phases. The derivations for a trapezoid profile are in appendix \ref{sec:deriv-trapezoid-profile} and the derivations for an S-curve profile are in
appendix \ref{sec:deriv-s-curve-profile}.

\subsection{Other profile types}

The ultimate goal of any profile is to match the motion system characteristics
to the desired application. Trapezoidal and S-curve profiles work well when the
motion system's torque response curve is fairly flat. In other words, when the
output torque does not vary that much over the range of velocities the system
will be experiencing. This is true for most servo motor systems, whether DC
Brush or Brushless DC.

Step motors, however, do not have flat torque/speed curves. Torque output is
nonlinear, sometimes has a large drop at a location called the ``mid-range
instability", and generally drops off at higher velocities.

Mid-range instability occurs at the step frequency when the motor's natural
resonance frequency matches the current step rate. To address mid-range
instability, the most common technique is to use a non-zero starting velocity.
This means that the profile instantly ``jumps" to a programmed velocity upon
initial acceleration, and while decelerating. While crude, this technique
sometimes provides better results than a smooth ramp for zero, particularly for
systems that do not use a microstepping drive technique.

To address torque drop-off at higher velocities, a parabolic profile can be
used. The corresponding acceleration curve has the smallest acceleration when
the velocity is highest. This is a good match for step-motor systems because
there is less torque available at higher speeds. However, notice that starting
and ending accelerations are very high, and there is no ``S" phase where the
acceleration smoothly transitions to zero. If load oscillation is a problem,
parabolic profiles may not work as well as an S-curve despite the fact that a
standard S-curve profile is not optimized for a step motor from the standpoint
of the torque/speed curve.

\subsection{Further reading}

FRC teams 254 and 971 gave a talk at FIRST World Championships in 2015 about
motion profiles \cite{bib:motion-profiles}.
