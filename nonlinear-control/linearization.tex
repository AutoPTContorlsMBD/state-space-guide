\section{Linearization}
\index{Nonlinear control!linearization}

One way to control nonlinear \glspl{system} is to
\glslink{linearization}{linearize} the \gls{model} around a reference point.
Then, all the powerful tools that exist for linear controls can be applied. This
is done by taking the partial derivative of the functions.

\begin{align*}
  \begin{array}{cccc}
    \mtx{A} = \frac{\partial f(\mtx{x}, \mtx{0}, \mtx{0})}{\partial \mtx{x}} &
    \mtx{B} = \frac{\partial f(\mtx{0}, \mtx{u}, \mtx{0})}{\partial \mtx{u}} &
    \mtx{C} = \frac{\partial h(\mtx{x}, \mtx{0}, \mtx{0})}{\partial \mtx{x}} &
    \mtx{D} = \frac{\partial h(\mtx{0}, \mtx{u}, \mtx{0})}{\partial \mtx{u}}
  \end{array}
\end{align*}

Higher order partial derivatives can be added to better approximate the
nonlinear dynamics. We typically only \glslink{linearization}{linearize} around
equilibrium points because we are interested in how the \gls{system} behaves
when perturbed from equilibrium. An FAQ on this goes into more detail
\cite{bib:linearize_equilibrium_point}. To be clear though,
\glslink{linearization}{linearizing} the \gls{system} around the current
\gls{state} as the \gls{system} evolves does give a closer approximation over
time.

Note that linearization with static matrices (that is, with a time-invariant
linear \gls{system}) only works if the original \gls{system} in question is
feedback linearizable.
