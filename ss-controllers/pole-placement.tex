\section{Pole placement}
\index{Controller design!pole placement}

This is the practice of placing the poles of a closed-loop system directly to
produce a desired response. This can be done manually for state feedback
controllers with controllable canonical form (see section \ref{sec:ctrl-canon}).
This can also be done manually for state observers with observable canonical
form (see section \ref{sec:obsv-canon}).

In general, pole placement should only be used if you know what you're doing.
It's much easier to let LQR place the poles for you, then use those as a
starting point for pole placement.
