\section{State-space observers and localization}

State-space observers are used to estimate \glspl{state} which cannot be
measured directly. This can be due to noisy measurements or the state not being
measurable (a hidden state). This information can be used for
\gls{localization}, which is the process of using external measurements to
determine an \gls{agent}'s pose\footnote{An agent is a system-agnostic term for
independent controlled actors like robots or aircraft.}, or orientation in the
world.

One type of state estimator is LQE. ``LQE" stands for ``Linear-Quadratic
Estimator". Similar to LQR, it places the estimator poles such that it minimizes
the sum of squares of the error. The Luenberger observer and Kalman filter are
examples of these.

Computer vision can also be used for \gls{localization}. By extracting features
from an image taken by the \gls{agent}'s camera, like a retroreflective target
in FRC, and comparing them to known dimensions, one can determine where the
\gls{agent}'s camera would have to be to see that image. This can be used to
correct your state estimate in the same way we do with an encoder or gyroscope.

\subsection{Luenberger observer}
\index{State-space observers!Luenberger observer}

\begin{theorem}[Luenberger observer]
  \begin{align}
    \dot{\hat{\mtx{x}}} &= \mtx{A}\hat{\mtx{x}} + \mtx{B}\mtx{u} +
      \mtx{L} (\mtx{y} - \hat{\mtx{y}}) \label{eq:s_obsv_x} \\
    \hat{\mtx{y}} &= \mtx{C}\hat{\mtx{x}} + \mtx{D}\mtx{u} \label{eq:s_obsv_y}
  \end{align}

  \begin{align}
    \hat{\mtx{x}}_{k+1} &= \mtx{A}\hat{\mtx{x}}_k + \mtx{B}\mtx{u}_k +
      \mtx{L}(\mtx{y}_k - \hat{\mtx{y}}_k) \label{eq:z_obsv_x} \\
    \hat{\mtx{y}}_k &= \mtx{C}\hat{\mtx{x}}_k + \mtx{D}\mtx{u}_k
      \label{eq:z_obsv_y} \\ \nonumber
  \end{align}

  \begin{figurekey}
    \begin{tabulary}{\linewidth}{LLLL}
      $\mtx{A}$ & system matrix      & $\hat{\mtx{x}}$ & state estimate vector \\
      $\mtx{B}$ & input matrix       & $\mtx{u}$ & input vector \\
      $\mtx{C}$ & output matrix      & $\mtx{y}$ & output vector \\
      $\mtx{D}$ & feedthrough matrix & $\hat{\mtx{y}}$ & output estimate vector \\
      $\mtx{L}$ & estimator gain matrix & & \\
    \end{tabulary}
  \end{figurekey}
\end{theorem}

\begin{booktable}
  \begin{tabular}{|ll|ll|}
    \hline
    \rowcolor{headingbg}
    \textbf{Matrix} & \textbf{Rows $\times$ Columns} &
    \textbf{Matrix} & \textbf{Rows $\times$ Columns} \\
    \hline
    $\mtx{A}$ & states $\times$ states & $\hat{\mtx{x}}$ & states $\times$ 1 \\
    $\mtx{B}$ & states $\times$ inputs & $\mtx{u}$ & inputs $\times$ 1 \\
    $\mtx{C}$ & outputs $\times$ states & $\mtx{y}$ & outputs $\times$ 1 \\
    $\mtx{D}$ & outputs $\times$ inputs & $\hat{\mtx{y}}$ & outputs $\times$ 1 \\
    $\mtx{L}$ & states $\times$ outputs & & \\
    \hline
  \end{tabular}
  \caption{Luenberger observer matrix dimensions}
  \label{tab:luenberger_matrix_dims}
\end{booktable}

Variables denoted with a hat are estimates of the corresponding variable. For
example, $\hat{\mtx{x}}$ is the estimate of the true state $\mtx{x}$.

Notice that a Luenberger observer has an extra term in the state evolution
equation. This term uses the difference between the estimated outputs and
measured outputs to steer the estimated state toward the true state. Large
values of $\mtx{L}$ trust the measurements more while small values trust the
model more.

\begin{remark}
  Using an estimator forfeits the performance guarantees from earlier, but the
  responses are still generally very good if the process and measurement noises
  are small enough. See John Doyle's paper \textit{Guaranteed Margins for LQG
  Regulators} for a proof.
\end{remark}

A Luenberger observer combines the prediction and update steps of an estimator.
To run them separately, use the equations in theorem \ref{thm:luenberger}
instead.

\begin{theorem}[Luenberger observer with separate predict/update]
  \label{thm:luenberger}

  \begin{align}
    \text{Predict step} \nonumber \\
    \hat{\mtx{x}}_{k+1}^- &= \mtx{A}\hat{\mtx{x}}_k^- + \mtx{B}\mtx{u}_k \\
    \text{Update step} \nonumber \\
    \hat{\mtx{x}}_{k+1}^+ &= \hat{\mtx{x}}_{k+1}^- + \mtx{A}^{-1}\mtx{L}
      (\mtx{y}_k - \hat{\mtx{y}}_k) \\
    \hat{\mtx{y}}_k &= \mtx{C} \hat{\mtx{x}}_k^-
  \end{align}
\end{theorem}

See appendix \ref{subsec:deriv-luenberger-separate} for a derivation.

\subsubsection{Eigenvalues of closed-loop observer}
\index{Stability!eigenvalues}

The eigenvalues of the system matrix can be used to determine whether a
\gls{state} observer's estimate will converge to the true \gls{state}.

Plugging equation (\ref{eq:z_obsv_y}) into equation (\ref{eq:z_obsv_x}) gives

\begin{align*}
  \hat{\mtx{x}}_{k+1} &= \mtx{A}\hat{\mtx{x}}_k + \mtx{B}\mtx{u}_k +
    \mtx{L} (\mtx{y}_k - \hat{\mtx{y}}_k) \\
  \hat{\mtx{x}}_{k+1} &= \mtx{A}\hat{\mtx{x}}_k + \mtx{B}\mtx{u}_k +
    \mtx{L} (\mtx{y}_k - (\mtx{C}\hat{\mtx{x}}_k + \mtx{D}\mtx{u}_k)) \\
  \hat{\mtx{x}}_{k+1} &= \mtx{A}\hat{\mtx{x}}_k + \mtx{B}\mtx{u}_k +
    \mtx{L} (\mtx{y}_k - \mtx{C}\hat{\mtx{x}}_k - \mtx{D}\mtx{u}_k)
\end{align*}

Plugging in equation (\ref{eq:ssz_ctrl_y}) gives

\begin{align*}
  \hat{\mtx{x}}_{k+1} &= \mtx{A}\hat{\mtx{x}}_k + \mtx{B}\mtx{u}_k +
    \mtx{L}((\mtx{C}\mtx{x}_k + \mtx{D}\mtx{u}_k) - \mtx{C}\hat{\mtx{x}}_k -
    \mtx{D}\mtx{u}_k) \\
  \hat{\mtx{x}}_{k+1} &= \mtx{A}\hat{\mtx{x}}_k + \mtx{B}\mtx{u}_k +
    \mtx{L}(\mtx{C}\mtx{x}_k + \mtx{D}\mtx{u}_k - \mtx{C}\hat{\mtx{x}}_k -
    \mtx{D}\mtx{u}_k) \\
  \hat{\mtx{x}}_{k+1} &= \mtx{A}\hat{\mtx{x}}_k + \mtx{B}\mtx{u}_k +
    \mtx{L}(\mtx{C}\mtx{x}_k - \mtx{C}\hat{\mtx{x}}_k) \\
  \hat{\mtx{x}}_{k+1} &= \mtx{A}\hat{\mtx{x}}_k + \mtx{B}\mtx{u}_k +
    \mtx{L}\mtx{C}(\mtx{x}_k - \hat{\mtx{x}}_k)
\end{align*}

Let $E_k = \mtx{x}_k - \hat{\mtx{x}}_k$ be the error in the estimate
$\hat{\mtx{x}}_k$.

\begin{equation*}
  \hat{\mtx{x}}_{k+1} = \mtx{A}\hat{\mtx{x}}_k + \mtx{B}\mtx{u}_k +
    \mtx{L}\mtx{C}\mtx{E}_k
\end{equation*}

Subtracting this from equation (\ref{eq:ssz_ctrl_x}) gives

\begin{align}
  \mtx{x}_{k+1} - \hat{\mtx{x}}_{k+1} &= \mtx{A}\mtx{x}_k + \mtx{B}\mtx{u}_k -
    (\mtx{A}\hat{\mtx{x}}_k + \mtx{B}\mtx{u}_k +
     \mtx{L}\mtx{C}\mtx{E}_k) \nonumber \\
  \mtx{E}_{k+1} &= \mtx{A}\mtx{x}_k + \mtx{B}\mtx{u}_k -
    (\mtx{A}\hat{\mtx{x}}_k + \mtx{B}\mtx{u}_k + \mtx{L}\mtx{C}\mtx{E}_k)
    \nonumber \\
  \mtx{E}_{k+1} &= \mtx{A}\mtx{x}_k + \mtx{B}\mtx{u}_k -
    \mtx{A}\hat{\mtx{x}}_k - \mtx{B}\mtx{u}_k - \mtx{L}\mtx{C}\mtx{E}_k
    \nonumber \\
  \mtx{E}_{k+1} &= \mtx{A}\mtx{x}_k - \mtx{A}\hat{\mtx{x}}_k -
    \mtx{L}\mtx{C}\mtx{E}_k \nonumber \\
  \mtx{E}_{k+1} &= \mtx{A}(\mtx{x}_k - \hat{\mtx{x}}_k) -
    \mtx{L}\mtx{C}\mtx{E}_k \nonumber \\
  \mtx{E}_{k+1} &= \mtx{A}\mtx{E}_k - \mtx{L}\mtx{C}\mtx{E}_k \nonumber \\
  \mtx{E}_{k+1} &= (\mtx{A} - \mtx{L}\mtx{C})\mtx{E}_k \label{eq:obsv_eig_calc}
\end{align}

For equation (\ref{eq:obsv_eig_calc}) to have a bounded output, the eigenvalues
of $\mtx{A} - \mtx{L}\mtx{C}$ must be within the unit circle. These eigenvalues
represent how fast the estimator converges to the true state of the given
\gls{model}. A fast estimator converges quickly while a slow estimator avoids
amplifying noise in the measurements used to produce a state estimate.

As stated before, the controller and estimator are dual problems. Controller
gains can be found assuming perfect estimator (i.e., perfect knowledge of all
\glspl{state}). Estimator gains can be found assuming an accurate \gls{model}
and a controller with perfect \gls{tracking}.

The effect of noise can be seen if it is modeled
\glslink{stochastic process}{stochastically} as

\begin{equation*}
  \hat{\mtx{x}}_{k+1} = \mtx{A}\hat{\mtx{x}}_k + \mtx{B}\mtx{u}_k +
    \mtx{L} ((\mtx{y}_k + \mtx{\nu}_k) - \hat{\mtx{y}}_k) \\
\end{equation*}

where $\mtx{\nu}_k$ is the measurement noise. Rearranging this equation yields

\begin{align*}
  \hat{\mtx{x}}_{k+1} &= \mtx{A}\hat{\mtx{x}}_k + \mtx{B}\mtx{u}_k +
    \mtx{L} (\mtx{y}_k - \hat{\mtx{y}}_k + \mtx{\nu}_k) \\
  \hat{\mtx{x}}_{k+1} &= \mtx{A}\hat{\mtx{x}}_k + \mtx{B}\mtx{u}_k +
    \mtx{L} (\mtx{y}_k - \hat{\mtx{y}}_k) + \mtx{L}\mtx{\nu}_k
\end{align*}

As $\mtx{L}$ increases, the measurement noise is amplified.
