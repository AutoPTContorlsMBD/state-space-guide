\chapterimage{going-digital.jpg}{Chaparral by Merril Apartments at UCSC}

\chapter{Going digital}

The complex plane discussed so far deals with continuous \glspl{system}. In
decades past, \glspl{plant} and controllers were implemented using analog
electronics, which are continuous in nature. Nowadays, microprocessors can be
used to achieve cheaper, less complex controller designs.
\glslink{discretization}{Discretization} converts the continuous model we've
worked with so far from a set of differential equations like

\begin{equation}
  \dot{x} = x - 3 \label{eq:differential_equ_example}
\end{equation}

to a set of difference equations like

\begin{equation}
  x_{k+1} = x_k + (x_k - 3) \Delta T \label{eq:difference_equ_example}
\end{equation}

where the difference equation is run a some update rate denoted by $T$,
$\Delta T$, or sometimes sloppily as $dt$.

While higher order terms of a differential equation are derivatives of the state
variable (e.g., $\ddot{x}$ in relation to equation
(\ref{eq:differential_equ_example})), higher order terms of a difference
equation are delayed copies of the state variable (e.g., $x_{k-1}$ with respect
to $x_k$ in equation (\ref{eq:difference_equ_example})).

\section{Phase loss}

However, \gls{discretization} has drawbacks. Since a microcontroller performs
discrete steps, there is a sample delay that introduces phase loss in the
controller. Phase loss is the reduction of phase margin (see section
\ref{sec:gain-phase-margin}) that occurs in digital implementations of feedback
controllers from sampling the continuous system at discrete time intervals. As
the sample rate of the controller decreases, the phase margin decreases rapidly
and will lead to instability if the phase margin reaches zero. Large amounts of
phase loss can make a stable controller in the continuous domain become unstable
in the discrete domain. Here are a few ways to combat this.

\begin{itemize}
  \item Run the controller with a high sample rate.
  \item Designing the controller in the analog domain with enough phase margin
    to compensate for any phase loss that occurs as part of discretization.
  \item Convert the \gls{plant} to the digital domain and design the controller
    completely in the digital domain.
\end{itemize}

\section{s-plane to z-plane}

Transfer functions are converted to impulse responses using the Z-transform. The
s-plane's LHP maps to the inside of a unit circle in the z-plane. Table
\ref{tab:s-plane2z-plane} contains a few common points.

\begin{booktable}
  \begin{tabular}{|cc|}
    \hline
    \rowcolor{headingbg}
    \textbf{s-plane} & \textbf{z-plane} \\
    \hline
    $(0, 0)$ & $(0, 1)$ \\
    imaginary axis & edge of unit circle \\
    $(0, -\infty)$ & $(0, 0)$ \\
    \hline
  \end{tabular}
  \caption{Mapping from s-plane to z-plane}
  \label{tab:s-plane2z-plane}
\end{booktable}

You may notice that poles can be placed at $(0, 0)$ in the z-plane. This is
known as a deadbeat controller. An $\rm N^{th}$ order deadbeat controller decays
to the \gls{reference} in N timesteps. While this sounds great, there are other
considerations like actuation effort, \gls{robustness}, and
\gls{noise immunity}. These will be discussed in detail with LQR and LQE.

\section{Discretization methods}
\index{Discretization}

Discretization is done using a zero-order hold. That is, the system state is
only updated at discrete intervals and the value is held constant between
samples. The exact method of applying this uses the matrix exponential, but this
can be computationaly expensive. Instead, approximations such as the following
are used.

\begin{enumerate}
  \item Forward Euler method. This is defined as
    $y_{n+1} = y_n + f(t_n, y_n) \Delta t$.
    \index{Discretization!forward Euler method}
  \item Backward Euler method. This is defined as
    $y_{n+1} = y_n + f(t_{n+1}, y_{n+1}) \Delta t$.
    \index{Discretization!backward Euler method}
  \item Bilinear transform. The first-order bilinear approximation is
    $s = \frac{2}{T} \frac{1 - z^{-1}}{1 + z^{-1}}$.
    \index{Discretization!bilinear transform}
\end{enumerate}

where $T$ is the sample period for the discrete system. Since these are
approximations, there is distortion between the real discrete system's poles and
the approximate poles. For fast-changing systems, this distortion can quickly
lead to instability.

\begin{figure}
  \begin{minisvg}{build/code/zoh}
    \caption{Zero-order hold of a system response.}
  \end{minisvg}
  \hfill
  \begin{minisvg}{build/code/discretization_methods}
    \caption{Discretization methods of a system response.}
  \end{minisvg}
\end{figure}

\section{Zero-order hold for state-space}
\index{Discretization!zero-order hold}

Assume the following continuous-time state space model

\begin{align*}
  \dot{\mtx{x}} &= \mtx{A}_c\mtx{x} + \mtx{B}_c\mtx{u} + \mtx{w} \\
  \mtx{y} &= \mtx{C}_c\mtx{x} + \mtx{D}_c\mtx{u} + \mtx{v}
\end{align*}

where $\mtx{w}$ is the process noise, $\mtx{v}$ is the measurement noise, and
both are zero-mean white noise sources with covariances of $\mtx{Q}_c$ and
$\mtx{R}_c$ respectively

\begin{align*}
  \mtx{w} &\sim N(0, \mtx{Q}_c) \\
  \mtx{v} &\sim N(0, \mtx{R}_c)
\end{align*}

The model can be discretized as follows

\begin{align*}
  \mtx{x}_{k+1} &= \mtx{A}_d \mtx{x}_k + \mtx{B}_d \mtx{u}_k + \mtx{w}_k \\
   \mtx{y}_k &= \mtx{C}_d \mtx{x}_k + \mtx{D}_d \mtx{u}_k + \mtx{v}_k
\end{align*}

with covariances

\begin{align*}
  \mtx{w}_k &\sim N(0, \mtx{Q}_d) \\
  \mtx{v}_k &\sim N(0, \mtx{R}_d)
\end{align*}

\begin{theorem}[Zero-order hold for state-space]
  \begin{align}
    \mtx{A}_d &= e^{\mtx{A}_c T} \\
    \mtx{B}_d &= \int_0^T e^{\mtx{A}_c \tau} d\tau \mtx{B}_c =
      \mtx{A}_c^{-1} (\mtx{A}_d - \mtx{I}) \mtx{B}_c \\
    \mtx{C}_d &= \mtx{C}_c \\
    \mtx{D}_d &= \mtx{D}_c \\
    \mtx{Q}_d &= \int_{\tau = 0}^{T} e^{\mtx{A}_c\tau} \mtx{Q}_c
      e^{\mtx{A}_c^T\tau} d\tau \\
    \mtx{R}_d &= \frac{1}{T} \mtx{R}_c
  \end{align}

  where a subscript of $d$ denotes discrete, a subscript of $c$ denotes the
  continuous version of the corresponding matrix, $T$ is the sample period for
  the discrete system, and $e^{\mtx{A}_c T}$ is the matrix exponential of
  $\mtx{A}_c$.
\end{theorem}

$\mtx{Q}_d$ is computed as

\begin{equation*}
  e^{
  \begin{bmatrix}
    -\mtx{A}^T & \mtx{Q}_c \\
    \mtx{0} & \mtx{A}
  \end{bmatrix}T} =
  \begin{bmatrix}
    -\mtx{A}_d^T & \mtx{A}_d^{-1} \mtx{Q}_d \\
    \mtx{0} & \mtx{A}_d
  \end{bmatrix}
\end{equation*}

and $\mtx{Q}_d$ is the lower-right quadrant multiplied by the upper-right
quadrant \cite{bib:integral_matrix_exp}.

See the Wikipedia page on discretization for derivations
\cite{bib:discretization}.

To understand why the matrix exponential is used in the discretization process,
consider the set of differential equations $\dot{\mtx{x}} = \mtx{A}\mtx{x}$ we
use to describe systems (systems also have a $\mtx{B}\mtx{u}$ term, but we'll
ignore it for clarity). The solution to this type of differential equation uses
an exponential. Since we are using matrices and vectors here, we use the matrix
exponential.

\begin{equation*}
  \mtx{x}(t) = e^{\mtx{A}t} \mtx{x}_0
\end{equation*}

where $\mtx{x}_0$ contains the initial conditions. If the initial state is the
current system state, then we can describe the system's state over time as

\begin{equation*}
  \mtx{x}_{k+1} = e^{\mtx{A}T} \mtx{x}_k
\end{equation*}

where $T$ is the time between samples $\mtx{x}_k$ and $\mtx{x}_{k+1}$.

To compute $\mtx{A}_d$ and $\mtx{B}_d$ in one step, one can utilize the
following property.

\begin{equation*}
  e^{
  \begin{bmatrix}
    \mtx{A} & \mtx{B} \\
    \mtx{0} & \mtx{0}
  \end{bmatrix}T} =
  \begin{bmatrix}
    \mtx{A}_d & \mtx{B}_d \\
    \mtx{0} & \mtx{I}
  \end{bmatrix}
\end{equation*}

\section{Effects of discretization on controller performance}

Running a feedback controller at a faster update rate doesn't always mean better
control. In fact, you may be using more computational resources than you need.
However, here are some reasons for running at a faster update rate.

Firstly, if you have a discrete model of the system, that model can more
accurately approximate the underlying continuous system. Most applications of
PID control don't have a model though.

Secondly, the controller can better handle fast system dynamics. If the system
can move from its initial state to the desired one in under 250ms, you obviously
want to run the controller with a period less than this. When you reduce the
sample period, you're making the discrete controller more accurately reflect
what the equivalent continuous controller would do (controllers built from
analog circuit components like op-amps are continuous).

Running at a lower sample rate only causes problems if you don't take into
account the response time of your system. Some systems like heaters have outputs
(output = measurement) that change on the order of minutes. Running a control
loop at 1kHz doesn't make sense for this because the plant input the controller
computes won't change much, if at all, in 1ms.

Figure \ref{fig:sampling_simulation} shows simulations of the same controller
for different sampling methods and sample rates, which have varying levels of
fidelity to the real system.

\begin{svg}{build/code/sampling_simulation}
  \caption{Sampling methods for system simulation.}
  \label{fig:sampling_simulation}
\end{svg}

Forward Euler is numerically unstable for low sample rates. The bilinear
transform is a significant improvement due to it being a second-order
approximation, but zero-order hold performs best due to the matrix exponential
including much higher orders (it is, in effect, exact).

\section{Computing the matrix exponential}
\index{Discretization!matrix exponential}

The matrix exponential (and system discretization in general) is typically
solved with a computer. Python Control's \texttt{StateSpace.sample()} with the
``zoh" method (the default) does this.

\begin{definition}[Matrix exponential]
  Let $\mtx{X}$ be an $n \times n$ matrix. The exponential of $\mtx{X}$ denoted
  by $e^{\mtx{X}}$ is the $n \times n$ matrix given by the following power
  series.

  \begin{equation}
    e^{\mtx{X}} = \sum_{k=0}^\infty \frac{1}{k!} \mtx{X}^k
  \end{equation}

  where $\mtx{X}^0$ is defined to be the identity matrix $\mtx{I}$ with the same
  dimensions as $\mtx{X}$.
\end{definition}
