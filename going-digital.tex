\chapterimage{going-digital.jpg}

\chapter{Going digital}

The complex plane discussed so far deals with continuous \glspl{system}. In
decades past, \glspl{plant} and controllers were implemented using analog
electronics, which are continuous in nature. Nowadays, microprocessors can be
used to achieve cheaper, less complex controller designs.
\glslink{discretization}{Discretization} converts the continuous model we've
worked with so far from a set of differential equations like

\begin{equation}
  \dot{x} = x - 3 \label{eq:differential_equ_example}
\end{equation}

to a set of difference equations like

\begin{equation}
  x_{k+1} = x_k + (x_k - 3) \Delta T \label{eq:difference_equ_example}
\end{equation}

where the difference equation is run a some update rate denoted by $T$,
$\Delta T$, or sometimes sloppily as $dt$.

While higher order terms of a differential equation are derivatives of the state
variable (e.g., $\ddot{x}$ in relation to equation
(\ref{eq:differential_equ_example})), higher order terms of a difference
equation are delayed copies of the state variable (e.g., $x_{k-1}$ with respect
to $x_k$ in equation (\ref{eq:difference_equ_example})).

\section{Phase loss}

However, \gls{discretization} has drawbacks. Since a microcontroller performs
discrete steps, there is a sample delay that introduces phase loss in the
controller. Phase loss is the reduction of phase margin (see section
\ref{sec:gain-phase-margin}) that occurs in digital implementations of feedback
controllers from sampling the continuous system at discrete time intervals. As
the sample rate of the controller decreases, the phase margin decreases rapidly
and will lead to instability if the phase margin reaches zero. Large amounts of
phase loss can make a stable controller in the continuous domain become unstable
in the discrete domain. Here are a few ways to combat this.

\begin{itemize}
  \item Run the controller with a high sample rate.
  \item Designing the controller in the analog domain with enough phase margin
    to compensate for any phase loss that occurs as part of discretization.
  \item Convert the \gls{plant} to the digital domain and design the controller
    completely in the digital domain.
\end{itemize}

\section{s-plane to z-plane}

Transfer functions are converted to impulse responses using the Z-transform. The
s-plane's LHP maps to the inside of a unit circle in the z-plane. Table
\ref{tab:s-plane2z-plane} contains a few common points.

\begin{table}
  \renewcommand{\arraystretch}{1.3}
  \centering
  \begin{tabular}{|cc|}
    \hline
    \rowcolor{headingbg}
    \textbf{s-plane} & \textbf{z-plane} \\
    \hline
    $(0, 0)$ & $(0, 1)$ \\
    imaginary axis & edge of unit circle \\
    $(0, -\infty)$ & $(0, 0)$ \\
    \hline
  \end{tabular}
  \caption{Mapping from s-plane to z-plane}
  \label{tab:s-plane2z-plane}
\end{table}

You may notice that poles can be placed at $(0, 0)$ in the z-plane. This is
known as a deadbeat controller. An $\rm N^{th}$ order deadbeat controller decays
to the \gls{reference} in N timesteps. While this sounds great, there are other
considerations like actuation effort, \gls{robustness}, and
\gls{noise immunity}. These will be discussed in detail with LQR and LQE.

\section{Discretization methods}

Discretization is done using a zero-order hold. That is, the system state is
only updated at discrete intervals and the value is held constant between
samples. The exact method of applying this uses the matrix exponential, but this
can be computationaly expensive. Instead, approximations such as the following
are used.

\begin{enumerate}
  \item Forward Euler method. This is defined as
    $y_{n+1} = y_n + f(t_n, y_n) \Delta t$.
  \item Backward Euler method. This is defined as
    $y_{n+1} = y_n + f(t_{n+1}, y_{n+1}) \Delta t$.
  \item Bilinear transform. The first-order bilinear approximation is
    $s = \frac{2}{T} \frac{1 - z^{-1}}{1 + z^{-1}}$.
\end{enumerate}

where $T$ is the sample period for the discrete system. Since these are
approximations, there is distortion between the real discrete system's poles and
the approximate poles. For fast-changing systems, this distortion can quickly
lead to instability.

The exact method of computing the zero-order hold for state-space is shown
below.

\begin{align}
  \mtx{A}_d &= e^{\mtx{A}_c T} \\
  \mtx{B}_d &= \int_0^T e^{\mtx{A}_c \tau} d\tau \mtx{B}_c \\
  \mtx{C}_d &= \mtx{C}_c \\
  \mtx{D}_d &= \mtx{D}_c
\end{align}

where a subscript of $d$ denotes discrete, a subscript of $c$ denotes
continuous, and $T$ is the sample period for the discrete system.
$e^{\mtx{A}_c T}$ and others are referred to as matrix exponentials. We will use
Python Control's \texttt{sample\_system()} function to perform these
calculations later.

See the Wikipedia page on discretization for more \cite{bib:discretization}.

\subsection{Computing the matrix exponential}

Let $\mtx{X}$ be an $n \times n$ matrix. The exponential of $\mtx{X}$ denoted by
$e^{\mtx{X}}$ is the $n \times n$ matrix given by the power series below.

\begin{equation}
  e^{\mtx{X}} = \sum_{k=0}^\infty \frac{1}{k!} \mtx{X}^k
\end{equation}

where $\mtx{X}^0$ is defined to be the identity matrix $\mtx{I}$ with the same
dimensions as $\mtx{X}$.
