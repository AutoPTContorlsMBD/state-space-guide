\chapterimage{review-of-pid.jpg}{Treeline by Crown/Merril bus stop at UCSC}

\chapter{Review of PID controller mathematics}

\section{PID basics and theory}
\index{PID control}

Negative feedback loops drive the difference between the \gls{reference} and
\gls{output} to zero.

\textbf{Proportional} gain compensates for current \gls{error}. \\
\textbf{Integral} gain compensates for past error (i.e.,
\gls{steady-state error}). \\
\textbf{Derivative} gain compensates for future error by slowing controller down
  if error decreases over time.

\begin{bookfigure}
  \begin{tikzpicture}[auto, >=latex']
    \fontsize{9pt}{10pt}

    % Place the blocks
    \node [name=input] {$r(t)$};
    \node [sum, right=0.5cm of input] (errorsum) {};
    \node [coordinate, right=0.75cm of errorsum] (branch) {};
    \node [block, right=0.5cm of branch] (I) { $K_i \int_0^t e(\tau) \,d\tau$ };
    \node [block, above=0.5cm of I] (P) { $K_p e(t)$ };
    \node [block, below=0.5cm of I] (D) { $K_d \frac{de(t)}{dt}$ };
    \node [sum, right=0.5cm of I] (ctrlsum) {};
    \node [block, right=0.75cm of ctrlsum] (plant) {Plant};
    \node [right=0.75cm of plant] (output) {};
    \node [coordinate, below=0.5cm of D] (measurements) {};

    % Connect the nodes
    \draw [arrow] (input) -- node[pos=0.9] {$+$} (errorsum);
    \draw [-] (errorsum) -- node {$e(t)$} (branch);
    \draw [arrow] (branch) |- (P);
    \draw [arrow] (branch) -- (I);
    \draw [arrow] (branch) |- (D);
    \draw [arrow] (P) -| node[pos=0.95, left] {$+$} (ctrlsum);
    \draw [arrow] (I) -- node[pos=0.9, below] {$+$} (ctrlsum);
    \draw [arrow] (D) -| node[pos=0.95, right] {$+$} (ctrlsum);
    \draw [arrow] (ctrlsum) -- node {$u(t)$} (plant);
    \draw [arrow] (plant) -- node [name=y] {$y(t)$} (output);
    \draw [-] (y) |- (measurements);
    \draw [arrow] (measurements) -| node[pos=0.99, right] {$-$} (errorsum);
  \end{tikzpicture}

  \caption{PID controller diagram}
  \label{fig:pid_ctrl_diag}
\end{bookfigure}

\begin{figurekey}
  \begin{tabulary}{\linewidth}{LLLL}
    $r(t)$ & \gls{reference} input & $u(t)$ & control input \\
    $e(t)$ & error & $y(t)$ & \gls{output} \\
  \end{tabulary}
\end{figurekey}

\section{Types of PID controllers}

PID controller inputs of different orders of derivatives, such as position and
velocity, affect the \gls{system} response differently. Below is the standard
form for a position PID controller.

\begin{definition}[Position PID controller]
  \begin{equation}
    u(t) = K_p e(t) + K_i \int_0^t e(\tau) \,d\tau + K_d \frac{de}{dt}
    \label{def:pos_pid}
  \end{equation}
\end{definition}

If the controller is measuring and controlling velocity instead, the control
input $u(t)$ becomes $\frac{du}{dt}$ and the error $e(t)$ becomes
$\frac{de}{dt}$. Substituting these into equation (\ref{def:pos_pid}) yields

\begin{align}
  \frac{du}{dt} &= K_p \frac{de}{dt} + K_i \int_0^t \frac{de}{d\tau} d\tau +
    K_d \frac{d^2e}{dt^2} \nonumber \\
  \frac{du}{dt} &= K_p \frac{de}{dt} + K_i e(t) + K_d \frac{d^2e}{dt^2}
\end{align}

This shows that the proportional action ($K_p$) of a velocity controller is
caused by the integral action ($K_i$) of a position controller. The derivative
action ($K_i$) of a velocity controller is caused by the proportional action
($K_p$) of a position controller. Integral action ($K_i$) of a velocity
controller has no equivalent in the position formulation. If we were to
implement one, it would use a double integral.

Since for the velocity controller, the proportional term is controlled by an
integral action and the derivative term is controlled by a proportional action,
the coefficients can be relabelled as follows.

\begin{theorem}[Velocity PID controller]
  \begin{equation}
    u = K_p \int_0^t e(\tau) \,d\tau + K_d e(t)
  \end{equation}
\end{theorem}

Integral control for the velocity is analogous to the throttle pedal on a car.
One must hold the throttle pedal (the control input) at a nonzero value to keep
the car traveling at the reference velocity.

Read \url{https://en.wikipedia.org/wiki/PID_controller} for more information on
PID control theory.

\section{PID control in terms of general control theory}

PID control defines \textit{setpoint} as the desired position and
\textit{process value} as the measured position. Control theory has more general
terms for these: \gls{reference} and \gls{output} respectively.

The derivative term is commonly used to ``slow down" the system if it's already
heading toward the \gls{reference}. We will explore what $K_p$ and $K_d$ are
really doing for a two-state system (position and velocity) and why $K_d$ acts
that way.

First, we will rearrange the equation for a PD controller.

\begin{equation*}
  u = K_p e_k + K_d \frac{e_k - e_{k-1}}{dt}
\end{equation*}

where $u$ is the control input and $e_k$ is the error at timestep $k$. $e_k$ is
defined as $e_k = r_k - x_k$ where $r_k$ is the reference and $x_k$ is the
current state at timestep $k$.

\begin{align*}
  u &= K_p (r_k - x_k) + K_d \frac{(r_k - x_k) - (r_{k-1} - x_{k-1})}{dt} \\
  u &= K_p (r_k - x_k) + K_d \frac{r_k - x_k - r_{k-1} + x_{k-1}}{dt} \\
  u &= K_p (r_k - x_k) + K_d \frac{r_k - r_{k-1} - x_k + x_{k-1}}{dt} \\
  u &= K_p (r_k - x_k) + K_d \frac{(r_k - r_{k-1}) - (x_k - x_{k-1})}{dt} \\
  u &= K_p (r_k - x_k) + K_d \left(\frac{r_k - r_{k-1}}{dt} -
    \frac{x_k - x_{k-1}}{dt}\right)
\end{align*}

Notice how $\frac{r_k - r_{k-1}}{dt}$ is the velocity of the reference. By the
same reasoning, $\frac{x_k - x_{k-1}}{dt}$ is the system's velocity at a given
timestep. That means the $K_d$ term is driving the estimated velocity to the
reference velocity. If the reference is constant, that means the $K_d$ term is
trying to drive the velocity of the system to zero, but it can't because $K_p$
is trying to make the system move and $K_p$ and $K_d$ are controlling the same
actuator. If $K_p$ is larger than $K_d$, one is in effect slowing down the
response of the controller during transients with the hope of decreasing
overshoot and settling time. If one makes $K_d$ much larger than $K_p$, $K_d$
overpowers $K_p$ to bring the system to a stop. However, when the velocity is
low enough, $K_p$ is stronger and starts accelerating the system again. This
oscillatory behavior in the velocity repeats as the system moves toward the
reference.

\section{Limitations of PID control}

PID's heuristic method of tuning is fine when there is no knowledge of the
\gls{system}. However, controllers with much better response can be developed if
a \glslink{model}{dynamical model} of the \gls{system} is known.
