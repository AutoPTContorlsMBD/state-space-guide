\section{Review of PID controller mathematics}

\subsection{PID basics and theory}

Negative feedback loops drive the difference between the \gls{reference} and
\gls{output} to zero. \\

\textbf{Proportional} gain compensates for current \gls{error}. \\
\textbf{Integral} gain compensates for past error (i.e.,
\gls{steady-state error}). \\
\textbf{Derivative} gain compensates for future error by slowing controller down
  if error decreases over time.

\begin{figure}[H]
  \centering

  \begin{tikzpicture}[auto, >=latex']
    \fontsize{9pt}{10pt}

    % Place the blocks
    \node [name=input] {$r(t)$};
    \node [sum, right=0.5cm of input] (errorsum) {};
    \node [coordinate, right=0.75cm of errorsum] (branch) {};
    \node [block, right=0.5cm of branch] (I) { $K_i \int_0^t e(\tau) d\tau$ };
    \node [block, above=0.5cm of I] (P) { $K_p e(t)$ };
    \node [block, below=0.5cm of I] (D) { $K_d \frac{de(t)}{dt}$ };
    \node [sum, right=0.5cm of I] (ctrlsum) {};
    \node [block, right=0.75cm of ctrlsum] (plant) {Plant};
    \node [right=0.75cm of plant] (output) {};
    \node [coordinate, below=0.5cm of D] (measurements) {};

    % Connect the nodes
    \draw [arrow] (input) -- node[pos=0.9] {$+$} (errorsum);
    \draw [-] (errorsum) -- node {$e(t)$} (branch);
    \draw [arrow] (branch) |- (P);
    \draw [arrow] (branch) -- (I);
    \draw [arrow] (branch) |- (D);
    \draw [arrow] (P) -| node[pos=0.95, left] {$+$} (ctrlsum);
    \draw [arrow] (I) -- node[pos=0.9, below] {$+$} (ctrlsum);
    \draw [arrow] (D) -| node[pos=0.95, right] {$+$} (ctrlsum);
    \draw [arrow] (ctrlsum) -- node {$u(t)$} (plant);
    \draw [arrow] (plant) -- node [name=y] {$y(t)$} (output);
    \draw [-] (y) |- (measurements);
    \draw [arrow] (measurements) -| node[pos=0.99, right] {$-$} (errorsum);
  \end{tikzpicture}

  \caption{PID controller diagram}
  \label{fig:pid_ctrl_diag}
\end{figure}

\begin{center}
  \renewcommand{\arraystretch}{1.3}
  \begin{tabulary}{\linewidth}{LLLL}
    $r(t)$ & \gls{reference} input & $u(t)$ & control input \\
    $e(t)$ & error & $y(t)$ & \gls{output} \\
  \end{tabulary}
\end{center}

\begin{table}
  \caption{Plant versus controller}
  \renewcommand{\arraystretch}{1.3}
  \centering
  \begin{tabular}{|l|ll|}
    \hline
    \rowcolor{lightblue}
    & \textbf{Plant} & \textbf{Controller} \\
    \hline
    Input & $u(t)$ & $r(t)$, $y(t)$ \\
    Output & $y(t)$ & $u(t)$ \\
    \hline
  \end{tabular}
  \label{tab:plant_v_controller}
\end{table}

\subsection{Types of PID controllers}

PID controller inputs of different orders of derivatives, such as position and
velocity, affect the \gls{system} response differently. The position PID
controller is defined as

\begin{equation}
  u(t) = K_p e(t) + K_i \int_0^t e(\tau) d\tau + K_d \frac{de}{dt}
\end{equation}

If a velocity is passed instead, which is a change in position, the equation
becomes

\begin{align}
  \frac{du}{dt} &= K_p \frac{de}{dt} + K_i \int_0^t \frac{de}{d\tau} d\tau +
    K_d \frac{d^2e}{dt^2} \nonumber \\
  \frac{du}{dt} &= K_p \frac{de}{dt} + K_i e(t) + K_d \frac{d^2e}{dt^2}
    \label{eq:pid_vel}
\end{align}

This shows that $K_i$ and $K_p$ from the position controller act as proportional
and derivative terms respectively in the velocity controller. $K_i$ from the
position controller has no equivalent in the velocity controller. If we were to
implement one, it would use a double integral. However, it would be of limited
use since the $K_i$ term in equation (\ref{eq:pid_vel}) also eliminates
steady-state error for step changes in \gls{reference}. Relabelling the
coefficients to match the position PID controller gives

\begin{equation}
  \frac{du}{dt} = K_p \int_0^t e(\tau) d\tau + K_d e(t)
\end{equation}

Read \url{https://en.wikipedia.org/wiki/PID_controller} for more information.

\subsection{Limitations of PID control}

PID's heuristic method of tuning is fine when there is no knowledge of the
\gls{system}. However, controllers with much better response can be developed if
a \glslink{model}{dynamical model} of the \gls{system} is known.
