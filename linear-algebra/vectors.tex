\section{Vectors}

\subsection{What is a vector?}

The fundamental building block for linear algebra is the vector. Broadly
speaking, there are three distinct but related ideas about vectors: the physics
student perspective, the computer science student perspective, and the
mathmetician's perspective.

The physics student perspective is that vectors are arrows pointing in space. A
given vector is defined by its length and direction, but as long as those two
facts are the same, you can move it around and it's still the same vector.
Vectors in the flat plane are two-dimensional, and those sitting in broader
space that we live in are three-dimensional.

The computer science perspective is that vectors are ordered lists of numbers.
For example, let's say you were doing some analytics about house prices and the
only features you cared about where square footage and price. You might model
each house with a pair of numbers where the first indicates square footage and
the second indicates price. Notice the order matters here. In the lingo, you'd
be modeling houses as two-dimensional vectors where, in this context, vector is
a synonym for list, and what makes it two-dimensional is that the length of that
list is two.

The mathematician, on the other hand, seeks to generalize both these views by
saying that a vector can be anything where there's a sensible notion of adding
two vectors and multiplying a vector by a number (operations that we'll talk
about later on). The details of this view are rather abstract and won't be
needed for this book as we'll favor a more concrete setting. We bring it up here
because it hints at the fact that the ideas of vector addition and
multiplication by numbers will play an important role throughout linear algebra.

\subsection{Geometric interpretation of vectors}

Before we talk about vector addition and multiplication by numbers, let's settle
on a specific thought to have in mind when we say the word ``vector". Given the
geometric focus that we're intending here, whenever we introduce a new topic
involving vectors, we want you to first think about an arrow, and specifically,
think about that arrow inside a coordinate system, like the x-y plane with its
tail sitting at the origin. This is slightly different from the physics student
perspective where vectors can freely sit anywhere they want in space. In linear
algebra, it's almost always the case that your vector will be rooted at the
origin. Then, once you understand a new concept in the concept of arrows in
space, we'll translate it over to the ``list of numbers" point of view, which we
can do by considering the coordinates of the vector.

While you may already be familiar with this coordinate system, it's worth
walking through explicitly since this is where all of the important
back-and-forth happens between the two perspectives of linear algebra. Focusing
your attention on two dimensions for the moment, you have a horizontal line
called the x-axis and a vertical line called the y-axis. The point at which they
intersect is called the origin, which you should think of as the center of
space and the root of all vectors. After choosing an arbitrary length to
represent one, you make tick marks on each axis to represent this distance. The
coordinates of a vector is a pair of numbers that essentially gives instructions
for getting from the tail of that vector at the origin to its tip. The first
number is how far to move along the x-axis (positive numbers indicating
rightward motion and negative numbers indicating leftward motion), and the
second number is how far to move parallel to the y-axis after that (positive
numbers indicating upward motion and negative numbers indicating downward
motion). To distinguish vectors from points, the convention is to write this
pair of numbers vertically with square brackets around them. For example:

\begin{equation*}
  \begin{bmatrix}
    3 \\
    -1
  \end{bmatrix}
\end{equation*}

Every pair of numbers represents one and only one vector, and every vector is
associated with one and only one pair of numbers. In three dimensions, there is
a third axis called the z-axis which is perpendicular to both the x and y axes.
In this case, each vector is associated with an ordered triplet of numbers. The
first is how far to move along the x-axis, the second is how far to move
parallel to the y-axis, and the third is how far to then move parallel to this
new z-axis. For example:

\begin{equation*}
  \begin{bmatrix}
    2 \\
    1 \\
    3
  \end{bmatrix}
\end{equation*}

Every triplet of numbers represents one unique vector in space, and every vector
in space represents exactly one triplet of numbers.

\subsection{Vector addition}

Back to vector addition and multiplication by numbers. Afterall, every topic in
linear algebra is going to center around these two operations. Luckily, each
one is straightforward to define. Let's say we have two vectors, one pointing up
and a little to the right, and the other one pointing right and down a bit. To
add these two vectors, move the second one so that its tail sits at the tip of
the first one. Then, if you draw a new vector from the tail of the first one to
where the tip of the second one now sits, that new vector is their sum.

This definition of addition, by the way, is one of the only times in linear
algebra where we let vectors stray away from the origin, but why is this a
reasonable thing to do? Why this definition of addition and not some other one?
Each vector represents a sort of movement, a step with a certain distance and
direction in space. If you take a step along the first vector, then take a step
in the direction and distance described by the second vector, the overall effect
is just the same as if you moved along the sum of those two vectors to start
with.

You could think about this as an extension of how we think about adding numbers
on a number line. One way that we teach students to think about this, say with
$2 + 5$, is to think of moving two steps to the right, followed by another 5
steps to the right. The overall effect is the same as if you just took 7 steps
to the right. In fact, let's see how vector addition looks numerically. The
first vector here has coordinates $(1, 2)$ and the second has coordinates
$(3, -1)$.

\begin{equation*}
  \begin{bmatrix}
    1 \\
    2
  \end{bmatrix} + \begin{bmatrix}
    3 \\
    -1
  \end{bmatrix}
\end{equation*}

When you take the vector sum using this tip-to-tail method, you can think of a
four-step path from the origin to the tip of the second vector: ``walk 1 to the
right, then 2 up, then 3 to the right, then 1 down." Reorganizing these steps so
that you first do all of the rightward motion, then do all of the vertical
motion, you can read it as saying, ``first move $1 + 3$ to the right, then move
$2 + (-1)$ up," so the new vector has coordinates $1 + 3$ and $2 + (-1)$.

\begin{equation*}
  \begin{bmatrix}
    1 \\
    2
  \end{bmatrix} + \begin{bmatrix}
    3 \\
    -1
  \end{bmatrix} = \begin{bmatrix}
    1 + 3 \\
    2 + (-1)
  \end{bmatrix}
\end{equation*}

In general, vector addition in this list-of-numbers conception looks like
matching up their terms, and adding each one together.

\begin{equation*}
  \begin{bmatrix}
    x_1 \\
    y_1
  \end{bmatrix} + \begin{bmatrix}
    x_2 \\
    y_2
  \end{bmatrix} = \begin{bmatrix}
    x_1 + x_2 \\
    y_1 + y_2
  \end{bmatrix}
\end{equation*}

\subsection{Scalar-vector multiplication}

The other fundamental vector operation is multiplication by a number. Now this
is best understood just by looking at a few examples. If you take the number
$2$, and multiply it by a given vector, you stretch out that vector so that it's
two times as long as when you started. If you multiply that vector by, say,
$\frac{1}{3}$, you compress it down so that it's $\frac{1}{3}$ of the original
length. When you multiply it by a negative number, like $-1.8$, then the vector
is first flipped around, then stretched out by that factor of $1.8$.

This process of stretching, compressing, or reversing the direction of a vector
is called ``scaling", and whenver a number like $2$ or $\frac{1}{3}$ or $-1.8$
acting like this--scaling some vector--we call it a ``scalar". In fact,
throughout linear algebra, one of the main things numbers do is scale vectors,
so it's common to use the word ``scalar" interchangeably with the word
``number". Numerically, stretching out a vector by a factor of, say, $2$,
corresponds to multiplying each of its components by that factor, $2$.

\begin{equation*}
  2 \cdot \begin{bmatrix}
    3 \\
    1
  \end{bmatrix} = \begin{bmatrix}
    6 \\
    2
  \end{bmatrix}
\end{equation*}

So in the conception of vectors as lists of numbers, multiplying a given vector
by a scalar means multiplying each one of those components by that scalar.

\begin{equation*}
  2 \cdot \begin{bmatrix}
    x \\
    y
  \end{bmatrix} = \begin{bmatrix}
    2x \\
    2y
  \end{bmatrix}
\end{equation*}

\begin{remark}
  See the corresponding \textit{Essence of Linear Algebra} video for a more
  visual presentation (5 minutes) \cite{bib:linalg_vectors}.
\end{remark}
