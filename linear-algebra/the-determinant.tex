\section{The determinant}
\index{Matrices!determinant}

So, moving forward, we will be assuming you have a visual understanding of
linear transformations and how they're represented with matrices.

\subsection{Scaling areas}

If you think about a couple linear transformations, you might notice how some of
them seem to stretch space out while others compress it. It's useful for
understanding these transformations to measure exactly how much it stretches or
compresses things (more specifically, to measure the factor by which areas are
scaled). For example, look at the matrix with columns $(3, 0)$, and $(0, 2)$.

\begin{equation*}
  \begin{bmatrix}
    3 & 0 \\
    0 & 2
  \end{bmatrix}
\end{equation*}

It scales $\hat{i}$ by a factor of $3$ and scales $\hat{j}$ by a factor of $2$.
Now, if we focus our attention on the one-by-one square whose bottom sits on
$\hat{i}$ and whose left side sits on $\hat{j}$, after the transformation, this
turns into a $2$ by $3$ rectangle. Since this region started out with area $1$
and ended up with area $6$, we can say the linear transformation has scaled its
area by a factor of $6$. Compare that to a shear whose matrix has columns
$(1, 0)$ and $(1, 1)$ meaning $\hat{i}$ stays in place and $\hat{j}$ moves over
to $(1, 1)$.

\begin{equation*}
  \begin{bmatrix}
    1 & 1 \\
    0 & 1
  \end{bmatrix}
\end{equation*}

That same unit square determined by $\hat{i}$ and $\hat{j}$ gets slanted and
turned into a parallelogram, but the area of that parallelogram is still $1$
since its base and height each continue to each have length $1$. Even though
this transformation pushes things about, it seems to leave areas unchanged (at
least in the case of that one uint square).

Actually though, if you know how much the area of that one single unit square
changes, you can know how the area of any possible region in space changes.
First off, notice that whatever happens to one square in the grid has to happen
to any other square in the grid no matter the size. This follows from the fact
that grid lines remain parallel and evenly spaced. Then, any shape that's not a
grid square can be approximated by grid squares pretty well with arbitrarily
good approximations if you use small enough grid squares. So, since the areas of
all those tiny grid squares are being scaled by some single amount, the area of
the shape as a whole will also be scaled by that same single amount.

\subsection{Exploring the determinant}

This special scaling factor, the factor by which a linear transformation changes
any area, is called the \textit{determinant} of that transformation. We'll show
how to compute the determinant of a transformation using its matrix later on,
but understanding what it represents is much more important than the
computation. For example, the determinant of a transformation would be $3$ if
that transformation increases the area of the region by a factor of $3$.

\begin{equation*}
  det\left(\begin{bmatrix}
    0 & 2 \\
    -1.5 & 1
  \end{bmatrix}\right) = 3
\end{equation*}

The determinant of a matrix is commonly denoted by vertical bars instead of
square brackets.

\begin{equation*}
  \begin{vmatrix}
    0 & 2 \\
    -1.5 & 1
  \end{vmatrix} = 3
\end{equation*}

The determinant of a transformation would be $\frac{1}{2}$ if it compresses all
areas by a factor of $\frac{1}{2}$.

\begin{equation*}
  \begin{vmatrix}
    0.5 & 0.5 \\
    -0.5 & 0.5
  \end{vmatrix} = 0.5
\end{equation*}

The determinant of a 2D transformation is zero if it compresses all of space
onto a line or even onto a single point since then, the area of any region would
become zero.

\begin{equation*}
  \begin{vmatrix}
    4 & 2 \\
    2 & 1
  \end{vmatrix} = 0
\end{equation*}

That last example proved to be pretty important. It means checking if the
determinant of a given matrix is zero will give a way of computing whether the
transformation associated with that matrix compresses everything into a smaller
dimension.

This analogy so far isn't quite right. The full concept of a determinant allows
for negative values.

\begin{equation*}
  \begin{vmatrix}
    1 & 2 \\
    3 & 4
  \end{vmatrix} = -2
\end{equation*}

What would scaling an area by a negative amount even mean? This has to do with
the idea of orientation. A 2D transformation with a negative determinant
essentially flips space over. Any transformations that do this are said to
``invert the orientation of space". Another way to think about it is in terms of
$\hat{i}$ and $\hat{j}$. In their starting positions, $\hat{j}$ is to the left
of $\hat{i}$. If, after a transformation, $\hat{j}$ is now on the right side of
$\hat{i}$, the orientation of space has been inverted. Whenever this happens,
the determinant will be negative. The absolute value of the determinant still
tells you the factor by which areas have been scaled.

For example, the matrix with columns $(1, 1)$ and $(2, -1)$ encodes a
transformation that has determinant $-3$.

\begin{equation*}
  \begin{vmatrix}
    1 & 2 \\
    1 & -1
  \end{vmatrix} = -3
\end{equation*}

This means that space gets flipped over and areas are scaled by a factor of $3$.

Why would this idea of a negative area scaling factor be a natural way to
describe orientation-flipping? Think about the series of transformations you get
by slowly letting $\hat{i}$ rotate closer and closer to $\hat{j}$. As $\hat{i}$
gets closer, all the areas in space are getting compressed more and more meaning
the determinant approaches zero. Once $\hat{i}$ lines up perfectly with
$\hat{j}$, the determinant is zero. Then, if $\hat{i}$ continues, doesn't it
feel natural for the determinant to keep decreasing into negative numbers?

\subsection{The determinant in 3D}

That's the understanding of determinants in two dimensions. What should it mean
for three dimensions? The determinant of a $3 \times 3$ matrix tells you how
much volumes get scaled. A determinant of zero would mean that all of space is
compressed onto something with zero volume meaning either a flat plane, a line,
or in the most extreme case, a single point. This means that the columns of the
matrix are linearly dependent.

What should negative determinants mean for three dimensions? One way to describe
orientation in 3D is with the right-hand rule. Point the forefinger of your
right hand in the direction if $\hat{i}$, stick out your middle finger in the
direction of $\hat{j}$, and notice how when you point your thumb up, it is in
the direction of $\hat{k}$. If you can still do that after the transformation,
orientation has not changed and the determinant is positive. Otherwise, if after
the transformation it only makes sense to do that with your left hand,
orientation has been flipped and the determinant is negative.

\subsection{Computing the determinant}

How do you actually compute the determinant? For a $2 \times 2$ matrix with
entries $a$, $b$, $c$, $d$, the formula is as follows.

\begin{equation*}
  \begin{vmatrix}
    a & b \\
    c & d
  \end{vmatrix} = ad - bc
\end{equation*}

Here's part of an intution for where this formula comes from. Let's say that the
terms $b$ and $c$ were both zero. Then, the term $a$ tells you how much
$\hat{i}$ is stretched in the x-direction and the term $d$ tells you how much
$\hat{j}$ is stretched in the y-direction. Since those other terms are zero, it
should make sense that $ad$ gives the area of the rectangle that the unit square
turns into. Even if only one of $b$ or $c$ are zero, you'll have a parallelogram
with a base of $a$ and a height $d$, so the area should still be $ad$. Loosely
speaking, if both $b$ and $c$ are nonzero, then that $bc$ term tells you how
much this parallelogram is stretched or compressed in the diagonal direction.

If you feel like computing determinants by hand is something that you need to
know (you won't for this book), the only way to get it down is to just practice
it with a few. This is all triply true for 3D determinants. There is a formula,
and if you feel like that's something you need to know, you should practice with
a few matrices.

\begin{equation*}
  \begin{vmatrix}
    a & b & c \\
    d & e & f \\
    g & h & i
  \end{vmatrix} =
  a \begin{vmatrix}
    e & f \\
    h & i
  \end{vmatrix}
  - b \begin{vmatrix}
    d & f \\
    g & i
  \end{vmatrix}
  + c \begin{vmatrix}
    d & e \\
    g & h
  \end{vmatrix}
\end{equation*}

We don't think those computations fall within the essence of linear algebra, but
understanding what the determinant represents falls within that essence.

\begin{remark}
  See the corresponding \textit{Essence of linear algebra} video for a more
  visual presentation (10 minutes) \cite{bib:linalg_the_determinant}.
\end{remark}
