\section{Why feedback control?}

Let's say we are controlling a DC brushed motor. With just a
\glslink{model}{mathematical model} and knowledge of all current \glspl{state}
of the \gls{system} (i.e., angular velocity), we can predict all future
\glspl{state} given the future voltage \glspl{input}. Why then do we need
feedback control? If the \gls{system} is \glslink{disturbance}{disturbed} in any
way that isn't modeled by our equations, like a load was applied to the
armature, or voltage sag in the rest of the circuit caused the commanded voltage
to not match the actual applied voltage, the angular velocity of the motor will
deviate from the \gls{model} over time.

To combat this, we can take measurements of the \gls{system} and the environment
to detect this deviation and account for it. For example, we could measure the
current position and estimate an angular velocity from it. We can then give the
motor corrective commands as well as steer our \gls{model} back to reality. This
feedback allows us to account for uncertainty and be
\glslink{robustness}{robust} to it.
