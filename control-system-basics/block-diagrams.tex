\section{Block diagrams}
\index{Block diagrams}

When designing or analyzing a \gls{control system}, it is useful to model it
graphically. Block diagrams are used for this purpose. They can be manipulated
and simplified systematically (see appendix
\ref{ch:simplifying_block_diagrams}). Figure \ref{fig:gain_nomenclature} is an
example of one.

\begin{bookfigure}
  \begin{tikzpicture}[auto, >=latex']
    % Place the blocks
    \node [name=input] {input};
    \node [sum, right=of input] (sum) {};
    \node [block, right=of sum] (P1) {open-loop};
    \node [right=of P1] (output) {output};
    \node [block, below=of P1] (P2) {feedback};

    % Connect the nodes
    \draw [arrow] (input) -- node[pos=0.85] {$+$} (sum);
    \draw [arrow] (sum) -- node {} (P1);
    \draw [arrow] (P1) -- node[name=y] {} (output);
    \draw [arrow] (y) |- (P2);
    \draw [arrow] (P2) -| node[pos=0.97, right] {$\mp$} (sum);
  \end{tikzpicture}

  \caption{Block diagram with nomenclature}
  \label{fig:gain_nomenclature}
\end{bookfigure}

The \gls{open-loop gain} is the total gain from the sum node at the input to the
output branch. The \gls{feedback gain} is the total gain from the output back to
the input sum node. The circle's output is the sum of its inputs.

Figure \ref{fig:feedback_block_diagram} is a block diagram with more formal
notation in a feedback configuration.

\begin{bookfigure}
  \begin{tikzpicture}[auto, >=latex']
    % Place the blocks
    \node [name=input] {$X(s)$};
    \node [sum, right=of input] (sum) {};
    \node [block, right=of sum] (P1) {$P_1$};
    \node [right=of P1] (output) {$Y(s)$};
    \node [block, below=of P1] (P2) {$P_2$};

    % Connect the nodes
    \draw [arrow] (input) -- node[pos=0.85] {$+$} (sum);
    \draw [arrow] (sum) -- node {} (P1);
    \draw [arrow] (P1) -- node[name=y] {} (output);
    \draw [arrow] (y) |- (P2);
    \draw [arrow] (P2) -| node[pos=0.97, right] {$\mp$} (sum);
  \end{tikzpicture}

  \caption{Feedback block diagram}
  \label{fig:feedback_block_diagram}
\end{bookfigure}

\begin{theorem}[Closed-loop gain for a feedback loop]
  \begin{equation}
    \frac{Y(s)}{X(s)} = \frac{P_1}{1 \mp P_1 P_2}
  \end{equation}
\end{theorem}

See appendix \ref{sec:deriv_tf_feedback} for a derivation.
