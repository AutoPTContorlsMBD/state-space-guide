\chapterimage{kinematics-and-dynamics.jpg}{Hills by freeway between Santa Maria and Ventura}

\chapter{Linear algebra}

Modern control theory borrows concepts from linear algebra. At first, linear
algebra may appear very abstract, but there are simple geometric intuitions
underlying it. First, watch 3Blue1Brown's preview video for the
\textit{Essence of linear algebra} video series (5 minutes)
\cite{bib:linalg_preview}. The goal here is to provide an intuitive, geometric
understanding of linear algebra as a method of linear transformations.

While only a subset of the material from the videos will be presented here that
we think is relevant to this book, we highly suggest watching the whole series
\cite{bib:essence_of_linalg}.

\begin{remark}
  The following sections are essentially transcripts of the video content for
  those who don't want to watch an hour of YouTube videos. However, we suggest
  watching the videos instead because animations are better at conveying the
  geometric intuition involved than text.
\end{remark}

\renewcommand*{\chapterpath}{\partpath/linear-algebra}
\section{Vectors}

\subsection{What is a vector?}

The fundamental building block for linear algebra is the vector. Broadly
speaking, there are three distinct but related ideas about vectors: the physics
student perspective, the computer science student perspective, and the
mathmetician's perspective.

The physics student perspective is that vectors are arrows pointing in space. A
given vector is defined by its length and direction, but as long as those two
facts are the same, you can move it around and it's still the same vector.
Vectors in the flat plane are two-dimensional, and those sitting in broader
space that we live in are three-dimensional.

The computer science perspective is that vectors are ordered lists of numbers.
For example, let's say you were doing some analytics about house prices and the
only features you cared about where square footage and price. You might model
each house with a pair of numbers where the first indicates square footage and
the second indicates price. Notice the order matters here. In the lingo, you'd
be modeling houses as two-dimensional vectors where, in this context, vector is
a synonym for list, and what makes it two-dimensional is that the length of that
list is two.

The mathematician, on the other hand, seeks to generalize both these views by
saying that a vector can be anything where there's a sensible notion of adding
two vectors and multiplying a vector by a number (operations that we'll talk
about later on). The details of this view are rather abstract and won't be
needed for this book as we'll favor a more concrete setting. We bring it up here
because it hints at the fact that the ideas of vector addition and
multiplication by numbers will play an important role throughout linear algebra.

\subsection{Geometric interpretation of vectors}

Before we talk about vector addition and multiplication by numbers, let's settle
on a specific thought to have in mind when we say the word ``vector". Given the
geometric focus that we're intending here, whenever we introduce a new topic
involving vectors, we want you to first think about an arrow, and specifically,
think about that arrow inside a coordinate system, like the x-y plane with its
tail sitting at the origin. This is slightly different from the physics student
perspective where vectors can freely sit anywhere they want in space. In linear
algebra, it's almost always the case that your vector will be rooted at the
origin. Then, once you understand a new concept in the concept of arrows in
space, we'll translate it over to the ``list of numbers" point of view, which we
can do by considering the coordinates of the vector.

While you may already be familiar with this coordinate system, it's worth
walking through explicitly since this is where all of the important
back-and-forth happens between the two perspectives of linear algebra. Focusing
your attention on two dimensions for the moment, you have a horizontal line
called the x-axis and a vertical line called the y-axis. The point at which they
intersect is called the origin, which you should think of as the center of
space and the root of all vectors. After choosing an arbitrary length to
represent one, you make tick marks on each axis to represent this distance. The
coordinates of a vector is a pair of numbers that essentially gives instructions
for getting from the tail of that vector at the origin to its tip. The first
number is how far to move along the x-axis (positive numbers indicating
rightward motion and negative numbers indicating leftward motion), and the
second number is how far to move parallel to the y-axis after that (positive
numbers indicating upward motion and negative numbers indicating downward
motion). To distinguish vectors from points, the convention is to write this
pair of numbers vertically with square brackets around them. For example:

\begin{equation*}
  \begin{bmatrix}
    3 \\
    -1
  \end{bmatrix}
\end{equation*}

Every pair of numbers represents one and only one vector, and every vector is
associated with one and only one pair of numbers. In three dimensions, there is
a third axis called the z-axis which is perpendicular to both the x and y axes.
In this case, each vector is associated with an ordered triplet of numbers. The
first is how far to move along the x-axis, the second is how far to move
parallel to the y-axis, and the third is how far to then move parallel to this
new z-axis. For example:

\begin{equation*}
  \begin{bmatrix}
    2 \\
    1 \\
    3
  \end{bmatrix}
\end{equation*}

Every triplet of numbers represents one unique vector in space, and every vector
in space represents exactly one triplet of numbers.

\subsection{Vector addition}

Back to vector addition and multiplication by numbers. Afterall, every topic in
linear algebra is going to center around these two operations. Luckily, each
one is straightforward to define. Let's say we have two vectors, one pointing up
and a little to the right, and the other one pointing right and down a bit. To
add these two vectors, move the second one so that its tail sits at the tip of
the first one. Then, if you draw a new vector from the tail of the first one to
where the tip of the second one now sits, that new vector is their sum.

This definition of addition, by the way, is one of the only times in linear
algebra where we let vectors stray away from the origin, but why is this a
reasonable thing to do? Why this definition of addition and not some other one?
Each vector represents a sort of movement, a step with a certain distance and
direction in space. If you take a step along the first vector, then take a step
in the direction and distance described by the second vector, the overall effect
is just the same as if you moved along the sum of those two vectors to start
with.

You could think about this as an extension of how we think about adding numbers
on a number line. One way that we teach students to think about this, say with
$2 + 5$, is to think of moving two steps to the right, followed by another 5
steps to the right. The overall effect is the same as if you just took 7 steps
to the right. In fact, let's see how vector addition looks numerically. The
first vector here has coordinates $(1, 2)$ and the second has coordinates
$(3, -1)$.

\begin{equation*}
  \begin{bmatrix}
    1 \\
    2
  \end{bmatrix} + \begin{bmatrix}
    3 \\
    -1
  \end{bmatrix}
\end{equation*}

When you take the vector sum using this tip-to-tail method, you can think of a
four-step path from the origin to the tip of the second vector: ``walk 1 to the
right, then 2 up, then 3 to the right, then 1 down." Reorganizing these steps so
that you first do all of the rightward motion, then do all of the vertical
motion, you can read it as saying, ``first move $1 + 3$ to the right, then move
$2 + (-1)$ up," so the new vector has coordinates $1 + 3$ and $2 + (-1)$.

\begin{equation*}
  \begin{bmatrix}
    1 \\
    2
  \end{bmatrix} + \begin{bmatrix}
    3 \\
    -1
  \end{bmatrix} = \begin{bmatrix}
    1 + 3 \\
    2 + (-1)
  \end{bmatrix}
\end{equation*}

In general, vector addition in this list-of-numbers conception looks like
matching up their terms, and adding each one together.

\begin{equation*}
  \begin{bmatrix}
    x_1 \\
    y_1
  \end{bmatrix} + \begin{bmatrix}
    x_2 \\
    y_2
  \end{bmatrix} = \begin{bmatrix}
    x_1 + x_2 \\
    y_1 + y_2
  \end{bmatrix}
\end{equation*}

\subsection{Scalar-vector multiplication}

The other fundamental vector operation is multiplication by a number. Now this
is best understood just by looking at a few examples. If you take the number
$2$, and multiply it by a given vector, you stretch out that vector so that it's
two times as long as when you started. If you multiply that vector by, say,
$\frac{1}{3}$, you compress it down so that it's $\frac{1}{3}$ of the original
length. When you multiply it by a negative number, like $-1.8$, then the vector
is first flipped around, then stretched out by that factor of $1.8$.

This process of stretching, compressing, or reversing the direction of a vector
is called ``scaling", and whenver a number like $2$ or $\frac{1}{3}$ or $-1.8$
acting like this--scaling some vector--we call it a ``scalar". In fact,
throughout linear algebra, one of the main things numbers do is scale vectors,
so it's common to use the word ``scalar" interchangeably with the word
``number". Numerically, stretching out a vector by a factor of, say, $2$,
corresponds to multiplying each of its components by that factor, $2$.

\begin{equation*}
  2 \cdot \begin{bmatrix}
    3 \\
    1
  \end{bmatrix} = \begin{bmatrix}
    6 \\
    2
  \end{bmatrix}
\end{equation*}

So in the conception of vectors as lists of numbers, multiplying a given vector
by a scalar means multiplying each one of those components by that scalar.

\begin{equation*}
  2 \cdot \begin{bmatrix}
    x \\
    y
  \end{bmatrix} = \begin{bmatrix}
    2x \\
    2y
  \end{bmatrix}
\end{equation*}

\begin{remark}
  See the corresponding \textit{Essence of Linear Algebra} video for a more
  visual presentation (5 minutes) \cite{bib:linalg_vectors}.
\end{remark}

\section{Linear combinations, span, and basis vectors}

Vector coordinates were probably already familiar to you, but there's another
interesting way to think about these coordinates which is central to linear
algebra. When given a pair of numbers that's meant to describe a vector, like
$(3, 2)$, we want you to think about each coordinate as a scalar, meaning, think
about how each one stretches or compresses vectors.

\subsection{Basis vectors}
\index{Linear algebra!basis vectors}

In the xy-coordinate system, there are two special vectors: the one pointing to
the right with length $1$, commonly called ``i-hat", or the unit vector in the
x-direction ($\hat{i}$), and the one pointing straight up, with length $1$,
commonly called ``j-hat", or the unit vector in the y-direction ($\hat{j}$).

Now think of the x-coordinate of our vector as a scalar that scales $\hat{i}$,
stretching it by a factor of $3$, and the y-coordinate as a scalar that scales
$\hat{j}$, flipping it and stretching it by a factor of $2$. In this sense, the
vectors that these coordinates describe is the sum of two scaled vectors
$(3)\hat{i} + (-2)\hat{j}$. This idea of adding together two scaled vectors is a
surprisingly important concept. Those two vectors, $\hat{i}$ and $\hat{j}$, have
a special name, by the way. Together they're called the \textit{basis} of a
coordinate system ($\hat{i}$ and $\hat{j}$ are the ``basis vectors" of the
xy-coordinate system). When you think about coordinates as scalars, the basis
vectors are what those scalars actually scale.

By framing our coordinate system in terms of these two special basis vectors, it
raises an interesting and subtle point: we could have chosen different basis
vectors and had a completely reasonable, new coordinate system system. For
example, take some vector pointing up and to the right, along with some other
vector pointing down and to the right, in some way. Take a moment to think about
all the different vectors that you can get by choosing two scalars, using each
one to scale one of the vectors, then adding together what you get. Which
two-dimensional vectors can you reach by altering the choices of scalars? The
answer is that you can reach every possible two-dimensional vector. A new pair
of basis vectors like this still gives us a valid way to go back and forth
between pairs of numbers and two-dimensional vectors, but the association is
definitely different from the one that you get using the more standard basis of
$\hat{i}$ and $\hat{j}$.

\subsection{Linear combination}
\index{Linear algebra!linear combination}

Any time we describe vectors numerically, it depends on an implicit choice of
what basis vectors we're using. So any time that you're scaling two vectors and
adding them like this, it's called a \textit{linear combination} of those two
vectors. Below is a linear combination of vectors $\vec{v}$ and $\vec{w}$ with
scalars $a$ and $b$.

\begin{equation*}
  a \vec{v} + b \vec{w}
\end{equation*}

Where does this word ``linear" come from? Why does this have anything to do with
lines? This isn't the etymology, but if you fix one of those scalars and let the
other one change its value freely, the tip of the resulting vector draws a
straight line.

\subsection{Span}

Now, if you let both scalars range freely and consider every possible resultant
vector, there are three things that can happen. For most pairs of vectors,
you'll be able to reach every possible point in the plane; every two-dimensional
vector is within your grasp. However, in the unlucky case where your two original
vectors happen to line up, the tip of the resulting vector is limited to just a
single line passing through the origin. The vectors could also both be zero, in
which case the resultant vector is just at the origin.

The set of all possible vectors that you can reach with a linear combination of
a given pair of vectors is called the \textit{span} of those two vectors. So,
restating what we just discussed in this lingo, the span of most pairs of 2D
vectors is all vectors in 2D space, but when they line up, their span is all
vectors whose tip sits on a certain line.

Remember how we said that linear algebra revolves around vector addition and
scalar multiplication? The span of two vectors is a way of asking, ``What are
all the possible vectors one can reach using only these two fundamental
operations, vector addition and scalar multiplication?"

Thinking about a whole collection of vectors sitting on a line gets crowded, and
even more so to think about all two-dimensional vectors at once filling up the
plane. When dealing with collections of vectors like this, it's common to
represent each one with just a point in space where the tip of the vector was.
This way, if you want to think about every possible vector whose tip sits on a
certain line, just think about the line itself.

Likewise, to think about all possible two-dimensional vectors at once,
conceptualize each one as the point where its tip sits. In effect, you're
thinking about the infinite, flat sheet of two-dimensional space itself, leaving
the arrows out of it.

In general, if you're thinking about a vector on its own, think of it as an
arrow, and if you're dealing with a collection of vectors, it's convenient to
think of them all as points. Therefore, for our span example, the span of most
pairs of vectors ends up being the entire infinite sheet of two-dimensional
space, but if they line up, their span is just a line.

The idea of span gets more interesting if we start thinking about vectors in
three-dimensional space. For example, given two vectors in 3D space that are not
pointing in the same direction, what does it mean to take their span? Their span
is the collection of all possible linear combinations of those two vectors,
meaning all possible vectors you get by scaling each vector in some way, then
adding them together.

You can imagine turning two different knobs to change the two scalars defining
the linear combination, adding the scaled vectors and following the tip of the
resulting vector. That tip will trace out a flat sheet cutting through the
origin of three-dimensional space. This flat sheet is the span of the two
vectors. More precisely, the span of the two vectors is the set of all possible
vectors whose tips sit on that flat sheet.

So what happens if we add a third vector and consider the span of all three? A
linear combination of three vectors is defined similarly as it is for two;
you'll choose three different scalars, scale each of those vectors, then add
them all together. The linear combination of $\vec{v}$, $\vec{w}$, and $\vec{u}$
looks like

\begin{equation*}
  a\vec{v} + b\vec{w} + c\vec{u}
\end{equation*}

where $a$, $b$, and $c$ are allowed to vary. Again, the span of these vectors is
the set of all possible linear combinations.

Two different things could happen here. If your third vector happens to be
sitting on the span of the first two, then the span doesn't change; you're
trapped on that same flat sheet. In other words, adding a scaled version of that
third vector to the linear combination doesn't give you access to any new
vectors. However, if you just randomly choose a third vector, it's almost
certainly not sitting on the span of those first two. Then, since it's pointing
in a separate direction, it unlocks access to every possible three-dimensional
vector. As you scale that new third vector, it moves around that span sheet of
the first two, sweeping it through all of space.

Another way to think about it is that you're making full use of the three,
freely-changing scalars that you have at your disposal to acces the full three
dimensions of space.

\subsection{Linear dependence and independence}

In the case where the third vector was already sitting on the span of the first
two, or the case where two vectors happen to line up, we want some terminology
to describe the fact that at least one of these vectors is redundant--not adding
anything to our span. When there are multiple vectors and one could be removed
without reducing the span, the relevant terminology is to say that they are
\textit{linearly dependent}.

In other words, one of the vectors can be expressed as a linear combination of
the others since it's already in the span of the others.

\begin{equation*}
  \vec{u} = a\vec{v} + b\vec{w} \text{ for some values of $a$ and $b$}
\end{equation*}

On the other hand, if each vector really does add another dimension to the span,
they're said to be \textit{linearly independent}.

\begin{equation*}
  \vec{w} \neq a\vec{v} \text{ for all values of $a$}
\end{equation*}

Now with all that terminology, and hopefully some good mental images to go with
it, the technical definition of a basis of a space is as follows.

\begin{definition}[Basis of a vector space]
  The \textit{basis} of a vector space is a set of \textit{linearly independent}
  vectors that \textit{span} the full space.
\end{definition}

\begin{remark}
  See the corresponding \textit{Essence of linear algebra} video for a more
  visual presentation (10 minutes) \cite{bib:linalg_linear_combinations}.
\end{remark}

\section{Linear transformations and matrices}

This section focuses on what linear transformations look like in the case of two
dimensions and how they relate to the idea of a matrix-vector multiplication.

\begin{equation*}
  \begin{bmatrix}
    \textcolor{red}{1} & \textcolor{orange}{-3} \\
    \textcolor{green}{2} & \textcolor{cyan}{4}
  \end{bmatrix}
  \begin{bmatrix}
    \textcolor{blue}{5} \\
    \textcolor{purple}{7}
  \end{bmatrix} = \begin{bmatrix}
    (\textcolor{red}{1})(\textcolor{blue}{5}) +
      (\textcolor{orange}{-3})(\textcolor{purple}{7}) \\
    (\textcolor{green}{2})(\textcolor{blue}{5}) +
      (\textcolor{cyan}{4})(\textcolor{purple}{7})
  \end{bmatrix}
\end{equation*}

In particular, we want to show you a way to think about matrix-vector
multiplication that doesn't rely on memorization of the procedure shown above.

\subsection{What is a linear transformation?}

To start, let's just parse this term ``linear transformation". ``Transformation"
is essentially another name for ``function". It's something that takes in inputs
and returns an output for each one. Specifically in the context of linear
algebra, we consider transformations that take in some vector and spit out
another vector.

\begin{figure}[H]
  \centering

  \begin{tikzpicture}[auto, >=latex']
    % Place the nodes
    \node [name=input] {
      $\begin{bmatrix}
        5 \\
        7
      \end{bmatrix}$
    };
    \node [name=inputlabel, below=of input] {Vector input};
    \node [name=func, right=of input] {$L(\vec{v})$};
    \node [name=output, right=of func] {
      $\begin{bmatrix}
        2 \\
        -3
      \end{bmatrix}$
    };
    \node [name=outputlabel, below=of output] {Vector output};

    % Connect the nodes
    \draw [arrow] (input) -- node {} (func);
    \draw [arrow] (func) -- node {} (output);
  \end{tikzpicture}
\end{figure}

So why use the word ``transformation" instead of ``function" if they mean the
same thing? It's to be suggestive of a certain way to visualize this
input-output relation. You see, a great way to understand functions of vectors
is to use movement. If a transformation takes some input vector to some output
vector, we imagine that input vector moving over to the output vector. Then to
understand the transformation as a whole, we might imagine watching every
possible input vector move over to its corresponding output vector. it gets
really crowded to think about all the vectors all at once, where each one is an
arrow. Therefore, as we mentioned in the previous section, it's useful to
conceptualize each vector as a single point where its tip sits rather than an
arrow. To think about a transformation taking every possible input vector to
some output vector, we watch every point in space moving to some other point.

The effect of various transformations moving around all of the points in space
gives the feeling of compressing and morphing space itself. As you can imagine
though, arbitrary transformations can look complicated. Luckily, linear algebra
limits itself to a special type of transformation, ones that are easier to
understand, called ``linear" transformations. Visually speaking, a
transformation is linear if it has two properties: all straight lines must
remain as such, and the origin must remain fixed in place. In general, you
should think of linear transformations as keeping grid lines parallel and evenly
spaced.

\subsection{Describing transformations numerically}

Some transformations are simple to think about, like rotations about the origin.
Others are more difficult to describe with words. So how could one describe
these transformations numerically? If you were, say, programming some animations
to make a video teaching the topic, what formula could you give the computer so
that if you give it the coordinates of a vector, it would return the coordinates
of where that vectors lands?

\begin{figure}[H]
  \centering

  \begin{tikzpicture}[auto, >=latex']
    % Place the nodes
    \node [name=input] {
      $\begin{bmatrix}
        x_{in} \\
        y_{in}
      \end{bmatrix}$
    };
    \node [name=func, right=of input] {$????$};
    \node [name=output, right=of func] {
      $\begin{bmatrix}
        x_{out} \\
        y_{out}
      \end{bmatrix}$
    };

    % Connect the nodes
    \draw [arrow] (input) -- node {} (func);
    \draw [arrow] (func) -- node {} (output);
  \end{tikzpicture}
\end{figure}

You only need to record where the two basis vectors, $\hat{i}$ and $\hat{j}$,
each land, and everything else will follow from that. For example, consider the
vector $v$ with coordinates $(-1, 2)$, meaning $\vec{v} = -1\hat{i} + 2\hat{j}$.
If we play some transformation and follow where all three of these vectors go,
the property that grid lines remain parallel and evenly spaced has a really
important consequence: the place where $\vec{v}$ lands will be $-1$ times the
vector where $\hat{i}$ landed plus $2$ times the vector where $\hat{j}$ landed.
In other words, it started off as a certain linear combination of $\hat{i}$ and
$\hat{j}$ and it ends up as that same linear combination of where those two
vectors landed. This means you can deduce where $\vec{v}$ must go based only on
where $\hat{i}$ and $\hat{j}$ each land. For this transformation, $\hat{i}$
lands on the coordinates $(1, -2)$ and $\hat{j}$ lands on the x-axis at the
coordinates $(3, 0)$.

\begin{align*}
  \text{Transformed } \vec{v} &= -1(\text{Transformed } \hat{i}) +
    2(\text{Transformed } \hat{j}) \\
  \text{Transformed } \vec{v} &= -1\begin{bmatrix}
    1 \\
    -2
  \end{bmatrix} + 2\begin{bmatrix}
    3 \\
    0
  \end{bmatrix}
\end{align*}

Adding that all together, you can deduce that $\vec{v}$ has to land on the
vector $(5, 2)$.

\begin{align*}
  \text{Transformed } \vec{v} &= \begin{bmatrix}
    -1(1) + 2(3) \\
    -1(-2) + 2(0)
  \end{bmatrix} \\
  \text{Transformed } \vec{v} &= \begin{bmatrix}
    5 \\
    2
  \end{bmatrix}
\end{align*}

This is a good point to pause and ponder, because it's pretty important. This
gives us a technique to deduce where any vectors land, so long as we have a
record of where $\hat{i}$ and $\hat{j}$ each land, without needing to watch the
transformation itself.

Given a vector with more general coordinates $x$ and $y$, it will land on $x$
times the vector where $\hat{i}$ lands $(1, -2)$, plus $y$ times the vector
where $\hat{j}$ lands $(3, 0)$. Carrying out that sum, you see that it lands at
$(1x + 3y, -2x + 0y)$.

\begin{equation*}
  \begin{array}{cc}
    \hat{i} \rightarrow \begin{bmatrix}
      1 \\
      -2
    \end{bmatrix} &
    \hat{j} \rightarrow \begin{bmatrix}
      3 \\
      0
    \end{bmatrix}
  \end{array}
\end{equation*}

\begin{equation*}
  \begin{bmatrix}
    x \\
    y
  \end{bmatrix} \rightarrow x\begin{bmatrix}
    1 \\
    -2
  \end{bmatrix} + y\begin{bmatrix}
    3 \\
    0
  \end{bmatrix} = \begin{bmatrix}
    1x + 3y \\
    -2x + 0y
  \end{bmatrix}
\end{equation*}

Given any vector, this formula will describe where that vector lands.

What all of this is saying is that a two dimensional linear transformation is
completely described by just four numbers: the two coordinates for where
$\hat{i}$ lands and the two coordinates for where $\hat{j}$ lands. It's common
to package these coordinates into a two-by-two grid of numbers, called a
two-by-two matrix, where you can interpret the columns as the two special
vectors where $\hat{i}$ and $\hat{j}$ each land. If $\hat{i}$ lands on the
vector $(3, -2)$ and $\hat{j}$ lands on the vector $(2, 1)$, this two-by-two
matrix would be

\begin{equation*}
  \begin{bmatrix}
    3 & 2 \\
    -2 & 1
  \end{bmatrix}
\end{equation*}

If you're given a two-by-two matrix describing a linear transformation and some
specific vector, say $(5, 7)$, and you want to know where that linear
transformation takes that vector, you can multiply the coordinates of the vector
by the corresponding columns of the matrix, then add together the result.

\begin{equation*}
  \begin{bmatrix}
    3 & 2 \\
    -2 & 1
  \end{bmatrix}
  \begin{bmatrix}
    5 \\
    7
  \end{bmatrix} = 5\begin{bmatrix}
    3 \\
    -2
  \end{bmatrix} + 7\begin{bmatrix}
    2 \\
    1
  \end{bmatrix}
\end{equation*}

This corresponds with the idea of adding the scaled versions of our new basis
vectors.

Let's see what this looks like in the most general case where your matrix has
entries $a$, $b$, $c$, $d$.

\begin{equation*}
  \begin{bmatrix}
    a & b \\
    c & d
  \end{bmatrix}
\end{equation*}

Remember, this matrix is just a way of packaging the information needed to
describe a linear transformation. Always remember to interpret that first
column, $(a, c)$, as the place where the first basis vector lands and that
second column, $(b, d)$, as the place where the second basis vector lands.

When we apply this transformation to some vector $(x, y)$, the result will be
$x$ times $(a, c)$ plus $y$ times $(b, d)$. Together, this gives a vector
$(ax + by, cx + dy)$.

\begin{equation*}
  \begin{bmatrix}
    a & b \\
    c & d
  \end{bmatrix} \begin{bmatrix}
    x \\
    y
  \end{bmatrix} = x\begin{bmatrix}
    a \\
    c
  \end{bmatrix} + y\begin{bmatrix}
    b \\
    d
  \end{bmatrix} = \begin{bmatrix}
    ax + by \\
    cx + dy
  \end{bmatrix}
\end{equation*}

You could even define this as matrix-vector multiplication when you put the
matrix on the left of the vector like it's a function. Then, you could make high
schoolers memorize this, without showing them the crucial part that makes it
feel intuitive (yes, that was sarcasm). Isn't it more fun to think about these
columns as the transformed versions of your basis vectors and to think about the
result as the appropriate linear combination of those vectors?

\subsection{Examples of linear transformations}

Let's practice describing a few linear transformations with matrices. For
example, if we rotate all of space $90\degree$ counterclockwise then $\hat{i}$
lands on the coordinates $(0, 1)$ and $\hat{j}$ lands on the coordinates
$(-1, 0)$. So the matrix we end up with has the columns $(0, 1)$, $(-1, 0)$.

\begin{equation*}
  \begin{bmatrix}
    0 & -1 \\
    1 & 0
  \end{bmatrix}
\end{equation*}

To ascertain what happens to any vector after a $90\degree$ rotation, you could
just multiply its coordinates by this matrix.

\begin{equation*}
  \begin{bmatrix}
    0 & -1 \\
    1 & 0
  \end{bmatrix} \begin{bmatrix}
    x \\
    y
  \end{bmatrix}
\end{equation*}

Here's a fun transformation with a special name, called a ``shear". In it,
$\hat{i}$ remains fixed so the first column of the matrix is $(1, 0)$, but
$\hat{j}$ moves over to the coordinates $(1, 1)$ which become the second column
of the matrix.

\begin{equation*}
  \begin{bmatrix}
    1 & 1 \\
    0 & 1
  \end{bmatrix}
\end{equation*}

And, at the risk of being redundant here, figuring out how shear transforms a
given vector comes down to multiplying this matrix by that vector.

\begin{equation*}
  \begin{bmatrix}
    1 & 1 \\
    0 & 1
  \end{bmatrix} \begin{bmatrix}
    x \\
    y
  \end{bmatrix}
\end{equation*}

Let's say we want to go the other way around, starting with a matrix, say with
columns $(1, 2)$ and $(3, 1)$, and we want to deduce what its transformation
looks like. Pause and take a moment to see if you can imagine it.

\begin{equation*}
  \begin{bmatrix}
    1 & 3 \\
    2 & 1
  \end{bmatrix}
\end{equation*}

One way to do this is to first move $\hat{i}$ to $(1, 2)$. Then, move $\hat{j}$
to $(3, 1)$, always moving the rest of space in such a way that that keeps grid
lines parallel and evenly spaced.

Suppose that the vectors that $\hat{i}$ and $\hat{j}$ land on are linearly
dependent as in the following matrix (that is, it has linearly dependent
columns).

\begin{equation*}
  \begin{bmatrix}
    2 & -2 \\
    1 & -1
  \end{bmatrix}
\end{equation*}

If you recall from last section, this means that one vector is a scaled version
of the other, so that linear transformation compresses all of 2D space onto the
line where those two vectors sit. This is also known as the one-dimensional span
of those two linearly dependent vectors.

To sum up, linear transformations are a way to move around space such that the
grid lines remain parallel and evenly spaced and such that the origin remains
fixed. Delightfully, these transformations can be described using only a handful
of numbers: the coordinates of where each basis vector lands. Matrices give us a
language to describe these transformations where the columns represent those
coordinates and matrix-vector multiplication is just a way to compute what that
transformation does to a given vector. The important take-away here is that
every time you see a matrix, you can interpret it as a certain transformation of
space. Once you really digest this idea, you're in a great position to
understand linear algebra deeply. Almost all of the topics coming up, from
matrix multiplication to determinants, eigenvalues, etc. will become easier to
understand once you start thinking about matrices as transformations of space.

\begin{remark}
  See the corresponding \textit{Essence of Linear Algebra} video for a more
  visual presentation (11 minutes)
  \cite{bib:linalg_linear_transformations_and_matrices}.
\end{remark}

\section{Matrix multiplication as composition}
\index{Matrices!multiplication}

Often-times you find yourself wanting to describe the effect of applying one
transformation and then another. For example, you may want to describe what
happens when you first rotate the plane $90\degree$ counterclockwise then apply
a shear. The overall effect here, from start to finish, is another linear
transformation distinct from the rotation and the shear. This new linear
transformation is commonly called the ``composition" of the two separate
transformations we applied, and like any linear transformation, it can be
described with a matrix all its own by following $\hat{i}$ and $\hat{j}$. In
this example, the ultimate landing spot for $\hat{i}$ after both transformations
is $(1, 1)$, so that's the first column of the matrix. Likewise, $\hat{j}$
ultimately ends up at the location $(-1, 0)$, so we make that the second column
of the matrix.

\begin{equation*}
  \begin{bmatrix}
    1 & -1 \\
    1 & -0
  \end{bmatrix}
\end{equation*}

This new matrix captures the overall effect of applying a rotation then a shear
but as one single action rather than two successive ones.

Here's one way to think about that new matrix: if you were to feed some vector
through the rotation then the shear, the long way to compute where it ends up is
to, first, multiply it on the left by the rotation matrix; then, take whatever
you get and multiply that on the left by the shear matrix.

\begin{equation*}
  \begin{bmatrix}
    1 & 1 \\
    0 & 1
  \end{bmatrix}\left(
  \begin{bmatrix}
    0 & -1 \\
    1 & 0
  \end{bmatrix}
  \begin{bmatrix}
    x \\
    y
  \end{bmatrix}\right)
\end{equation*}

This is, numerically speaking, what it means to apply a rotation then a shear to
a given vector, but the result should be the same as just applying this new
composition matrix we found to that same vector. This applies to any vector
because this new matrix is supposed to capture the same overall effect as the
rotation-then-shear action.

\begin{equation*}
  \begin{bmatrix}
    1 & 1 \\
    0 & 1
  \end{bmatrix}\left(
  \begin{bmatrix}
    0 & -1 \\
    1 & 0
  \end{bmatrix}
  \begin{bmatrix}
    x \\
    y
  \end{bmatrix}\right) =
  \begin{bmatrix}
    1 & -1 \\
    1 & 0
  \end{bmatrix} \begin{bmatrix}
    x \\
    y
  \end{bmatrix}
\end{equation*}

Based on how things are written down here, it's reasonable to call this new
matrix the ``product" of the original two matrices.

\begin{equation*}
  \begin{bmatrix}
    1 & 1 \\
    0 & 1
  \end{bmatrix}
  \begin{bmatrix}
    0 & -1 \\
    1 & 0
  \end{bmatrix} =
  \begin{bmatrix}
    1 & -1 \\
    1 & 0
  \end{bmatrix}
\end{equation*}

We can think about how to compute that product more generally in just a moment,
but it's way too easy to get lost in the forest of numbers. Always remember
that multiplying two matrices like this has the geometric meaning of applying
one transformation then another.

One oddity here is that we are reading the transformations from right to left;
you first apply the transformation represented by the matrix on the right, then
you apply the transformation represented by the matrix on the left. This stems
from function notation, since we write functions on the left of variables, so
every time you compose two functions, you always have to read it right to left.

Let's look at another example. Take the matrix with columns $(1, 1)$ and
$(-2, 0)$.

\begin{equation*}
  M_1 = \begin{bmatrix}
    1 & -2 \\
    1 & 0
  \end{bmatrix}
\end{equation*}

Next, take the matrix with columns $(0, 1)$ and $(2, 0)$.

\begin{equation*}
  M_2 = \begin{bmatrix}
    0 & 2 \\
    1 & 0
  \end{bmatrix}
\end{equation*}

The total effect of applying $M_1$ then $M_2$ gives us a new transformation, so
let's find its matrix. First, we need to determine where $\hat{i}$ goes. After
applying $M_1$, the new coordinates of $\hat{i}$, by definition, are given by
that first column of $M_1$, namely, $(1, 1)$. To see what happens after applying
$M_2$, multiply the matrix for $M_2$ by that vector $(1, 1)$. Working it out the
way described in the last section, you'll get the vector $(2, 1)$.

\begin{equation*}
  \begin{bmatrix}
    0 & 2 \\
    1 & 0
  \end{bmatrix}
  \begin{bmatrix}
    1 \\
    1
  \end{bmatrix} = 1
  \begin{bmatrix}
    0 \\
    1
  \end{bmatrix} + 1
  \begin{bmatrix}
    2 \\
    0
  \end{bmatrix} =
  \begin{bmatrix}
    2 \\
    1
  \end{bmatrix}
\end{equation*}

This will be the first column of the composition matrix. Likewise, to follow
$\hat{j}$, the second column of $M_1$ tells us that it first lands on $(-2, 0)$.
Then, when we apply $M_2$ to that vector, you can work out the matrix-vector
product to get $(0, -2)$.

\begin{equation*}
  \begin{bmatrix}
    0 & 2 \\
    1 & 0
  \end{bmatrix}
  \begin{bmatrix}
    -2 \\
    0
  \end{bmatrix} = -2
  \begin{bmatrix}
    0 \\
    1
  \end{bmatrix} + 0
  \begin{bmatrix}
    2 \\
    0
  \end{bmatrix} =
  \begin{bmatrix}
    0 \\
    -2
  \end{bmatrix}
\end{equation*}

This will be the second column of the composition matrix.

\begin{equation*}
  \begin{bmatrix}
    0 & 2 \\
    1 & 0
  \end{bmatrix}
  \begin{bmatrix}
    1 & -2 \\
    1 & 0
  \end{bmatrix} =
  \begin{bmatrix}
    2 & 0 \\
    1 & -2
  \end{bmatrix}
\end{equation*}

\subsection{General matrix multiplication}

Let's go through that same process again, but this time, we'll use variable
entries in each matrix, just to show that the same line of reasoning works for
any matrices. This is more symbol-heavy, but it should be pretty satisfying for
anyone who has previously been taught matrix multiplication the more rote way.

\begin{equation*}
  \begin{bmatrix}
    a & b \\
    c & d
  \end{bmatrix}
  \begin{bmatrix}
    e & f \\
    g & h
  \end{bmatrix} =
  \begin{bmatrix}
    ? & ? \\
    ? & ?
  \end{bmatrix}
\end{equation*}

To follow where $\hat{i}$ goes, start by looking at the first column of the
matrix on the right, since this is where $\hat{i}$ initially lands. Multiplying
that column by the matrix on the left is how you can tell where the intermediate
version of $\hat{i}$ ends up after applying the second transformation.

\begin{equation*}
  \begin{bmatrix}
    a & b \\
    c & d
  \end{bmatrix}
  \begin{bmatrix}
    e \\
    g
  \end{bmatrix} = e
  \begin{bmatrix}
    a \\
    c
  \end{bmatrix} + g
  \begin{bmatrix}
    b \\
    d
  \end{bmatrix} =
  \begin{bmatrix}
    ae + bg \\
    ce + dg
  \end{bmatrix}
\end{equation*}

So the first column of the composition matrix will always equal the left matrix
times the first column of the right matrix. Likewise, $\hat{j}$ will always
initially land on the second column of the right matrix, so multiplying by this
second column will give its final location, and hence, that's the second column
of the composition matrix.

\begin{equation*}
  \begin{bmatrix}
    a & b \\
    c & d
  \end{bmatrix}
  \begin{bmatrix}
    f \\
    h
  \end{bmatrix} = f
  \begin{bmatrix}
    a \\
    c
  \end{bmatrix} + h
  \begin{bmatrix}
    b \\
    d
  \end{bmatrix} =
  \begin{bmatrix}
    af + bh \\
    cf + dh
  \end{bmatrix}
\end{equation*}

So the complete composition matrix is

\begin{equation*}
  \begin{bmatrix}
    a & b \\
    c & d
  \end{bmatrix}
  \begin{bmatrix}
    e & f \\
    g & h
  \end{bmatrix} =
  \begin{bmatrix}
    ae + bg & af + bh \\
    ce + dg & cf + dh
  \end{bmatrix}
\end{equation*}

Notice there's a lot of symbols here, and it's common to be taught this formula
as something to memorize along with a certain algorithmic process to help
remember it. Before memorizing that process, you should get in the habit of
thinking about what matrix multiplication really represents: applying one
transformation after another. This will give you a much better conceptual
framework that makes the properties of matrix multiplication much easier to
understand.

\subsection{Matrix multiplication associativity}

For example, here's a question: does it matter what order we put the two
matrices in when we multiply them? Let's think through a simple example. Take a
shear which fixes $\hat{i}$ and moves $\hat{j}$ over to the right, and a
$90\degree$ rotation. If you first do the shear then rotate, we can see that
$\hat{i}$ ends up at $(0, 1)$ and $\hat{j}$ ends up at $(-1, 1)$. Both are
generally pointing close together. If you first rotate then do the shear,
$\hat{i}$ ends up over at $(1, 1)$ and $\hat{j}$ is off on a different direction
at $(-1, 0)$ and they're pointing farther apart. The overall effect here is
clearly different, so evidently, order totally does matter. Notice by thinking
in terms of transformations, that's the kind of thing that you can do in your
head by visualizing. No matrix multiplication necessary.

Let's consider trying to prove that matrix multiplication is associative. This
means that if you have three matrices $A$, $B$, and $C$, and you multiply them
all together, it shouldn't matter if you first compute $A$ times $B$ then
multiply the result by $C$, or if you first multiply $B$ times $C$ then multiply
that result by $A$ on the left. In other words, it doesn't matter where you put
the parenthesis.

\begin{equation*}
  (AB)C \stackrel{?}{=} A(BC)
\end{equation*}

If you try to work through this numerically, it's horrible, and unenligthening
for that matter. However, when you think about matrix multiplication as applying
one transformation after another, this property is just trivial. Can you see
why? What it's saying is that if you first apply $C$ then $B$, then $A$, it's
the same as applying $C$, then $B$ then $A$. There's nothing to prove, you're
just applying the same three things one after the other all in the same order.
This might feel like cheating, but it's not. This is a valid proof that matrix
multiplication is associative, and even better than that, it's a good
explanation for why that property should be true.

\begin{remark}
  See the corresponding \textit{Essence of Linear Algebra} video for a more
  visual presentation (10 minutes)
  \cite{bib:linalg_matrix_multiplication_as_composition}.
\end{remark}

\section{The determinant}

So, moving forward, we will be assuming you have a visual understanding of
linear transformations and how they're represented with matrices.

\subsection{Scaling areas}

If you think about a couple linear transformations, you might notice how some of
them seem to stretch space out while others compress it. It's useful for
understanding these transformations to measure exactly how much it stretches or
compresses things (more specifically, to measure the factor by which areas are
scaled). For example, look at the matrix with columns $(3, 0)$, and $(0, 2)$.

\begin{equation*}
  \begin{bmatrix}
    3 & 0 \\
    0 & 2
  \end{bmatrix}
\end{equation*}

It scales $\hat{i}$ by a factor of $3$ and scales $\hat{j}$ by a factor of $2$.
Now, if we focus our attention on the one-by-one square whose bottom sits on
$\hat{i}$ and whose left side sits on $\hat{j}$, after the transformation, this
turns into a $2$ by $3$ rectangle. Since this region started out with area $1$
and ended up with area $6$, we can say the linear transformation has scaled its
area by a factor of $6$. Compare that to a shear whose matrix has columns
$(1, 0)$ and $(1, 1)$ meaning $\hat{i}$ stays in place and $\hat{j}$ moves over
to $(1, 1)$.

\begin{equation*}
  \begin{bmatrix}
    1 & 1 \\
    0 & 1
  \end{bmatrix}
\end{equation*}

That same unit square determined by $\hat{i}$ and $\hat{j}$ gets slanted and
turned into a parallelogram, but the area of that parallelogram is still $1$
since its base and height each continue to each have length $1$. Even though
this transformation pushes things about, it seems to leave areas unchanged (at
least in the case of that one uint square).

Actually though, if you know how much the area of that one single unit square
changes, you can know how the area of any possible region in space changes.
First off, notice that whatever happens to one square in the grid has to happen
to any other square in the grid no matter the size. This follows from the fact
that grid lines remain parallel and evenly spaced. Then, any shape that's not a
grid square can be approximated by grid squares pretty well with arbitrarily
good approximations if you use small enough grid squares. So, since the areas of
all those tiny grid squares are being scaled by some single amount, the area of
the shape as a whole will also be scaled by that same single amount.

\subsection{Exploring the determinant}

This special scaling factor, the factor by which a linear transformation changes
any area, is called the \textit{determinant} of that transformation. We'll show
how to compute the determinant of a transformation using its matrix later on,
but understanding what it represents is much more important than the
computation. For example, the determinant of a transformation would be $3$ if
that transformation increases the area of the region by a factor of $3$.

\begin{equation*}
  det\left(\begin{bmatrix}
    0 & 2 \\
    -1.5 & 1
  \end{bmatrix}\right) = 3
\end{equation*}

The determinant of a matrix is commonly denoted by vertical bars instead of
square brackets.

\begin{equation*}
  \begin{vmatrix}
    0 & 2 \\
    -1.5 & 1
  \end{vmatrix} = 3
\end{equation*}

The determinant of a transformation would be $\frac{1}{2}$ if it compresses all
areas by a factor of $\frac{1}{2}$.

\begin{equation*}
  \begin{vmatrix}
    0.5 & 0.5 \\
    -0.5 & 0.5
  \end{vmatrix} = 0.5
\end{equation*}

The determinant of a 2D transformation is zero if it compresses all of space
onto a line or even onto a single point since then, the area of any region would
become zero.

\begin{equation*}
  \begin{vmatrix}
    4 & 2 \\
    2 & 1
  \end{vmatrix} = 0
\end{equation*}

That last example proved to be pretty important. It means checking if the
determinant of a given matrix is zero will give a way of computing whether the
transformation associated with that matrix compresses everything into a smaller
dimension.

This analogy so far isn't quite right. The full concept of a determinant allows
for negative values.

\begin{equation*}
  \begin{vmatrix}
    1 & 2 \\
    3 & 4
  \end{vmatrix} = -2
\end{equation*}

What would scaling an area by a negative amount even mean? This has to do with
the idea of orientation. A 2D transformation with a negative determinant
essentially flips space over. Any transformations that do this are said to
"invert the orientation of space". Another way to think about it is in terms of
$\hat{i}$ and $\hat{j}$. In their starting positions, $\hat{j}$ is to the left
of $\hat{i}$. If, after a transformation, $\hat{j}$ is now on the right side of
$\hat{i}$, the orientation of space has been inverted. Whenever this happens,
the determinant will be negative. The absolute value of the determinant still
tells you the factor by which areas have been scaled.

For example, the matrix with columns $(1, 1)$ and $(2, -1)$ encodes a
transformation that has determinant $-3$.

\begin{equation*}
  \begin{vmatrix}
    1 & 2 \\
    1 & -1
  \end{vmatrix} = -3
\end{equation*}

This means that space gets flipped over and areas are scaled by a factor of $3$.

Why would this idea of a negative area scaling factor be a natural way to
describe orientation-flipping? Think about the series of transformations you get
by slowly letting $\hat{i}$ rotate closer and closer to $\hat{j}$. As $\hat{i}$
gets closer, all the areas in space are getting compressed more and more meaning
the determinant approaches zero. Once $\hat{i}$ lines up perfectly with
$\hat{j}$, the determinant is zero. Then, if $\hat{i}$ continues, doesn't it
feel natural for the determinant to keep decreasing into negative numbers?

\subsection{The determinant in 3D}

That's the understanding of determinants in two dimensions. What should it mean
for three dimensions? The determinant of a $3 \times 3$ matrix tells you how
much volumes get scaled. A determinant of zero would mean that all of space is
compressed onto something with zero volume meaning either a flat plane, a line,
or in the most extreme case, a single point. This means that the columns of the
matrix are linearly dependent.

What should negative determinants mean for three dimensions? One way to describe
orientation in 3D is with the right-hand rule. Point the forefinger of your
right hand in the direction if $\hat{i}$, stick out your middle finger in the
direction of $\hat{j}$, and notice how when you point your thumb up, it is in
the direction of $\hat{k}$. If you can still do that after the transformation,
orientation has not changed and the determinant is positive. Otherwise, if after
the transformation it only makes sense to do that with your left hand,
orientation has been flipped and the determinant is negative.

\subsection{Computing the determinant}

How do you actually compute the determinant? For a $2 \times 2$ matrix with
entries $a$, $b$, $c$, $d$, the formula is as follows.

\begin{equation*}
  \begin{vmatrix}
    a & b \\
    c & d
  \end{vmatrix} = ad - bc
\end{equation*}

Here's part of an intution for where this formula comes from. Let's say that the
terms $b$ and $c$ were both zero. Then, the term $a$ tells you how much
$\hat{i}$ is stretched in the x-direction and the term $d$ tells you how much
$\hat{j}$ is stretched in the y-direction. Since those other terms are zero, it
should make sense that $ad$ gives the area of the rectangle that the unit square
turns into. Even if only one of $b$ or $c$ are zero, you'll have a parallelogram
with a base of $a$ and a height $d$, so the area should still be $ad$. Loosely
speaking, if both $b$ and $c$ are nonzero, then that $bc$ term tells you how
much this parallelogram is stretched or compressed in the diagonal direction.

If you feel like computing determinants by hand is something that you need to
know (you won't for this book), the only way to get it down is to just practice
it with a few. This is all triply true for 3D determinants. There is a formula,
and if you feel like that's something you need to know, you should practice with
a few matrices.

\begin{equation*}
  \begin{vmatrix}
    a & b & c \\
    d & e & f \\
    g & h & i
  \end{vmatrix} =
  a \begin{vmatrix}
    e & f \\
    h & i
  \end{vmatrix}
  - b \begin{vmatrix}
    d & f \\
    g & i
  \end{vmatrix}
  + c \begin{vmatrix}
    d & e \\
    g & h
  \end{vmatrix}
\end{equation*}

We don't think those computations fall within the essence of linear algebra, but
understanding what the determinant represents falls within that essence.

\begin{remark}
  See the corresponding \textit{Essence of Linear Algebra} video for a more
  visual presentation (10 minutes) \cite{bib:linalg_the_determinant}.
\end{remark}

\section{Inverse matrices, column space, and null space}

As you can probably tell by now, the bulk of this chapter is on understanding
matrix and vector operations through that more visual lens of linear
transformations. This section is no exception, describing the concepts of
inverse matrices, columns space, rank, and null space through that lens. A fair
warning though: we're not going to talk about the methods for actually computing
these things, and some would argue that that's pretty important. There are a lot
of very good resources for learning those methods outside of this chapter.
Keywords: "Gaussian elimination" and "row echelon form". Most of the value that
we actually have to add here is on the intuition half. Plus, in practice, we
usually use software to compute these things for us anyway.

\subsection{Linear systems of equations}

First, a few words on the usefulness of linear algebra. By now, you already have
a hint for how it's used in describing the manipulation of space, which is
useful for computer graphics and robotics. However, one of the main reasons that
linear algebra is more broadly applicable, and required for just about any
technical discipline, is that it lets us solve certain systems of equations.
When we say "system of equations", we mean there is a list of variables, things
you don't know, and a list of equations relating them. For example,

\begin{align*}
  6x - 3y + 2z &= 7 \\
  x + 2y + 5z &= 0 \\
  2x - 8y - z &= -2
\end{align*}

is a system of equations with the unknowns $x$, $y$, and $z$.

In a lot of situations, those equations can get very complicated, but, if you're
lucky, they might take on a certain special form. Within each equation, the only
thing happening to each variable is that it's scaled by some constant, and the
only thing happening to each of those scaled variables is that they're added to
each other, so no exponents or fancy functions, or multiplying two variables
together.

The typical way to organize this sort of special system of equations is to throw
all the variables on the left and put any lingering constants on the right. It's
also nice to vertically line up the common variables, and to do that, you might
need to throw in some zero coefficients whenever the variable doesn't show up in
one of the equations.

\begin{align*}
  2x + 5y + 3z &= -3 \\
  4x + 0y + 8z &= 0 \\
  1x + 3y + 0z &= 2
\end{align*}

This is called a "linear system of equations". You might notice that this looks
a lot like matrix-vector multiplication. In fact, you can package all of the
equations together into a single vector equation, where you have the matrix
containing all the constant coefficients, a vector containing all the constant
coefficients, and a vector containing all the variables. Their matrix-vector
product equals some different constant vector.

\begin{equation*}
  \begin{bmatrix}
    2 & 5 & 3 \\
    4 & 0 & 8 \\
    1 & 3 & 0
  \end{bmatrix}
  \begin{bmatrix}
    x \\
    y \\
    z
  \end{bmatrix} =
  \begin{bmatrix}
    -3 \\
    0 \\
    2
  \end{bmatrix}
\end{equation*}

Let's name that constant matrix $\mtx{A}$, denote the vector holding the
variables with $\mtx{x}$, and call the constant vector on the right-hand side
$\mtx{v}$. This is more than just a notational trick to get our system of
equations written on one line. It sheds light on a pretty cool geometric
interpretation for the problem.

\begin{equation*}
  \mtx{A}\mtx{x} = \mtx{v}
\end{equation*}

The matrix $\mtx{A}$ corresponds with some linear transformation, so solving
$\mtx{A}\mtx{x} = \mtx{v}$ means we're looking for a vector $\mtx{x}$ which,
after applying the transformation $\mtx{A}$, lands on $\mtx{v}$.

Think about what's happening here for a moment. You can hold in your head this
really complicated idea of multiple variables all intermingling with each other
just by thinking about compressing or morphing space and trying to determine
which vector lands on another.

To start simple, let's say you have a system with two equations and two
unknowns. This means the matrix $\mtx{A}$ is a $2 \times 2$ matrix, and
$\mtx{v}$ and $\mtx{x}$ are each two-dimensional vectors.

\begin{align*}
  2x + 2y &= -4 \\
  1x + 3y &= -1 \\
  \begin{bmatrix}
    2 & 2 \\
    1 & 3
  \end{bmatrix}
  \begin{bmatrix}
    x \\
    y
  \end{bmatrix} &=
  \begin{bmatrix}
    -4 \\
    -1
  \end{bmatrix}
\end{align*}

\subsection{Inverse}

How we think about the solutions to this equation depends on whether the
transformation associated with $\mtx{A}$ compresses all of space into a lower
dimension, like a line or a point, or if it leaves everything spanning the full
two dimensions where it started. In the language of the last section, we
subdivide into the case where $\mtx{A}$ has zero determinant and the case where
$\mtx{A}$ has nonzero determinant.

Let's start with the most likely case where the determinant is nonzero, meaning
space does not get compressed into a zero area region. In this case, there will
always be one and only one vector that lands on $\mtx{v}$, and you can find it
by playing the transformation in reverse. Following where $\mtx{v}$ goes as we
undo the transformation, you'll find the vector $\mtx{x}$ such that $\mtx{A}$
times $\mtx{x}$ equals $\mtx{v}$.

When you play the transformation in reverse, it actually corresponds to a
separate linear transformation, commonly called "the inverse of $\mtx{A}$"
denoted $\mtx{A}^{-1}$.

\begin{equation*}
  \mtx{A}^{-1} =
  \begin{bmatrix}
    3 & 1 \\
    0 & 2
  \end{bmatrix}^{-1}
\end{equation*}

For example, if $\mtx{A}$ was a counterclockwise rotation by $90 \degree$, then
the inverse of $\mtx{A}$ would be a clockwise rotation by $90 \degree$.

\begin{equation*}
  \begin{array}{cc}
  \mtx{A} =
  \begin{bmatrix}
    0 & -1 \\
    1 & 0
  \end{bmatrix} &
  \mtx{A}^{-1} =
  \begin{bmatrix}
    0 & 1 \\
    -1 & 0
  \end{bmatrix}
  \end{array}
\end{equation*}

If $\mtx{A}$ was a rightward shear that pushes $\hat{j}$ one unit to the right,
the inverse of $\mtx{A}$ would be a leftward shear that pushes $\hat{j}$ one
unit to the left.

In general, $\mtx{A}^{-1}$ is the unique transformation with the property that
if you first apply $\mtx{A}$, then follow it with the transformation
$\mtx{A}^{-1}$, you end up back where you started. Applying one transformation
after another is captured algebraically with matrix multiplication, so the core
property of this transformation $\mtx{A}^{-1}$ is that $\mtx{A}^{-1}\mtx{A}$
equals the matrix that corresponds to doing nothing.

The transformation that does nothing is called the "identity transformation". It
leaves $\hat{i}$ and $\hat{j}$ each where they are, unmoved, so its columns are
$(1, 0)$ and $(0, 1)$.

\begin{equation*}
  \mtx{A}^{-1}\mtx{A} =
  \begin{bmatrix}
    1 & 0 \\
    0 & 1
  \end{bmatrix}
\end{equation*}

Once you find this inverse, which in practice, you do with a computer, you can
solve your equation by multipling this inverse matrix by $\mtx{v}$.

\begin{align*}
  \mtx{A}\mtx{x} &= \mtx{v} \\
  \mtx{A}^{-1}\mtx{A}\mtx{x} &= \mtx{A}^{-1}\mtx{v} \\
  \mtx{x} &= \mtx{A}^{-1}\mtx{v}
\end{align*}

Again, what this means geometrically is that you're playing the transformation
in reverse and following $\mtx{v}$. This nonzero determinant case, which for a
random choice of matrix is by far the most likely one, corresponds with the idea
that if you have two unknowns and two equations, it's almost certainly the case
that there's a single, unique solution.

This idea also makes sense in higher dimensions when the number of equations
equals the number of unknowns. Again, the system of equations can be translated
to the geometric interpretation where you have some transformation, $\mtx{A}$,
some vector $\mtx{v}$, and you're looking for the vector $\mtx{x}$ that lands on
$\mtx{v}$. As long as the transformation $\mtx{A}$ doesn't compress all of
space into a lower dimension, meaning, its determinant is nonzero, there will be
an inverse transformation, $\mtx{A}^{-1}$ , with the property that if you first
do $\mtx{A}$, then you do $\mtx{A}^{-1}$, it's the same as doing nothing. To
solve your equation, you just have to multiply that reverse transformation
matrix by the vector $\mtx{v}$.

When the determinant is zero and the transformation associated with this system
of equations compresses space into a smaller dimension, there is no inverse. You
cannot uncompress a line to turn it into a plane. At least, that's not something
that a function can do. That would require transforming each individual vector
into a whole line full of vectors, but functions can only take a single input to
a single output.

Similarly, for three equations and three unknowns, there will be no inverse if
the corresponding transformation compresses 3D space onto the plane, or even if
it compresses it onto a line, or a point. Those all correspond to a determinant
of zero since any region is compressed into something with zero volume.

It's still possible that a solution exists even when there is no inverse. It's
just that when your transformation compresses space onto, say, a line, you have
to be lucky enough that the vector $\mtx{v}$ exists somewhere on that line.

\subsection{Rank and column space}

You might notice that some of these zero determinant cases feel a lot more
restrictive than others. Given a $3 \times 3$ matrix, for example, it seems a
lot harder for a solution to exist when it compresses space onto a line compared
to when it compresses space onto a plane even though both of those have zero
determinant. We have some language that's more specific than just saying "zero
determinant". When the output of a transformation is a line, meaning it's
one-dimensional, we say the transformation has a \textit{rank} of one. If all
the vectors land on some two-dimensional plane, we say the transformation has a
rank of two. The word "rank" means the number of dimensions in the output of a
transformation.

For instance, in the case of $2 \times 2$ matrices, the highest possible rank is
$2$. It means the basis vectors continue to span the full two dimensions of
space, and the determinant is nonzero. For $3 \times 3$ matrices, rank $2$ means
that we've collapsed, but not as much as we would have collapsed for a rank 1
situation. If a 3D transformation has a nonzero determinant and its output fills
all of 3D space, it has a rank of $3$.

This set of all possible outputs for your matrix, whether it's a line, a plane,
3D space, whatever, is called the \textit{column space} of your matrix. You can
probably guess where that name comes from. The columns of your matrix tell you
where the basis vectors land, and the span of those transformed basis vectors
gives you all possible outputs. In other words, the column space is the span of
the columns of your matrix, so a more precise definition of rank would be that
it's the number of dimensions in the column space. When this rank is as high as
it can be, meaning it equals the number of columns, we call the matrix "full
rank".

\subsection{Null space}

Notice, the zero vector will always be included in the column space since linear
transformations must keep the origin fixed in place. For a full rank
transformation, the only vector that lands at the origin is the zero vector
itself, but for matrices that aren't full rank, which compress to a smaller
dimension, you can have a whole bunch of vectors that land on zero. If a 2D
transformation compresses space onto a line, for example, there is a separate
line in a different direction full of vectors that get compressed onto the
origin. If a 3D transformation compresses space onto a plane, there's also a
full line of vectors that land on the origin. If a 3D transformation compresses
all the space onto a line, then there's a whole plane full of vectors that land
on the origin.

This set of vectors that lands on the origin is called the \textit{null space}
or the \textit{kernel} of your matrix. It's the space of all vectors that become
null in the sense that they land on the zero vector. In terms of the linear
system of equations $\mtx{A}\mtx{x} = \mtx{v}$, when $\mtx{v}$ happens to be the
zero vector, the null space gives you all the possible solutions to the
equation.

\subsection{Closing remarks}

That's a high-level overview of how to think about linear systems of equations
geometrically. Each system has some kind of linear transformation associated
with it, and when that transformation has an inverse, you can use that inverse
to solve your system. Otherwise, the idea of column space lets us understand
when a solution even exists, and the idea of a null space helps us understand
what the set of all possible solutions can look like.

Again, there's a lot not covered here, most notably how to compute these things.
We also had to limit the scope to examples where the number of equations equals
the number of unknowns. The goal here is not to try to teach everything: it's
that you come away with a strong intuition for inverse matrices, column space,
and null space, and that those intuitions make any future learning that you do
more fruitful.

\begin{remark}
  See the corresponding \textit{Essence of Linear Algebra} video for a more
  visual presentation (12 minutes)
  \cite{bib:linalg_inverse_matrices_column_space_and_null_space}.
\end{remark}

\section{Nonsquare matrices as transformations between dimensions}

When we've talked about linear transformations so far, we've only really talked
about transformations from 2D vectors to other 2D vectors, represented with
$2 \times 2$ matrices; or from 3D vectors to other 3D vectors, represented with
$3 \times 3$ matrices. What about nonsquare matrices? We'll take a moment to
discuss what those mean geometrically.

By now, you have most of the background you need to start pondering a question
like this on your own, but we'll start talking through it, just to give a little
mental momentum.

It's perfectly reasonable to talk about transformations between dimensions, such
as one that takes 2D vectors to 3D vectors. Again, what makes one of these
linear is that grid lines remain parallel and evenly spaced, and that the origin
maps to the origin.

Encoding one of these transformations with a matrix the same as what we've done
before. You look at where each basis vector lands and write the coordinates of
the landing spots as the coordinates of the landing spots as the columns of a
matrix. For example, the following is a transformation that takes $\hat{i}$ to
the coordinates $(2, -1, -2)$ and $\hat{j}$ to the coordinates $(0, 1, 1)$.

\begin{equation*}
  \begin{bmatrix}
    2 & 0 \\
    -1 & 1 \\
    -2 & 1
  \end{bmatrix}
\end{equation*}

Notice, this means the matrix encoding our transformation has three rows and two
columns, which, to use standard terminology, makes it a $3 \times 2$ matrix. In
the language of last section, the column space of this matrix, the place where
all the vectors land, is a 2D plane slicing through the origin of 3D space. The
matrix is still full rank since the number of dimensions in this column space is
the same as the number of dimensions of the input space.

If you see a $3 \times 2$ matrix out in the wild, you can know that it has the
geometric interpretation of mapping two dimensions to three dimensions since the
two columns indicate that the input space has two basis vectors, and the three
rows indicate that the landing spots for each of those basis vectors is
described with three separate coordinates.

For a $2 \times 3$ matrix, the three columns indicate a starting space that has
three basis vectors, so it starts in three dimensions; and the two rows indicate
that the landing spot for each of those three basis vectors is described with
only two coordinates, so they must be landing in two dimensions. It's a
transformation from 3D space onto the 2D plane.

You could also have a transformation from two dimensions to one dimension.
One-dimensional space is really just the number line, so a transformation like
this takes in 2D vectors and returns numbers. Thinking about gridlines remaining
parallel and evenly spaced is messy due to all the compression happening here,
so in this case, the visual understanding for what linearity means is that if
you have a line of evenly spaced dots, it would remain evenly spaced once
they're mapped onto the number line.

One of these transformations is encoded with a $1 \times 2$ matrix, each of
whose two columns has just a single entry. The two columns represent where the
basis vectors land, and each one of those columns requires just one number, the
number that that basis vector landed on.

\begin{remark}
  See the corresponding \textit{Essence of linear algebra} video for a more
  visual presentation (4 minutes)
  \cite{bib:linalg_nonsquare_matrices_as_transformations_between_dimensions}.
\end{remark}

\section{Eigenvectors and eigenvalues}

\subsection{What is an eigenvector?}

To start, consider some linear transformation in two dimensions that moves the
basis vector $\hat{i}$ to the coordinates $(3, 0)$ and $\hat{j}$ to $(1, 2)$, so
it's represented with a matrix whose columns are $(3, 0)$ and $(1, 2)$.

\begin{equation*}
  \begin{bmatrix}
    3 & 1 \\
    0 & 2
  \end{bmatrix}
\end{equation*}

Focus in on what it does to one particular vector and think about the span of
that vector, the line passing through its origin and its tip. Most vectors are
going to get knocked off their span during the transformation, but some special
vectors do remain on their own span meaning the effect that the matrix has on
such a vector is just to stretch it or compress it like a scalar.

For this specific example, the basis vector $\hat{i}$ is one such special
vector. The span of $\hat{i}$ is the x-axis, and from the first column of the
matrix, we can see that $\hat{i}$ moves over to three times itself still on that
x-axis. What's more, due to the way linear transformations work, any other
vector on the x-axis is also just stretched by a factor of $3$, and hence,
remains on its own span.

A slightly sneakier vector that remains on its own span during this
transformation is $(-1, 1)$. It ends up getting stretched by a factor of $2$.
Again, linearity is going to imply that any other vector on the diagonal line
spanned by this vector is just going to get stretched out by a factor of $2$.

For this transformation, those are all the vectors with this special property of
staying on their span. Those on the x-axis get stretched out by a factor of $3$
and those on the diagonal line get stretched out by a factor of $2$. Any other
vector is going to get rotated somewhat during the transformation and knocked
off the line that it spans. As you might have guessed by now, these special
vectors are called the \textit{eigenvectors} of the transformation, and each
eigenvector has associated with it an \textit{eigenvalue}, which is just the
factor by which it's stretched or compressed during the transformation.

Of course, there's nothing special about stretching vs compressing or the fact
that these eigenvalues happen to be positive. In another example, you could have
an eigenvector with eigenvalue $-\frac{1}{2}$, meaning that the vector gets
flipped and compressed by a factor of $\frac{1}{2}$.

\begin{equation*}
  \begin{bmatrix}
    0.5 & -1 \\
    -1 & 0.5
  \end{bmatrix}
\end{equation*}

The important part here is that it stays on the line that it spans out without
getting rotated off of it.

\subsection{Eigenvectors in 3D rotation}

For a glimpse of why this might be a useful thing to think about, consider some
three-dimensional rotation. If you can find an eigenvector for that rotation, a
vector that remains on its own span, you have found the axis of rotation. It's
much easier to think about a 3D rotation in terms of some axis of rotation and
an angle by which it's rotating rather than thinking about the full $3 \times 3$
matrix associated with that transformation. In this case, by the way, the
corresponding eigenvalue would have to be $1$ since rotations never stretch or
compress anything, so the length of the vector would remain the same.

\subsection{Finding eigenvalues}
\index{Matrices!eigenvalues}

The following pattern shows up a lot in linear algebra. With any linear
transformation described by a matrix, you could understand what it's doing by
reading off the columns of this matrix as the landing spots for basis vectors,
but often a better way to get at the heart of what the linear transformation
actually does, less dependent on your particular coordinate system, is to find
the eigenvectors and eigenvalues.

I won't cover the full details on methods for computing eigenvectors and
eigenvalues here, but I'll try to give an overview of the computational ideas
that are most important for a conceptual understanding. Symbolically, an
eigenvector look like the following

\begin{equation*}
  \mtx{A}\mtx{v} = \lambda \mtx{v}
\end{equation*}

$\mtx{A}$ is the matrix representing some transformation, $\mtx{v}$ is the
eigenvector, and $\lambda$ is a number, namely the corresponding eigenvalue.
This expression is saying that the matrix-vector product $\mtx{A}\mtx{v}$ gives
the same result as just scaling the eigenvector $\mtx{v}$ by some value
$\lambda$. Finding the eigenvectors and their eigenvalues of the matrix
$\mtx{A}$ involves finding the values of $\mtx{v}$ and $\lambda$ that make this
expression true. It's awkward to work with at first because that left-hand side
represents matrix-vector multiplication, but the right-hand side is
scalar-vector multiplication. Let's rewrite the right-hand side as some kind of
matrix-vector multiplication using a matrix which has the effect of scaling any
vector by a factor of $\lambda$. The columns of such a matrix will represent
what happens to each basis vector, and each basis vector is simply multiplied by
$\lambda$, so this matrix will have the number $\lambda$ down the diagonal and
zeroes everywhere else.

\begin{equation*}
  \begin{bmatrix}
    \lambda & 0 & 0 \\
    0 & \lambda & 0 \\
    0 & 0 & \lambda \\
  \end{bmatrix}
\end{equation*}

The common way to write this is to factor out $\lambda$ and write it as
$\lambda\mtx{I}$ where $\mtx{I}$ is the identity matrix with ones down the
diagonal.

\begin{equation*}
  \mtx{A}\mtx{v} = (\lambda\mtx{I}) \mtx{v}
\end{equation*}

With both sides looking like matrix-vector multiplication, we can subtract off
that right-hand side and factor out $\mtx{v}$.

\begin{align*}
  \mtx{A}\mtx{v} - (\lambda\mtx{I}) \mtx{v} &= \mtx{0} \\
  (\mtx{A} - \lambda\mtx{I})\mtx{v} &= \mtx{0}
\end{align*}

We now have a new matrix $\mtx{A} - \lambda\mtx{I}$, and we're looking for a
vector $\mtx{v}$ such that this new matrix times $\mtx{v}$ gives the zero
vector. This will always be true if $\mtx{v}$ itself is the zero vector, but
that's boring. We want a nonzero eigenvector. The only way it's possible for the
product of a matrix with a nonzero vector to become zero is if the
transformation associated with that matrix compresses space into a lower
dimension. That compression corresponds to a zero determinant for the matrix.

\begin{equation*}
  det(\mtx{A} - \lambda\mtx{I}) = 0
\end{equation*}

To be concrete, let's say your matrix $\mtx{A}$ has columns $(2, 1)$ and
$(2, 3)$, and think about subtracting off a variable amount $\lambda$.

\begin{equation*}
  det\left(\begin{bmatrix}
    2 - \lambda & 2 \\
    1 & 3 - \lambda
  \end{bmatrix}\right) = 0
\end{equation*}

The goal is to find a value of $\lambda$ that will make this determinant zero
meaning the tweaked transformation compresses space into a lower dimension. In
this case, that value is $\lambda = 1$. Of course, if we had chosen some other
matrix, the eigenvalue might not necessarily be $1$.

This is kind of a lot, but let's unravel what this is saying. When
$\lambda = 1$, the matrix $\mtx{A} - \lambda\mtx{I}$ compresses space onto a
line. That means there's a nonzero vector $\mtx{v}$ such that
$(\mtx{A} - \lambda\mtx{I})\mtx{v}$ equals the zero vector.

\begin{equation*}
  (\mtx{A} - \lambda\mtx{I})\mtx{v} = \mtx{0}
\end{equation*}

Remember, we care about that because it means
$\mtx{A}\mtx{v} = \lambda\mtx{v}$, which you can read off as saying that the
vector $\mtx{v}$ is an eigenvector of $\mtx{A}$ staying on its own span during
the transformation $\mtx{A}$. For the following example

\begin{equation*}
  \begin{bmatrix}
    2 & 2 \\
    1 & 3
  \end{bmatrix}
  \mtx{v} = 1\mtx{v}
\end{equation*}

the corresponding eigenvalue is $1$, so $\mtx{v}$ would actually just stay fixed
in place.

To summarize our line of reasoning:

\begin{align*}
  \mtx{A}\mtx{v} &= \lambda\mtx{v} \\
  \mtx{A}\mtx{v} - \lambda\mtx{I}\mtx{v} &= \mtx{0} \\
  (\mtx{A} - \lambda\mtx{I})\mtx{v} &= \mtx{0} \\
  det(\mtx{A} - \lambda\mtx{I}) &= \mtx{0}
\end{align*}

To see this in action, let's visit the example from the start with a matrix
whose columns are $(3, 0)$ and $(1, 2)$. To determine if a value $\lambda$ is an
eigenvalue, subtract it from the diagonals of this matrix and compute the
determinant.

\begin{align*}
  \begin{bmatrix}
    3 & 1 \\
    0 & 2
  \end{bmatrix} & \\
  \begin{bmatrix}
    3 - \lambda & 1 \\
    0 & 2 - \lambda
  \end{bmatrix} & \\
  det\left(\begin{bmatrix}
    3 - \lambda & 1 \\
    0 & 2 - \lambda
  \end{bmatrix}\right) &=
  (3 - \lambda)(2 - \lambda) - 1 \cdot 0 \\
  &= (3 - \lambda)(2 - \lambda)
\end{align*}

We get a certain quadratic polynomial in $\lambda$. Since $\lambda$ can only be
an eigenvalue if this determinant happens to be zero, you can conclude that the
only possible eigenvalues are $\lambda = 2$ and $\lambda = 3$.

To determine what the eigenvectors are that actually have one of these
eigenvalues, say $\lambda = 2$, plug in that value of $\lambda$ to the matrix
and then solve for which vectors this diagonally altered matrix sends to zero.

\begin{equation*}
  \begin{bmatrix}
    3 - 2 & 1 \\
    0 & 2 - 2
  \end{bmatrix}
  \begin{bmatrix}
    x \\
    y
  \end{bmatrix} =
  \begin{bmatrix}
    0 \\
    0
  \end{bmatrix}
\end{equation*}

If you computed this the way you would any other linear system, you'd see that
the solutions are all the vectors on the diagonal line spanned by $(-1, 1)$.
This corresponds to the fact that the unaltered matrix has the effect of
stretching all those vectors by a factor of $2$.

\subsection{Transformations with no eigenvectors}

A 2D transformation doesn't have to have eigenvectors. For example, consider a
rotation by $90 \degree$.

\begin{equation*}
  \begin{bmatrix}
    0 & -1 \\
    1 & 0
  \end{bmatrix}
\end{equation*}

This doesn't have any eigenvectors since it rotates every vector off its own
span. If you actually tried computing the eigenvalues of a rotation like this,
notice what happens.

\begin{align*}
  \begin{bmatrix}
    0 & -1 \\
    1 & 0
  \end{bmatrix} & \\
  \begin{bmatrix}
    - \lambda & -1 \\
    1 & -\lambda
  \end{bmatrix} & \\
  det\left(\begin{bmatrix}
    -\lambda & -1 \\
    1 & -\lambda
  \end{bmatrix}\right) &=
  (-\lambda)(-\lambda) - (-1)(1) \\
  &= \lambda^2 + 1 = 0
\end{align*}

The only roots of that polynomial are the imaginary numbers $i$ and $-i$. The
fact that there are no real number solutions indicates that there are no
eigenvectors.

\begin{remark}
  Interestingly though, the fact that multiplication by $i$ in the complex plane
  looks like a $90 \degree$ rotation is related to the fact that $i$ is an
  eigenvalue of this transformation of 2D real vectors. The specifics of this
  are out of scope, but note that eigenvalues which are complex numbers
  generally correspond to some kind of rotation in the transformation.
\end{remark}

\subsection{Repeated eigenvalues}

Another interesting example is a shear which fixes $\hat{i}$ in place and moves
$\hat{j}$ over by $1$.

\begin{equation*}
  \begin{bmatrix}
    1 & 1 \\
    0 & 1
  \end{bmatrix}
\end{equation*}

All the vectors on the x-axis are eigenvectors with eigenvalue $1$. In fact,
these are the only eigenvectors.

\begin{align*}
  det\left(\begin{bmatrix}
    1 - \lambda & 1 \\
    0 & 1 - \lambda
  \end{bmatrix}\right) &= (1 - \lambda)(1 - \lambda) = 0 \\
  (1 - \lambda)^2 &= 0
\end{align*}

The only root of this expression is $\lambda = 1$.

\subsection{Transformations with larger eigenvector spans}

Keep in mind it's also possible to have just one eigenvalue but with more than
just a line full of eigenvectors. A simple example is a matrix that scales
everything by $2$.

\begin{equation*}
  \begin{bmatrix}
    2 & 0 \\
    0 & 2
  \end{bmatrix}
\end{equation*}

The only eigenvalue is $2$, but every vector in the plane gets to be an
eigenvector with that eigenvalue.

\begin{remark}
  See the corresponding \textit{Essence of linear algebra} video for a more
  visual presentation (17 minutes)
  \cite{bib:linalg_eigenvectors_and_eigenvalues}.
\end{remark}

\section{Miscellaneous notation}

This book works with two-dimensional matrices in the sense that they only have
rows and columns. The dimensionality of these matrices is specified by row
first, then column. For example, a matrix with two rows and three columns would
be a two-by-three matrix. A square matrix has the same number of rows as
columns. Matrices commonly use capital letters while vectors use lowercase
letters.

The matrix $\mtx{I}$ is known as the identity matrix, which is a square matrix
with ones along its diagonal and zeroes elsewhere. For example

\begin{equation*}
  \begin{bmatrix}
    1 & 0 & 0 \\
    0 & 1 & 0 \\
    0 & 0 & 1
  \end{bmatrix}
\end{equation*}

The matrix denoted by $\mtx{0}_{m \times n}$ is a matrix filled with zeroes with
$m$ rows and $n$ columns.

The $^T$ in $\mtx{A}^T$ denotes transpose, which flips the matrix across its
diagonal such that the rows become columns and vice versa.

The $^\dagger$ in $\mtx{B}^\dagger$ denotes the Moore-Penrose pseudoinverse
given by $\mtx{B}^\dagger = (\mtx{B}^T\mtx{B})^{-1}\mtx{B}^T$. The pseudoinverse
is used when the matrix is nonsquare and thus not invertible to produce a close
approximation of an inverse.

