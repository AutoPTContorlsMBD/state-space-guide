\section{Nonlinear observers}

In this book, we have covered the Kalman filter, which is the optimal unbiased
estimator for linear \glspl{system}. It isn't optimal for nonlinear
\glspl{system}, but several extensions to it have been developed to make it more
accurate.

\subsection{Extended Kalman filter}
\index{Nonlinear control!extended Kalman filter}
\index{State-space observers!Kalman filter!extended Kalman filter}

This method \glslink{linearization}{linearizes} the matrices used during the
prediction step. In addition to the $\mtx{A}$, $\mtx{B}$, $\mtx{C}$, and
$\mtx{D}$ matrices above, the process noise intensity vector $\mtx{\Gamma}$ is
\glslink{linearization}{linearized} as follows:

\begin{equation*}
  \mtx{\Gamma} = \frac{\partial f(\mtx{x}, \mtx{0}, \mtx{0})}{\partial \mtx{w}}
\end{equation*}

where $\mtx{w}$ is the process noise included in the stochastic model.

From there, the continuous Kalman filter equations are used like normal to
compute the error covariance matrix $\mtx{P}$ and Kalman gain matrix. The
\gls{state} estimate update can still use the function $h(\mtx{x})$ for
accuracy.

\begin{equation*}
  \hat{\mtx{x}}_{k+1}^+ = \hat{\mtx{x}}_{k+1}^- +
    \mtx{K}_{k+1}(\mtx{y}_{k+1} - h(\hat{\mtx{x}}_{k+1}^-))
\end{equation*}

\subsection{Unscented Kalman filter}
\index{Nonlinear control!unscented Kalman filter}
\index{State-space observers!Kalman filter!unscented Kalman filter}

This method \glslink{linearization}{linearizes} around carefully chosen points
to minimize the modeling error. There's a lot of detail to cover, so we
recommend just reading a paper on it \cite{bib:unscented_kalman_filter}.

Here's a paper on a quaternion-based Unscented Kalman filter for orientation
tracking \cite{bib:ukf_state_tracking}.
