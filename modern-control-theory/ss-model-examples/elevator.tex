\section{Elevator}

\subsection{Continuous state-space model}
\index{FRC models!elevator equations}

The position and velocity of the elevator can be written as

\begin{align}
  \dot{x}_m &= v_m \label{eq:elevator_cont_ss_pos} \\
  \dot{v}_m &= a_m \label{eq:elevator_cont_ss_vel}
\end{align}

where by equation (\ref{eq:elevator_accel}),

\begin{equation*}
  a_m = \frac{GK_t}{Rrm} V - \frac{G^2 K_t}{Rr^2 m K_v} v_m
\end{equation*}

Substitute this into equation (\ref{eq:elevator_cont_ss_vel}).

\begin{align}
  \dot{v}_m &= \frac{GK_t}{Rrm} V - \frac{G^2 K_t}{Rr^2 m K_v} v_m \nonumber \\
  \dot{v}_m &= -\frac{G^2 K_t}{Rr^2 m K_v} v_m + \frac{GK_t}{Rrm} V
\end{align}

\begin{theorem}[Elevator state-space model]
  \begin{align*}
    \dot{\mtx{x}} &= \mtx{A} \mtx{x} + \mtx{B} \mtx{u} \\
    \mtx{y} &= \mtx{C} \mtx{x} + \mtx{D} \mtx{u}
  \end{align*}
  \begin{equation*}
    \begin{array}{ccc}
      \mtx{x} =
      \begin{bmatrix}
        x \\
        v_m
      \end{bmatrix} &
      \mtx{y} = x &
      \mtx{u} = V
    \end{array}
  \end{equation*}
  \begin{equation}
    \begin{array}{cccc}
      \mtx{A} =
      \begin{bmatrix}
        0 & 1 \\
        0 & -\frac{G^2 K_t}{Rr^2 mK_v}
      \end{bmatrix} &
      \mtx{B} =
      \begin{bmatrix}
        0 \\
        \frac{GK_t}{Rrm}
      \end{bmatrix} &
      \mtx{C} =
      \begin{bmatrix}
        1 & 0
      \end{bmatrix} &
      \mtx{D} = 0
    \end{array}
  \end{equation}
\end{theorem}

\subsection{Model augmentation}

As per subsection \ref{subsec:u_error_estimation}, we will now augment the
\gls{model} so a $u_{error}$ term is added to the \gls{control input}.

The \gls{plant} and \gls{observer} augmentations should be performed before the
\gls{model} is \glslink{discretization}{discretized}. After the \gls{controller}
gain is computed with the unaugmented discrete \gls{model}, the controller may
be augmented. Therefore, the \gls{plant} and \gls{observer} augmentations assume
a continuous \gls{model} and the \gls{controller} augmentation assumes a
discrete \gls{controller}.

\begin{equation*}
  \begin{array}{ccc}
    \mtx{x}_{aug} =
    \begin{bmatrix}
      x \\
      v_m \\
      u_{error}
    \end{bmatrix} &
    \mtx{y} = x &
    \mtx{u} = V
  \end{array}
\end{equation*}

\begin{equation}
  \begin{array}{cccc}
    \mtx{A}_{aug} =
    \begin{bmatrix}
      \mtx{A} & \mtx{B} \\
      \mtx{0}_{1 \times 2} & 0
    \end{bmatrix} &
    \mtx{B}_{aug} =
    \begin{bmatrix}
      \mtx{B} \\
      0
    \end{bmatrix} &
    \mtx{C}_{aug} = \begin{bmatrix}
      \mtx{C} & 0
    \end{bmatrix} &
    \mtx{D}_{aug} = \mtx{D}
  \end{array}
\end{equation}

\begin{equation}
  \begin{array}{cc}
    \mtx{K}_{aug} = \begin{bmatrix}
      \mtx{K} & 1
    \end{bmatrix} &
    \mtx{r}_{aug} = \begin{bmatrix}
      \mtx{r} \\
      0
    \end{bmatrix}
  \end{array}
\end{equation}

This will compensate for unmodeled dynamics such as gravity. However, using a
constant feedforward to counteract gravity is preferred over this method.

\subsection{Simulation}

Python Control will be used to \glslink{discretization}{discretize} the
\gls{model} and simulate it. One of the frccontrol
examples\footnote{\url{https://github.com/calcmogul/frccontrol/blob/master/examples/elevator.py}}
creates and tests a controller for it.

\begin{remark}
  Python Control currently doesn't support finding the transmission zeroes of
  MIMO \glspl{system} with a different number of \glspl{input} than
  \glspl{output}, so \texttt{control.pzmap()} and
  \texttt{frccontrol.System.plot\_pzmaps()} fail with an error if Slycot isn't
  installed.
\end{remark}

%Figure \ref{fig:elevator_pzmaps} shows the pole-zero maps for the open-loop
%\gls{system}, closed-loop \gls{system}, and \gls{observer}. Figure
%\ref{fig:elevator_response} shows the \gls{system} response with them.

%\begin{svg}{build/frccontrol/examples/elevator_pzmaps}
%  \caption{Elevator pole-zero maps}
%  \label{fig:elevator_pzmaps}
%\end{svg}

Figure \ref{fig:elevator_response} shows the \gls{system} response.

\begin{svg}{build/frccontrol/examples/elevator_response}
  \caption{Elevator response}
  \label{fig:elevator_response}
\end{svg}

\subsection{Implementation}

The script linked above also generates two files: ElevatorCoeffs.h and
ElevatorCoeffs.cpp. These can be used with the WPILib StateSpacePlant,
StateSpaceController, and StateSpaceObserver classes in C++ and Java. A C++
implementation of this elevator controller is available online\footnote{
\url{https://github.com/calcmogul/allwpilib/tree/state-space/wpilibcExamples/src/main/cpp/examples/StateSpaceElevator}}.

\begin{remark}
  Instead of implementing $u_{error}$ estimation to compensate for gravity, one
  can apply a constant voltage feedforward since input voltage is proportional
  to force and gravity is a constant force.
\end{remark}
