\section{Flywheel}

\subsection{Continuous state-space model}
\index{FRC models!flywheel equations}

By equation (\ref{eq:dot_omega_f})

\begin{equation*}
  \dot{\omega}_f = -\frac{G^2 K_t}{K_v RJ} \omega_f + \frac{G K_t}{RJ} V
\end{equation*}

\begin{theorem}[Flywheel state-space model]
  \begin{align*}
    \dot{\mtx{x}} &= \mtx{A} \mtx{x} + \mtx{B} \mtx{u} \\
    \mtx{y} &= \mtx{C} \mtx{x} + \mtx{D} \mtx{u}
  \end{align*}
  \begin{equation*}
    \begin{array}{ccc}
      \mtx{x} = \omega_f &
      \mtx{y} = \omega_f &
      \mtx{u} = V
    \end{array}
  \end{equation*}
  \begin{equation}
    \begin{array}{cccc}
      \mtx{A} = -\frac{G^2 K_t}{K_v RJ} &
      \mtx{B} = \frac{G K_t}{RJ} &
      \mtx{C} = 1 &
      \mtx{D} = 0
    \end{array}
  \end{equation}
\end{theorem}

\subsection{Model augmentation}

As per subsection \ref{subsec:u_error_estimation}, we will now augment the
\gls{model} so a $u_{error}$ term is added to the \gls{control input}.

The \gls{plant} and \gls{observer} augmentations should be performed before the
\gls{model} is \glslink{discretization}{discretized}. After the \gls{controller}
gain is computed with the unaugmented discrete \gls{model}, the controller may
be augmented. Therefore, the \gls{plant} and \gls{observer} augmentations assume
a continuous \gls{model} and the \gls{controller} augmentation assumes a
discrete \gls{controller}.

\begin{equation*}
  \begin{array}{ccc}
    \mtx{x} =
    \begin{bmatrix}
      \omega_f \\
      u_{error}
    \end{bmatrix} &
    \mtx{y} = \omega_f &
    \mtx{u} = V
  \end{array}
\end{equation*}

\begin{equation}
  \begin{array}{cccc}
    \mtx{A}_{aug} =
    \begin{bmatrix}
      \mtx{A} & \mtx{B} \\
      0 & 0
    \end{bmatrix} &
    \mtx{B}_{aug} =
    \begin{bmatrix}
      \mtx{B} \\
      0
    \end{bmatrix} &
    \mtx{C}_{aug} = \begin{bmatrix}
      \mtx{C} & 0
    \end{bmatrix} &
    \mtx{D}_{aug} = \mtx{D}
  \end{array}
\end{equation}

\begin{equation}
  \begin{array}{cc}
    \mtx{K}_{aug} = \begin{bmatrix}
      \mtx{K} & 1
    \end{bmatrix} &
    \mtx{r}_{aug} = \begin{bmatrix}
      \mtx{r} \\
      0
    \end{bmatrix}
  \end{array}
\end{equation}

This will compensate for unmodeled dynamics such as projectiles slowing down the
flywheel.

\subsection{Simulation}

Python Control will be used to \glslink{discretization}{discretize} the
\gls{model} and simulate it. One of the frccontrol
examples\footnote{\url{https://github.com/calcmogul/frccontrol/blob/master/examples/flywheel.py}}
creates and tests a controller for it.

Figure \ref{fig:flywheel_pzmaps} shows the pole-zero maps for the open-loop
\gls{system}, closed-loop \gls{system}, and \gls{observer}. Figure
\ref{fig:flywheel_response} shows the \gls{system} response with them.

\begin{svg}{build/frccontrol/examples/flywheel_pzmaps}
  \caption{Flywheel pole-zero maps}
  \label{fig:flywheel_pzmaps}
\end{svg}

\begin{svg}{build/frccontrol/examples/flywheel_response}
  \caption{Flywheel response}
  \label{fig:flywheel_response}
\end{svg}

Notice how the \gls{control effort} when the \gls{reference} is reached is
nonzero. This is the two-state feedforward compensating for the \gls{system}
dynamics attempting to slow the flywheel down when no voltage is applied.

\subsection{Implementation}

The script linked above also generates two files: FlywheelCoeffs.h and
FlywheelCoeffs.cpp. These can be used with the WPILib StateSpacePlant,
StateSpaceController, and StateSpaceObserver classes in C++ and Java. A C++
implementation of this flywheel controller is available online\footnote{
\url{https://github.com/calcmogul/allwpilib/tree/state-space/wpilibcExamples/src/main/cpp/examples/StateSpaceFlywheel}}.
