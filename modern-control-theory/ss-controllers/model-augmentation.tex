\section{Model augmentation}

This section will teach various tricks for manipulating state-space
\glspl{model} with the goal of demystifying the matrix algebra at play. We will
use the augmentation techniques discussed here in the section on integral
control.

Matrix augmentation is the process of appending rows or columns to a matrix. In
state-space, there are several common types of augmentation used: \gls{plant}
augmentation, controller augmentation, and \gls{observer} augmentation.

\subsection{Plant augmentation}
\index{Model augmentation!of plant}

Plant augmentation is the process of adding a state to a model's state vector
and adding a corresponding row to the $\mtx{A}$ and $\mtx{B}$ matrices.

\subsection{Controller augmentation}
\index{Model augmentation!of controller}

Controller augmentation is the process of adding a column to a controller's
$\mtx{K}$ matrix. This is often done in combination with \gls{plant}
augmentation to add controller dynamics relating to a newly added \gls{state}.

\subsection{Observer augmentation}
\index{Model augmentation!of observer}

Observer augmentation is closely related to \gls{plant} augmentation. In
addition to adding entries to the \gls{observer} matrix $\mtx{L}$, the
\gls{observer} is using this augmented \gls{plant} for estimation purposes. This
is better explained with an example.

By augmenting the \gls{plant} with a bias term with no dynamics (represented by
zeroes in its rows in $\mtx{A}$ and $\mtx{B}$, the \gls{observer} will attempt
to estimate a value for this bias term that makes the \gls{model} best reflect
the measurements taken of the real \gls{system}. Note that we're not collecting
any data on this bias term directly; it's what's known as a hidden \gls{state}.
Rather than our \glspl{input} and other \glspl{state} affecting it directly, the
\gls{observer} determines a value for it based on what is most likely given the
\gls{model} and current measurements. We just tell the \gls{plant} what kind of
dynamics the term has and the \gls{observer} will estimate it for us.

\subsection{Output augmentation}
\index{Model augmentation!of output}

Output augmentation is the process of adding rows to the $\mtx{C}$ matrix. This
is done to help the controls designer visualize the behavior of internal states
or other aspects of the \gls{system} in MATLAB or Python Control. $\mtx{C}$
matrix augmentation doesn't affect \gls{state} feedback, so the designer has a
lot of freedom here. Noting that the \gls{output} is defined as
$\mtx{y} = \mtx{C}\mtx{x} + \mtx{D}\mtx{u}$, The following row augmentations of
$\mtx{C}$ may prove useful. Of course, $\mtx{D}$ needs to be augmented with
zeroes as well in these cases to maintain the correct matrix dimensionality.

Since $\mtx{u} = -\mtx{K}\mtx{x}$, augmenting $\mtx{C}$ with $-\mtx{K}$ makes
the \gls{observer} estimate the \gls{control input} $\mtx{u}$ applied.

\begin{align*}
  \mtx{y} &= \mtx{C}\mtx{x} + \mtx{D}\mtx{u} \\
  \begin{bmatrix}
    \mtx{y} \\
    \mtx{u}
  \end{bmatrix} &=
  \begin{bmatrix}
    \mtx{C} \\
    -\mtx{K}
  \end{bmatrix}
  \mtx{x} +
  \begin{bmatrix}
    \mtx{D} \\
    \mtx{0}
  \end{bmatrix}
  \mtx{u}
\end{align*}

This works because $\mtx{K}$ has the same number of columns as \glspl{state}.

Various \glspl{state} can also be produced in the \gls{output} with $\mtx{I}$
matrix augmentation.

\subsection{Examples}

Snippet \ref{lst:augment} shows how one packs together the following augmented
matrix in Python.

\begin{equation*}
  \begin{bmatrix}
    \mtx{A} - \mtx{I} & \mtx{B} \\
    \mtx{C} & \mtx{D}
  \end{bmatrix}
\end{equation*}

\begin{code}{Python}{code/snippets/augment.py}
  \caption{Matrix augmentation example}
  \label{lst:augment}
\end{code}

Section \ref{sec:integral_control} demonstrates \gls{model} augmentation for
different types of integral control.
