\chapterimage{ss-controllers.jpg}{Night sky above Dufour Street in Santa Cruz, CA}

\chapter{State-space controllers}

\begin{remark}
  Chapters from here on use the \texttt{frccontrol} Python package to
  demonstrate the concepts discussed and perform the complex math required. See
  appendix \ref{ch:installing_python_packages} for how to install it.
\end{remark}

When we want to command a \gls{system} to a set of \glspl{state}, we design a
controller with certain \glspl{control law} to do it. PID controllers use the
system \glspl{output} with proportional, integral, and derivative
\glspl{control law}. In state-space, we also have knowledge of the system
\glspl{state} so we can do better.

Modern control theory uses state-space representation to model and control
systems. State-space representation models \glspl{system} as a set of
\gls{state}, \gls{input}, and \gls{output} variables related by first-order
differential equations that describe how the \gls{system}'s \gls{state} changes
over time given the current \glspl{state} and \glspl{input}.

\renewcommand*{\chapterpath}{\partpath/ss-controllers}
\section{From PID control to model-based control}
\index{PID control}

As mentioned before, controls engineers have a more general framework to
describe control theory than just PID control. PID controller designers are
focused on fiddling with controller parameters relating to the current, past,
and future error rather than the underlying system states. Integral control is a
commonly used tool, and some people use integral action as the majority of the
control action. While this approach works in a lot of situations, it is an
incomplete view of the world.

Model-based control has a completely different mindset. Controls designers using
model-based control care about developing an accurate model of the system, then
driving the states they care about to zero (or to a \gls{reference}). Integral
control is added with $u_{error}$ estimation if needed to handle model
uncertainty, but we prefer not to use it because its response is hard to tune
and some of its destabilizing dynamics aren't visible during simulation.

\section{What is a dynamical system?}

A dynamical system is a \gls{system} whose motion varies according to a set of
differential equations. A dynamical system is considered \textit{linear} if the
differential equations describing its dynamics consist only of linear operators.
Linear operators are things like constant gain multiplications, derivatives, and
integrals. You can define reasonably accurate linear \glspl{model} for pretty
much everything you'll see in FRC with just those relations.

But let's say you have a DC brushed motor hooked up to a power supply and you
applied a constant voltage to it from rest. The motor approaches a steady-state
angular velocity, but the shape of the angular velocity curve over time isn't a
line. In fact, it's a decaying exponential curve akin to

\begin{equation*}
  \omega = \omega_{max}\left(1 - e^{-t}\right)
\end{equation*}

where $\omega$ is the angular velocity and $\omega_{max}$ is the maximum angular
velocity. If DC brushed motors are said to behave linearly, then why is this?

Linearity refers to a \gls{system}'s equations of motion, not its time domain
response. The equation defining the motor's change in angular velocity over time
looks like

\begin{equation*}
  \dot{\omega} = -a\omega + bV
\end{equation*}

where $\dot{\omega}$ is the derivative of $\omega$ with respect to time, $V$ is
the input voltage, and $a$ and $b$ are constants specific to the motor. This
equation, unlike the one shown before, is actually linear because it only
consists of multiplications and additions relating the \gls{input} $V$ and
current \gls{state} $\omega$.

Also of note is that the relation between the input voltage and the angular
velocity of the output shaft is a linear regression. You'll see why if you model
a DC brushed motor as a voltage source and generator producing back-EMF (in the
equation above, $bV$ corresponds to the voltage source and $-a\omega$
corresponds to the back-EMF). As you increase the input voltage, the back-EMF
increases linearly with the motor's angular velocity. If there was a friction
term that varied with the angular velocity squared (air resistance is one
example), the relation from input to output would be a curve. Friction that
scales with just the angular velocity would result in a lower maximum angular
velocity, but because that term can be lumped into the back-EMF term, the
response is still linear.

\section{State-space notation}

\subsection{What is state-space?}

Recall from last chapter that 2D space has two axes: $x$ and $y$. We represent
locations within this space as a pair of numbers packaged in a vector, and each
coordinate is a measure of how far to move along the corresponding axis.
State-space is a Cartesian coordinate system with an axis for each \gls{state}
variable, and we represent locations within it the same way we do for 2D space:
with a list of numbers in a vector. Each element in the vector corresponds to a
\gls{state} of the \gls{system}.

In addition to the \gls{state}, \glspl{input} and \glspl{output} are represented
as vectors. Since the mapping from the current \glspl{state} and \glspl{input}
to the change in \gls{state} is a system of equations, it's natural to write it
in matrix form.

\subsection{Benefits over classical control}

State-space notation provides a more convenient and compact way to model and
analyze \glspl{system} with multiple \glspl{input} and \glspl{output}. For a
\gls{system} with $p$ \glspl{input} and $q$ \glspl{output}, we would have to
write $q \times p$ Laplace transforms to represent it. Not only is the resulting
algebra unwieldy, but it only works for linear \glspl{system}. Including nonzero
initial conditions complicates the algebra even more. State-space representation
uses the time domain instead of the Laplace domain, so it can model nonlinear
\glspl{system}\footnote{This book focuses on analysis and control of linear
\glspl{system}. See chapter \ref{ch:nonlinear_control} for more on nonlinear
control.} and trivially supports nonzero initial conditions.

Students are still taught classical control first because it provides a
framework within which to understand the results we get from the fancy
mathematical machinery of modern control.

\subsection{The equations}

Below are the continuous and discrete versions of state-space notation.

\begin{definition}[State-space notation]%
  \index{State-space controllers!open-loop}

  \begin{align}
    \dot{\mtx{x}} &= \mtx{A}\mtx{x} + \mtx{B}\mtx{u} \label{eq:ss_ctrl_x} \\
    \mtx{y} &= \mtx{C}\mtx{x} + \mtx{D}\mtx{u} \label{eq:ss_ctrl_y} \\
    \nonumber \\
    \mtx{x}_{k+1} &= \mtx{A}\mtx{x}_k + \mtx{B}\mtx{u}_k \label{eq:ssz_ctrl_x} \\
    \mtx{y}_{k+1} &= \mtx{C}\mtx{x}_k + \mtx{D}\mtx{u}_k \label{eq:ssz_ctrl_y}
  \end{align}

  \begin{figurekey}
    \begin{tabulary}{\linewidth}{LLLL}
      $\mtx{A}$ & system matrix      & $\mtx{x}$ & state vector \\
      $\mtx{B}$ & input matrix       & $\mtx{u}$ & input vector \\
      $\mtx{C}$ & output matrix      & $\mtx{y}$ & output vector \\
      $\mtx{D}$ & feedthrough matrix &  &  \\
    \end{tabulary}
  \end{figurekey}
\end{definition}

\begin{booktable}
  \begin{tabular}{|ll|ll|}
    \hline
    \rowcolor{headingbg}
    \textbf{Matrix} & \textbf{Rows $\times$ Columns} &
    \textbf{Matrix} & \textbf{Rows $\times$ Columns} \\
    \hline
    $\mtx{A}$ & states $\times$ states & $\mtx{x}$ & states $\times$ 1 \\
    $\mtx{B}$ & states $\times$ inputs & $\mtx{u}$ & inputs $\times$ 1 \\
    $\mtx{C}$ & outputs $\times$ states & $\mtx{y}$ & outputs $\times$ 1 \\
    $\mtx{D}$ & outputs $\times$ inputs &  &  \\
    \hline
  \end{tabular}
  \caption{State-space matrix dimensions}
  \label{tab:ss_matrix_dims}
\end{booktable}

In the continuous case, the change in \gls{state} and the \gls{output} are
linear combinations of the \gls{state} vector and the \gls{input} vector. The
$\mtx{A}$ and $\mtx{B}$ matrices are used to map the \gls{state} vector
$\mtx{x}$ and the \gls{input} vector $\mtx{u}$ to a change in the \gls{state}
vector $\dot{\mtx{x}}$. The $\mtx{C}$ and $\mtx{D}$ matrices are used to map the
\gls{state} vector $\mtx{x}$ and the \gls{input} vector $\mtx{u}$ to an
\gls{output} vector $\mtx{y}$.

\section{Controllability}
\index{Controller design!controllability}

\Gls{state} controllability implies that it is possible -- by admissible inputs
-- to steer the \glspl{state} from any initial value to any final value within
some finite time window.

\begin{theorem}[Controllability]
  A continuous \gls{time-invariant} linear state-space \gls{model} is
  controllable if and only if

  \begin{equation}
    \rank\left(
    \begin{bmatrix}
      \mtx{B} & \mtx{A}\mtx{B} & \mtx{A}^2\mtx{B} & \cdots &
      \mtx{A}^{n-1}\mtx{B}
    \end{bmatrix}
    \right) = n
    \label{eq:ctrl_rank}
  \end{equation}

  where rank is the number of linearly independent rows in a matrix and $n$ is
  the number of \gls{state} variables.
\end{theorem}

The matrix in equation (\ref{eq:ctrl_rank}) being rank-deficient means the
\glspl{input} cannot apply transforms along all axes in the state-space; the
transformation the matrix represents is collapsed into a lower dimension.

The condition number of the controllability matrix $\mathbb{C}$ is defined as
$\frac{\sigma_{max}(\mathbb{C})}{\sigma_{min}(\mathbb{C})}$ where $\sigma_{max}$
is the maximum singular value\footnote{\label{footn:singular_val}Singular values
are a generalization of eigenvalues for nonsquare matrices.} and $\sigma_{min}$
is the minimum singular value. As this number approaches infinity, one or more
of the \glspl{state} becomes uncontrollable. This number can also be used to
tell us which actuators are better than others for the given \gls{system}; a
lower condition number means that the actuators have more control authority.

\section{Observability}
\index{Controller design!observability}

Observability is a measure for how well internal \glspl{state} of a \gls{system}
can be inferred by knowledge of its external \glspl{output}. The observability
and controllability of a \gls{system} are mathematical duals (i.e., as
controllability proves that an \gls{input} is available that brings any initial
\gls{state} to any desired final \gls{state}, observability proves that knowing
enough \gls{output} values provides enough information to predict the initial
\gls{state} of the \gls{system}).

\begin{theorem}[Observability]
  A continuous \gls{time-invariant} linear state-space \gls{model} is observable
  if and only if

  \begin{equation}
    \rank\left(
    \begin{bmatrix}
      C \\
      CA \\
      \vdots \\
      CA^{n-1}
    \end{bmatrix}\right) = n \label{eq:obsv_rank}
  \end{equation}

  where rank is the number of linearly independent rows in a matrix and $n$ is
  the number of \gls{state} variables.
\end{theorem}

The matrix in equation (\ref{eq:obsv_rank}) being rank-deficient means the
\glspl{output} do not contain contributions from every \gls{state}. That is, not
all \glspl{state} are mapped to a linear commbination in the \gls{output}.
Therefore, the \glspl{output} alone are insufficient to estimate all the
\glspl{state}.

The condition number of the observability matrix $\mathbb{O}$ is defined as
$\frac{\sigma_{max}(\mathbb{O})}{\sigma_{min}(\mathbb{O})}$ where $\sigma_{max}$
is the maximum singular value\footref{footn:singular_val} and $\sigma_{min}$ is
the minimum singular value. As this number approaches infinity, one or more of
the \glspl{state} becomes unobservable. This number can also be used to tell us
which sensors are better than others for the given \gls{system}; a lower
condition number means the \glspl{output} produced by the sensors are better
indicators of the \gls{system} \gls{state}.

\section{Closed-loop controller}

With the \gls{control law} $\mtx{u} = \mtx{K}(\mtx{r} - \mtx{x})$, we can derive
the closed-loop state-space equations. We'll discuss where this control law
comes from in subsection \ref{sec:lqr}.

First is the state update equation. Substitute the control law into equation
(\ref{eq:ss_ctrl_x}).

\begin{align}
  \dot{\mtx{x}} &= \mtx{A}\mtx{x} + \mtx{B}\mtx{K}(\mtx{r} - \mtx{x}) \nonumber
    \\
  \dot{\mtx{x}} &= \mtx{A}\mtx{x} + \mtx{B}\mtx{K}\mtx{r} -
    \mtx{B}\mtx{K}\mtx{x} \nonumber \\
  \dot{\mtx{x}} &= (\mtx{A} - \mtx{B}\mtx{K})\mtx{x} + \mtx{B}\mtx{K}\mtx{r}
\end{align}

Now for the output equation. Substitute the control law into equation
(\ref{eq:ss_ctrl_y}).

\begin{align}
  \mtx{y} &= \mtx{C}\mtx{x} + \mtx{D}(\mtx{K}(\mtx{r} - \mtx{x})) \nonumber \\
  \mtx{y} &= \mtx{C}\mtx{x} + \mtx{D}\mtx{K}\mtx{r} - \mtx{D}\mtx{K}\mtx{x}
    \nonumber \\
  \mtx{y} &= (\mtx{C} - \mtx{D}\mtx{K})\mtx{x} + \mtx{D}\mtx{K}\mtx{r}
\end{align}

Now, we'll do the same for the discrete system. We'd like to know whether the
\gls{system} defined by equation (\ref{eq:ssz_ctrl_x}) operating with the
\gls{control law} $\mtx{u}_k = \mtx{K}(\mtx{r}_k - \mtx{x}_k)$ converges to the
\gls{reference} $\mtx{r}_k$.

\begin{align*}
  \mtx{x}_{k+1} &= \mtx{A}\mtx{x}_k + \mtx{B}\mtx{u}_k \\
  \mtx{x}_{k+1} &= \mtx{A}\mtx{x}_k + \mtx{B}(\mtx{K}(\mtx{r}_k - \mtx{x}_k)) \\
  \mtx{x}_{k+1} &= \mtx{A}\mtx{x}_k + \mtx{B}\mtx{K}\mtx{r}_k -
    \mtx{B}\mtx{K}\mtx{x}_k \\
  \mtx{x}_{k+1} &= \mtx{A}\mtx{x}_k - \mtx{B}\mtx{K}\mtx{x}_k +
    \mtx{B}\mtx{K}\mtx{r}_k \\
  \mtx{x}_{k+1} &= (\mtx{A} - \mtx{B}\mtx{K})\mtx{x}_k + \mtx{B}\mtx{K}\mtx{r}_k
\end{align*}

\begin{theorem}[Closed-loop state-space controller]
  \index{State-space controllers!closed-loop}

  \begin{align}
    \dot{\mtx{x}} &= (\mtx{A} - \mtx{B}\mtx{K})\mtx{x} + \mtx{B}\mtx{K}\mtx{r}
      \label{eq:s_ref_ctrl_x} \\
    \mtx{y} &= (\mtx{C} - \mtx{D}\mtx{K})\mtx{x} + \mtx{D}\mtx{K}\mtx{r}
      \label{eq:s_ref_ctrl_y}
  \end{align}

  \begin{align}
    \mtx{x}_{k+1} &= (\mtx{A} - \mtx{B}\mtx{K})\mtx{x}_k +
      \mtx{B}\mtx{K}\mtx{r}_k \label{eq:z_ref_ctrl_x} \\
    \mtx{y}_k &= (\mtx{C} - \mtx{D}\mtx{K})\mtx{x}_k + \mtx{D}\mtx{K}\mtx{r}_k
      \label{eq:z_ref_ctrl_y}
  \end{align}

  \begin{figurekey}
    \begin{tabulary}{\linewidth}{LLLL}
      $\mtx{A}$ & system matrix      & $\mtx{K}$ & controller gain matrix \\
      $\mtx{B}$ & input matrix       & $\mtx{x}$ & state vector \\
      $\mtx{C}$ & output matrix      & $\mtx{r}$ & \gls{reference} vector \\
      $\mtx{D}$ & feedthrough matrix & $\mtx{y}$ & output vector \\
    \end{tabulary}
  \end{figurekey}
\end{theorem}

\begin{booktable}
  \begin{tabular}{|ll|ll|}
    \hline
    \rowcolor{headingbg}
    \textbf{Matrix} & \textbf{Rows $\times$ Columns} &
    \textbf{Matrix} & \textbf{Rows $\times$ Columns} \\
    \hline
    $\mtx{A}$ & states $\times$ states & $\mtx{x}$ & states $\times$ 1 \\
    $\mtx{B}$ & states $\times$ inputs & $\mtx{u}$ & inputs $\times$ 1 \\
    $\mtx{C}$ & outputs $\times$ states & $\mtx{y}$ & outputs $\times$ 1 \\
    $\mtx{D}$ & outputs $\times$ inputs & $\mtx{r}$ & states $\times$ 1 \\
    $\mtx{K}$ & inputs $\times$ states &  &  \\
    \hline
  \end{tabular}
  \caption{Controller matrix dimensions}
  \label{tab:ctrl_matrix_dims}
\end{booktable}

\index{Stability!eigenvalues}
Instead of commanding the system to a state using the vector $\mtx{u}$ directly,
we can now specify a vector of desired states through $\mtx{r}$ and the system
will choose values of $\mtx{u}$ for us over time to make the system converge to
the reference. For equation (\ref{eq:s_ref_ctrl_x}) to reach steady-state, the
eigenvalues of $\mtx{A} - \mtx{B}\mtx{K}$ must be in the left-half plane. For
equation (\ref{eq:z_ref_ctrl_x}) to have a bounded output, the eigenvalues of
$\mtx{A} - \mtx{B}\mtx{K}$ must be within the unit circle.

The eigenvalues of $\mtx{A} - \mtx{B}\mtx{K}$ are the poles of the closed-loop
system. Therefore, the rate of convergence and stability of the closed-loop
system can be changed by moving the poles via the eigenvalues of
$\mtx{A} - \mtx{B}\mtx{K}$. $\mtx{A}$ and $\mtx{B}$ are inherent to the system,
but $\mtx{K}$ can be chosen arbitrarily by the controller designer.

\section{Pole placement}
\index{Controller design!pole placement}

This is the practice of placing the poles of a closed-loop system directly to
produce a desired response. This can be done manually for state feedback
controllers with controllable canonical form (see section \ref{sec:ctrl-canon}).
This can also be done manually for state observers with observable canonical
form (see section \ref{sec:obsv-canon}).

In general, pole placement should only be used if you know what you're doing.
It's much easier to let LQR place the poles for you, then use those as a
starting point for pole placement.

\section{LQR} \label{sec:lqr}
\index{Controller design!LQR}
\index{Optimal control!LQR}

Instead of placing the poles of a closed-loop \gls{system} manually, LQR design
places the poles for us based on acceptable \gls{error} and \gls{control effort}
constraints. ``LQR" stands for ``Linear-Quadratic
\glslink{regulator}{Regulator}". This method of controller design uses a
quadratic function for the cost-to-go defined as the sum of the \gls{error} and
\gls{control effort} over time for the linear \gls{system}
$\dot{\mtx{x}} = \mtx{A}\mtx{x} + \mtx{B}\mtx{u}$.

\begin{equation*}
  J = \int\limits_0^\infty \left(\mtx{x}^T\mtx{Q}\mtx{x} +
    \mtx{u}^T\mtx{R}\mtx{u}\right) dt
\end{equation*}

where $J$ represents a tradeoff between \gls{state} excursion and
\gls{control effort} with the weighting factors $\mtx{Q}$ and $\mtx{R}$. LQR
finds a \gls{control law} $\mtx{u}$ that minimizes the cost function. $\mtx{Q}$
and $\mtx{R}$ slide the cost along a Pareto boundary between state tracking and
\gls{control effort} (see figure \ref{fig:pareto_boundary}). Pareto optimality
for this problem means that an improvement in state \gls{tracking} cannot be
obtained without using more \gls{control effort} to do so. Also, a reduction in
\gls{control effort} cannot be obtained without sacrificing state \gls{tracking}
performance. Pole placement, on the other hand, will have a cost anywhere on,
above, or to the right of the Pareto boundary (no cost can be inside the
boundary).

\begin{svg}{build/code/pareto_boundary}
  \caption{Pareto boundary for LQR}
  \label{fig:pareto_boundary}
\end{svg}

The minimum of LQR's cost function is found by setting the derivative of the
cost function to zero and solving for the \gls{control law} $\mtx{u}$. However,
matrix calculus is used instead of normal calculus to take the derivative.

The feedback \gls{control law} that minimizes $J$, which we'll call the
``optimal control law", is shown in theorem \ref{thm:optimal_control_law}.

\begin{theorem}[Optimal control law]
  \label{thm:optimal_control_law}

  \begin{equation}
    \mtx{u} = -\mtx{K}\mtx{x}
  \end{equation}
\end{theorem}
\index{Controller design!LQR!optimal control law}
\index{Optimal control!LQR!optimal control law}

This means that optimal control can be achieved with simply a set of
proportional gains on all the \glspl{state}. This \gls{control law} will make
all \glspl{state} converge to zero assuming the \gls{system} is controllable. To
converge to nonzero \glspl{state}, a \gls{reference} vector $\mtx{r}$ can be
added to the \gls{state} $\mtx{x}$.

\begin{theorem}[Optimal control law with nonzero reference]
  \begin{equation}
    \mtx{u} = \mtx{K}(\mtx{r} - \mtx{x})
  \end{equation}
\end{theorem}

To use the \gls{control law}, we need knowledge of the full \gls{state} of the
\gls{system}. That means we either have to measure all our \glspl{state}
directly or estimate those we do not measure.

See appendix \ref{sec:deriv-optimal-control-law} for how $\mtx{K}$ is calculated
in Python. If the result is finite, the controller is guaranteed to be stable
and \glslink{robustness}{robust} with a \gls{phase margin} of 60 degrees
\cite{bib:lqr-phase-margin}.

\begin{remark}
  LQR design's $\mtx{Q}$ and $\mtx{R}$ matrices don't need \gls{discretization},
  but the $\mtx{K}$ calculated for continuous time and discrete time
  \glspl{system} will be different.
\end{remark}

\subsection{Bryson's rule}
\index{Controller design!LQR!Bryson's rule}
\index{Optimal control!LQR!Bryson's rule}

The next obvious question is what values to choose for $\mtx{Q}$ and $\mtx{R}$.
With Bryson's rule, the diagonals of the $\mtx{Q}$ and $\mtx{R}$ matrices are
chosen based on the maximum acceptable value for each \gls{state} and actuator.
The nondiagonal elements are zero. The balance between $\mtx{Q}$ and $\mtx{R}$
can be slid along the Pareto boundary using a weighting factor $\rho$.

\begin{equation*}
  J = \int\limits_0^\infty \left(\rho \left[
    \left(\frac{x_1}{x_{1,max}}\right)^2 + \ldots +
    \left(\frac{x_n}{x_{n,max}}\right)^2\right] + \left[
    \left(\frac{u_1}{u_{1,max}}\right)^2 + \ldots +
    \left(\frac{u_n}{u_{n,max}}\right)^2\right]\right) dt
\end{equation*}

\begin{equation*}
  \begin{array}{cc}
    \mtx{Q} = \begin{bmatrix}
      \frac{\rho}{x_{1,max}^2} & 0 & \ldots & 0 \\
      0 & \frac{\rho}{x_{2,max}^2} & & \vdots \\
      \vdots & & \ddots & 0 \\
      0 & \ldots & 0 & \frac{\rho}{x_{n,max}^2}
    \end{bmatrix} &
    \mtx{R} = \begin{bmatrix}
      \frac{1}{u_{1,max}^2} & 0 & \ldots & 0 \\
      0 & \frac{1}{u_{2,max}^2} & & \vdots \\
      \vdots & & \ddots & 0 \\
      0 & \ldots & 0 & \frac{1}{u_{n,max}^2}
    \end{bmatrix}
  \end{array}
\end{equation*}

Small values of $\rho$ penalize \gls{control effort} while large values of
$\rho$ penalize \gls{state} excursions. Large values would be chosen in
applications like fighter jets where performance is necessary. Spacecrafts would
use small values to conserve their limited fuel supply.

\section{Case studies of controller design methods}

This example uses the following second-order model for a CIM motor (a DC brushed
motor).

\begin{align*}
  \begin{array}{cccc}
    \mtx{A} = \begin{bmatrix}
      -\frac{b}{J} & \frac{K_t}{J} \\
      -\frac{K_e}{L} & -\frac{R}{L}
    \end{bmatrix} &
    \mtx{B} = \begin{bmatrix}
      0 \\
      \frac{1}{L}
    \end{bmatrix} &
    \mtx{C} = \begin{bmatrix}
      1 & 0
    \end{bmatrix} &
    \mtx{D} = \begin{bmatrix}
      0
    \end{bmatrix}
  \end{array}
\end{align*}

Figure \ref{fig:case_study_pp_lqr} shows the response using poles placed at
$(0.1, 0)$ and $(0.9, 0)$ and LQR with the following cost matrices.

\begin{align*}
  \begin{array}{cc}
    \mtx{Q} = \begin{bmatrix}
      \frac{1}{20^2} & 0 \\
      0 & \frac{1}{40^2}
    \end{bmatrix} &
    \mtx{R} = \begin{bmatrix}
      \frac{1}{12^2}
    \end{bmatrix}
  \end{array}
\end{align*}

\begin{svg}{build/code/case_study_pp_lqr}
  \caption{Second-order CIM motor response with pole placement and LQR}
  \label{fig:case_study_pp_lqr}
\end{svg}

LQR selected poles at $(0.593, 0)$ and $(0.955, 0)$. Notice with pole placement
that as the current pole moves left, the control effort becomes more aggressive.

\section{Model augmentation}

This section will teach various tricks for manipulating state-space models with
the goal of demystifying the matrix algebra at play. We will use the
augmentation techniques discussed here in the section on integral control.

Matrix augmentation is the process of appending rows or columns to a matrix. In
state-space, there are several common types of augmentation used: plant
augmentation, controller augmentation, and observer augmentation.

\subsection{Plant augmentation}

Plant augmentation is the process of adding a state to a model's state vector
and adding a corresponding row to the $\mtx{A}$ and $\mtx{B}$ matrices.

\subsection{Controller augmentation}

Controller augmentation is the process of adding a column to a controller's
$\mtx{K}$ matrix. This is often done in combination with plant augmentation to
add controller dynamics relating to a newly added state.

\subsection{Observer augmentation}

Observer augmentation is closely related to plant augmentation. In addition to
adding entries to the observer matrix $\mtx{L}$, the observer is using this
augmented plant for estimation purposes. This is better explained with an
example.

By augmenting the plant with a bias term with no dynamics (represented by zeroes
in its rows in $\mtx{A}$ and $\mtx{B}$, the observer will attempt to estimate a
value for this bias term that makes the model best reflect the measurements
taken of the real system. Note that we're not collecting any data on this bias
term directly; it's what's known as a hidden state. Rather than our inputs and
other states affecting it directly, the observer determines a value for it based
on what is most likely given the model and current measurements. We just tell
the plant what kind of dynamics the term has and the observer will estimate it
for us.

\subsection{Output augmentation}

Output augmentation is the process of adding rows to the $\mtx{C}$ matrix. This
is done to help the controls designer visualize the behavior of internal states
or other aspects of the system in MATLAB or Python Control. $\mtx{C}$ matrix
augmentation doesn't affect state feedback, so the designer has a lot of freedom
here. Noting that the output is defined as
$\mtx{y} = \mtx{C}\mtx{x} + \mtx{D}\mtx{u}$, The following row augmentations of
$\mtx{C}$ may prove useful. Of course, $\mtx{D}$ needs to be augmented with
zeroes as well in these cases to maintain the correct matrix dimensionality.

Since $\mtx{u} = -\mtx{K}\mtx{x}$, augmenting $\mtx{C}$ with $-\mtx{K}$ makes
the observer estimate the control input $\mtx{u}$ applied.

\begin{align*}
  \mtx{y} &= \mtx{C}\mtx{x} + \mtx{D}\mtx{u} \\
  \begin{bmatrix}
    \mtx{y} \\
    \mtx{u}
  \end{bmatrix} &=
  \begin{bmatrix}
    \mtx{C} \\
    -\mtx{K}
  \end{bmatrix}
  \mtx{x} +
  \begin{bmatrix}
    \mtx{D} \\
    \mtx{0}
  \end{bmatrix}
  \mtx{u}
\end{align*}

This works because $\mtx{K}$ has the same number of columns as states.

Various states can also be produced in the output with $\mtx{I}$ matrix
augmentation.

\section{Feedforwards}

Feedforwards are used to inject information about either the system's dynamics
(like a model does) or the intended movement into a controller. Feedforward is
generally used to handle the control actions we already know must be applied to
make a system track a reference, then let the feedback controller correct for
what we do not or cannot know about the system at runtime. We will present
two ways of implementing feedforward for state feedback.

\subsection{Steady-state feedforward}

Steady-state feedforwards apply the control effort required to keep a system at
the reference if it is no longer moving (i.e., the system is at steady-state).
The first steady-state feedforward converts a desired output to a desired state.

\begin{equation*}
  \mtx{x}_c = \mtx{N}_x\mtx{y}_c
\end{equation*}

$\mtx{N}_x$ converts a desired output $\mtx{y}_c$ to a desired state
$\mtx{x}_c$ (also known as $\mtx{r}$). For steady-state, that is

\begin{equation}
  \mtx{x}_{ss} = \mtx{N}_x\mtx{y}_{ss} \label{eq:x_ss}
\end{equation}

The second steady-state feedforward converts the desired output $\mtx{y}$ to the
control input required at steady-state.

\begin{equation*}
  \mtx{u}_c = \mtx{N}_u\mtx{y}_c
\end{equation*}

$\mtx{N}_u$ converts the desired output $\mtx{y}$ to the control input $\mtx{u}$
required at steady-state. For steady-state, that is

\begin{equation}
  \mtx{u}_{ss} = \mtx{N}_u\mtx{y}_{ss} \label{eq:u_ss}
\end{equation}

To find the control input required at steady-state, set equation
(\ref{eq:ss_ctrl_x}) to zero.

\begin{align*}
  \dot{\mtx{x}} &= \mtx{A}\mtx{x} + \mtx{B}\mtx{u} \\
  \mtx{y} &= \mtx{C}\mtx{x} + \mtx{D}\mtx{u}
\end{align*}

\begin{align*}
  \mtx{0} &= \mtx{A}\mtx{x}_{ss} + \mtx{B}\mtx{u}_{ss} \\
  \mtx{y}_{ss} &= \mtx{C}\mtx{x}_{ss} + \mtx{D}\mtx{u}_{ss}
\end{align*}

\begin{align*}
  \mtx{0} &= \mtx{A}\mtx{N}_x\mtx{y}_{ss} + \mtx{B}\mtx{N}_u\mtx{y}_{ss} \\
  \mtx{y}_{ss} &= \mtx{C}\mtx{N}_x\mtx{y}_{ss} + \mtx{D}\mtx{N}_u\mtx{y}_{ss}
\end{align*}

\begin{align*}
  \begin{bmatrix}
    \mtx{0} \\
    \mtx{y}_{ss}
  \end{bmatrix} &=
  \begin{bmatrix}
    \mtx{A}\mtx{N}_x + \mtx{B}\mtx{N}_u \\
    \mtx{C}\mtx{N}_x + \mtx{D}\mtx{N}_u
  \end{bmatrix}
  \mtx{y}_{ss} \\
  \begin{bmatrix}
    \mtx{0} \\
    \mtx{1}
  \end{bmatrix} &=
  \begin{bmatrix}
    \mtx{A}\mtx{N}_x + \mtx{B}\mtx{N}_u \\
    \mtx{C}\mtx{N}_x + \mtx{D}\mtx{N}_u
  \end{bmatrix} \\
  \begin{bmatrix}
    \mtx{0} \\
    \mtx{1}
  \end{bmatrix} &=
  \begin{bmatrix}
    \mtx{A} & \mtx{B} \\
    \mtx{C} & \mtx{D}
  \end{bmatrix}
  \begin{bmatrix}
    \mtx{N}_x \\
    \mtx{N}_u
  \end{bmatrix} \\
  \begin{bmatrix}
    \mtx{N}_x \\
    \mtx{N}_u
  \end{bmatrix} &=
  \begin{bmatrix}
    \mtx{A} & \mtx{B} \\
    \mtx{C} & \mtx{D}
  \end{bmatrix}^{\dagger}
  \begin{bmatrix}
    \mtx{0} \\
    \mtx{1}
  \end{bmatrix}
\end{align*}

where $^\dagger$ is the Moore-Penrose pseudoinverse.

Now, we'll find an expression that uses $\mtx{N}_x$ and $\mtx{N}_u$ to convert
the reference $\mtx{r}$ to a control input feedforward $\mtx{u}_{ff}$. Let's
start with equation (\ref{eq:x_ss}).

\begin{align*}
  \mtx{x}_{ss} &= \mtx{N}_x \mtx{y}_{ss} \\
  \mtx{N}_x^\dagger \mtx{x}_{ss} &= \mtx{y}_{ss}
\end{align*}

Now substitute this into equation (\ref{eq:u_ss}).

\begin{align*}
  \mtx{u}_{ss} &= \mtx{N}_u \mtx{y}_{ss} \\
  \mtx{u}_{ss} &= \mtx{N}_u (\mtx{N}_x^\dagger \mtx{x}_{ss}) \\
  \mtx{u}_{ss} &= \mtx{N}_u \mtx{N}_x^\dagger \mtx{x}_{ss}
\end{align*}

$\mtx{u}_{ss}$ and $\mtx{x}_{ss}$ are also known as $\mtx{u}_{ff}$ and $\mtx{r}$
respectively.

\begin{align*}
  \mtx{u}_{ff} = \mtx{N}_u \mtx{N}_x^\dagger \mtx{r}
\end{align*}

So all together, we get theorem \ref{thm:steady-state_ff}.

\index{Feedforward!steady-state feedforward}
\begin{theorem}[Steady-state feedforward]
  \begin{align}
    \begin{bmatrix}
      \mtx{N}_x \\
      \mtx{N}_u
    \end{bmatrix} &=
    \begin{bmatrix}
      \mtx{A} & \mtx{B} \\
      \mtx{C} & \mtx{D}
    \end{bmatrix}^{\dagger}
    \begin{bmatrix}
      \mtx{0} \\
      \mtx{1}
    \end{bmatrix} \\
    \mtx{u}_{ff} &= \mtx{N}_u \mtx{N}_x^\dagger \mtx{r}
  \end{align}

  where $^\dagger$ is the Moore-Penrose pseudoinverse.

  In the augmented matrix, $\mtx{B}$ should contain one column corresponding to
  an actuator and $\mtx{C}$ should contain one row whose output will be driven
  by that actuator. More than one actuator or output can be included in the
  computation at once, but the result won't be the same as if they were computed
  independently and summed afterward.

  After computing the feedforward for each actuator-output pair, the respective
  collections of $\mtx{N}_x$ and $\mtx{N}_u$ matrices can summed to produce the
  combined feedforward.

  \label{thm:steady-state_ff}
\end{theorem}

If the augmented matrix in theorem \ref{thm:steady-state_ff} is square (number
of inputs = number of outputs), the normal inverse can be used instead.

\subsection{Two-state feedforward}

Let's start with the equation for the reference dynamics

\begin{equation*}
  \mtx{r}_{k+1} = \mtx{A}\mtx{r}_k + \mtx{B}\mtx{u}_{ff}
\end{equation*}

where $\mtx{u}_{ff}$ is the feedforward input. Note that this feedforward
equation does not and should not take into account any feedback terms. We want
to find the optimal $\mtx{u}_{ff}$ such that we minimize the tracking error
between $\mtx{r}_{k+1}$ and $\mtx{r}_k$.

\begin{equation*}
  \mtx{r}_{k+1} - \mtx{A}\mtx{r}_k = \mtx{B}\mtx{u}_{ff}
\end{equation*}

To solve for $\mtx{u}_{ff}$, we need to take the inverse of the nonsquare matrix
$\mtx{B}$. This isn't possible, but we can find the pseudoinverse given some
constraints on the state tracking error and control effort. To find the optimal
solution for these sorts of trade-offs, one can define a cost function and
attempt to minimize it. To do this, we'll first solve the expression for
$\mtx{0}$.

\begin{equation*}
  \mtx{0} = \mtx{B}\mtx{u}_{ff} - (\mtx{r}_{k+1} - \mtx{A}\mtx{r}_k)
\end{equation*}

This expression will be the state tracking cost we use in our cost function.

Our cost function will use an $H_2$ norm with $\mtx{Q}$ as the state cost matrix
with dimensionality $states \times states$ and $\mtx{R}$ as the control input
cost matrix with dimensionality $inputs \times inputs$.

\begin{equation*}
  \mtx{J} = (\mtx{B}\mtx{u}_{ff} - (\mtx{r}_{k+1} - \mtx{A}\mtx{r}_k))^T \mtx{Q}
    (\mtx{B}\mtx{u}_{ff} - (\mtx{r}_{k+1} - \mtx{A}\mtx{r}_k)) +
    \mtx{u}_{ff}^T\mtx{R}\mtx{u}_{ff}
\end{equation*}

\begin{remark}
  $\mtx{r}_{k+1} - \mtx{A}\mtx{r}_k$ will only return a nonzero vector if the
  reference isn't following the system dynamics. If it is, the feedback
  controller already compensates for it. This feedforward compensates for any
  unmodeled dynamics reflected in how the reference is changing (or not
  changing). In the case of a constant reference, the feedforward opposes any
  system dynamics that would change the state over time.
\end{remark}

The following theorems will be needed to find the minimum of $\mtx{J}$.

\begin{theorem}
  $\frac{\partial \mtx{x}^T\mtx{A}\mtx{x}}{\partial\mtx{x}} =
    2\mtx{A}\mtx{x}$ where $\mtx{A}$ is symmetric.
  \label{thm:partial_xax}
\end{theorem}

\begin{theorem}
  $\frac{\partial (\mtx{A}\mtx{x} + \mtx{b})^T\mtx{C}
    (\mtx{D}\mtx{x} + \mtx{e})}{\partial\mtx{x}} =
    \mtx{A}^T\mtx{C}(\mtx{D}\mtx{x} + \mtx{e}) + \mtx{D}^T\mtx{C}^T
    (\mtx{A}\mtx{x} + \mtx{b})$
  \label{thm:partial_ax_b}
\end{theorem}

\begin{corollary}
  $\frac{\partial (\mtx{A}\mtx{x} + \mtx{b})^T\mtx{C}
    (\mtx{A}\mtx{x} + \mtx{b})}{\partial\mtx{x}} =
    2\mtx{A}^T\mtx{C}(\mtx{A}\mtx{x} + \mtx{b})$ where $\mtx{C}$ is symmetric.
  \label{cor:partial_ax_b}

  Proof:
  \begin{align*}
    \frac{\partial (\mtx{A}\mtx{x} + \mtx{b})^T\mtx{C}
      (\mtx{A}\mtx{x} + \mtx{b})}{\partial\mtx{x}} &=
      \mtx{A}^T\mtx{C}(\mtx{A}\mtx{x} + \mtx{b}) + \mtx{A}^T\mtx{C}^T
      (\mtx{A}\mtx{x} + \mtx{b}) \\
    \frac{\partial (\mtx{A}\mtx{x} + \mtx{b})^T\mtx{C}
      (\mtx{A}\mtx{x} + \mtx{b})}{\partial\mtx{x}} &=
      (\mtx{A}^T\mtx{C} + \mtx{A}^T\mtx{C}^T)(\mtx{A}\mtx{x} + \mtx{b})
  \end{align*}

  $\mtx{C}$ is symmetric, so

  \begin{align*}
    \frac{\partial (\mtx{A}\mtx{x} + \mtx{b})^T\mtx{C}
      (\mtx{A}\mtx{x} + \mtx{b})}{\partial\mtx{x}} &=
      (\mtx{A}^T\mtx{C} + \mtx{A}^T\mtx{C})(\mtx{A}\mtx{x} + \mtx{b}) \\
    \frac{\partial (\mtx{A}\mtx{x} + \mtx{b})^T\mtx{C}
      (\mtx{A}\mtx{x} + \mtx{b})}{\partial\mtx{x}} &=
      2\mtx{A}^T\mtx{C}(\mtx{A}\mtx{x} + \mtx{b})
  \end{align*}
\end{corollary}

Given theorem \ref{thm:partial_xax} and corollary \ref{cor:partial_ax_b}, find
the minimum of $\mtx{J}$ by taking the partial derivative with respect to
$\mtx{u}_{ff}$ and setting the result to $\mtx{0}$.

\begin{align*}
  \frac{\partial\mtx{J}}{\partial\mtx{u}_{ff}} &= 2\mtx{B}^T\mtx{Q}
    (\mtx{B}\mtx{u}_{ff} - (\mtx{r}_{k+1} - \mtx{A}\mtx{r}_k)) +
    2\mtx{R}\mtx{u}_{ff} \\
  \mtx{0} &= 2\mtx{B}^T\mtx{Q}
    (\mtx{B}\mtx{u}_{ff} - (\mtx{r}_{k+1} - \mtx{A}\mtx{r}_k)) +
    2\mtx{R}\mtx{u}_{ff} \\
  \mtx{0} &= \mtx{B}^T\mtx{Q}
    (\mtx{B}\mtx{u}_{ff} - (\mtx{r}_{k+1} - \mtx{A}\mtx{r}_k)) +
    \mtx{R}\mtx{u}_{ff} \\
  \mtx{0} &= \mtx{B}^T\mtx{Q}\mtx{B}\mtx{u}_{ff} -
    \mtx{B}^T\mtx{Q}(\mtx{r}_{k+1} - \mtx{A}\mtx{r}_k) + \mtx{R}\mtx{u}_{ff} \\
  \mtx{B}^T\mtx{Q}\mtx{B}\mtx{u}_{ff} + \mtx{R}\mtx{u}_{ff} &=
    \mtx{B}^T\mtx{Q}(\mtx{r}_{k+1} - \mtx{A}\mtx{r}_k) \\
  (\mtx{B}^T\mtx{Q}\mtx{B} + \mtx{R})\mtx{u}_{ff} &=
    \mtx{B}^T\mtx{Q}(\mtx{r}_{k+1} - \mtx{A}\mtx{r}_k) \\
  \mtx{u}_{ff} &= (\mtx{B}^T\mtx{Q}\mtx{B} + \mtx{R})^{-1}
    \mtx{B}^T\mtx{Q}(\mtx{r}_{k+1} - \mtx{A}\mtx{r}_k)
\end{align*}

\begin{theorem}[Two-state feedforward]
  \begin{align}
    &\mtx{u}_{ff} = \mtx{K}_{ff} (\mtx{r}_{k+1} - \mtx{A}\mtx{r}_k) \\
    &\text{where } \mtx{K}_{ff} =
      (\mtx{B}^T\mtx{Q}\mtx{B} + \mtx{R})^{-1}\mtx{B}^T\mtx{Q}
  \end{align}
  \label{thm:two-state_ff}
\end{theorem}
\index{Feedforward!two-state feedforward}
\index{Optimal control!two-state feedforward}

If control effort is considered inexpensive, $\mtx{R} \ll \mtx{Q}$ and
$\mtx{u}_{ff}$ approaches corollary \ref{cor:two-state_ff_no_r}.

\begin{corollary}[Two-state feedforward with inexpensive control effort]
  \begin{align}
    &\mtx{u}_{ff} = \mtx{K}_{ff} (\mtx{r}_{k+1} - \mtx{A}\mtx{r}_k) \\
    &\text{where } \mtx{K}_{ff} =
      (\mtx{B}^T\mtx{Q}\mtx{B})^{-1}\mtx{B}^T\mtx{Q}
  \end{align}
  \label{cor:two-state_ff_no_r}
\end{corollary}

\begin{remark}
  If the cost matrix $\mtx{Q}$ isn't included in the cost function (that is,
  $\mtx{Q}$ is set to the identity matrix), $\mtx{K}_{ff}$ becomes the
  Moore-Penrose pseudoinverse of $\mtx{B}$ given by
  $\mtx{B}^\dagger = (\mtx{B}^T\mtx{B})^{-1}\mtx{B}^T$.
\end{remark}

\section{Integral control}

A common way of implementing integral control is to add an additional state that
is the integral of the error of the variable intended to have zero steady-state
error.

There are two drawbacks to this method. First, there is integral windup on a
unit step input. That is, the integrator accumulates even if the system is
tracking the model correctly. The second is demonstrated by an example from
Jared Russell of FRC team 254. Say there is a position/velocity trajectory for
some plant to follow. Without integral control, one can calculate a desired
$\mtx{K}\mtx{x}$ to use as the reference input to the controller. As a result of
using both desired position and velocity, reference tracking is good. With
integral control added, the reference is always the desired position, but there
is no way to tell the controller the desired velocity.

Consider carefully whether integral control is necessary. One can get relatively
close without integral control, and integral adds all the issues listed above.
Below, it is assumed that the controls designer has determined that integral
control will be worth the inconvenience.

There are three methods FRC team 971 has used over the years:

\begin{enumerate}
  \item Augment the plant as described earlier. For an arm, one would add an
    ``integral of position" state.
  \item Add an integrator to the output of the controller, then estimate the
    control effort being applied. 971 has called this Delta U control. The
    upside is that it doesn't have the windup issue described above; the
    integrator only acts if the system isn't behaving like the model, which was
    the original intent. The downside is working with it is very confusing.
  \item Estimate an ``error" in the observer and compensate for it. This
    quantity is the difference between what was applied and what was observed to
    happen. To use it, you simply add it to your control input and it will
    converge. This is 971's primary method.
\end{enumerate}

We'll present the first and third methods since the third is strictly better
than the second.

\subsection{Plant augmentation}

We want to augment the system with an integral term that integrates the error
$\mtx{e} = \mtx{r} - \mtx{y} = \mtx{r} - \mtx{C}\mtx{x}$.

\begin{align*}
  \mtx{x}_I &= \int \mtx{e} \,dt \\
  \dot{\mtx{x}}_I &= \mtx{e} = \mtx{r} - \mtx{C}\mtx{x}
\end{align*}

The plant is augmented as

\begin{align*}
  \dot{\begin{bmatrix}
    \mtx{x} \\
    \mtx{x}_I
  \end{bmatrix}} &=
  \begin{bmatrix}
    \mtx{A} & \mtx{0} \\
    -\mtx{C} & \mtx{0}
  \end{bmatrix}
  \begin{bmatrix}
    \mtx{x} \\
    \mtx{x}_I
  \end{bmatrix} +
  \begin{bmatrix}
    \mtx{B} \\
    \mtx{0}
  \end{bmatrix}
  \mtx{u} +
  \begin{bmatrix}
    \mtx{0} \\
    \mtx{I}
  \end{bmatrix}
  \mtx{r} \\
  \dot{\begin{bmatrix}
    \mtx{x} \\
    \mtx{x}_I
  \end{bmatrix}} &=
  \begin{bmatrix}
    \mtx{A} & \mtx{0} \\
    -\mtx{C} & \mtx{0}
  \end{bmatrix}
  \begin{bmatrix}
    \mtx{x} \\
    \mtx{x}_I
  \end{bmatrix} +
  \begin{bmatrix}
    \mtx{B} & \mtx{0} \\
    \mtx{0} & \mtx{I}
  \end{bmatrix}
  \begin{bmatrix}
    \mtx{u} \\
    \mtx{r}
  \end{bmatrix}
\end{align*}

The controller is augmented as

\begin{align*}
  \mtx{u} &= \mtx{K} (\mtx{r} - \mtx{x}) - \mtx{K}_I\mtx{x}_I \\
  \mtx{u} &=
  \begin{bmatrix}
    \mtx{K} & \mtx{K}_I
  \end{bmatrix}
  \left(\begin{bmatrix}
    \mtx{r} \\
    \mtx{0}
  \end{bmatrix} -
  \begin{bmatrix}
    \mtx{x} \\
    \mtx{x}_I
  \end{bmatrix}\right)
\end{align*}

\subsection{U error estimation}

Let $u_{error}$ be the error in a system's input. The $u_{error}$ term is then
added to the system as follows.

\begin{align*}
  \dot{\mtx{x}} &= \mtx{A}\mtx{x} + \mtx{B}\left(\mtx{u} + u_{error}\right) \\
  \dot{\mtx{x}} &= \mtx{A}\mtx{x} + \mtx{B}\mtx{u} + \mtx{B}u_{error}
\end{align*}

For a multiple-output system, this would be

\begin{equation*}
  \dot{\mtx{x}} = \mtx{A}\mtx{x} + \mtx{B}\mtx{u} + \mtx{B}_{error}u_{error}
\end{equation*}

where $\mtx{B}_{error}$ is the column vector that maps $u_{error}$ to changes in
the rest of the state the same way $\mtx{B}$ does for $\mtx{u}$.
$\mtx{B}_{error}$ is only a column of $\mtx{B}$ if $u_{error}$ corresponds to an
existing input within $\mtx{u}$.

The plant is augmented as

\begin{align*}
  \dot{\begin{bmatrix}
    \mtx{x} \\
    u_{error}
  \end{bmatrix}} &=
  \begin{bmatrix}
    \mtx{A} & \mtx{B}_{error} \\
    0 & 0
  \end{bmatrix}
  \begin{bmatrix}
    \mtx{x} \\
    u_{error}
  \end{bmatrix} +
  \begin{bmatrix}
    \mtx{B} \\
    0
  \end{bmatrix}
  \mtx{u}
\end{align*}

With this model, the observer will estimate both the state and the $u_{error}$
term. The controller is augmented similarly. $\mtx{r}$ is augmented with a zero
for the goal $u_{error}$ term.

\begin{align*}
  \mtx{u} &= \mtx{K} \left(\mtx{r} - \mtx{x}\right) - \mtx{k}_{error}u_{error}
    \\
  \mtx{u} &=
  \begin{bmatrix}
    \mtx{K} & \mtx{k}_{error}
  \end{bmatrix}
  \left(\begin{bmatrix}
    \mtx{r} \\
    0
  \end{bmatrix} -
  \begin{bmatrix}
    \mtx{x} \\
    u_{error}
  \end{bmatrix}\right)
\end{align*}

where $\mtx{k}_{error}$ is a column vector with a $1$ in a given row if
$u_{error}$ should be applied to that input or a $0$ otherwise.

This process can be repeated for an arbitrary error which can be corrected via
some linear combination of the inputs.

