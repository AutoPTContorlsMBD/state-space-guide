\newglossaryentry{agent}{
  name={agent},
  description={An independent actor being controlled through autonomy or
    human-in-the-loop (e.g., a robot, aircraft, etc.).}}
\newglossaryentry{control effort}{
  name={control effort},
  description={A term describing how much force, pressure, etc. an actuator is
    exerting.}}
\newglossaryentry{control input}{
  name={control input},
  description={The input of a plant used for the purpose of controlling it.}}
\newglossaryentry{control law}{
  name={control law},
  description={Also known as control policy, is a mathematical formula used by
    the controller to determine the input u that is sent to the plant. This
    control law is designed to drive the system from its current state to some
    other desired state.}}
\newglossaryentry{control system}{
  name={control system},
  description={Monitors and controls the behavior of a system.}}
\newglossaryentry{controller}{
  name={controller},
  description={Used in positive or negative feedback with a plant to bring about
    a desired system state by driving the difference between a reference signal
    and the output to zero.}}
\newglossaryentry{discretization}{
  name={discretization},
  description={The process by which a continuous (e.g., analog) system or
    controller design is converted to discrete (e.g., digital).}}
\newglossaryentry{disturbance}{
  name={disturbance},
  description={An external force acting on a system that isn't included in the
    system's model.}}
\newglossaryentry{disturbance rejection}{
  name={disturbance rejection},
  description={The quality of a feedback control system to compensate for
    external forces to reach a desired reference.}}
\newglossaryentry{error}{
  name={error},
  description={Reference minus input.}}
\newglossaryentry{gain margin}{
  name={gain margin},
  description={See section \ref{sec:gain-phase-margin} on gain and phase
    margin.}}
\newglossaryentry{impulse response}{
  name={impulse response},
  description={The response of a system to the Dirac delta function.}}
\newglossaryentry{input}{
  name={input},
  description={An input to the plant (hence the name) that can be used to change
  the plant's state.}}
\newglossaryentry{linearization}{
  name={linearization},
  description={A method by which a nonlinear system's dynamics are approximated
  by a linear system.}}
\newglossaryentry{localization}{
  name={localization},
  description={The process of using external measurements to determine an
    agent's pose.}}
\newglossaryentry{model}{
  name={model},
  description={A set of mathematical equations that reflects some aspect of a
    physical system's behavior.}}
\newglossaryentry{noise immunity}{
  name={noise immunity},
  description={The quality of a system to have its performance or stability
    unaffected by noise in the outputs (see also: \gls{robustness}).}}
\newglossaryentry{observer}{
  name={observer},
  description={In control theory, a system that provides an estimate of the
    internal state of a given real system from measurements of the input and
    output of the real system.}}
\newglossaryentry{open-loop gain}{
  name={open-loop gain},
  description={The gain directly from the input to the output, ignoring loops.}}
\newglossaryentry{output}{
  name={output},
  description={Measurements from sensors.}}
\newglossaryentry{output-based control}{
  name={output-based control},
  description={Controls the system's state via the outputs.}}
\newglossaryentry{overshoot}{
  name={overshoot},
  description={The amount by which a system's state surpasses the reference
    after rising toward it.}}
\newglossaryentry{phase margin}{
  name={phase margin},
  description={See section \ref{sec:gain-phase-margin} on gain and phase
    margin.}}
\newglossaryentry{plant}{
  name={plant},
  description={The system or collection of actuators being controlled.}}
\newglossaryentry{pose}{
  name={pose},
  description={The orientation of an agent in the world, which is represented by
    all or part of the agent's state.}}
\newglossaryentry{process value}{
  name={process value},
  description={The term used to describe the output of a PID controller.}}
\newglossaryentry{realization}{
  name={realization},
  description={In control theory, this is an implementation of a given
    input-output behavior as a state-space model.}}
\newglossaryentry{reference}{
  name={reference},
  description={The desired state.}}
\newglossaryentry{regulator}{
  name={regulator},
  description={A controller that attempts to minimize the error from a constant
    reference in the presence of disturbances.}}
\newglossaryentry{rise time}{
  name={rise time},
  description={The time a system takes to initially reach the reference after
    applying a step input.}}
\newglossaryentry{robustness}{
  name={robustness},
  description={The quality of a feedback control system to remain stable in
    response to disturbances and uncertainty.}}
\newglossaryentry{setpoint}{
  name={setpoint},
  description={The term used to describe the reference of a PID controller.}}
\newglossaryentry{settling time}{
  name={settling time},
  description={The time a system takes to settle at the reference after a step
    input is applied.}}
\newglossaryentry{state}{
  name={state},
  description={A characteristic of a system (e.g., velocity) that can be used to
    determine the system's future behavior.}}
\newglossaryentry{state feedback}{
  name={state feedback},
  description={Uses state instead of output in feedback.}}
\newglossaryentry{steady-state error}{
  name={steady-state error},
  description={Error after system reaches equilibrium.}}
\newglossaryentry{step input}{
  name={step input},
  description={A system input that is $0$ for $t < 0$ and $1$ for $t \geq 0$.}}
\newglossaryentry{step response}{
  name={step response},
  description={The response of a system to a step input.}}
\newglossaryentry{stochastic process}{
  name={stochastic process},
  description={A process whose model is partially or completely defined by
    random variables.}}
\newglossaryentry{system}{
  name={system},
  description={Maps inputs to outputs through a linear combination of states.}}
\newglossaryentry{system response}{
  name={system response},
  description={The behavior of a system over time for a given input.}}
\newglossaryentry{time-invariant}{
  name={time-invariant},
  description={The system's fundamental response does not change over time.}}
\newglossaryentry{tracking}{
  name={tracking},
  description={In control theory, the process of making the output of a control
    system follow the reference.}}
\newglossaryentry{unity feedback}{
  name={unity feedback},
  description={A feedback network in a control system diagram with a feedback
    gain of 1.}}
