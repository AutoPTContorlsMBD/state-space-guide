\newglossaryentry{controller}{
  name={controller},
  description={Used in positive or negative feedback with a plant to bring about
    a desired system state by driving the difference between a reference signal
    and the output to zero.}}
\newglossaryentry{control law}{
  name={control law},
  description={Also known as control policy, is a mathematical formula used by
  the controller to determine the input u that is sent to the plant. This
  control law is designed to drive the system from its current state to some
  other desired state.}}
\newglossaryentry{discretization}{
  name={discretization},
  description={The process by which a continuous (e.g., analog) system or
  controller design is converted to discrete (e.g., digital).}}
\newglossaryentry{disturbance}{
  name={disturbance},
  description={An external force acting on a system that isn't included in the
    system's model.}}
\newglossaryentry{disturbance rejection}{
  name={disturbance rejection},
  description={The quality of a feedback control system to compensate for
    external forces to reach a desired reference.}}
\newglossaryentry{error}{
  name={error},
  description={Reference minus input.}}
\newglossaryentry{gain margin}{
  name={gain margin},
  description={See section \ref{sec:gain-phase-margin} on gain and phase
    margin.}}
\newglossaryentry{input}{
  name={input},
  description={An input to the plant (hence the name) that can be used to change
  the plant's state.}}
\newglossaryentry{model}{
  name={model},
  description={A set of mathematical equations that reflects some aspect of a
    physical system's behavior.}}
\newglossaryentry{noise immunity}{
  name={noise immunity},
  description={The quality of a system to have its performance or stability
    unaffected by noise in the outputs (see also: \gls{robustness}).}}
\newglossaryentry{open-loop gain}{
  name={open-loop gain},
  description={The gain directly from the input to the output, ignoring loops.}}
\newglossaryentry{output}{
  name={output},
  description={Measurements from sensors.}}
\newglossaryentry{output-based control}{
  name={output-based control},
  description={Controls the system's state via the outputs.}}
\newglossaryentry{phase margin}{
  name={phase margin},
  description={See section \ref{sec:gain-phase-margin} on gain and phase
    margin.}}
\newglossaryentry{plant}{
  name={plant},
  description={The system or collection of actuators being controlled.}}
\newglossaryentry{realization}{
  name={realization},
  description={In systems theory, this is an implementation of a given
  input-output behavior as a state-space model.}}
\newglossaryentry{reference}{
  name={reference},
  description={The desired state.}}
\newglossaryentry{reference tracking}{
  name={state tracking},
  description={The ability of a feedback control system to make the state follow
    the reference.}}
\newglossaryentry{robustness}{
  name={robustness},
  description={The quality of a feedback control system to remain stable in
    response to disturbances and uncertainty.}}
\newglossaryentry{state}{
  name={state},
  description={A characteristic of a system (e.g., velocity) that can be used to
    determine the system's future behavior.}}
\newglossaryentry{state feedback}{
  name={state feedback},
  description={Uses state instead of output in feedback.}}
\newglossaryentry{steady-state error}{
  name={steady-state error},
  description={Error after system reaches equilibrium.}}
\newglossaryentry{stochastic process}{
  name={stochastic process},
  description={A process whose model is partially or completely defined by
    random variables.}}
\newglossaryentry{system}{
  name={system},
  description={Maps inputs to outputs through linear combination of states.}}
\newglossaryentry{time-invariant}{
  name={time-invariant},
  description={The system's fundamental response does not change over time.}}
\newglossaryentry{unity feedback}{
  name={unity feedback},
  description={A feedback network in a control system diagram with a feedback
    gain of 1.}}
