\section{Manual tuning}

These steps apply for both position and velocity PID controllers.

\begin{enumerate}
  \item Set $K_p$, $K_i$, and $K_d$ to zero.
  \item Increase $K_p$ until the \gls{output} starts to oscillate around the
    \gls{reference}.
  \item Increase $K_d$ as much as possible without introducing jittering in the
    \gls{system response}.
\end{enumerate}

If the \gls{controller} settles at an \gls{output} above or below the
\gls{reference}, increase $K_i$ such that the \gls{controller} reaches the
\gls{reference} in a reasonable amount of time. However, a steady-state
feedforward is preferred to integral control (especially for velocity PID
control).

\begin{remark}
  \textit{Note:} Adding an integral gain to the \gls{controller} is an incorrect
  way to eliminate \gls{steady-state error}. A better approach would be to tune
  it with an integrator added to the \gls{plant}, but this requires a
  \gls{model}. Since we are doing output-based rather than model-based control,
  our only option is to add an integrator to the \gls{controller}.
\end{remark}

Beware that if $K_i$ is too large, integral windup can occur. Following a large
change in \gls{reference}, the integral term can accumulate an error larger than
the maximal \gls{control input}. As a result, the system overshoots and
continues to increase until this accumulated error is unwound.
