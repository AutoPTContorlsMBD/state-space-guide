\subsection{Single-jointed arm}

\subsubsection{Equations of motion}

This single-jointed arm consists of a DC brushed motor attached to a pulley that
spins a straight bar in pitch.

\begin{figure}[H]
  \centering

  \begin{tikzpicture}[auto, >=latex', circuit ee IEC,
                      set resistor graphic=var resistor IEC graphic]
    % \draw [help lines] (-1,-3) grid (7,4);

    % Electrical equivalent circuit
    \draw (0,2) to [voltage source={direction info'={->}, info'=$V$}] (0,0);
    \draw (0,2) to [current direction={info=$I$}] (0,3);
    \draw (0,3) -- (0.5,3);
    \draw (0.5,3) to [resistor={info={$R$}}] (2,3);

    \draw (2,3) -- (2.5,3);
    \draw (2.5,3) to [voltage source={direction info'={->}, info'=$V_{emf}$}]
      (2.5,0);
    \draw (0,0) -- (2.5,0);

    % Motor
    \draw[fill=black] (2.4,1.05) rectangle (2.6,1.95);
    \draw[fill=white] (2.5,1.5) ellipse (0.3 and 0.3);

    % Transmission gear one
    \draw[fill=black!50] (3.95,1.5) ellipse (0.08 and 0.33);
    \draw[fill=black!50, color=black!50] (3.75,1.17) rectangle (3.95,1.83);
    \draw[fill=white] (3.75,1.5) ellipse (0.08 and 0.33);
    \draw (3.75,1.83) -- (3.95,1.83);
    \draw (3.75,1.17) -- (3.95,1.17) node[pos=0.5,below] {$G$};

    % Output shaft of motor
    \draw[fill=black!50] (2.8,1.45) rectangle (3.75,1.55);

    % Angular velocity arrow of drive -> transmission
    \draw[line width=0.7pt,<-] (3.2,1) arc (-30:30:1) node[above] {$\omega_m$};

    % Transmission gear two
    \draw[fill=black!50] (3.95,2.51) ellipse (0.13 and 0.67);
    \draw[fill=black!50, color=black!50] (3.75,1.83) rectangle (3.95,3.18);
    \draw[fill=white] (3.75,2.51) ellipse (0.13 and 0.67);
    \draw (3.75,3.18) -- (3.95,3.18);
    \draw (3.75,1.83) -- (3.95,1.83);

    \begin{scope}[xshift=5.075cm,yshift=2.4cm]
      % Single-jointed arm
      \draw[fill=white] (0,0) -- (0.1,-0.05) -- (0.35,1.45) -- (0.25,1.5)
        -- cycle;
      \draw[fill=black!50] (0.1,-0.05) -- (0.3,-0.05) -- (0.55,1.45) --
        (0.35,1.45) -- cycle;
      \draw[fill=white] (0.25,1.5) -- (0.35,1.45) -- (0.55,1.45) -- (0.45,1.5)
        -- cycle;

      % Arm length arrow
      \draw[line width=0.7pt,<->] (0.55,-0.05) -- node[right] {$l$} (0.8,1.45);

      % Mass label
      \draw (-0.05,1.2) node {$m$};
    \end{scope}

    % Transmission shaft from gear two to arm
    \draw[fill=black!50] (4.09,2.42) rectangle (5.15,2.52);

    % Angular velocity arrow between transmission and arm
    \draw[line width=0.7pt,->] (4.54,1.97) arc (-30:30:1) node[above]
      {$\omega_{arm}$};

    % Descriptions of subsystems under graphic
    \draw[decorate,decoration={brace,amplitude=10pt}]
      (3,-0.28) -- (-0.5,-0.28) node[midway,yshift=-20pt] {circuit};
    \draw[decorate,decoration={brace,amplitude=10pt}]
      (6.05,-0.28) -- (3.25,-0.28) node[midway,yshift=-20pt] {mechanics};
  \end{tikzpicture}

  \caption{Single-jointed arm system diagram}
  \label{fig:single_jointed_arm}
\end{figure}

Gear ratios are written as output over input, so $G$ is greater than one in
figure \ref{fig:single_jointed_arm}. \\

We will start with the equation derived earlier for a DC brushed motor, equation
(\ref{eq:motor_tau_V}).

\begin{equation*}
  V = \frac{\tau_m}{K_t} R + \frac{\omega_m}{K_v}
\end{equation*}

Solve for the angular acceleration. First, we'll rearrange the terms because
from inspection, $V$ is the model input, $\omega_m$ is the state, and $\tau_m$
contains the angular acceleration.

\begin{equation*}
  V = \frac{R}{K_t} \tau_m + \frac{1}{K_v} \omega_m
\end{equation*}

Solve for $\tau_m$.

\begin{align*}
  V &= \frac{R}{K_t} \tau_m + \frac{1}{K_v} \omega_m \\
  \frac{R}{K_t} \tau_m &= V - \frac{1}{K_v} \omega_m \\
  \tau_m &= \frac{K_t}{R} V - \frac{K_t}{K_v R} \omega_m
\end{align*}

Since $\tau_m G = \tau_{arm}$ and $\omega_m = G \omega_{arm}$

\begin{align}
  \left(\frac{\tau_{arm}}{G}\right) &= \frac{K_t}{R} V -
    \frac{K_t}{K_v R} (G \omega_f) \nonumber \\
  \frac{\tau_{arm}}{G} &= \frac{K_t}{R} V - \frac{G K_t}{K_v R} \omega_{arm}
    \nonumber \\
  \tau_{arm} &= \frac{G K_t}{R} V - \frac{G^2 K_t}{K_v R} \omega_{arm}
    \label{eq:tau_arm}
\end{align}

The angular velocity of the arm is defined as

\begin{equation}
  \tau_{arm} = J \dot{\omega}_{arm} \label{eq:tau_arm_def}
\end{equation}

where $J$ is the moment of inertia of the arm and $\dot{\omega}_{arm}$ is the
angular acceleration. Substitute equation (\ref{eq:tau_arm_def}) into equation
(\ref{eq:tau_arm}).

\begin{align}
  (J \dot{\omega}_{arm}) &= \frac{G K_t}{R} V - \frac{G^2 K_t}{K_v R}
    \omega_{arm} \nonumber \\
  \dot{\omega}_{arm} &= \frac{G K_t}{RJ} V - \frac{G^2 K_t}{K_v RJ} \omega_{arm}
    \label{eq:dot_omega_arm}
\end{align}

$J$ can be approximated as the moment of inertia of a thin rod rotating around
one end. Therefore

\begin{equation}
  J = \frac{1}{3}ml^2
\end{equation}

where $m$ is the mass of the arm and $l$ is the length of the arm.

\subsubsection{Continuous state-space model}

The position and velocity of the elevator can be written as

\begin{align}
  \dot{\theta}_{arm} &= \omega_{arm} \label{eq:arm_cont_ss_pos} \\
  \dot{\omega}_{arm} &= \dot{\omega}_{arm} \label{eq:arm_cont_ss_vel}
\end{align}

By equation (\ref{eq:dot_omega_arm})

\begin{equation*}
  \dot{\omega}_{arm} = -\frac{G^2 K_t}{K_v RJ} \omega_{arm} + \frac{G K_t}{RJ} V
\end{equation*}

\begin{align*}
  \dot{\mtx{x}} &= \mtx{A} \mtx{x} + \mtx{B} \mtx{u} \\
  \mtx{y} &= \mtx{C} \mtx{x} + \mtx{D} \mtx{u}
\end{align*}

\begin{align*}
  \mtx{x} &= \left[
  \begin{array}{c}
    \theta_{arm} \\
    \omega_{arm}
  \end{array}
  \right] \\
  \mtx{y} &= \omega_{arm} \\
  \mtx{u} &= V
\end{align*}

\begin{align}
  \mtx{A} &= \left[
  \begin{array}{cc}
    0 & 1 \\
    0 & -\frac{G^2 K_t}{K_v RJ}
  \end{array}
  \right] \\
  \mtx{B} &= \left[
  \begin{array}{c}
    0 \\
    \frac{G K_t}{RJ}
  \end{array}
  \right] \\
  \mtx{C} &= \left[
  \begin{array}{cc}
    1 & 0
  \end{array}
  \right] \\
  \mtx{D} &= 0
\end{align}

\subsubsection{Discrete state-space model}

The angle and angular velocity of the arm can be written as

\begin{align}
  \theta_{arm,k+1} &= \theta_{arm,k} + \omega_{arm,k} \Delta t
    \label{eq:arm_disc_ss_pos} \\
  \omega_{arm,k+1} &= \omega_{arm,k} + \dot{\omega}_{arm,k} \Delta t
    \label{eq:arm_disc_ss_vel}
\end{align}

where by equation (\ref{eq:dot_omega_arm})

\begin{equation*}
  \dot{\omega}_{arm,k} = -\frac{G^2 K_t}{K_v RJ} \omega_{arm,k} +
    \frac{G K_t}{RJ} V_k
\end{equation*}

Substitute this into equation (\ref{eq:arm_disc_ss_vel}).

\begin{align}
  \omega_{arm,k+1} &= \omega_{arm,k} + \left(-\frac{G^2 K_t}{K_v RJ}
    \omega_{arm,k} + \frac{G K_t}{RJ} V_k\right) \Delta t \nonumber \\
  \omega_{arm,k+1} &= \omega_{arm,k} - \frac{G^2 K_t}{K_v RJ} \omega_{arm,k}
    \Delta t + \frac{G K_t}{RJ} \Delta t V_k \nonumber \\
  \omega_{arm,k+1} &= \left(1 - \frac{G^2 K_t}{K_v RJ} \Delta t\right)
    \omega_{arm,k} + \frac{G K_t}{RJ} \Delta t V_k \nonumber \\
\end{align}

\begin{align*}
  \mtx{x}_{k+1} &= \mtx{A} \mtx{x}_k + \mtx{B} \mtx{u}_k \\
  \mtx{y}_{k+1} &= \mtx{C} \mtx{x}_k + \mtx{D} \mtx{u}_k
\end{align*}

\begin{align*}
  \mtx{x}_k &= \left[
  \begin{array}{c}
    \theta_{arm,k} \\
    \omega_{arm,k}
  \end{array}
  \right] \\
  \mtx{y}_k &= \omega_{arm,k} \\
  \mtx{u}_k &= V_k
\end{align*}

\begin{align}
  \mtx{A} &= \left[
  \begin{array}{cc}
    1 & \Delta t \\
    0 & 1 - \frac{G^2 K_t}{K_v RJ} \Delta t
  \end{array}
  \right] \\
  \mtx{B} &= \left[
  \begin{array}{c}
    0 \\
    \frac{G K_t}{RJ} \Delta t
  \end{array}
  \right] \\
  \mtx{C} &= \left[
  \begin{array}{cc}
    1 & 0
  \end{array}
  \right] \\
  \mtx{D} &= 0
\end{align}
